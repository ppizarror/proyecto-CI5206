% Template:     Informe/Reporte LaTeX
% Documento:    Archivo principal
% Versión:      6.0.1 (21/10/2018)
% Codificación: UTF-8
%
% Autor: Pablo Pizarro R. @ppizarror
%        Facultad de Ciencias Físicas y Matemáticas
%        Universidad de Chile
%        pablo.pizarro@ing.uchile.cl, ppizarror.com
%
% Manual template: [http://latex.ppizarror.com/Template-Informe/]
% Licencia MIT:    [https://opensource.org/licenses/MIT/]

% CREACIÓN DEL DOCUMENTO
\documentclass[letterpaper,11pt]{article} % Articulo tamaño carta, 11pt
\usepackage[utf8]{inputenc} % Codificación UTF-8

% INFORMACIÓN DEL DOCUMENTO
\def\titulodelinforme {Cubicaciones}
\def\temaatratar {Proyecto de Hormigón Armado - Entrega N°3}

\def\autordeldocumento {Grupo N1}
\def\nombredelcurso {Proyecto de Hormigón Armado}
\def\codigodelcurso {CI5206-2}

\def\nombreuniversidad {Universidad de Chile}
\def\nombrefacultad {Facultad de Ciencias Físicas y Matemáticas}
\def\departamentouniversidad {Departamento de Ingeniería Civil}
\def\imagendepartamento {departamentos/dic}
\def\imagendepartamentoescala {0.2}
\def\localizacionuniversidad {Santiago, Chile}

% INTEGRANTES, PROFESORES Y FECHAS
\def\tablaintegrantes {
\begin{tabular}{ll}
	Integrantes:
	& \begin{tabular}[t]{@{}l@{}}
		Mauricio Leal V. \\
		Pablo Pizarro R. \\
		Ignacio Yáñez G.
	\end{tabular} \\
	Profesor:
	& \begin{tabular}[t]{@{}l@{}}
		Juan Mendoza V.
	\end{tabular} \\
	Auxiliar:
	& \begin{tabular}[t]{@{}l@{}}
		Felipe Andrade T.
	\end{tabular} \\
	& \\
	\multicolumn{2}{l}{Fecha de entrega: 24 de octubre de 2018} \\
	\multicolumn{2}{l}{\localizacionuniversidad}
\end{tabular}}{
}

% CONFIGURACIONES
\input{lib/config}

% IMPORTACIÓN DE LIBRERÍAS
\input{lib/env/imports}

% IMPORTACIÓN DE FUNCIONES Y ENTORNOS
\input{lib/cmd/all}

% IMPORTACIÓN DE ESTILOS
\input{lib/style/all}

% CONFIGURACIÓN INICIAL DEL DOCUMENTO
\input{lib/cfg/init}
\usepackage{adjustbox}

% INICIO DE LAS PÁGINAS
\begin{document}

% PORTADA
\input{lib/page/portrait} % Se puede borrar

% CONFIGURACIÓN DE PÁGINA Y ENCABEZADOS
\input{lib/cfg/page}

% TABLA DE CONTENIDOS - ÍNDICE
% Template:     Informe/Reporte LaTeX
% Documento:    Índice
% Versión:      6.0.1 (21/10/2018)
% Codificación: UTF-8
%
% Autor: Pablo Pizarro R. @ppizarror
%        Facultad de Ciencias Físicas y Matemáticas
%        Universidad de Chile
%        pablo.pizarro@ing.uchile.cl, ppizarror.com
%
% Manual template: [http://latex.ppizarror.com/Template-Informe/]
% Licencia MIT:    [https://opensource.org/licenses/MIT/]

\ifthenelse{\equal{\showindex}{true}}{
	\newpage
	\begingroup
	\sectionfont{\color{\indextitlecolor} \fontsizetitlei \styletitlei \selectfont}
	\ifthenelse{\equal{\addindextobookmarks}{true}}{
		\belowpdfbookmark{\nomltcont}{contents}}{
	}
	\tocloftpagestyle{fancy}
	\ifthenelse{\equal{\showdotontitles}{true}}{
		\def\cftsecaftersnum {.}
		\def\cftsubsecaftersnum {.}
		\def\cftsubsubsecaftersnum {.}
		\def\cftsubsubsubsecaftersnum {.}
		}{
	}
	\def\cftfigaftersnum {\charafterobjectindex\enspace}
	\def\cftsubfigaftersnum {\charafterobjectindex\enspace}
	\def\cfttabaftersnum {\charafterobjectindex\enspace}
	\def\cftlstlistingaftersnum {\charafterobjectindex\enspace}
	\renewcommand{\cftdot}{\charnumpageindex}
	\ifthenelse{\equal{\showlinenumbers}{true}}{
		\nolinenumbers}{
	}
	\ifthenelse{\equal{\objectindexindent}{true}}{
		\def\cftlstlistingindent {1.495em}
	}{
		\setlength{\cfttabindent}{0in}
		\setlength{\cftfigindent}{0in}
		\setlength{\cftsubfigindent}{0in}
		\setlength{\cftfigindent}{0in}
		\def\cftlstlistingindent {0.01em}
	}
	\ifthenelse{\equal{\equalmarginnumobject}{true}}{
		\ifthenelse{\equal{\showsectioncaption}{none}}{
			\def\cftdefautnumwidth {2.3em}
		}{
		\ifthenelse{\equal{\showsectioncaption}{sec}}{
			\def\cftdefautnumwidth {3.0em}
		}{
		\ifthenelse{\equal{\showsectioncaption}{ssec}}{
			\def\cftdefautnumwidth {3.8em}
		}{
		\ifthenelse{\equal{\showsectioncaption}{sssec}}{
			\def\cftdefautnumwidth {4.3em}
		}{
			\throwbadconfig{Valor configuracion incorrecto}{\showsectioncaption}{none,sec,ssec,sssec}}}}
		}
		\def\cftfignumwidth {\cftdefautnumwidth}
		\def\cftsubfignumwidth {\cftdefautnumwidth}
		\def\cfttabnumwidth {\cftdefautnumwidth}
		\def\cftlstlistingnumwidth {\cftdefautnumwidth}}{
	}
	\ifthenelse{\equal{\showindexofcontents}{true}}{\tableofcontents}{}
	\iftotalfigures
		\ifthenelse{\equal{\showindexoffigures}{true}}{
			\ifthenelse{\equal{\indexforcenewpage}{true}}{\newpage}{}
			\listoffigures
		}{}
	\fi
	\iftotaltables
		\ifthenelse{\equal{\showindexoftables}{true}}{
			\ifthenelse{\equal{\indexforcenewpage}{true}}{\newpage}{}
			\listoftables
		}{}
	\fi
	\iftotallstlistings
		\ifthenelse{\equal{\showindexofcode}{true}}{
			\ifthenelse{\equal{\indexforcenewpage}{true}}{\newpage}{}
			\lstlistoflistings
		}{}
	\fi
	\endgroup
	\ifthenelse{\equal{\addemptypagetwosides}{true}}{
		\vfill
		\checkoddpage
		\ifoddpage
		\else
			\newpage
			\null
			\thispagestyle{empty}
			\newpage
			\addtocounter{page}{-1}
		\fi}{
	}
}{}
 % Se puede borrar

% CONFIGURACIONES FINALES
% Template:     Informe/Reporte LaTeX
% Documento:    Configuraciones finales
% Versión:      6.1.6 (14/12/2018)
% Codificación: UTF-8
%
% Autor: Pablo Pizarro R. @ppizarror
%        Facultad de Ciencias Físicas y Matemáticas
%        Universidad de Chile
%        pablo.pizarro@ing.uchile.cl, ppizarror.com
%
% Manual template: [https://latex.ppizarror.com/Template-Informe/]
% Licencia MIT:    [https://opensource.org/licenses/MIT/]

\markboth{}{}
\newpage
\ifthenelse{\equal{\disablehfrightmark}{false}}{
	\ifthenelse{\equal{\hfstyle}{style1}}{
		\fancyhead[L]{\nouppercase{\leftmark}}}{
	}
	\ifthenelse{\equal{\hfstyle}{style2}}{
		\fancyhead[L]{\nouppercase{\leftmark}}}{
	}
	\ifthenelse{\equal{\hfstyle}{style4}}{
		\fancyhead[L]{\nouppercase{\leftmark}}}{
	}
	\ifthenelse{\equal{\hfstyle}{style5}}{
		\fancyhead[R]{\nouppercase{\leftmark}}}{
	}
	\ifthenelse{\equal{\hfstyle}{style9}}{
		\fancyhead[L]{\nouppercase{\leftmark}}}{
	}
	\ifthenelse{\equal{\hfstyle}{style10}}{
		\fancyhead[L]{\nouppercase{\leftmark}}}{
	}
\ifthenelse{\equal{\hfstyle}{style11}}{
		\fancyhead[L]{\nouppercase{\leftmark}}}{
	}
\ifthenelse{\equal{\hfstyle}{style14}}{
		\fancyhead[L]{\nouppercase{\leftmark}}}{
	}}{
}
\sectionfont{\color{\titlecolor} \fontsizetitle \styletitle \selectfont}
\subsectionfont{\color{\subtitlecolor} \fontsizesubtitle \stylesubtitle \selectfont}
\subsubsectionfont{\color{\subsubtitlecolor} \fontsizesubsubtitle \stylesubsubtitle \selectfont}
\titleformat{\subsubsubsection}{\color{\ssstitlecolor} \normalfont \fontsizessstitle \stylessstitle}{\thesubsubsubsection}{1em}{}
\titlespacing*{\subsubsubsection}{0pt}{3.25ex plus 1ex minus .2ex}{1.5ex plus .2ex}
\ifthenelse{\equal{\showsectioncaption}{none}}{
}{
\ifthenelse{\equal{\showsectioncaption}{sec}}{
	\counterwithin{equation}{section}
	\counterwithin{figure}{section}
	\counterwithin{lstlisting}{section}
	\counterwithin{table}{section}
}{
\ifthenelse{\equal{\showsectioncaption}{ssec}}{
	\counterwithin{equation}{subsection}
	\counterwithin{figure}{subsection}
	\counterwithin{lstlisting}{subsection}
	\counterwithin{table}{subsection}
}{
\ifthenelse{\equal{\showsectioncaption}{sssec}}{
	\counterwithin{equation}{subsubsection}
	\counterwithin{figure}{subsubsection}
	\counterwithin{lstlisting}{subsubsection}
	\counterwithin{table}{subsubsection}
}{
\ifthenelse{\equal{\showsectioncaption}{ssssec}}{
	\counterwithin{equation}{subsubsubsection}
	\counterwithin{figure}{subsubsubsection}
	\counterwithin{lstlisting}{subsubsubsection}
	\counterwithin{table}{subsubsubsection}
}{
	\throwbadconfig{Valor configuracion incorrecto}{\showsectioncaption}{none,sec,ssec,sssec,ssssec}
}}}}}
\ifthenelse{\equal{\predocuseromannumber}{true}}{
	\renewcommand{\thepage}{\arabic{page}}}{
}
\ifthenelse{\equal{\resetpagnumafterindex}{true}}{
	\setcounter{page}{1}}{
}
\setcounter{section}{0}
\setcounter{footnote}{0}
\ifthenelse{\equal{\showlinenumbers}{true}}{
	\linenumbers}{
}


% ======================= INICIO DEL DOCUMENTO =======================

\section{Pesos involucrados en las losas}

Para el cálculo de cubicaciones y pesos que reciben las losas de cada piso se consideran los siguientes elementos:
\begin{itemize}
    \item Muros y Tabiques.
    \item Vigas normales, invertidas y semi-invertidas.
    \item Estuco, Enlucido y Sobrelosa.
\end{itemize}

Para el caso de los muros, estos deben llevar una capa de estuco tal y como se muestra a continuación:

\insertimage[]{img/estucos_muro}{width=9cm}{Disposición de capa de estucos en muros.}

Para las vigas, dependiendo de si estas son normales o invertidas, se tendrán dos configuraciones distintas para el estuco y enlucido que estas llevan:

\begin{images}[]{Disposición de estuco y enlucido.}
    \addimage{img/vigainvertida}{width=5.5cm}{Viga Invertida.}
    \addimage{img/viganormal}{width=6.5cm}{Viga Normal.}
\end{images}

Por último, para las losas, la disposición de enlucido y sobrelosa se ejemplifica a continuación:

\insertimage[]{img/losa}{width=10cm}{Disposición de enlucido y sobrelosa.}
\section{Cubicaciones de elementos}
    \begin{itemize}
        \item Características de los materiales \\
        A continuación se muestran los pesos  y espesores de los elementos utilizados.
         
            \begin{table}[H]
              \centering
              \caption{Pesos Volumétricos.}
                \begin{tabular}{|c|c|}
                \hline
                \textbf{Elemento} &
                  \boldmath{}\textbf{Densidad [$T/m^3$]}\unboldmath{}
                  \bigstrut\\
                \hline
                $\gamma_{Hormigon}$ &
                  2,5
                  \bigstrut\\
                \hline
                $\gamma_{Estuco}$ &
                  2,0
                  \bigstrut\\
                \hline
                $\gamma_{Enlucido}$ &
                  2,0
                  \bigstrut\\
                \hline
                $\gamma_{Sobrelosa}$ &
                  1,5
                  \bigstrut\\
                \hline
                \end{tabular}%
              \label{volumen}%
            \end{table}%

            \begin{table}[H]
              \centering
              \caption{Pesos por área.}
                \begin{tabular}{|c|c|}
                \hline
                \textbf{Elemento} & \boldmath{}\textbf{Peso [$T/m^2$]}\unboldmath{} \bigstrut\\
                \hline
                Peso tabique & 0.100 \bigstrut\\
                \hline
                Sobrecarga uso vivienda & 0.200 \bigstrut\\
                \hline
                Sobrecarga escala uso público & 0.400 \bigstrut\\
                \hline
                Sobrecarga estacionamientos & 0.500 \bigstrut\\
                \hline
                \end{tabular}%
              \label{Area}%
            \end{table}

            \begin{table}[H]
              \centering
              \caption{Espesores elementos.}
                \begin{tabular}{|c|c|}
                \hline
                \textbf{Elemento} &
                  \textbf{Espesor (m)}
                  \bigstrut\\
                \hline
                Estuco &
                  0,025
                  \bigstrut\\
                \hline
                Enlucido &
                  0,02
                  \bigstrut\\
                \hline
                Sobrelosa &
                  0,05
                  \bigstrut\\
                \hline
                \end{tabular}%
              \label{Espesor}%
            \end{table}%

        \newpage
        \item Cubicaciones Losas\\
        Para la determinación del Peso Sísmico de las losas, se obtienen los siguientes resultados por piso:
         
            \begin{table}[H]
              \centering
              \caption{Espesores de losa por piso.}
                \begin{tabular}{|c|c|}
            \cline{2-2}    \multicolumn{1}{c|}{} &
                  \textbf{Espesor (cm)}
                  \bigstrut\\
                \hline
                Piso 1 &
                  0,17
                  \bigstrut[t]\\
                Piso 2  &
                  0,16
                  \\
                Piso 3  &
                  0,16
                  \\
                Piso Tipo &
                  0,16
                  \\
                Cubierta &
                  0,16
                  \bigstrut[b]\\
                \hline
                \end{tabular}%
              \label{Espesor losas}%
            \end{table}%
         
            \begin{table}[H]
              \centering
              \caption{Cubicación y pesos sísmicos losas.}
                \resizebox{\textwidth}{!}{ 
                \begin{tabular}{ccccc|c|c|c|c|c|}
                \hline
                \multicolumn{1}{|p{5.39em}|}{\textbf{Losa}} &
                  \multicolumn{1}{p{5.39em}|}{\textbf{Espesor (m)}} &
                  \multicolumn{1}{p{5.39em}|}{\textbf{Área losa interior (m2)}} &
                  \multicolumn{1}{p{5.39em}|}{\textbf{Descuento viga (m2)}} &
                  \multicolumn{1}{p{5.39em}|}{\textbf{Área losa interior final (m2)}} &
                  \multicolumn{1}{p{5.39em}|}{\textbf{Área estacionamientos (m2)}} &
                  \multicolumn{1}{p{5.39em}|}{\textbf{Peso Hormigón (T)}} &
                  \multicolumn{1}{p{5.39em}|}{\textbf{Peso Enlucido+SL (T)}} &
                  \multicolumn{1}{p{5.39em}|}{\textbf{SC (T)}} &
                  \multicolumn{1}{p{5.39em}|}{\textbf{PSl Losa (T)}}
                  \bigstrut\\
                \hline
                \multicolumn{1}{|c|}{24} &
                  \multicolumn{1}{c|}{0,16} &
                  \multicolumn{1}{c|}{68,70} &
                  \multicolumn{1}{c|}{26,37} &
                  42,33 &
                  0 &
                  27,48 &
                  4,87 &
                  13,74 &
                  46,09
                  \bigstrut[t]\\
                \multicolumn{1}{|c|}{4-23} &
                  \multicolumn{1}{c|}{0,16} &
                  \multicolumn{1}{c|}{410,20} &
                  \multicolumn{1}{c|}{15,17} &
                  395,03 &
                  0 &
                  164,08 &
                  45,43 &
                  82,04 &
                  291,55
                  \\
                \multicolumn{1}{|c|}{3} &
                  \multicolumn{1}{c|}{0,16} &
                  \multicolumn{1}{c|}{410,70} &
                  \multicolumn{1}{c|}{14,93} &
                  395,77 &
                  0 &
                  164,28 &
                  45,51 &
                  82,14 &
                  291,93
                  \\
                \multicolumn{1}{|c|}{2} &
                  \multicolumn{1}{c|}{0,16} &
                  \multicolumn{1}{c|}{177,55} &
                  \multicolumn{1}{c|}{26,77} &
                  150,77 &
                  255,93 &
                  173,39 &
                  17,34 &
                  86,70 &
                  277,42
                  \\
                \multicolumn{1}{|c|}{1} &
                  \multicolumn{1}{c|}{0,17} &
                  \multicolumn{1}{c|}{205,29} &
                  \multicolumn{1}{c|}{38,33} &
                  166,96 &
                  514,18 &
                  305,77 &
                  19,20 &
                  143,89 &
                  468,86
                  \bigstrut[b]\\
                \hline
                 &
                   &
                   &
                   &
                   &
                  TOTAL &
                  3952,55 &
                  995,50 &
                  1967,28 &
                  6915,32
                  \bigstrut\\
            \cline{6-10}    \end{tabular}}%
              \label{Pesoslosa}%
            \end{table}%

          
        \item Cubicaciones Muros y Tabiques\\
        Para la determinación de los pesos sísmicos de muros y tabiques en cada piso se obtienen los siguientes resultados:
        
        \begin{table}[H]
          \centering
          \caption{Cubicaciones y pesos sísmicos de muros.}
          \resizebox{\textwidth}{!}{ 
            \begin{tabular}{cccccc|c|c|c|c|}
            \hline
            \multicolumn{1}{|c|}{\textbf{Piso}} &
              \multicolumn{1}{c|}{\textbf{ex (m)}} &
              \multicolumn{1}{c|}{\textbf{ey (m)}} &
              \multicolumn{1}{c|}{\textbf{Lx (m)}} &
              \multicolumn{1}{c|}{\textbf{Ly (m)}} &
              \textbf{A planta muro (m2)} &
              \textbf{Área Estuco (m2)} &
              \textbf{Peso Hormigón Muro (T)} &
              \textbf{Peso Estuco (T)} &
              \textbf{Peso Muro + estuco(T)}
              \bigstrut\\
            \hline
            \multicolumn{1}{|c|}{24} &
              \multicolumn{1}{c|}{0,2} &
              \multicolumn{1}{c|}{0,2} &
              \multicolumn{1}{c|}{26,92} &
              \multicolumn{1}{c|}{17,6} &
              8,90 &
              2,25 &
              51,20 &
              0,11 &
              51,31
              \bigstrut[t]\\
            \multicolumn{1}{|c|}{8-23} &
              \multicolumn{1}{c|}{0,2} &
              \multicolumn{1}{c|}{0,2} &
              \multicolumn{1}{c|}{54,48} &
              \multicolumn{1}{c|}{58,86} &
              22,67 &
              5,69 &
              130,34 &
              0,28 &
              130,63
              \\
            \multicolumn{1}{|c|}{7} &
              \multicolumn{1}{c|}{0,25} &
              \multicolumn{1}{c|}{0,2} &
              \multicolumn{1}{c|}{54,48} &
              \multicolumn{1}{c|}{58,86} &
              25,39 &
              5,69 &
              146,00 &
              0,28 &
              146,29
              \\
            \multicolumn{1}{|c|}{6} &
              \multicolumn{1}{c|}{0,25} &
              \multicolumn{1}{c|}{0,25} &
              \multicolumn{1}{c|}{54,48} &
              \multicolumn{1}{c|}{58,86} &
              28,34 &
              5,70 &
              162,93 &
              0,28 &
              163,21
              \\
            \multicolumn{1}{|c|}{5} &
              \multicolumn{1}{c|}{0,25} &
              \multicolumn{1}{c|}{0,25} &
              \multicolumn{1}{c|}{54,48} &
              \multicolumn{1}{c|}{58,86} &
              28,34 &
              5,70 &
              162,93 &
              0,28 &
              163,21
              \\
            \multicolumn{1}{|c|}{4} &
              \multicolumn{1}{c|}{0,25} &
              \multicolumn{1}{c|}{0,25} &
              \multicolumn{1}{c|}{54,48} &
              \multicolumn{1}{c|}{58,86} &
              28,34 &
              5,70 &
              162,93 &
              0,28 &
              163,21
              \\
            \multicolumn{1}{|c|}{3} &
              \multicolumn{1}{c|}{0,25} &
              \multicolumn{1}{c|}{0,25} &
              \multicolumn{1}{c|}{58,32} &
              \multicolumn{1}{c|}{58,86} &
              29,30 &
              5,89 &
              168,45 &
              0,29 &
              168,74
              \\
            \multicolumn{1}{|c|}{2} &
              \multicolumn{1}{c|}{0,25} &
              \multicolumn{1}{c|}{0,3} &
              \multicolumn{1}{c|}{65,35} &
              \multicolumn{1}{c|}{50,48} &
              31,48 &
              5,82 &
              181,02 &
              0,29 &
              181,31
              \\
            \multicolumn{1}{|c|}{1} &
              \multicolumn{1}{c|}{0,25} &
              \multicolumn{1}{c|}{0,3} &
              \multicolumn{1}{c|}{69,98} &
              \multicolumn{1}{c|}{54,98} &
              33,99 &
              6,28 &
              195,44 &
              0,31 &
              195,75
              \\
            \multicolumn{1}{|c|}{-1} &
              \multicolumn{1}{c|}{0,25} &
              \multicolumn{1}{c|}{0,3} &
              \multicolumn{1}{c|}{78,44} &
              \multicolumn{1}{c|}{61,53} &
              38,07 &
              7,03 &
              218,90 &
              0,35 &
              219,25
              \bigstrut[b]\\
            \hline
             &
               &
               &
               &
               &
               &
              \textbf{Total} &
              \textbf{3535,24} &
              \textbf{7,06} &
              \textbf{3542,29}
              \bigstrut\\
        \cline{7-10}    \end{tabular}}%
          \label{muros}%
        \end{table}%

        \begin{table}[H]
          \centering
          \caption{Cubicaciones y pesos sísmicos tabiques.}
          \resizebox{\textwidth}{!}{ 
            \begin{tabular}{ccccc|c|c|}
            \hline
            \multicolumn{1}{|c|}{\textbf{Piso}} &
              \multicolumn{1}{c|}{\textbf{ex (m)}} &
              \multicolumn{1}{c|}{\textbf{ey (m)}} &
              \multicolumn{1}{c|}{\textbf{Lx (m)}} &
              \textbf{Ly (m)} &
              \textbf{Área tabiques (m2)} &
              \textbf{Peso Tabique (T)}
              \bigstrut\\
            \hline
            \multicolumn{1}{|c|}{24} &
              \multicolumn{1}{c|}{0,2} &
              \multicolumn{1}{c|}{0,2} &
              \multicolumn{1}{c|}{10,25} &
              8,7 &
              3,79 &
              0,379
              \bigstrut[t]\\
            \multicolumn{1}{|c|}{8-23} &
              \multicolumn{1}{c|}{0,2} &
              \multicolumn{1}{c|}{0,2} &
              \multicolumn{1}{c|}{60,46} &
              39,19 &
              19,93 &
              1,993
              \\
            \multicolumn{1}{|c|}{7} &
              \multicolumn{1}{c|}{0,2} &
              \multicolumn{1}{c|}{0,2} &
              \multicolumn{1}{c|}{60,46} &
              39,19 &
              19,93 &
              1,993
              \\
            \multicolumn{1}{|c|}{6} &
              \multicolumn{1}{c|}{0,2} &
              \multicolumn{1}{c|}{0,2} &
              \multicolumn{1}{c|}{60,46} &
              39,19 &
              19,93 &
              1,993
              \\
            \multicolumn{1}{|c|}{5} &
              \multicolumn{1}{c|}{0,25} &
              \multicolumn{1}{c|}{0,25} &
              \multicolumn{1}{c|}{60,46} &
              39,19 &
              24,91 &
              2,49
              \\
            \multicolumn{1}{|c|}{4} &
              \multicolumn{1}{c|}{0,25} &
              \multicolumn{1}{c|}{0,25} &
              \multicolumn{1}{c|}{60,46} &
              39,19 &
              24,91 &
              2,49
              \\
            \multicolumn{1}{|c|}{3} &
              \multicolumn{1}{c|}{0,25} &
              \multicolumn{1}{c|}{0,25} &
              \multicolumn{1}{c|}{60,46} &
              39,19 &
              24,91 &
              2,49
              \\
            \multicolumn{1}{|c|}{2} &
              \multicolumn{1}{c|}{0,25} &
              \multicolumn{1}{c|}{0,3} &
              \multicolumn{1}{c|}{36,07} &
              38,7 &
              20,63 &
              2,06
              \\
            \multicolumn{1}{|c|}{1} &
              \multicolumn{1}{c|}{0,25} &
              \multicolumn{1}{c|}{0,3} &
              \multicolumn{1}{c|}{38,11} &
              31,64 &
              19,02 &
              1,90
              \\
            \multicolumn{1}{|c|}{-1} &
              \multicolumn{1}{c|}{0,25} &
              \multicolumn{1}{c|}{0,3} &
              \multicolumn{1}{c|}{22,62} &
              16,9 &
              10,73 &
              1,07
              \bigstrut[b]\\
            \hline
             &
               &
               &
               &
               &
              \textbf{Total} &
              \textbf{48,76}
              \bigstrut\\
        \cline{6-7}    \end{tabular}}%
          \label{tabiques}%
        \end{table}%
        
        \item Cubicaciones Vigas\\
        Para la determinación de los pesos sísmicos de vigas en cada piso se obtienen los siguientes resultados:
        
        \begin{table}[H]
          \centering
          \caption{Cubicaciones y Pesos Sísmicos de vigas.}
          \resizebox{\textwidth}{!}{
            \begin{tabular}{ccc|c|c|c|c|}
            \hline
            \multicolumn{1}{|c|}{\textbf{Piso}} &
              \multicolumn{1}{c|}{\textbf{Volumen vigas (m3)}} &
              \textbf{Área estuco (m2)} &
              \textbf{Volumen Estuco (m3)} &
              \textbf{Peso Hormigón Vigas (T)} &
              \textbf{Peso Estuco Vigas (T)} &
              \textbf{Peso Total (T)}
              \bigstrut\\
            \hline
            \multicolumn{1}{|c|}{24} &
              \multicolumn{1}{c|}{23,73} &
              242,60 &
              6,07 &
              59,33 &
              12,13 &
              71,46
              \bigstrut[t]\\
            \multicolumn{1}{|c|}{4-23} &
              \multicolumn{1}{c|}{14,77} &
              150,72 &
              3,77 &
              36,92 &
              7,54 &
              44,46
              \\
            \multicolumn{1}{|c|}{3} &
              \multicolumn{1}{c|}{14,55} &
              148,52 &
              3,71 &
              36,38 &
              7,43 &
              43,81
              \\
            \multicolumn{1}{|c|}{2} &
              \multicolumn{1}{c|}{16,18} &
              56,52 &
              1,41 &
              40,45 &
              2,83 &
              43,27
              \\
            \multicolumn{1}{|c|}{1} &
              \multicolumn{1}{c|}{19,74} &
              41,16 &
              1,03 &
              49,35 &
              2,06 &
              51,41
              \bigstrut[b]\\
            \hline
             &
               &
               &
              \textbf{TOTAL} &
              \textbf{923,96} &
              \textbf{175,16} &
              \textbf{1099,13}
              \bigstrut\\
        \cline{4-7}    \end{tabular}}%
          \label{vigas}%
        \end{table}%

    \end{itemize}
   
\newpage 
\section{Planilla resumen}

A continuación se entrega un resumen de las cubicaciones anteriores con los parámetros más relevantes.

\begin{table}[H]
  \centering
   \caption{Planilla resumen de cubicaciones.}
    \begin{adjustbox}{angle=90}
    \resizebox{1\textwidth}{!}{
    \begin{tabular}{|ccccccccccccc|}
    \hline
    \multicolumn{1}{|p{3em}}{\textbf{Losa}} & \textbf{Altura (m)} & \textbf{Área losa (m2)} & \textbf{Xg losa (m)} & \textbf{Yg losa (m)} & \textbf{I polar (m4)} & \textbf{Ps losa (T)} & \textbf{P vigas (T)} & \textbf{P muros (T)} & \textbf{Ps nivel (T)} & \boldmath{}\textbf{M tras (T $\cdot$ s2/m)}\unboldmath{} & \boldmath{}\textbf{M rot (T $\cdot$ s2$\cdot$ m)}\unboldmath{} & \textbf{q (T/m2)} \bigstrut\\
    \hline
    24    & 2.46  & 42.33 & 19.07 & 15.60 & 726.26 & 37.64 & 71.46 & 51.69 & 201.25 & 20.536 & 352.36 & 2.93 \bigstrut[t]\\
    \textit{23} & 2.46  & 395.03 & 15.95 & 16.26 & 37835.55 & 258.89 & 44.46 & 132.62 & 435.97 & 44.486 & 4260.82 & 1.06 \\
    \textit{22} & 2.46  & 395.03 & 15.95 & 16.26 & 37835.55 & 258.89 & 44.46 & 132.62 & 435.97 & 44.486 & 4260.82 & 1.06 \\
    \textit{21} & 2.46  & 395.03 & 15.95 & 16.26 & 37835.55 & 258.89 & 44.46 & 132.62 & 435.97 & 44.486 & 4260.82 & 1.06 \\
    \textit{20} & 2.46  & 395.03 & 15.95 & 16.26 & 37835.55 & 258.89 & 44.46 & 132.62 & 435.97 & 44.486 & 4260.82 & 1.06 \\
    \textit{19} & 2.46  & 395.03 & 15.95 & 16.26 & 37835.55 & 258.89 & 44.46 & 132.62 & 435.97 & 44.486 & 4260.82 & 1.06 \\
    \textit{18} & 2.46  & 395.03 & 15.95 & 16.26 & 37835.55 & 258.89 & 44.46 & 132.62 & 435.97 & 44.486 & 4260.82 & 1.06 \\
    \textit{17} & 2.46  & 395.03 & 15.95 & 16.26 & 37835.55 & 258.89 & 44.46 & 132.62 & 435.97 & 44.486 & 4260.82 & 1.06 \\
    \textit{16} & 2.46  & 395.03 & 15.95 & 16.26 & 37835.55 & 258.89 & 44.46 & 132.62 & 435.97 & 44.486 & 4260.82 & 1.06 \\
    \textit{15} & 2.46  & 395.03 & 15.95 & 16.26 & 37835.55 & 258.89 & 44.46 & 132.62 & 435.97 & 44.486 & 4260.82 & 1.06 \\
    \textit{14} & 2.46  & 395.03 & 15.95 & 16.26 & 37835.55 & 258.89 & 44.46 & 132.62 & 435.97 & 44.486 & 4260.82 & 1.06 \\
    \textit{13} & 2.46  & 395.03 & 15.95 & 16.26 & 37835.55 & 258.89 & 44.46 & 132.62 & 435.97 & 44.486 & 4260.82 & 1.06 \\
    \textit{12} & 2.46  & 395.03 & 15.95 & 16.26 & 37835.55 & 258.89 & 44.46 & 132.62 & 435.97 & 44.486 & 4260.82 & 1.06 \\
    \textit{11} & 2.46  & 395.03 & 15.95 & 16.26 & 37835.55 & 258.89 & 44.46 & 132.62 & 435.97 & 44.486 & 4260.82 & 1.06 \\
    \textit{10} & 2.46  & 395.03 & 15.95 & 16.26 & 37835.55 & 258.89 & 44.46 & 132.62 & 435.97 & 44.486 & 4260.82 & 1.06 \\
    \textit{9} & 2.46  & 395.03 & 15.95 & 16.26 & 37835.55 & 258.89 & 44.46 & 132.62 & 435.97 & 44.486 & 4260.82 & 1.06 \\
    \textit{8} & 2.46  & 395.03 & 15.95 & 16.26 & 37835.55 & 258.89 & 44.46 & 132.62 & 435.97 & 44.486 & 4260.82 & 1.06 \\
    \textit{7} & 2.46  & 395.03 & 15.95 & 16.26 & 37835.55 & 258.89 & 44.46 & 132.62 & 452.26 & 46.149 & 4420.06 & 1.10 \\
    \textit{6} & 2.46  & 395.03 & 15.95 & 16.26 & 37835.55 & 258.89 & 44.46 & 165.20 & 468.80 & 47.837 & 4581.73 & 1.14 \\
    \textit{5} & 2.46  & 395.03 & 15.95 & 16.26 & 37835.55 & 258.89 & 44.46 & 165.70 & 469.05 & 47.862 & 4584.16 & 1.14 \\
    \textit{4} & 2.46  & 395.03 & 15.95 & 16.26 & 37835.55 & 258.89 & 44.46 & 165.70 & 471.81 & 48.144 & 4611.18 & 1.15 \\
    3     & 2.46  & 395.77 & 15.95 & 16.26 & 37835.55 & 259.27 & 43.81 & 171.23 & 480.38 & 49.018 & 4686.09 & 1.17 \\
    2     & 2.46  & 406.70 & 10.25 & 20.00 & 55075.84 & 241.57 & 43.27 & 183.37 & 475.35 & 48.505 & 6568.66 & 1.10 \\
    1     & 2.47  & 681.13 & 16.41 & 19.81 & 94413.26 & 410.11 & 51.41 & 197.65 & 670.51 & 68.419 & 9483.75 & 0.93 \bigstrut[b]\\
    \hline
    \end{tabular}
    }
    \end{adjustbox}
 \label{plresumen}%
\end{table}

\newpage
\section{Comentarios y conclusiones}

%- Resultados por piso para valor de q promedio resultante, respecto de valor Edificios Chilenos.
%- Comentar que función cumple el centro de gravedad de la losa.
%- Uso estucos en la actualidad.
\begin{itemize}
    \item Es posible observar en la tabla resumen \ref{plresumen} que los cortes promedio por pisos son parecidos a los vistos en clase, el cual oscila entre 1 y 1.1 $T/m^2$ para edificios habitacionales Chilenos de hormigón armado. Un caso interesante es el corte obtenido en el nivel -1 (losa piso 1), el cual dio un corte bajo 1, esto dado la gran área que posee dicha planta.
    \item El Estuco es uno de los revestimientos finales más usados para embellecer muros y techos; consiste en una masa o pasta muy fina compuesto de un material base: cal, yeso o cemento que se mezcla con otros materiales como polvo de mármol etc. \\
    Como recomendación de un fabricante de estuco (Drymix) se debe emplear el material en dos capas no mayores a 1.5cm cada una sobre los muros.\\
    Existen muchos tipos de estucos, cada uno utilizado según lo que se requiera. Hay estucos interiores, exteriores y especiales, todos ellos con, compuestos adicionales, espesores tipos y cantidad de manos necesarias distintas. % creepypasta esa wea
    % \item Centro de gravedad de la losa.
\end{itemize}
    
% FIN DEL DOCUMENTO
\end{document}
