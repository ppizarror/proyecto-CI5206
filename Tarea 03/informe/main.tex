% Template:     Informe/Reporte LaTeX
% Documento:    Archivo principal
% Versión:      6.0.1 (21/10/2018)
% Codificación: UTF-8
%
% Autor: Pablo Pizarro R. @ppizarror
%        Facultad de Ciencias Físicas y Matemáticas
%        Universidad de Chile
%        pablo.pizarro@ing.uchile.cl, ppizarror.com
%
% Manual template: [http://latex.ppizarror.com/Template-Informe/]
% Licencia MIT:    [https://opensource.org/licenses/MIT/]

% CREACIÓN DEL DOCUMENTO
\documentclass[letterpaper,11pt]{article} % Articulo tamaño carta, 11pt
\usepackage[utf8]{inputenc} % Codificación UTF-8

% INFORMACIÓN DEL DOCUMENTO
\def\titulodelinforme {Cubicaciones}
\def\temaatratar {Proyecto de Hormigón Armado - Entrega informe N°3 corregido}

\def\autordeldocumento {Grupo N1}
\def\nombredelcurso {Proyecto de Hormigón Armado}
\def\codigodelcurso {CI5206-2}

\def\nombreuniversidad {Universidad de Chile}
\def\nombrefacultad {Facultad de Ciencias Físicas y Matemáticas}
\def\departamentouniversidad {Departamento de Ingeniería Civil}
\def\imagendepartamento {dic}
\def\imagendepartamentoescala {0.2}
\def\localizacionuniversidad {Santiago, Chile}

% INTEGRANTES, PROFESORES Y FECHAS
\def\tablaintegrantes {
\begin{tabular}{ll}
    Grupo:
	& \begin{tabular}[t]{@{}l@{}}
		N°1
	\end{tabular} \\
	Integrantes:
	& \begin{tabular}[t]{@{}l@{}}
		Mauricio Leal V. \\
		Pablo Pizarro R. \\
		Ignacio Yáñez G.
	\end{tabular} \\
	Profesor:
	& \begin{tabular}[t]{@{}l@{}}
		Juan Mendoza V.
	\end{tabular} \\
	Auxiliar:
	& \begin{tabular}[t]{@{}l@{}}
		Felipe Andrade T.
	\end{tabular} \\
	& \\
	\multicolumn{2}{l}{Fecha de entrega: 19 de Diciembre de 2018} \\
	\multicolumn{2}{l}{\localizacionuniversidad}
\end{tabular}}{
}

% CONFIGURACIONES
\input{lib/config}

% IMPORTACIÓN DE LIBRERÍAS
\input{lib/env/imports}

% IMPORTACIÓN DE FUNCIONES Y ENTORNOS
\input{lib/cmd/all}

% IMPORTACIÓN DE ESTILOS
\input{lib/style/all}

% CONFIGURACIÓN INICIAL DEL DOCUMENTO
\input{lib/cfg/init}
\usepackage{adjustbox}

% INICIO DE LAS PÁGINAS
\begin{document}

% PORTADA
\input{lib/page/portrait} % Se puede borrar

% CONFIGURACIÓN DE PÁGINA Y ENCABEZADOS
\input{lib/cfg/page}

% TABLA DE CONTENIDOS - ÍNDICE
% Template:     Informe/Reporte LaTeX
% Documento:    Índice
% Versión:      6.0.1 (21/10/2018)
% Codificación: UTF-8
%
% Autor: Pablo Pizarro R. @ppizarror
%        Facultad de Ciencias Físicas y Matemáticas
%        Universidad de Chile
%        pablo.pizarro@ing.uchile.cl, ppizarror.com
%
% Manual template: [http://latex.ppizarror.com/Template-Informe/]
% Licencia MIT:    [https://opensource.org/licenses/MIT/]

\ifthenelse{\equal{\showindex}{true}}{
	\newpage
	\begingroup
	\sectionfont{\color{\indextitlecolor} \fontsizetitlei \styletitlei \selectfont}
	\ifthenelse{\equal{\addindextobookmarks}{true}}{
		\belowpdfbookmark{\nomltcont}{contents}}{
	}
	\tocloftpagestyle{fancy}
	\ifthenelse{\equal{\showdotontitles}{true}}{
		\def\cftsecaftersnum {.}
		\def\cftsubsecaftersnum {.}
		\def\cftsubsubsecaftersnum {.}
		\def\cftsubsubsubsecaftersnum {.}
		}{
	}
	\def\cftfigaftersnum {\charafterobjectindex\enspace}
	\def\cftsubfigaftersnum {\charafterobjectindex\enspace}
	\def\cfttabaftersnum {\charafterobjectindex\enspace}
	\def\cftlstlistingaftersnum {\charafterobjectindex\enspace}
	\renewcommand{\cftdot}{\charnumpageindex}
	\ifthenelse{\equal{\showlinenumbers}{true}}{
		\nolinenumbers}{
	}
	\ifthenelse{\equal{\objectindexindent}{true}}{
		\def\cftlstlistingindent {1.495em}
	}{
		\setlength{\cfttabindent}{0in}
		\setlength{\cftfigindent}{0in}
		\setlength{\cftsubfigindent}{0in}
		\setlength{\cftfigindent}{0in}
		\def\cftlstlistingindent {0.01em}
	}
	\ifthenelse{\equal{\equalmarginnumobject}{true}}{
		\ifthenelse{\equal{\showsectioncaption}{none}}{
			\def\cftdefautnumwidth {2.3em}
		}{
		\ifthenelse{\equal{\showsectioncaption}{sec}}{
			\def\cftdefautnumwidth {3.0em}
		}{
		\ifthenelse{\equal{\showsectioncaption}{ssec}}{
			\def\cftdefautnumwidth {3.8em}
		}{
		\ifthenelse{\equal{\showsectioncaption}{sssec}}{
			\def\cftdefautnumwidth {4.3em}
		}{
			\throwbadconfig{Valor configuracion incorrecto}{\showsectioncaption}{none,sec,ssec,sssec}}}}
		}
		\def\cftfignumwidth {\cftdefautnumwidth}
		\def\cftsubfignumwidth {\cftdefautnumwidth}
		\def\cfttabnumwidth {\cftdefautnumwidth}
		\def\cftlstlistingnumwidth {\cftdefautnumwidth}}{
	}
	\ifthenelse{\equal{\showindexofcontents}{true}}{\tableofcontents}{}
	\iftotalfigures
		\ifthenelse{\equal{\showindexoffigures}{true}}{
			\ifthenelse{\equal{\indexforcenewpage}{true}}{\newpage}{}
			\listoffigures
		}{}
	\fi
	\iftotaltables
		\ifthenelse{\equal{\showindexoftables}{true}}{
			\ifthenelse{\equal{\indexforcenewpage}{true}}{\newpage}{}
			\listoftables
		}{}
	\fi
	\iftotallstlistings
		\ifthenelse{\equal{\showindexofcode}{true}}{
			\ifthenelse{\equal{\indexforcenewpage}{true}}{\newpage}{}
			\lstlistoflistings
		}{}
	\fi
	\endgroup
	\ifthenelse{\equal{\addemptypagetwosides}{true}}{
		\vfill
		\checkoddpage
		\ifoddpage
		\else
			\newpage
			\null
			\thispagestyle{empty}
			\newpage
			\addtocounter{page}{-1}
		\fi}{
	}
}{}
 % Se puede borrar

% CONFIGURACIONES FINALES
% Template:     Informe/Reporte LaTeX
% Documento:    Configuraciones finales
% Versión:      6.1.6 (14/12/2018)
% Codificación: UTF-8
%
% Autor: Pablo Pizarro R. @ppizarror
%        Facultad de Ciencias Físicas y Matemáticas
%        Universidad de Chile
%        pablo.pizarro@ing.uchile.cl, ppizarror.com
%
% Manual template: [https://latex.ppizarror.com/Template-Informe/]
% Licencia MIT:    [https://opensource.org/licenses/MIT/]

\markboth{}{}
\newpage
\ifthenelse{\equal{\disablehfrightmark}{false}}{
	\ifthenelse{\equal{\hfstyle}{style1}}{
		\fancyhead[L]{\nouppercase{\leftmark}}}{
	}
	\ifthenelse{\equal{\hfstyle}{style2}}{
		\fancyhead[L]{\nouppercase{\leftmark}}}{
	}
	\ifthenelse{\equal{\hfstyle}{style4}}{
		\fancyhead[L]{\nouppercase{\leftmark}}}{
	}
	\ifthenelse{\equal{\hfstyle}{style5}}{
		\fancyhead[R]{\nouppercase{\leftmark}}}{
	}
	\ifthenelse{\equal{\hfstyle}{style9}}{
		\fancyhead[L]{\nouppercase{\leftmark}}}{
	}
	\ifthenelse{\equal{\hfstyle}{style10}}{
		\fancyhead[L]{\nouppercase{\leftmark}}}{
	}
\ifthenelse{\equal{\hfstyle}{style11}}{
		\fancyhead[L]{\nouppercase{\leftmark}}}{
	}
\ifthenelse{\equal{\hfstyle}{style14}}{
		\fancyhead[L]{\nouppercase{\leftmark}}}{
	}}{
}
\sectionfont{\color{\titlecolor} \fontsizetitle \styletitle \selectfont}
\subsectionfont{\color{\subtitlecolor} \fontsizesubtitle \stylesubtitle \selectfont}
\subsubsectionfont{\color{\subsubtitlecolor} \fontsizesubsubtitle \stylesubsubtitle \selectfont}
\titleformat{\subsubsubsection}{\color{\ssstitlecolor} \normalfont \fontsizessstitle \stylessstitle}{\thesubsubsubsection}{1em}{}
\titlespacing*{\subsubsubsection}{0pt}{3.25ex plus 1ex minus .2ex}{1.5ex plus .2ex}
\ifthenelse{\equal{\showsectioncaption}{none}}{
}{
\ifthenelse{\equal{\showsectioncaption}{sec}}{
	\counterwithin{equation}{section}
	\counterwithin{figure}{section}
	\counterwithin{lstlisting}{section}
	\counterwithin{table}{section}
}{
\ifthenelse{\equal{\showsectioncaption}{ssec}}{
	\counterwithin{equation}{subsection}
	\counterwithin{figure}{subsection}
	\counterwithin{lstlisting}{subsection}
	\counterwithin{table}{subsection}
}{
\ifthenelse{\equal{\showsectioncaption}{sssec}}{
	\counterwithin{equation}{subsubsection}
	\counterwithin{figure}{subsubsection}
	\counterwithin{lstlisting}{subsubsection}
	\counterwithin{table}{subsubsection}
}{
\ifthenelse{\equal{\showsectioncaption}{ssssec}}{
	\counterwithin{equation}{subsubsubsection}
	\counterwithin{figure}{subsubsubsection}
	\counterwithin{lstlisting}{subsubsubsection}
	\counterwithin{table}{subsubsubsection}
}{
	\throwbadconfig{Valor configuracion incorrecto}{\showsectioncaption}{none,sec,ssec,sssec,ssssec}
}}}}}
\ifthenelse{\equal{\predocuseromannumber}{true}}{
	\renewcommand{\thepage}{\arabic{page}}}{
}
\ifthenelse{\equal{\resetpagnumafterindex}{true}}{
	\setcounter{page}{1}}{
}
\setcounter{section}{0}
\setcounter{footnote}{0}
\ifthenelse{\equal{\showlinenumbers}{true}}{
	\linenumbers}{
}


% ======================= INICIO DEL DOCUMENTO =======================
\section{Introducción}
Mediante el uso de los planos de arquitectura, en el presente informe, correspondiente a la tercera entrega del curso de proyecto de hormigón, se entregan cubicaciones y pesos sísmicos de todas las losas del edificio. Para ello, se detallan los pesos de los elementos estructurales que recibe cada losa, tales como muros, tabiques, vigas, estuco y sobrelosas. De forma complementaria a los resultados, se utilizan como apoyo esquemas de las disposiciones en las losas de los elementos previamente mencionados.\\

Por último, con las cubicaciones obtenidas se realiza un análisis comparativo según los resultados expuestos durante las clases de cátedra, y de esta forma observar similitudes o diferencias entre los resultados obtenidos y los esperados. 

\newpage
\section{Pesos involucrados en las losas}

Para el cálculo de cubicaciones y pesos que reciben las losas de cada piso se consideran los siguientes elementos:
\begin{itemize}
    \item Muros y Tabiques.
    \item Vigas normales, invertidas y semi-invertidas.
    \item Estuco, Enlucido y sobrelosa.
\end{itemize}

Para el caso de los muros, estos deben llevar una capa de estuco tal y como se muestra a continuación:

\insertimage[]{img/estucos_muro}{width=9cm}{Disposición de capa de estucos en muros.}

Para las vigas, dependiendo de si estas son normales o invertidas, se tendrán dos configuraciones distintas para el estuco y enlucido que estas llevan:

\begin{images}[]{Disposición de estuco y enlucido.}
    \addimage{img/vigainvertida}{width=5.5cm}{Viga Invertida.}
    \addimage{img/viganormal}{width=6.5cm}{Viga Normal.}
\end{images}

Por último, para las losas, la disposición de enlucido y sobrelosa se ejemplifica a continuación:

\insertimage[]{img/losa}{width=10cm}{Disposición de enlucido y sobrelosa.}

\section{Cubicaciones de elementos}

\subsection{Características de los materiales}

A continuación se muestran los pesos  y espesores de los elementos utilizados.
 
    \begin{table}[H]
      \centering
      \caption{Pesos Volumétricos.}
        \begin{tabular}{|c|c|}
        \hline
        \textbf{Elemento} &
          \boldmath{}\textbf{Densidad [$T/m^3$]}\unboldmath{}
          \bigstrut\\
        \hline
        $\gamma_{Hormigon}$ &
          2,5
          \bigstrut\\
        \hline
        $\gamma_{Estuco}$ &
          2,0
          \bigstrut\\
        \hline
        $\gamma_{Enlucido}$ &
          2,0
          \bigstrut\\
        \hline
        $\gamma_{Sobrelosa}$ &
          1,5
          \bigstrut\\
        \hline
        \end{tabular}%
      \label{volumen}%
    \end{table}%
    
Para cada elemento estructural se consideraron los siguientes espesores, obtenidos a partir de los apuntes de clase.

\begin{table}[H]
  \centering
  \caption{Espesores elementos.}
    \begin{tabular}{|c|c|}
    \hline
    \textbf{Elemento} &
      \textbf{Espesor (m)}
      \bigstrut\\
    \hline
    Estuco &
      0,025
      \bigstrut\\
    \hline
    Enlucido &
      0,02
      \bigstrut\\
    \hline
    Sobrelosa &
      0,05
      \bigstrut\\
    \hline
    \end{tabular}%
  \label{Espesor}%
\end{table}%

Para el cálculo de las sobrecargas de cada losa se utilizaron las sobrecargas de la Tabla \ref{Area}, los valores utilizados fueron obtenidos a partir de la Tabla 3, \textit{Sobrecargas de uso uniformemente distribuidas para pisos}, Norma NCh1537. Cabe destacar que un 25\% de la sobrecarga (SC) se utiliza para el cálculo de los pesos sísmicos.

\begin{table}[H]
  \centering
  \caption{Pesos por área.}
    \begin{tabular}{|c|c|}
    \hline
    \textbf{Elemento} & \boldmath{}\textbf{Peso [$T/m^2$]}\unboldmath{} \bigstrut\\
    \hline
    Peso tabique & 0.100 \bigstrut\\
    \hline
    Sobrecarga uso vivienda & 0.200 \bigstrut\\
    \hline
    Sobrecarga escala uso público & 0.400 \bigstrut\\
    \hline
    Sobrecarga estacionamientos & 0.500 \bigstrut\\
    \hline
    \end{tabular}%
  \label{Area}%
\end{table}
            
\newpage
\subsection{Cubicaciones losas}

    Para la determinación del peso sísmico de las losas, se utilizan los espesores de losas de la tabla \ref{Espesor losas}. Estos fueron obtenidos a partir de un criterio de condición de apoyo y largo, trabajo realizado en la entrega N°2 del mismo curso.
    
    \begin{table}[H]
      \centering
      \caption{Espesores de losa por piso.}
        \begin{tabular}{|c|c|}
    \cline{2-2}    \multicolumn{1}{c|}{} &
          \textbf{Espesor (cm)}
          \bigstrut\\
        \hline
        Piso 24, Cubierta &
          0,16
          \bigstrut[t]\\
        Piso Tipo  &
          0,16
          \\
        Piso 2  &
          0,16
          \\
        Piso 1 &
          0,17
          \\
        Piso -1 (losa cielo subterráneo) &
          0,17
          \bigstrut[b]\\
        \hline
        \end{tabular}%
      \label{Espesor losas}%
    \end{table}%
    
    La Tabla \ref{weatablaqldelosas} detalla la cubicación y los pesos sísmico de las losas, en ella se diferencia entre área losa interior (afecta a una sobrecarga de vivienda/pasillos), y losa estacionamientos (afecta a sobrecarga de estacionamiento vehicular). La columna SC detalla la suma de las sobrecargas de losas ponderadas por un factor del 25\%.
    
    \begin{table}[H]
      \centering
      \caption{Cubicación y pesos sísmicos losas.}
      \itemresize{1}{
      
      \begin{tabular}{ccc|c|c|c|c|c|}
        \hline
        \multicolumn{1}{|P{3em}|}{\textbf{Losa}} & \multicolumn{1}{P{4em}|}{\textbf{Espesor (m)}} & \multicolumn{1}{P{5em}|}{\textbf{Área losa interior (m2)}} & \multicolumn{1}{P{7em}|}{\textbf{Área estacionamientos (m2)}} & \multicolumn{1}{P{5em}|}{\textbf{Peso hormigón (T)}} & \multicolumn{1}{P{7em}|}{\textbf{Peso enlucido + sobrelosa (T)}} & \multicolumn{1}{P{4em}|}{\textbf{SC (T)}} & \multicolumn{1}{P{5em}|}{\textbf{PSl Losa (T)}} \bigstrut\\
        \hline
        \multicolumn{1}{|c|}{24} & \multicolumn{1}{c|}{0.16} & 68.7  & 0.0   & 27.5  & 7.9   & 3.4   & 38.8 \bigstrut[t]\\
        \multicolumn{1}{|c|}{3-23} & \multicolumn{1}{c|}{0.16} & 410.2 & 0.0   & 164.1 & 47.2  & 20.5  & 231.8 \\
        \multicolumn{1}{|c|}{2} & \multicolumn{1}{c|}{0.16} & 410.7 & 0.0   & 164.3 & 47.2  & 20.5  & 232.0 \\
        \multicolumn{1}{|c|}{1} & \multicolumn{1}{c|}{0.16} & 177.5 & 255.9 & 173.4 & 20.4  & 40.9  & 234.7 \\
        \multicolumn{1}{|c|}{-1} & \multicolumn{1}{c|}{0.17} & 205.3 & 514.2 & 305.8 & 23.6  & 74.5  & 403.9 \bigstrut[b]\\
        \hline
              &       &       & TOTAL & 4116.6 & 1089.8 & 570.1 & 5776.5 \bigstrut\\
    \cline{4-8}  \end{tabular}
         
     }
     \label{weatablaqldelosas}
    \end{table}
          
\subsection{Cubicaciones muros y tabiques}

    Los pesos sísmicos de muros se detallan en la Tabla \ref{muros}, la Tabla \ref{tabiques} detalla los pesos de tabiquería.
 
    \begin{table}[H]
      \centering
      \caption{Cubicaciones y pesos sísmicos de muros.}
      \itemresize{1}{
      
      \begin{tabular}{cccccc|c|c|c|c|}
        \hline
        \multicolumn{1}{|c|}{\textbf{Piso}} & \multicolumn{1}{c|}{\textbf{ex (m)}} & \multicolumn{1}{c|}{\textbf{ey (m)}} & \multicolumn{1}{c|}{\textbf{Lx (m)}} & \multicolumn{1}{c|}{\textbf{Ly (m)}} & \multicolumn{1}{P{5.665em}|}{\textbf{Área planta muro (m2)}} & \multicolumn{1}{P{5.11em}|}{\textbf{Área Estuco (m2)}} & \multicolumn{1}{P{6.39em}|}{\textbf{Peso Hormigón muro (T)}} & \multicolumn{1}{P{5.165em}|}{\textbf{Peso estuco (T)}} & \multicolumn{1}{P{5.61em}|}{\textbf{Peso muro + estuco (T)}} \bigstrut\\
        \hline
        \multicolumn{1}{|c|}{24} & \multicolumn{1}{c|}{0.20} & \multicolumn{1}{c|}{0.20} & \multicolumn{1}{c|}{26.9} & \multicolumn{1}{c|}{17.6} & 8.9   & 2.3   & 51.2  & 0.11  & 51.3 \bigstrut[t]\\
        \multicolumn{1}{|c|}{15-23} & \multicolumn{1}{c|}{0.20} & \multicolumn{1}{c|}{0.20} & \multicolumn{1}{c|}{54.5} & \multicolumn{1}{c|}{58.9} & 22.7  & 5.7   & 130.3 & 0.28  & 130.6 \\
        \multicolumn{1}{|c|}{4-14} & \multicolumn{1}{c|}{0.25} & \multicolumn{1}{c|}{0.25} & \multicolumn{1}{c|}{54.5} & \multicolumn{1}{c|}{58.9} & 28.3  & 5.7   & 162.9 & 0.28  & 163.2 \\
        \multicolumn{1}{|c|}{3} & \multicolumn{1}{c|}{0.25} & \multicolumn{1}{c|}{0.25} & \multicolumn{1}{c|}{58.3} & \multicolumn{1}{c|}{58.9} & 29.3  & 5.9   & 168.4 & 0.29  & 168.7 \\
        \multicolumn{1}{|c|}{2} & \multicolumn{1}{c|}{0.25} & \multicolumn{1}{c|}{0.3} & \multicolumn{1}{c|}{65.4} & \multicolumn{1}{c|}{50.5} & 31.5  & 5.8   & 181.0 & 0.29  & 181.3 \\
        \multicolumn{1}{|c|}{1} & \multicolumn{1}{c|}{0.25} & \multicolumn{1}{c|}{0.3} & \multicolumn{1}{c|}{70.0} & \multicolumn{1}{c|}{55.0} & 34.0  & 6.3   & 195.4 & 0.31  & 195.8 \\
        \multicolumn{1}{|c|}{-1} & \multicolumn{1}{c|}{0.25} & \multicolumn{1}{c|}{0.3} & \multicolumn{1}{c|}{78.4} & \multicolumn{1}{c|}{61.5} & 38.1  & 7.0   & 218.9 & 0.35  & 219.2 \bigstrut[b]\\
        \hline
              &       &       &       &       &       & \textbf{Total} & \textbf{3670.8} & \textbf{6.9} & \textbf{3677.7} \bigstrut\\
    \cline{7-10}  \end{tabular}
      
      }
      \label{muros}
    \end{table}

    \begin{table}[H]
      \centering
      \caption{Cubicaciones y pesos sísmicos tabiques.}
      \itemresize{1}{
      
      \begin{tabular}{cccc|c|c|c|}
        \hline
        \multicolumn{1}{|c|}{\textbf{Piso}} & \multicolumn{1}{c|}{\textbf{ex (m)}} & \multicolumn{1}{c|}{\textbf{ey (m)}} & \textbf{Lx (m)} & \textbf{Ly (m)} & \textbf{Área tabiques (m2)} & \textbf{Peso Tabique (T)} \bigstrut\\
        \hline
        \multicolumn{1}{|c|}{24} & \multicolumn{1}{c|}{0.20} & \multicolumn{1}{c|}{0.20} & 10.3  & 8.7   & 3.8   & 0.4 \bigstrut[t]\\
        \multicolumn{1}{|c|}{15-23} & \multicolumn{1}{c|}{0.20} & \multicolumn{1}{c|}{0.20} & 60.5  & 39.2  & 19.9  & 2.0 \\
        \multicolumn{1}{|c|}{3-14} & \multicolumn{1}{c|}{0.25} & \multicolumn{1}{c|}{0.25} & 60.5  & 39.2  & 24.9  & 2.5 \\
        \multicolumn{1}{|c|}{2} & \multicolumn{1}{c|}{0.25} & \multicolumn{1}{c|}{0.30} & 36.1  & 38.7  & 20.6  & 2.1 \\
        \multicolumn{1}{|c|}{1} & \multicolumn{1}{c|}{0.25} & \multicolumn{1}{c|}{0.30} & 38.1  & 31.6  & 19.0  & 1.9 \\
        \multicolumn{1}{|c|}{-1} & \multicolumn{1}{c|}{0.25} & \multicolumn{1}{c|}{0.30} & 22.6  & 16.9  & 10.7  & 1.1 \bigstrut[b]\\
        \hline
              &       &       &       & \textbf{Total} & \textbf{52.7} & \textbf{48.8} \bigstrut\\
    \cline{5-7}  \end{tabular}
      
      }
      \label{tabiques}
    \end{table}
        
\subsection{ Cubicaciones Vigas}
    Para la determinación de los pesos sísmicos de vigas en cada piso se obtienen los siguientes resultados:
    
    \begin{table}[H]
  \centering
  \caption{Cubicaciones y pesos sísmicos de vigas.}
  \itemresize{1}{
  
  \begin{tabular}{ccc|c|c|c|c|}
    \hline
    \multicolumn{1}{|c|}{\textbf{Piso}} & \multicolumn{1}{P{5.165em}|}{\textbf{Volumen vigas (m3)}} & \multicolumn{1}{P{4em}|}{\textbf{Área estuco (m2)}} & \multicolumn{1}{P{4em}|}{\textbf{Volumen estuco (m3)}} & \multicolumn{1}{P{6.78em}|}{\textbf{Peso hormigón vigas (T)}} & \multicolumn{1}{P{5.39em}|}{\textbf{Peso estuco vigas (T)}} & \textbf{Peso total (T)} \bigstrut\\
    \hline
    \multicolumn{1}{|c|}{24} & \multicolumn{1}{c|}{23.7} & 242.6 & 6.1   & 59.3  & 12.1  & 71.5 \bigstrut[t]\\
    \multicolumn{1}{|c|}{3-23} & \multicolumn{1}{c|}{14.8} & 150.7 & 3.8   & 36.9  & 7.5   & 44.5 \\
    \multicolumn{1}{|c|}{2} & \multicolumn{1}{c|}{14.6} & 148.5 & 3.7   & 36.4  & 7.4   & 43.8 \\
    \multicolumn{1}{|c|}{1} & \multicolumn{1}{c|}{16.2} & 56.5  & 1.4   & 40.4  & 2.8   & 43.3 \\
    \multicolumn{1}{|c|}{-1} & \multicolumn{1}{c|}{19.7} & 41.2  & 1.0   & 49.4  & 2.1   & 51.4 \bigstrut[b]\\
    \hline
          &       &       & \textbf{TOTAL} & \textbf{960.9} & \textbf{182.7} & \textbf{1143.6} \bigstrut\\
\cline{4-7}  \end{tabular}
  
  }
  \label{pesovigas}
\end{table}
   
\newpage 
\section{Planilla resumen}

La Tabla \ref{plresumen} resume las cubicaciones de cada uno de los niveles, considerando el peso sísmico de losa, viga y muros. La columna $\sum Ps$ detalla la carga axial aditiva por cada piso. El peso sísmico total corresponde a 10650.5 toneladas.

\begin{table}[H]
  \centering
  \caption{Planilla resumen de cubicaciones.}
   \begin{adjustbox}{angle=90}
    \resizebox{1.1\textwidth}{!}{
  \begin{tabular}{|ccccccccccccc|c|}
    \hline
    \multicolumn{1}{|P{2em}}{\textbf{Losa}} & \textbf{Altura (m)} & \multicolumn{1}{P{3.5em}}{\textbf{Área losa (m2)}} & \multicolumn{1}{P{3em}}{\textbf{Xg losa (m)}} & 
    \multicolumn{1}{P{3em}}{\textbf{Yg losa (m)}} & \multicolumn{1}{P{3.945em}}{\textbf{I polar (m4)}} & \multicolumn{1}{P{3.11em}}{\textbf{Ps losa (T)}} & \multicolumn{1}{P{3.39em}}{\textbf{P vigas (T)}} & \multicolumn{1}{P{3.72em}}{\textbf{P muros (T)}} & \multicolumn{1}{P{3.61em}}{\textbf{Ps nivel (T)}} & \multicolumn{1}{P{4.555em}}{\boldmath{}\textbf{$\sum$ Ps nivel (T)}\unboldmath{}} & \multicolumn{1}{P{4em}}{\boldmath{}\textbf{M tras (T $\cdot$ s2/m)}\unboldmath{}} & 
    \multicolumn{1}{P{4em}|}{\boldmath{}\textbf{M rot (T $\cdot$ s2$\cdot$ m)}\unboldmath{}} & \multicolumn{1}{P{3.835em}|}{\textbf{q (T/m2)}} \bigstrut\\
    \hline
    24    & 2.5   & 68.7  & 19.1  & 15.6  & 726.3 & 38.8  & 71.5  & 51.7  & 136.1 & 136.1 & 13.9  & 146.8 & 1.98 \bigstrut[t]\\
    \textit{23} & 2.5   & 410.2 & 16.0  & 16.3  & 37835.5 & 231.8 & 44.5  & 51.7  & 368.4 & 504.5 & 37.6  & 3467.1 & 0.90 \\
    \textit{22} & 2.5   & 410.2 & 16.0  & 16.3  & 37835.5 & 231.8 & 44.5  & 132.6 & 408.8 & 913.3 & 41.7  & 3848.0 & 1.00 \\
    \textit{21} & 2.5   & 410.2 & 16.0  & 16.3  & 37835.5 & 231.8 & 44.5  & 132.6 & 408.8 & 1322.2 & 41.7  & 3848.0 & 1.00 \\
    \textit{20} & 2.5   & 410.2 & 16.0  & 16.3  & 37835.5 & 231.8 & 44.5  & 132.6 & 408.8 & 1731.0 & 41.7  & 3848.0 & 1.00 \\
    \textit{19} & 2.5   & 410.2 & 16.0  & 16.3  & 37835.5 & 231.8 & 44.5  & 132.6 & 408.8 & 2139.9 & 41.7  & 3848.0 & 1.00 \\
    \textit{18} & 2.5   & 410.2 & 16.0  & 16.3  & 37835.5 & 231.8 & 44.5  & 132.6 & 408.8 & 2548.7 & 41.7  & 3848.0 & 1.00 \\
    \textit{17} & 2.5   & 410.2 & 16.0  & 16.3  & 37835.5 & 231.8 & 44.5  & 132.6 & 408.8 & 2957.6 & 41.7  & 3848.0 & 1.00 \\
    \textit{16} & 2.5   & 410.2 & 16.0  & 16.3  & 37835.5 & 231.8 & 44.5  & 132.6 & 408.8 & 3366.4 & 41.7  & 3848.0 & 1.00 \\
    \textit{15} & 2.5   & 410.2 & 16.0  & 16.3  & 37835.5 & 231.8 & 44.5  & 132.6 & 408.8 & 3775.2 & 41.7  & 3848.0 & 1.00 \\
    \textit{14} & 2.5   & 410.2 & 16.0  & 16.3  & 37835.5 & 231.8 & 44.5  & 132.6 & 425.4 & 4200.6 & 43.4  & 4003.7 & 1.04 \\
    \textit{13} & 2.5   & 410.2 & 16.0  & 16.3  & 37835.5 & 231.8 & 44.5  & 165.7 & 441.9 & 4642.5 & 45.1  & 4159.3 & 1.08 \\
    \textit{12} & 2.5   & 410.2 & 16.0  & 16.3  & 37835.5 & 231.8 & 44.5  & 165.7 & 441.9 & 5084.5 & 45.1  & 4159.3 & 1.08 \\
    \textit{11} & 2.5   & 410.2 & 16.0  & 16.3  & 37835.5 & 231.8 & 44.5  & 165.7 & 441.9 & 5526.4 & 45.1  & 4159.3 & 1.08 \\
    \textit{10} & 2.5   & 410.2 & 16.0  & 16.3  & 37835.5 & 231.8 & 44.5  & 165.7 & 441.9 & 5968.3 & 45.1  & 4159.3 & 1.08 \\
    \textit{9} & 2.5   & 410.2 & 16.0  & 16.3  & 37835.5 & 231.8 & 44.5  & 165.7 & 441.9 & 6410.2 & 45.1  & 4159.3 & 1.08 \\
    \textit{8} & 2.5   & 410.2 & 16.0  & 16.3  & 37835.5 & 231.8 & 44.5  & 165.7 & 441.9 & 6852.2 & 45.1  & 4159.3 & 1.08 \\
    \textit{7} & 2.5   & 410.2 & 16.0  & 16.3  & 37835.5 & 231.8 & 44.5  & 165.7 & 441.9 & 7294.1 & 45.1  & 4159.3 & 1.08 \\
    \textit{6} & 2.5   & 410.2 & 16.0  & 16.3  & 37835.5 & 231.8 & 44.5  & 165.7 & 441.9 & 7736.0 & 45.1  & 4159.3 & 1.08 \\
    \textit{5} & 2.5   & 410.2 & 16.0  & 16.3  & 37835.5 & 231.8 & 44.5  & 165.7 & 441.9 & 8178.0 & 45.1  & 4159.3 & 1.08 \\
    \textit{4} & 2.5   & 410.2 & 16.0  & 16.3  & 37835.5 & 231.8 & 44.5  & 165.7 & 441.9 & 8619.9 & 45.1  & 4159.3 & 1.08 \\
    \textit{3} & 2.5   & 410.2 & 16.0  & 16.3  & 37835.5 & 231.8 & 44.5  & 165.7 & 444.7 & 9064.6 & 45.4  & 4185.4 & 1.08 \\
    2     & 2.5   & 410.7 & 16.0  & 16.3  & 37835.5 & 232.0 & 43.8  & 171.2 & 453.2 & 9517.7 & 46.2  & 4259.9 & 1.10 \\
    1     & 2.5   & 433.5 & 10.3  & 20.0  & 55075.8 & 234.7 & 43.3  & 183.4 & 468.5 & 9986.2 & 47.8  & 6073.6 & 1.08 \\
    -1    & 2.5   & 719.5 & 16.4  & 19.8  & 94413.3 & 403.9 & 51.4  & 197.7 & 664.3 & 10650.5 & 67.8  & 8895.5 & 0.92 \bigstrut[b]\\
    \hline
  \end{tabular}
  }
    \end{adjustbox}
  \label{plresumen}
\end{table}

\newpage
\section{Comentarios y conclusiones}

%- Resultados por piso para valor de q promedio resultante, respecto de valor Edificios Chilenos.
%- Comentar que función cumple el centro de gravedad de la losa.
%- Uso estucos en la actualidad.
\begin{itemize}
    \item Es posible observar en la tabla resumen \ref{plresumen} que los cortes promedio obtenidos por nivel son parecidos a los vistos en clase, el cual oscila entre 1 y 1.1 $T/m^2$ para edificios habitacionales Chilenos de hormigón armado. Un caso interesante es el corte obtenido en el nivel -1 (losa piso 1), el cual dio un corte bajo 1 $T/m^2$, esto dado la gran área que posee dicha planta. \\
    
    La Figura \ref{varq} ilustra la variación del corte por cada nivel y los rangos del corte promedio para estructuras Chilenas. Es claro que todo el corte oscila entre ambos límites, sólo el nivel último (Cubierta) posee un corte mayor dado la gran densidad de muros para la poca área que esta losa tiene.
    
    \insertimage[\label{varq}]{img/varq}{width=13cm}{Variación del corte promedio por cada nivel.}
    
    \item Al graficar el peso sísmico por cada nivel se obtiene la Figura \ref{varp}, es posible observar que el peso se incrementa a medida que se tiene un mayor cantidad de niveles, y además que este incremento es relativamente uniforme. Esto conversa bien con el hecho de que se hayan obtenido cortes por piso similares, dado que el corte es una función del peso sísmico. \\
    
    La Figura \ref{porp} detalla el porcentaje de participación del peso sísmico por cada uno de los niveles, calculado simplemente como $Part_i = \frac{Ps_i}{\sum Ps}$. Es claro observar que a medida que aumentan los niveles en profundidad existe un incremento en el porcentaje de participación, esto dado que para los niveles 3-14 y 15-13 existen diferencias en el ancho de los muros (25cm y 20cm respectivamente). Sin embargo el promedio de participación es cercano al 4\%, ello significa que cada piso aporta en promedio el mismo peso ($\frac{1}{23}$ = 4.3\%).
    
    \insertimage[\label{varp}]{varp}{width=13cm}{Peso sísmico por nivel.}
    
    \insertimage[\label{porp}]{porp}{width=13cm}{Porcentaje participación peso sísmico de cada nivel.}
    
    \newpage
    \item Al analizar la distribución del peso total en la suma de los pesos de los distintos componentes estructurales del edificio, muros, vigas y losas, se obtiene el gráfico de la Figura \ref{partp}. De ella se desprende que la mitad del peso sísmico total de la estructura corresponde a las losas, con un 54.2\%, le siguen los muros, con un 34.5\%.
    
    \insertimage[\label{partp}]{partp}{width=13cm}{Distribución del peso total en la suma de las componentes estructurales.}
    
    \item El estuco es uno de los revestimientos finales más usados para embellecer muros y techos; consiste en una masa o pasta muy fina compuesto de un material base: cal, yeso o cemento que se mezcla con otros materiales como polvo de mármol etc. Como recomendación de un fabricante de estuco (Drymix) se debe emplear el material en dos capas no mayores a 1.5cm cada una sobre los muros.\\
    
    Existen muchos tipos de estucos, cada uno utilizado según lo que se requiera. Hay estucos interiores, exteriores y especiales, todos ellos con, compuestos adicionales, espesores tipos y cantidad de manos necesarias distintas.
\end{itemize}
    
% FIN DEL DOCUMENTO
\end{document}