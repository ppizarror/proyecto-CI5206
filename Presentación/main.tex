% Template:     Informe/Reporte LaTeX
% Documento:    Archivo principal
% Versión:      6.1.0 (03/11/2018)
% Codificación: UTF-8
%
% Autor: Pablo Pizarro R. @ppizarror
%        Facultad de Ciencias Físicas y Matemáticas
%        Universidad de Chile
%        pablo.pizarro@ing.uchile.cl, ppizarror.com
%
% Manual template: [https://latex.ppizarror.com/Template-Informe/]
% Licencia MIT:    [https://opensource.org/licenses/MIT/]

% CREACIÓN DEL DOCUMENTO
\documentclass[letterpaper,11pt]{article} % Articulo tamaño carta, 11pt
\usepackage[utf8]{inputenc} % Codificación UTF-8

% INFORMACIÓN DEL DOCUMENTO
\def\titulodelinforme {Título del informe}
\def\temaatratar {Tema a tratar}

\def\autordeldocumento {Nombre del autor}
\def\nombredelcurso {Curso}
\def\codigodelcurso {CO-1234}

\def\nombreuniversidad {Universidad de Chile}
\def\nombrefacultad {Facultad de Ciencias Físicas y Matemáticas}
\def\departamentouniversidad {Departamento de Ingeniería Civil}
\def\imagendepartamento {departamentos/dic}
\def\imagendepartamentoescala {0.2}
\def\localizacionuniversidad {Santiago, Chile}

% INTEGRANTES, PROFESORES Y FECHAS
\def\tablaintegrantes {
\begin{tabular}{ll}
	Integrantes:
	& \begin{tabular}[t]{@{}l@{}}
		Integrante 1 \\
		Integrante 2
	\end{tabular} \\
	Profesor:
	& \begin{tabular}[t]{@{}l@{}}
		Profesor 1
	\end{tabular} \\
	Auxiliar:
	& \begin{tabular}[t]{@{}l@{}}
		Auxiliar 1
	\end{tabular} \\
	Ayudantes:
	& \begin{tabular}[t]{@{}l@{}}
		Ayudante 1 \\
		Ayudante 2
	\end{tabular} \\
	\multicolumn{2}{l}{Ayudante del laboratorio: Ayudante 1} \\
	& \\
	\multicolumn{2}{l}{Fecha de realización: \today} \\
	\multicolumn{2}{l}{Fecha de entrega: \today} \\
	\multicolumn{2}{l}{\localizacionuniversidad}
\end{tabular}}{
}

% CONFIGURACIONES
\input{lib/config}

% IMPORTACIÓN DE LIBRERÍAS
\input{lib/env/imports}

% IMPORTACIÓN DE FUNCIONES Y ENTORNOS
\input{lib/cmd/all}

% IMPORTACIÓN DE ESTILOS
\input{lib/style/all}

% CONFIGURACIÓN INICIAL DEL DOCUMENTO
\input{lib/cfg/init}

% INICIO DE LAS PÁGINAS
\begin{document}

% PORTADA
\input{lib/page/portrait} % Se puede borrar

% CONFIGURACIÓN DE PÁGINA Y ENCABEZADOS
\input{lib/cfg/page}


% TABLA DE CONTENIDOS - ÍNDICE
% Template:     Informe/Reporte LaTeX
% Documento:    Índice
% Versión:      6.0.1 (21/10/2018)
% Codificación: UTF-8
%
% Autor: Pablo Pizarro R. @ppizarror
%        Facultad de Ciencias Físicas y Matemáticas
%        Universidad de Chile
%        pablo.pizarro@ing.uchile.cl, ppizarror.com
%
% Manual template: [http://latex.ppizarror.com/Template-Informe/]
% Licencia MIT:    [https://opensource.org/licenses/MIT/]

\ifthenelse{\equal{\showindex}{true}}{
	\newpage
	\begingroup
	\sectionfont{\color{\indextitlecolor} \fontsizetitlei \styletitlei \selectfont}
	\ifthenelse{\equal{\addindextobookmarks}{true}}{
		\belowpdfbookmark{\nomltcont}{contents}}{
	}
	\tocloftpagestyle{fancy}
	\ifthenelse{\equal{\showdotontitles}{true}}{
		\def\cftsecaftersnum {.}
		\def\cftsubsecaftersnum {.}
		\def\cftsubsubsecaftersnum {.}
		\def\cftsubsubsubsecaftersnum {.}
		}{
	}
	\def\cftfigaftersnum {\charafterobjectindex\enspace}
	\def\cftsubfigaftersnum {\charafterobjectindex\enspace}
	\def\cfttabaftersnum {\charafterobjectindex\enspace}
	\def\cftlstlistingaftersnum {\charafterobjectindex\enspace}
	\renewcommand{\cftdot}{\charnumpageindex}
	\ifthenelse{\equal{\showlinenumbers}{true}}{
		\nolinenumbers}{
	}
	\ifthenelse{\equal{\objectindexindent}{true}}{
		\def\cftlstlistingindent {1.495em}
	}{
		\setlength{\cfttabindent}{0in}
		\setlength{\cftfigindent}{0in}
		\setlength{\cftsubfigindent}{0in}
		\setlength{\cftfigindent}{0in}
		\def\cftlstlistingindent {0.01em}
	}
	\ifthenelse{\equal{\equalmarginnumobject}{true}}{
		\ifthenelse{\equal{\showsectioncaption}{none}}{
			\def\cftdefautnumwidth {2.3em}
		}{
		\ifthenelse{\equal{\showsectioncaption}{sec}}{
			\def\cftdefautnumwidth {3.0em}
		}{
		\ifthenelse{\equal{\showsectioncaption}{ssec}}{
			\def\cftdefautnumwidth {3.8em}
		}{
		\ifthenelse{\equal{\showsectioncaption}{sssec}}{
			\def\cftdefautnumwidth {4.3em}
		}{
			\throwbadconfig{Valor configuracion incorrecto}{\showsectioncaption}{none,sec,ssec,sssec}}}}
		}
		\def\cftfignumwidth {\cftdefautnumwidth}
		\def\cftsubfignumwidth {\cftdefautnumwidth}
		\def\cfttabnumwidth {\cftdefautnumwidth}
		\def\cftlstlistingnumwidth {\cftdefautnumwidth}}{
	}
	\ifthenelse{\equal{\showindexofcontents}{true}}{\tableofcontents}{}
	\iftotalfigures
		\ifthenelse{\equal{\showindexoffigures}{true}}{
			\ifthenelse{\equal{\indexforcenewpage}{true}}{\newpage}{}
			\listoffigures
		}{}
	\fi
	\iftotaltables
		\ifthenelse{\equal{\showindexoftables}{true}}{
			\ifthenelse{\equal{\indexforcenewpage}{true}}{\newpage}{}
			\listoftables
		}{}
	\fi
	\iftotallstlistings
		\ifthenelse{\equal{\showindexofcode}{true}}{
			\ifthenelse{\equal{\indexforcenewpage}{true}}{\newpage}{}
			\lstlistoflistings
		}{}
	\fi
	\endgroup
	\ifthenelse{\equal{\addemptypagetwosides}{true}}{
		\vfill
		\checkoddpage
		\ifoddpage
		\else
			\newpage
			\null
			\thispagestyle{empty}
			\newpage
			\addtocounter{page}{-1}
		\fi}{
	}
}{}
 % Se puede borrar

% CONFIGURACIONES FINALES
% Template:     Informe/Reporte LaTeX
% Documento:    Configuraciones finales
% Versión:      6.1.6 (14/12/2018)
% Codificación: UTF-8
%
% Autor: Pablo Pizarro R. @ppizarror
%        Facultad de Ciencias Físicas y Matemáticas
%        Universidad de Chile
%        pablo.pizarro@ing.uchile.cl, ppizarror.com
%
% Manual template: [https://latex.ppizarror.com/Template-Informe/]
% Licencia MIT:    [https://opensource.org/licenses/MIT/]

\markboth{}{}
\newpage
\ifthenelse{\equal{\disablehfrightmark}{false}}{
	\ifthenelse{\equal{\hfstyle}{style1}}{
		\fancyhead[L]{\nouppercase{\leftmark}}}{
	}
	\ifthenelse{\equal{\hfstyle}{style2}}{
		\fancyhead[L]{\nouppercase{\leftmark}}}{
	}
	\ifthenelse{\equal{\hfstyle}{style4}}{
		\fancyhead[L]{\nouppercase{\leftmark}}}{
	}
	\ifthenelse{\equal{\hfstyle}{style5}}{
		\fancyhead[R]{\nouppercase{\leftmark}}}{
	}
	\ifthenelse{\equal{\hfstyle}{style9}}{
		\fancyhead[L]{\nouppercase{\leftmark}}}{
	}
	\ifthenelse{\equal{\hfstyle}{style10}}{
		\fancyhead[L]{\nouppercase{\leftmark}}}{
	}
\ifthenelse{\equal{\hfstyle}{style11}}{
		\fancyhead[L]{\nouppercase{\leftmark}}}{
	}
\ifthenelse{\equal{\hfstyle}{style14}}{
		\fancyhead[L]{\nouppercase{\leftmark}}}{
	}}{
}
\sectionfont{\color{\titlecolor} \fontsizetitle \styletitle \selectfont}
\subsectionfont{\color{\subtitlecolor} \fontsizesubtitle \stylesubtitle \selectfont}
\subsubsectionfont{\color{\subsubtitlecolor} \fontsizesubsubtitle \stylesubsubtitle \selectfont}
\titleformat{\subsubsubsection}{\color{\ssstitlecolor} \normalfont \fontsizessstitle \stylessstitle}{\thesubsubsubsection}{1em}{}
\titlespacing*{\subsubsubsection}{0pt}{3.25ex plus 1ex minus .2ex}{1.5ex plus .2ex}
\ifthenelse{\equal{\showsectioncaption}{none}}{
}{
\ifthenelse{\equal{\showsectioncaption}{sec}}{
	\counterwithin{equation}{section}
	\counterwithin{figure}{section}
	\counterwithin{lstlisting}{section}
	\counterwithin{table}{section}
}{
\ifthenelse{\equal{\showsectioncaption}{ssec}}{
	\counterwithin{equation}{subsection}
	\counterwithin{figure}{subsection}
	\counterwithin{lstlisting}{subsection}
	\counterwithin{table}{subsection}
}{
\ifthenelse{\equal{\showsectioncaption}{sssec}}{
	\counterwithin{equation}{subsubsection}
	\counterwithin{figure}{subsubsection}
	\counterwithin{lstlisting}{subsubsection}
	\counterwithin{table}{subsubsection}
}{
\ifthenelse{\equal{\showsectioncaption}{ssssec}}{
	\counterwithin{equation}{subsubsubsection}
	\counterwithin{figure}{subsubsubsection}
	\counterwithin{lstlisting}{subsubsubsection}
	\counterwithin{table}{subsubsubsection}
}{
	\throwbadconfig{Valor configuracion incorrecto}{\showsectioncaption}{none,sec,ssec,sssec,ssssec}
}}}}}
\ifthenelse{\equal{\predocuseromannumber}{true}}{
	\renewcommand{\thepage}{\arabic{page}}}{
}
\ifthenelse{\equal{\resetpagnumafterindex}{true}}{
	\setcounter{page}{1}}{
}
\setcounter{section}{0}
\setcounter{footnote}{0}
\ifthenelse{\equal{\showlinenumbers}{true}}{
	\linenumbers}{
}


% ======================= INICIO DEL DOCUMENTO =======================

\section{Definición elemento estructural}
    Se utilizan cuando existen cargas importantes, de modo que entre la placa y el pilar se construye un capitel (superior e inferior) que optimiza el trabajo de la losa, tanto a los esfuerzos de flexión como de corte. Su armado es similar a las de losas de espesor constante.
    \subsection{Tipos de capitel}
    
\section{Historia}

\section{Ejemplos de aplicación}

\section{Recomendaciones de diseño}

\section{Normativa}

\section{Ventajas y desventajas}

\section{Alternativas}

\section{Referencias}
    %DESCRIPCION GENERAL - página 18 - https://upcommons.upc.edu/bitstream/handle/2099.1/6081/07.pdf?sequence=8&isAllowed=y
    %ARMADURA ANTIPUNZONAMIENTO - https://www.plakagroup.com/getmedia/a3aa2687-d79e-4939-8677-254d41866530/01-006-ARMATEC_INT1306_ES_LR.pdf
    %DEFINICION - https://www.construmatica.com/construpedia/Losas_con_Capitel
    %PUNZONAMIENTO - https://e-struc.com/2017/05/23/punzonamiento-por-que-se-produce-y-cuales-son-sus-efectos/
    %PUNZONAMIENTO - https://e-struc.com/2017/06/06/punzonamiento-diseno-estructural/

% FIN DEL DOCUMENTO
\end{document}
