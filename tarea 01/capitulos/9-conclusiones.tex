\newpage
\section{Conclusiones}

\begin{itemize}
    % \item % comentarios sobre Documento Bases de Calculo 
    %que?
    \item Con la aproximación aplicada para la determinación del período sísmico de la estructura se obtuvo un resultado cercano a lo espera, según la experiencia con estructuras de este estilo. Es de esperar que este valor se vea modificado una vez analizada la estructura con la estructuración definitiva y las cargas correspondientes.
    
    \item Como se pudo apreciar en los antecedentes del análisis sísmico y en los parámetros de mecánica de suelos, la ubicación de la estructura influencia en un comienzo la zona sísmica en la cual se encuentra el edificio. Esta, en conjunto con los posteriores análisis sísmicos que deben efectuarse generan una variación directa del periodo de la estructura, peso sísmico, corte basal, entre otros. Lo anterior sumado a que según este tipo de suelo también ocurrirán variaciones en el dimensionamiento de las zapatas de la estructura, producen de igual forma variaciones en la constante de balasto y asentamientos que se producirán bajo el edificio.
    
    % comentarios sobre  Influencia en el diseño, tipo de suelo y zona sísmica de emplazamiento. 
    \item Se consideró la utilización de dos tipos de hormigón a modo de conseguir un diseño más costo-eficiente. Debido a la concentración de esfuerzos de compresión hacia la base del edificio, se opta por un hormigón H-40. Por otra parte, hacia los pisos superiores, para conservar la homogeneidad de los elementos sin caer en el sobre-dimensionamiento de estos, se escogió un hormigón H-35.
    
\end{itemize}