\newpage
\section {Combinaciones de carga}

Los elementos estructurales serán diseñados para aquella combinación de cargas que genere la condición más desfavorable, en cuanto a su resistencia límite requerida. \\

Los estados de carga serán modificados por factores de mayoración o minoración de acuerdo a la combinación general de cargas, según se indique en la norma NCh3171, apartado 9.2.1 combinaciones de cargas nominales que se usan en el método de tensiones admisibles. \\

Como criterios generales de combinación para todas las estructuras se establece que, a lo menos, se debe considerar lo siguiente: \\ % ¿ será necesario señalar que es para edificios?
%naaaa en la norma mencionan la mierda de ocupación, pero pa los combos no se considera na de eso
%mencionaba unos combos que iban o no, dependiendo de la situación, pero ninguno aplicaba pa este caso, le pongo por qué no? mejors weno
% los de 433 pico? bah nada

\indent C1          $=PP$\\
\indent C2          $=PP+SC$\\
\indent C3.1        $=PP+S$\\
\indent C3.2        $=PP+L_{r}$\\
\indent C4.1        $=PP+0.75SC+0.75S$\\
\indent C4.2        $=PP+0.75SC+0.75L_{r}$\\
\indent C5.a.1      $=PP+W^{+}$\\
\indent C5.a.2      $=PP+W^{-}$\\
\indent C5.b.1      $=PP+S+Ex+Ez$\\
\indent C5.b.2      $=PP+S+Ex-Ez$\\
\indent C5.b.3      $=PP+S-Ex+Ez$\\
\indent C5.b.4      $=PP+S-Ex-Ez$\\
\indent C5.b.1      $=PP+S+Ey+Ez$\\
\indent C5.b.2      $=PP+S+Ey-Ez$\\
\indent C5.b.3      $=PP+S-Ey+Ez$\\
\indent C5.b.4      $=PP+S-Ey-Ez$\\
\indent C6.a.1      $=PP+0,75SC+0,75W^{+}+0,75S$\\
\indent C6.a.2      $=PP+0,75SC+0,75W^{-}+0,75S$\\
\indent C6.a.3      $=PP+0,75SC+0,75W^{+}+0,75Lr$\\
\indent C6.a.4      $=PP+0,75SC+0,75W^{-}+0,75Lr$\\
\indent C6.b.1      $=PP+0,75SC+0,75W^{+}+0,75Ex+0,75Ez$\\
\indent C6.b.2      $=PP+0,75SC+0,75W^{+}+0,75Ex-0,75Ez$\\
\indent C6.b.3      $=PP+0,75SC+0,75W^{+}-0,75Ex+0,75Ez$\\
\indent C6.b.4      $=PP+0,75SC+0,75W^{+}-0,75Ex-0,75Ez$\\
\indent C6.b.5      $=PP+0,75SC+0,75W^{+}+0,75Ey+0,75Ez$\\
\indent C6.b.6      $=PP+0,75SC+0,75W^{+}+0,75Ey-0,75Ez$\\
\indent C6.b.7      $=PP+0,75SC+0,75W^{+}-0,75Ey+0,75Ez$\\
\indent C6.b.8      $=PP+0,75SC+0,75W^{+}-0,75Ey-0,75Ez$\\
\indent C7.1        $=0,6PP+W^{+}$\\
\indent C7.2        $=0,6PP+W^{-}$\\
\indent C8.1        $=0,6PP+Ex+Ez$\\
\indent C8.2        $=0,6PP+Ex-Ez$\\
\indent C8.3        $=0,6PP-Ex+Ez$\\
\indent C8.4        $=0,6PP-Ex-Ez$\\
\indent C8.5        $=0,6PP+Ey+Ez$\\
\indent C8.6        $=0,6PP+Ey-Ez$\\
\indent C8.7        $=0,6PP-Ey+Ez$\\
\indent C8.8        $=0,6PP-Ey-Ez$\\

%No se consideran las combinaciones 7 y 8 de la NCH 3171 por ser siempre menores a las combinaciones 5.a y 5.b; y por el sistema... FAKE
 
Donde:

\begin{table}[H]
  \centering
    \begin{tabular}{lp{12cm}}
    \textbf{PP} & Carga por peso propio \\
    \textbf{SC} & Sobrecarga de uso \\
    \textbf{W} & Carga de viento \\
    \textbf{Lr} & Carga de uso de techo \\
    \textbf{S} & Carga de nieve \\
    \textbf{Ex, Ey} & Sismo horizontal \\
    \textbf{Ez} & Sismo vertical \\
    \textbf{H} & Carga debido a la presión lateral de tierra, a la presión del agua subterránea, o a la presión lateral de materiales a granel, más el empuje sísmico de suelo u otros materiales en las combinaciones que incluyan el efecto sísmico \\
    \end{tabular}%
\end{table}%

De acuerdo al apartado 9.2.1 b) de la NCH 3171, cuando la carga H esta presente, aplicada sobre los muros del subterráneo en este caso, su factor debe ser 1,0 en las combinaciones 2, 3.n y 4.n, y cuando la acción de H se suma al efecto de las cargas E o W en las combinaciones 5.a.n, 5.b.n, 6.a.n y 6.b.n.