\newpage
\section{Determinación de losas a diseñar y condiciones de apoyo} 
A continuación, se dan a conocer las losas a diseñar por piso, junto con sus condiciones de apoyo, numeración y otros parámetros relevantes a considerar para su posterior diseño.

\begin{table}[H]
  \centering
  \caption{Características de losas nivel -1.}
  \begin{tabular}{ccccccc}
    \hline
    \textbf{N° Losa} &      \textbf{Lx (m)} &      \textbf{Ly (m)} &      \textbf{e (cm)} &      \textbf{Caso} &      \textbf{SC $(kgf/m^2)$} &      \textbf{Tipo H.}      \bigstrut\\
        \hline
        0101 &      5 &      5,33 &      17 &      6 &      500 &      G35      \bigstrut[t]\\
        0102 &      5 &      7,2 &      17 &      6 &      500 &      G35      \\
        0103 &      5 &      7,2 &      17 &      6 &      500 &      G35      \\
        0104 &      5 &      7,89 &      17 &      6 &      500 &      G35      \\
        0105 &      4,45 &      5 &      17 &      6 &      500 &      G35      \\
        0106 &      3,6 &      5 &      17 &      6 &      500 &      G35      \\
        0107 &      5,33 &      5,54 &      17 &      6 &      500 &      G35      \\
        0108 &      5,54 &      7,2 &      17 &      6 &      500 &      G35      \\
        0109 &      5,54 &      7,2 &      17 &      6 &      500 &      G35      \\
        0110 &      5,54 &      12,34 &      17 &      6 &      500 &      G35      \\
        0111 &      3,6 &      5,54 &      17 &      6 &      500 &      G35      \\
        0112 &      6,05 &      9,34 &      17 &      6 &      400 &      G35      \\
        0113 &      4,65 &      5,6 &      17 &      6 &      400 &      G35      \\
        0114 &      4,65 &      5,6 &      17 &      6 &      500 &      G35      \\
        0115 &      6,05 &      9,49 &      17 &      6 &      500 &      G35      \\
        0116 &      4,45 &      10,5 &      17 &      6 &      500 &      G35      \\
        0117 &      3,6 &      10,5 &      17 &      6 &      500 &      G35      \\
        0118 &      7,16 &      11,34 &      17 &      6 &      400 &      G35      \\
        0119 &      1,4 &      11,2 &      17 &      6 &      400 &      G35      \\
        0120 &      2,1 &      4,04 &      17 &      5b &      400 &      G35      \\
        0121 &      2,9 &      6,36 &      17 &      6 &      400 &      G35      \\
        0122 &      4,25 &      6,36 &      17 &      6 &      500 &      G35      \\
        0123 &      6,49 &      7,16 &      17 &      6 &      500 &      G35      \\
        0124 &      8,24 &      18,54 &      17 &      6 &      500 &      G35      \\
        0125 &      5 &      8,24 &      17 &      6 &      500 &      G35      \\
        0126 &      5,52 &      8,24 &      17 &      6 &      500 &      G35      \\
        0127 &      7,09 &      10,15 &      17 &      6 &      500 &      G35      \\
        0128 &      3,6 &      10,15 &      17 &      6 &      500 &      G35      \\
        0129 &      2,32 &      4,3 &      17 &      6 &      400 &      G35      \bigstrut[b]\\
        \hline
      \end{tabular}
      \label{losas-1}
    \end{table}

\newpage

El siguiente esquema muestra la disposición de las losas en el modelo ETABS:

\insertimage[]{img/Losa-1}{width=10cm}{Disposición de losas en el nivel -1.}

\newpage

Para el nivel 1 se tendrá:

\begin{table}[H]
  \centering
  \caption{Características de losas nivel 1.}
  \begin{tabular}{ccccccc}
    \hline
    \textbf{N° Losa} &      \textbf{Lx (m)} &      \textbf{Ly (m)} &      \textbf{e (cm)} &      \textbf{Caso} &      \textbf{SC $(kgf/m^2)$} &      \textbf{Tipo H.}      \bigstrut\\
    \hline
        101 &      5 &      5,33 &      16 &      5a &      500 &      G35      \bigstrut[t]\\
        102 &      5 &      7,2 &      16 &      6 &      500 &      G35      \\
        103 &      5 &      7,2 &      16 &      6 &      500 &      G35      \\
        104 &      5 &      7,89 &      16 &      6 &      500 &      G35      \\
        105 &      4,93 &      5,33 &      16 &      4 &      500 &      G35      \\
        106 &      5,54 &      7,2 &      16 &      6 &      500 &      G35      \\
        107 &      5,54 &      7,2 &      16 &      6 &      500 &      G35      \\
        108 &      5,54 &      7,89 &      16 &      6 &      500 &      G35      \\
        109 &      5,54 &      4,45 &      16 &      5b &      500 &      G35      \\
        110 &      4,65 &      5,6 &      16 &      6 &      400 &      G35      \\
        111 &      4,65 &      5,6 &      16 &      6 &      500 &      G35      \\
        112 &      6,05 &      9,49 &      16 &      6 &      500 &      G35      \\
        113 &      2 &      2,825 &      16 &      4 &      400 &      G35      \\
        114 &      1,4 &      9,2 &      16 &      6 &      400 &      G35      \\
        115 &      2,9 &      6,36 &      16 &      6 &      400 &      G35      \\
        116 &      5 &      5,02 &      16 &      6 &      200 &      G35      \\
        117 &      5,82 &      6,49 &      16 &      6 &      200 &      G35      \\
        118 &      1,34 &      7,01 &      16 &      2a &      300 &      G35      \\
        119 &      2,1 &      4,04 &      16 &      5b &      400 &      G35      \\
        120 &      2,32 &      4,3 &      16 &      6 &      400 &      G35      \\
        121 &      1,6 &      6,05 &      16 &      5b &      400 &      G35      \bigstrut[b]\\
        \hline
      \end{tabular}
      \label{losas1}
    \end{table}

\newpage

El siguiente esquema muestra la disposición de las losas en el modelo ETABS:
\insertimage[]{img/losa1}{width=10cm}{Disposición de losas en el nivel 1.}

\newpage

Para el nivel 2 se tendrá:

\begin{table}[H]
  \centering
  \caption{Características de losas nivel 2.}
  \begin{tabular}{ccccccc}
    \hline
    \textbf{N° Losa} &      \textbf{Lx (m)} &      \textbf{Ly (m)} &      \textbf{e (cm)} &      \textbf{Caso} &      \textbf{SC $(kgf/m^2)$} &      \textbf{Tipo H.}      \bigstrut\\
    \hline
    201 &      6,05 &      10 &      16 &      6 &      200 &      G35      \bigstrut[t]\\
    202 &      4,65 &      5,6 &      16 &      6 &      200 &      G35      \\
    203 &      4,65 &      5,6 &      16 &      6 &      200 &      G35      \\
    204 &      6,05 &      7,89 &      16 &      6 &      200 &      G35      \\
    205 &      5,82 &      6,96 &      16 &      6 &      200 &      G35      \\
    206 &      5 &      5,02 &      16 &      6 &      200 &      G35      \\
    207 &      2,1 &      4,04 &      16 &      5b &      400 &      G35      \\
    208 &      1,4 &      11,2 &      16 &      6 &      400 &      G35      \\
    209 &      2,9 &      6,36 &      16 &      6 &      400 &      G35      \\
    210 &      5 &      5,02 &      16 &      6 &      200 &      G35      \\
    211 &      5,82 &      6,49 &      16 &      6 &      200 &      G35      \\
    212 &      1,51 &      5,33 &      16 &      2a &      300 &      G35      \\
    213 &      1,51 &      5,6 &      16 &      5b &      300 &      G35      \\
    214 &      1,51 &      5,6 &      16 &      5b &      300 &      G35      \\
    215 &      1,51 &      5,25 &      16 &      2a &      300 &      G35      \\
    216 &      1,04 &      3,83 &      16 &      2a &      300 &      G35      \\
    217 &      1,04 &      3,83 &      16 &      2a &      300 &      G35      \\
    218 &      1,34 &      7,01 &      16 &      2a &      300 &      G35      \\
    219 &      1,34 &      7,09 &      16 &      2a &      300 &      G35      \\
    220 &      0,74 &      3,83 &      16 &      2a &      300 &      G35      \\
    221 &      0,7 &      3,83 &      16 &      2a &      300 &      G35      \\
    222 &      2,32 &      4,3 &      16 &      5b &      400 &      G35      \bigstrut[b]\\
    \hline
      \end{tabular}
      \label{losas2}
    \end{table}

\newpage

El siguiente esquema muestra la disposición de las losas en el modelo ETABS:

\insertimage[]{img/losa2}{width=9cm}{Disposición de losas en el nivel 2.}

\newpage

Para el nivel 3 al 22 se tendrán las siguientes características en común, teniendo solo una variación del tipo de hormigón ( H35 desde nivel 3 al 7, H30 desde nivel 8 al 13 y H20 desde el nivel 14 al 22):

\begin{table}[H]
  \centering
  \caption{Características de losas desde nivel 3 al 22.}
  \begin{tabular}{ccccccc}
    \hline
    \textbf{N° Losa} &      \textbf{Lx (m)} &      \textbf{Ly (m)} &      \textbf{e (cm)} &      \textbf{Caso} &      \textbf{SC $(kgf/m^2)$} &      \textbf{Tipo H.}      \bigstrut\\
    \hline
    (3-22)01 &      6,05 &      10 &      16 &      6 &      200 &      G35,30,20      \bigstrut[t]\\
    (3-22)02 &      4,65 &      5,6 &      16 &      6 &      200 &      G35,30,20      \\
    (3-22)03 &      4,65 &      5,6 &      16 &      6 &      200 &      G35,30,20      \\
    (3-22)04 &      6,05 &      7,89 &      16 &      6 &      200 &      G35,30,20      \\
    (3-22)05 &      5,82 &      6,96 &      16 &      6 &      200 &      G35,30,20      \\
    (3-22)06 &      5 &      5,02 &      16 &      6 &      200 &      G35,30,20      \\
    (3-22)07 &      2,1 &      4,04 &      16 &      5b &      400 &      G35,30,20      \\
    (3-22)08 &      1,4 &      11,2 &      16 &      6 &      400 &      G35,30,20      \\
    (3-22)09 &      2,9 &      6,36 &      16 &      6 &      400 &      G35,30,20      \\
    (3-22)10 &      5 &      5,02 &      16 &      6 &      200 &      G35,30,20      \\
    (3-22)11 &      5,82 &      6,49 &      16 &      6 &      200 &      G35,30,20      \\
    (3-22)12 &      1,51 &      5,33 &      16 &      2a &      300 &      G35,30,20      \\
    (3-22)13 &      1,51 &      5,6 &      16 &      5b &      300 &      G35,30,20      \\
    (3-22)14 &      1,51 &      5,6 &      16 &      5b &      300 &      G35,30,20      \\
    (3-22)15 &      1,51 &      5,25 &      16 &      2a &      300 &      G35,30,20      \\
    (3-22)16 &      1,04 &      3,83 &      16 &      2a &      300 &      G35,30,20      \\
    (3-22)17 &      1,04 &      3,83 &      16 &      2a &      300 &      G35,30,20      \\
    (3-22)18 &      1,34 &      7,01 &      16 &      2a &      300 &      G35,30,20      \\
    (3-22)19 &      1,34 &      7,09 &      16 &      2a &      300 &      G35,30,20      \\
    (3-22)20 &      0,74 &      3,83 &      16 &      2a &      300 &      G35,30,20      \\
    (3-22)21 &      0,7 &      3,83 &      16 &      2a &      300 &      G35,30,20      \\
    (3-22)22 &      2,32 &      4,3 &      16 &      5b &      400 &      G35,30,20      \bigstrut[b]\\
    \hline
  \end{tabular}
  \label{losas322}
\end{table}

\newpage

El siguiente esquema muestra la disposición de las losas en el modelo ETABS para el nivel 3, el cual se repite en el resto de los pisos:

\insertimage[]{img/losa3}{width=9cm}{Disposición de losas en el nivel 3.}

\newpage

Para el nivel 23 se tendrá:

\begin{table}[H]
  \centering
  \caption{Características de losas del nivel 23.}
  \begin{tabular}{ccccccc}
    \hline
    \textbf{N° Losa} &      \textbf{Lx (m)} &      \textbf{Ly (m)} &      \textbf{e (cm)} &      \textbf{Caso} &      \textbf{SC $(kgf/m^2)$} &      \textbf{Tipo H.}      \bigstrut\\
    \hline
    2301 &      6,05 &      10 &      16 &      6 &      100 &      G20      \bigstrut[t]\\
    2302 &      4,65 &      5,6 &      16 &      6 &      100 &      G20      \\
    2303 &      4,65 &      5,6 &      16 &      6 &      100 &      G20      \\
    2304 &      6,05 &      7,89 &      16 &      6 &      100 &      G20      \\
    2305 &      5,82 &      6,96 &      16 &      6 &      100 &      G20      \\
    2306 &      5 &      5,02 &      16 &      6 &      100 &      G20      \\
    2307 &      2,1 &      4,04 &      16 &      5b &      100 &      G20      \\
    2308 &      1,4 &      11,2 &      16 &      6 &      100 &      G20      \\
    2309 &      2,9 &      6,36 &      16 &      6 &      100 &      G20      \\
    2310 &      5 &      5,02 &      16 &      6 &      100 &      G20      \\
    2311 &      5,82 &      6,49 &      16 &      6 &      100 &      G20      \\
    2312 &      1,51 &      9,1 &      16 &      2a &      100 &      G20      \\
    2313 &      1,51 &      5,6 &      16 &      5b &      100 &      G20      \\
    2314 &      1,51 &      5,6 &      16 &      5b &      100 &      G20      \\
    2315 &      1,51 &      8,93 &      16 &      2a &      100 &      G20      \\
    2316 &      1,04 &      3,83 &      16 &      4 &      100 &      G20      \\
    2317 &      1,04 &      3,83 &      16 &      4 &      100 &      G20      \\
    2318 &      1,34 &      10,69 &      16 &      2a &      100 &      G20      \\
    2319 &      1,34 &      10,86 &      16 &      2a &      100 &      G20      \\
    2320 &      0,74 &      3,83 &      16 &      4 &      100 &      G20      \\
    2321 &      0,7 &      3,83 &      16 &      4 &      100 &      G20      \\
    2322 &      2,32 &      4,3 &      16 &      5b &      100 &      G20      \bigstrut[b]\\
    \hline
  \end{tabular}
  \label{losas23}
\end{table}

\newpage

El siguiente esquema muestra la disposición de las losas en el modelo ETABS:

\insertimage[]{img/losa23}{width=9cm}{Disposición de losas en el nivel 23.}

\newpage

Para el nivel 24 se tendrá:

\begin{table}[H]
  \centering
  \caption{Características de losas del nivel 24.}
  \begin{tabular}{ccccccc}
    \hline
    \textbf{N° Losa} &      \textbf{Lx (m)} &      \textbf{Ly (m)} &      \textbf{e (cm)} &      \textbf{Caso} &      \textbf{SC $(kgf/m^2)$} &      \textbf{Tipo H.}      \bigstrut\\
    \hline
    2401 &      2,1 &      4,04 &      16 &      5a &      100 &      G20      \bigstrut[t]\\
    2402 &      2,32 &      4,3 &      16 &      6 &      100 &      G20      \\
    2403 &      2,9 &      6,36 &      16 &      6 &      100 &      G20      \bigstrut[b]\\
    \hline
  \end{tabular}
  \label{losas24}
\end{table}

El siguiente esquema muestra la disposición de las losas en el modelo ETABS:

\insertimage[]{img/losa24}{width=10cm}{Disposición de losas en el nivel 24.}

\newpage
Para la cubierta se tendrá:

\begin{table}[H]
  \centering
  \caption{Características de losa de cubierta.}
  \begin{tabular}{ccccccc}
    \hline
    \textbf{N° Losa} &      \textbf{Lx (m)} &      \textbf{Ly (m)} &      \textbf{e (cm)} &      \textbf{Caso} &      \textbf{SC $(kgf/m^2)$} &      \textbf{Tipo H.}      \bigstrut\\
    \hline
    CU &      2,2 &      4,04 &      16 &      6 &      100 &      G20      \bigstrut\\
    \hline
  \end{tabular}
  \label{losacu}
\end{table}


El siguiente esquema muestra la disposición de la losa en el modelo ETABS:

\insertimage[]{img/losacubierta}{width=10cm}{Disposición de losa en la cubierta.}
