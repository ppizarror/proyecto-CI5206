% Template:     Informe/Reporte LaTeX
% Documento:    Archivo principal
% Versión:      6.1.0 (03/11/2018)
% Codificación: UTF-8
%
% Autor: Pablo Pizarro R. @ppizarror
%        Facultad de Ciencias Físicas y Matemáticas
%        Universidad de Chile
%        pablo.pizarro@ing.uchile.cl, ppizarror.com
%
% Manual template: [https://latex.ppizarror.com/Template-Informe/]
% Licencia MIT:    [https://opensource.org/licenses/MIT/]

% CREACIÓN DEL DOCUMENTO
\documentclass[letterpaper,11pt]{article} % Articulo tamaño carta, 11pt
\usepackage[utf8]{inputenc} % Codificación UTF-8

% INFORMACIÓN DEL DOCUMENTO
\def\titulodelinforme {Diseño y análisis de losas tradicionales de hormigón}
\def\temaatratar {Proyecto de Hormigón Armado - Entrega N°5}

\def\autordeldocumento {Grupo N1}
\def\nombredelcurso {Proyecto de Hormigón Armado}
\def\codigodelcurso {CI5206-2}

\def\nombreuniversidad {Universidad de Chile}
\def\nombrefacultad {Facultad de Ciencias Físicas y Matemáticas}
\def\departamentouniversidad {Departamento de Ingeniería Civil}
\def\imagendepartamento {dic}
\def\imagendepartamentoescala {0.2}
\def\localizacionuniversidad {Santiago, Chile}

% INTEGRANTES, PROFESORES Y FECHAS
\def\tablaintegrantes {
\begin{tabular}{ll}
	Integrantes:
	& \begin{tabular}[t]{@{}l@{}}
		Mauricio Leal V. \\
		Pablo Pizarro R. \\
		Ignacio Yáñez G.
	\end{tabular} \\
	Profesor:
	& \begin{tabular}[t]{@{}l@{}}
		Juan Mendoza V.
	\end{tabular} \\
	Auxiliar:
	& \begin{tabular}[t]{@{}l@{}}
		Felipe Andrade T.
	\end{tabular} \\
	& \\
	\multicolumn{2}{l}{Fecha de entrega: 21 de Noviembre de 2018} \\
	\multicolumn{2}{l}{\localizacionuniversidad}
\end{tabular}}{
}

% CONFIGURACIONES
\input{lib/config}

% IMPORTACIÓN DE LIBRERÍAS
\input{lib/env/imports}

% IMPORTACIÓN DE FUNCIONES Y ENTORNOS
\input{lib/cmd/all}

% IMPORTACIÓN DE ESTILOS
\input{lib/style/all}

% CONFIGURACIÓN INICIAL DEL DOCUMENTO
\input{lib/cfg/init}

% INICIO DE LAS PÁGINAS
\begin{document}

% PORTADA
\input{lib/page/portrait} % Se puede borrar

% CONFIGURACIÓN DE PÁGINA Y ENCABEZADOS
\input{lib/cfg/page}

% TABLA DE CONTENIDOS - ÍNDICE
% Template:     Informe/Reporte LaTeX
% Documento:    Índice
% Versión:      6.0.1 (21/10/2018)
% Codificación: UTF-8
%
% Autor: Pablo Pizarro R. @ppizarror
%        Facultad de Ciencias Físicas y Matemáticas
%        Universidad de Chile
%        pablo.pizarro@ing.uchile.cl, ppizarror.com
%
% Manual template: [http://latex.ppizarror.com/Template-Informe/]
% Licencia MIT:    [https://opensource.org/licenses/MIT/]

\ifthenelse{\equal{\showindex}{true}}{
	\newpage
	\begingroup
	\sectionfont{\color{\indextitlecolor} \fontsizetitlei \styletitlei \selectfont}
	\ifthenelse{\equal{\addindextobookmarks}{true}}{
		\belowpdfbookmark{\nomltcont}{contents}}{
	}
	\tocloftpagestyle{fancy}
	\ifthenelse{\equal{\showdotontitles}{true}}{
		\def\cftsecaftersnum {.}
		\def\cftsubsecaftersnum {.}
		\def\cftsubsubsecaftersnum {.}
		\def\cftsubsubsubsecaftersnum {.}
		}{
	}
	\def\cftfigaftersnum {\charafterobjectindex\enspace}
	\def\cftsubfigaftersnum {\charafterobjectindex\enspace}
	\def\cfttabaftersnum {\charafterobjectindex\enspace}
	\def\cftlstlistingaftersnum {\charafterobjectindex\enspace}
	\renewcommand{\cftdot}{\charnumpageindex}
	\ifthenelse{\equal{\showlinenumbers}{true}}{
		\nolinenumbers}{
	}
	\ifthenelse{\equal{\objectindexindent}{true}}{
		\def\cftlstlistingindent {1.495em}
	}{
		\setlength{\cfttabindent}{0in}
		\setlength{\cftfigindent}{0in}
		\setlength{\cftsubfigindent}{0in}
		\setlength{\cftfigindent}{0in}
		\def\cftlstlistingindent {0.01em}
	}
	\ifthenelse{\equal{\equalmarginnumobject}{true}}{
		\ifthenelse{\equal{\showsectioncaption}{none}}{
			\def\cftdefautnumwidth {2.3em}
		}{
		\ifthenelse{\equal{\showsectioncaption}{sec}}{
			\def\cftdefautnumwidth {3.0em}
		}{
		\ifthenelse{\equal{\showsectioncaption}{ssec}}{
			\def\cftdefautnumwidth {3.8em}
		}{
		\ifthenelse{\equal{\showsectioncaption}{sssec}}{
			\def\cftdefautnumwidth {4.3em}
		}{
			\throwbadconfig{Valor configuracion incorrecto}{\showsectioncaption}{none,sec,ssec,sssec}}}}
		}
		\def\cftfignumwidth {\cftdefautnumwidth}
		\def\cftsubfignumwidth {\cftdefautnumwidth}
		\def\cfttabnumwidth {\cftdefautnumwidth}
		\def\cftlstlistingnumwidth {\cftdefautnumwidth}}{
	}
	\ifthenelse{\equal{\showindexofcontents}{true}}{\tableofcontents}{}
	\iftotalfigures
		\ifthenelse{\equal{\showindexoffigures}{true}}{
			\ifthenelse{\equal{\indexforcenewpage}{true}}{\newpage}{}
			\listoffigures
		}{}
	\fi
	\iftotaltables
		\ifthenelse{\equal{\showindexoftables}{true}}{
			\ifthenelse{\equal{\indexforcenewpage}{true}}{\newpage}{}
			\listoftables
		}{}
	\fi
	\iftotallstlistings
		\ifthenelse{\equal{\showindexofcode}{true}}{
			\ifthenelse{\equal{\indexforcenewpage}{true}}{\newpage}{}
			\lstlistoflistings
		}{}
	\fi
	\endgroup
	\ifthenelse{\equal{\addemptypagetwosides}{true}}{
		\vfill
		\checkoddpage
		\ifoddpage
		\else
			\newpage
			\null
			\thispagestyle{empty}
			\newpage
			\addtocounter{page}{-1}
		\fi}{
	}
}{}
 % Se puede borrar

% CONFIGURACIONES FINALES
% Template:     Informe/Reporte LaTeX
% Documento:    Configuraciones finales
% Versión:      6.1.6 (14/12/2018)
% Codificación: UTF-8
%
% Autor: Pablo Pizarro R. @ppizarror
%        Facultad de Ciencias Físicas y Matemáticas
%        Universidad de Chile
%        pablo.pizarro@ing.uchile.cl, ppizarror.com
%
% Manual template: [https://latex.ppizarror.com/Template-Informe/]
% Licencia MIT:    [https://opensource.org/licenses/MIT/]

\markboth{}{}
\newpage
\ifthenelse{\equal{\disablehfrightmark}{false}}{
	\ifthenelse{\equal{\hfstyle}{style1}}{
		\fancyhead[L]{\nouppercase{\leftmark}}}{
	}
	\ifthenelse{\equal{\hfstyle}{style2}}{
		\fancyhead[L]{\nouppercase{\leftmark}}}{
	}
	\ifthenelse{\equal{\hfstyle}{style4}}{
		\fancyhead[L]{\nouppercase{\leftmark}}}{
	}
	\ifthenelse{\equal{\hfstyle}{style5}}{
		\fancyhead[R]{\nouppercase{\leftmark}}}{
	}
	\ifthenelse{\equal{\hfstyle}{style9}}{
		\fancyhead[L]{\nouppercase{\leftmark}}}{
	}
	\ifthenelse{\equal{\hfstyle}{style10}}{
		\fancyhead[L]{\nouppercase{\leftmark}}}{
	}
\ifthenelse{\equal{\hfstyle}{style11}}{
		\fancyhead[L]{\nouppercase{\leftmark}}}{
	}
\ifthenelse{\equal{\hfstyle}{style14}}{
		\fancyhead[L]{\nouppercase{\leftmark}}}{
	}}{
}
\sectionfont{\color{\titlecolor} \fontsizetitle \styletitle \selectfont}
\subsectionfont{\color{\subtitlecolor} \fontsizesubtitle \stylesubtitle \selectfont}
\subsubsectionfont{\color{\subsubtitlecolor} \fontsizesubsubtitle \stylesubsubtitle \selectfont}
\titleformat{\subsubsubsection}{\color{\ssstitlecolor} \normalfont \fontsizessstitle \stylessstitle}{\thesubsubsubsection}{1em}{}
\titlespacing*{\subsubsubsection}{0pt}{3.25ex plus 1ex minus .2ex}{1.5ex plus .2ex}
\ifthenelse{\equal{\showsectioncaption}{none}}{
}{
\ifthenelse{\equal{\showsectioncaption}{sec}}{
	\counterwithin{equation}{section}
	\counterwithin{figure}{section}
	\counterwithin{lstlisting}{section}
	\counterwithin{table}{section}
}{
\ifthenelse{\equal{\showsectioncaption}{ssec}}{
	\counterwithin{equation}{subsection}
	\counterwithin{figure}{subsection}
	\counterwithin{lstlisting}{subsection}
	\counterwithin{table}{subsection}
}{
\ifthenelse{\equal{\showsectioncaption}{sssec}}{
	\counterwithin{equation}{subsubsection}
	\counterwithin{figure}{subsubsection}
	\counterwithin{lstlisting}{subsubsection}
	\counterwithin{table}{subsubsection}
}{
\ifthenelse{\equal{\showsectioncaption}{ssssec}}{
	\counterwithin{equation}{subsubsubsection}
	\counterwithin{figure}{subsubsubsection}
	\counterwithin{lstlisting}{subsubsubsection}
	\counterwithin{table}{subsubsubsection}
}{
	\throwbadconfig{Valor configuracion incorrecto}{\showsectioncaption}{none,sec,ssec,sssec,ssssec}
}}}}}
\ifthenelse{\equal{\predocuseromannumber}{true}}{
	\renewcommand{\thepage}{\arabic{page}}}{
}
\ifthenelse{\equal{\resetpagnumafterindex}{true}}{
	\setcounter{page}{1}}{
}
\setcounter{section}{0}
\setcounter{footnote}{0}
\ifthenelse{\equal{\showlinenumbers}{true}}{
	\linenumbers}{
}


% ======================= INICIO DEL DOCUMENTO =======================
\section{Determinación de cargas de diseño.}

    A continuación se presentan las cargas a considerar en el diseño:
    
    \begin{table}[htbp]
      \centering
      \caption{Cargas de peso propio.}
        \begin{tabular}{cc}
        \hline
        \textbf{Elemento} & \boldmath{}\textbf{Peso Propio $[kgf/m^2]$}\unboldmath{} \bigstrut\\
        \hline
        \boldmath{}\textbf{$PP_{Tabique}$}\unboldmath{} & \textbf{100} \bigstrut[t]\\
        \textit{$\gamma_{yeso} [kgf/m2 \cdot cm]$} & \textit{10} \\
        \textit{$e_{yeso} [cm]$} & \textit{2,5} \\
        \boldmath{}\textbf{$PP_{Yeso}$}\unboldmath{} & \textbf{25} \\
        $\gamma_{sl} [kgf/m2 \cdot cm]$ & 20 \\
        $e_{sl} [cm]$ & 5 \\
        \boldmath{}\textbf{$PP_{Sobrelosa}$}\unboldmath{} & \textbf{100} \bigstrut[b]\\
        \hline
        \boldmath{}\textbf{$PP_{adic}$}\unboldmath{} & \textbf{225} \bigstrut\\
        \hline
        \end{tabular}%
      \label{peso_propio}%
    \end{table}%

    \begin{table}[htbp]
      \centering
      \caption{Sobrecargas.}
        \begin{tabular}{cc}
        \hline
        Ocupación & Sobrecargas $[kgf/m^2]$ \bigstrut\\
        \hline
        Habitacional & 200 \bigstrut[t]\\
        Área común y escalera & 400 \\
        Balcones & 300 \\
        Autos & 500 \\
        Techo & 100 \bigstrut[b]\\
        \hline
        \end{tabular}%
      \label{sobrecarga}%
    \end{table}%
    
    A esto se agrega el peso propio de cada losa y, luego de mayorar las cargas, se obtienen las solicitaciones ($qu$) en cada elemento.
\newpage
\section{Especificación de espesores de losa por piso, armaduras mínimas y calidad de materiales.}

    En etapas previas se definió el espesor de losas, con el que se determina la armadura mínima por piso, y la calidad de materiales a usar según se muestra en la siguiente tabla:
    
    \begin{table}[H]
      \centering
      \caption{Armaduras mínimas y calidad de materiales.}
         \begin{tabular}{ccrlcc}
            \hline
            \textbf{Nivel} & \textbf{e [cm]} & \multicolumn{2}{c}{\boldmath{}\textbf{$As_{min} \quad [cm^2/m]$}\unboldmath{}} & \textbf{Hormigón} & \textbf{Acero refuerzo} \bigstrut\\
            \hline
            Cubierta & 16    & 2,88  & $\phi8@17$ & G20   & A63-42H \bigstrut[t]\\
            24    & 16    & 2,88  & $\phi8@17$ & G20   & A63-42H \\
            23    & 16    & 2,88  & $\phi8@17$ & G20   & A63-42H \\
            22    & 16    & 2,88  & $\phi8@17$ & G20   & A63-42H \\
            21    & 16    & 2,88  & $\phi8@17$ & G20   & A63-42H \\
            20    & 16    & 2,88  & $\phi8@17$ & G20   & A63-42H \\
            19    & 16    & 2,88  & $\phi8@17$ & G20   & A63-42H \\
            18    & 16    & 2,88  & $\phi8@17$ & G20   & A63-42H \\
            17    & 16    & 2,88  & $\phi8@17$ & G20   & A63-42H \\
            16    & 16    & 2,88  & $\phi8@17$ & G20   & A63-42H \\
            15    & 16    & 2,88  & $\phi8@17$ & G20   & A63-42H \\
            14    & 16    & 2,88  & $\phi8@17$ & G20   & A63-42H \\
            13    & 16    & 2,88  & $\phi8@17$ & G30   & A63-42H \\
            12    & 16    & 2,88  & $\phi8@17$ & G30   & A63-42H \\
            11    & 16    & 2,88  & $\phi8@17$ & G30   & A63-42H \\
            10    & 16    & 2,88  & $\phi8@17$ & G30   & A63-42H \\
            9     & 16    & 2,88  & $\phi8@17$ & G30   & A63-42H \\
            8     & 16    & 2,88  & $\phi8@17$ & G30   & A63-42H \\
            7     & 16    & 2,88  & $\phi8@17$ & G35   & A63-42H \\
            6     & 16    & 2,88  & $\phi8@17$ & G35   & A63-42H \\
            5     & 16    & 2,88  & $\phi8@17$ & G35   & A63-42H \\
            4     & 16    & 2,88  & $\phi8@17$ & G35   & A63-42H \\
            3     & 16    & 2,88  & $\phi8@17$ & G35   & A63-42H \\
            2     & 16    & 2,88  & $\phi8@17$ & G35   & A63-42H \\
            1     & 16    & 2,88  & $\phi8@17$ & G35   & A63-42H \\
            -1    & 17    & 3,06  & $\phi8@16$ & G35   & A63-42H \bigstrut[b]\\
            \hline
        \end{tabular}%
      \label{materiales}%
    \end{table}%
\newpage
\section{Determinación de losas a diseñar y condiciones de apoyo} 
A continuación, se dan a conocer las losas a diseñar por piso, junto con sus condiciones de apoyo, numeración y otros parámetros relevantes a considerar para su posterior diseño.

\begin{table}[H]
  \centering
  \caption{Características de losas nivel -1.}
  \begin{tabular}{ccccccc}
    \hline
    \textbf{N° Losa} &      \textbf{Lx (m)} &      \textbf{Ly (m)} &      \textbf{e (cm)} &      \textbf{Caso} &      \textbf{SC $(kgf/m^2)$} &      \textbf{Tipo H.}      \bigstrut\\
        \hline
        0101 &      5 &      5,33 &      17 &      6 &      500 &      G35      \bigstrut[t]\\
        0102 &      5 &      7,2 &      17 &      6 &      500 &      G35      \\
        0103 &      5 &      7,2 &      17 &      6 &      500 &      G35      \\
        0104 &      5 &      7,89 &      17 &      6 &      500 &      G35      \\
        0105 &      4,45 &      5 &      17 &      6 &      500 &      G35      \\
        0106 &      3,6 &      5 &      17 &      6 &      500 &      G35      \\
        0107 &      5,33 &      5,54 &      17 &      6 &      500 &      G35      \\
        0108 &      5,54 &      7,2 &      17 &      6 &      500 &      G35      \\
        0109 &      5,54 &      7,2 &      17 &      6 &      500 &      G35      \\
        0110 &      5,54 &      12,34 &      17 &      6 &      500 &      G35      \\
        0111 &      3,6 &      5,54 &      17 &      6 &      500 &      G35      \\
        0112 &      6,05 &      9,34 &      17 &      6 &      400 &      G35      \\
        0113 &      4,65 &      5,6 &      17 &      6 &      400 &      G35      \\
        0114 &      4,65 &      5,6 &      17 &      6 &      500 &      G35      \\
        0115 &      6,05 &      9,49 &      17 &      6 &      500 &      G35      \\
        0116 &      4,45 &      10,5 &      17 &      6 &      500 &      G35      \\
        0117 &      3,6 &      10,5 &      17 &      6 &      500 &      G35      \\
        0118 &      7,16 &      11,34 &      17 &      6 &      400 &      G35      \\
        0119 &      1,4 &      11,2 &      17 &      6 &      400 &      G35      \\
        0120 &      2,1 &      4,04 &      17 &      5b &      400 &      G35      \\
        0121 &      2,9 &      6,36 &      17 &      6 &      400 &      G35      \\
        0122 &      4,25 &      6,36 &      17 &      6 &      500 &      G35      \\
        0123 &      6,49 &      7,16 &      17 &      6 &      500 &      G35      \\
        0124 &      8,24 &      18,54 &      17 &      6 &      500 &      G35      \\
        0125 &      5 &      8,24 &      17 &      6 &      500 &      G35      \\
        0126 &      5,52 &      8,24 &      17 &      6 &      500 &      G35      \\
        0127 &      7,09 &      10,15 &      17 &      6 &      500 &      G35      \\
        0128 &      3,6 &      10,15 &      17 &      6 &      500 &      G35      \\
        0129 &      2,32 &      4,3 &      17 &      6 &      400 &      G35      \bigstrut[b]\\
        \hline
      \end{tabular}
      \label{losas-1}
    \end{table}

\newpage

El siguiente esquema muestra la disposición de las losas en el modelo ETABS:

\insertimage[]{img/Losa-1}{width=10cm}{Disposición de losas en el nivel -1.}

\newpage

Para el nivel 1 se tendrá:

\begin{table}[H]
  \centering
  \caption{Características de losas nivel 1.}
  \begin{tabular}{ccccccc}
    \hline
    \textbf{N° Losa} &      \textbf{Lx (m)} &      \textbf{Ly (m)} &      \textbf{e (cm)} &      \textbf{Caso} &      \textbf{SC $(kgf/m^2)$} &      \textbf{Tipo H.}      \bigstrut\\
    \hline
        101 &      5 &      5,33 &      16 &      5a &      500 &      G35      \bigstrut[t]\\
        102 &      5 &      7,2 &      16 &      6 &      500 &      G35      \\
        103 &      5 &      7,2 &      16 &      6 &      500 &      G35      \\
        104 &      5 &      7,89 &      16 &      6 &      500 &      G35      \\
        105 &      4,93 &      5,33 &      16 &      4 &      500 &      G35      \\
        106 &      5,54 &      7,2 &      16 &      6 &      500 &      G35      \\
        107 &      5,54 &      7,2 &      16 &      6 &      500 &      G35      \\
        108 &      5,54 &      7,89 &      16 &      6 &      500 &      G35      \\
        109 &      5,54 &      4,45 &      16 &      5b &      500 &      G35      \\
        110 &      4,65 &      5,6 &      16 &      6 &      400 &      G35      \\
        111 &      4,65 &      5,6 &      16 &      6 &      500 &      G35      \\
        112 &      6,05 &      9,49 &      16 &      6 &      500 &      G35      \\
        113 &      2 &      2,825 &      16 &      4 &      400 &      G35      \\
        114 &      1,4 &      9,2 &      16 &      6 &      400 &      G35      \\
        115 &      2,9 &      6,36 &      16 &      6 &      400 &      G35      \\
        116 &      5 &      5,02 &      16 &      6 &      200 &      G35      \\
        117 &      5,82 &      6,49 &      16 &      6 &      200 &      G35      \\
        118 &      1,34 &      7,01 &      16 &      2a &      300 &      G35      \\
        119 &      2,1 &      4,04 &      16 &      5b &      400 &      G35      \\
        120 &      2,32 &      4,3 &      16 &      6 &      400 &      G35      \\
        121 &      1,6 &      6,05 &      16 &      5b &      400 &      G35      \bigstrut[b]\\
        \hline
      \end{tabular}
      \label{losas1}
    \end{table}

\newpage

El siguiente esquema muestra la disposición de las losas en el modelo ETABS:
\insertimage[]{img/losa1}{width=10cm}{Disposición de losas en el nivel 1.}

\newpage

Para el nivel 2 se tendrá:

\begin{table}[H]
  \centering
  \caption{Características de losas nivel 2.}
  \begin{tabular}{ccccccc}
    \hline
    \textbf{N° Losa} &      \textbf{Lx (m)} &      \textbf{Ly (m)} &      \textbf{e (cm)} &      \textbf{Caso} &      \textbf{SC $(kgf/m^2)$} &      \textbf{Tipo H.}      \bigstrut\\
    \hline
    201 &      6,05 &      10 &      16 &      6 &      200 &      G35      \bigstrut[t]\\
    202 &      4,65 &      5,6 &      16 &      6 &      200 &      G35      \\
    203 &      4,65 &      5,6 &      16 &      6 &      200 &      G35      \\
    204 &      6,05 &      7,89 &      16 &      6 &      200 &      G35      \\
    205 &      5,82 &      6,96 &      16 &      6 &      200 &      G35      \\
    206 &      5 &      5,02 &      16 &      6 &      200 &      G35      \\
    207 &      2,1 &      4,04 &      16 &      5b &      400 &      G35      \\
    208 &      1,4 &      11,2 &      16 &      6 &      400 &      G35      \\
    209 &      2,9 &      6,36 &      16 &      6 &      400 &      G35      \\
    210 &      5 &      5,02 &      16 &      6 &      200 &      G35      \\
    211 &      5,82 &      6,49 &      16 &      6 &      200 &      G35      \\
    212 &      1,51 &      5,33 &      16 &      2a &      300 &      G35      \\
    213 &      1,51 &      5,6 &      16 &      5b &      300 &      G35      \\
    214 &      1,51 &      5,6 &      16 &      5b &      300 &      G35      \\
    215 &      1,51 &      5,25 &      16 &      2a &      300 &      G35      \\
    216 &      1,04 &      3,83 &      16 &      2a &      300 &      G35      \\
    217 &      1,04 &      3,83 &      16 &      2a &      300 &      G35      \\
    218 &      1,34 &      7,01 &      16 &      2a &      300 &      G35      \\
    219 &      1,34 &      7,09 &      16 &      2a &      300 &      G35      \\
    220 &      0,74 &      3,83 &      16 &      2a &      300 &      G35      \\
    221 &      0,7 &      3,83 &      16 &      2a &      300 &      G35      \\
    222 &      2,32 &      4,3 &      16 &      5b &      400 &      G35      \bigstrut[b]\\
    \hline
      \end{tabular}
      \label{losas2}
    \end{table}

\newpage

El siguiente esquema muestra la disposición de las losas en el modelo ETABS:

\insertimage[]{img/losa2}{width=9cm}{Disposición de losas en el nivel 2.}

\newpage

Para el nivel 3 al 22 se tendrán las siguientes características en común, teniendo solo una variación del tipo de hormigón ( H35 desde nivel 3 al 7, H30 desde nivel 8 al 13 y H20 desde el nivel 14 al 22):

\begin{table}[H]
  \centering
  \caption{Características de losas desde nivel 3 al 22.}
  \begin{tabular}{ccccccc}
    \hline
    \textbf{N° Losa} &      \textbf{Lx (m)} &      \textbf{Ly (m)} &      \textbf{e (cm)} &      \textbf{Caso} &      \textbf{SC $(kgf/m^2)$} &      \textbf{Tipo H.}      \bigstrut\\
    \hline
    (3-22)01 &      6,05 &      10 &      16 &      6 &      200 &      G35,30,20      \bigstrut[t]\\
    (3-22)02 &      4,65 &      5,6 &      16 &      6 &      200 &      G35,30,20      \\
    (3-22)03 &      4,65 &      5,6 &      16 &      6 &      200 &      G35,30,20      \\
    (3-22)04 &      6,05 &      7,89 &      16 &      6 &      200 &      G35,30,20      \\
    (3-22)05 &      5,82 &      6,96 &      16 &      6 &      200 &      G35,30,20      \\
    (3-22)06 &      5 &      5,02 &      16 &      6 &      200 &      G35,30,20      \\
    (3-22)07 &      2,1 &      4,04 &      16 &      5b &      400 &      G35,30,20      \\
    (3-22)08 &      1,4 &      11,2 &      16 &      6 &      400 &      G35,30,20      \\
    (3-22)09 &      2,9 &      6,36 &      16 &      6 &      400 &      G35,30,20      \\
    (3-22)10 &      5 &      5,02 &      16 &      6 &      200 &      G35,30,20      \\
    (3-22)11 &      5,82 &      6,49 &      16 &      6 &      200 &      G35,30,20      \\
    (3-22)12 &      1,51 &      5,33 &      16 &      2a &      300 &      G35,30,20      \\
    (3-22)13 &      1,51 &      5,6 &      16 &      5b &      300 &      G35,30,20      \\
    (3-22)14 &      1,51 &      5,6 &      16 &      5b &      300 &      G35,30,20      \\
    (3-22)15 &      1,51 &      5,25 &      16 &      2a &      300 &      G35,30,20      \\
    (3-22)16 &      1,04 &      3,83 &      16 &      2a &      300 &      G35,30,20      \\
    (3-22)17 &      1,04 &      3,83 &      16 &      2a &      300 &      G35,30,20      \\
    (3-22)18 &      1,34 &      7,01 &      16 &      2a &      300 &      G35,30,20      \\
    (3-22)19 &      1,34 &      7,09 &      16 &      2a &      300 &      G35,30,20      \\
    (3-22)20 &      0,74 &      3,83 &      16 &      2a &      300 &      G35,30,20      \\
    (3-22)21 &      0,7 &      3,83 &      16 &      2a &      300 &      G35,30,20      \\
    (3-22)22 &      2,32 &      4,3 &      16 &      5b &      400 &      G35,30,20      \bigstrut[b]\\
    \hline
  \end{tabular}
  \label{losas322}
\end{table}

\newpage

El siguiente esquema muestra la disposición de las losas en el modelo ETABS para el nivel 3, el cual se repite en el resto de los pisos:

\insertimage[]{img/losa3}{width=9cm}{Disposición de losas en el nivel 3.}

\newpage

Para el nivel 23 se tendrá:

\begin{table}[H]
  \centering
  \caption{Características de losas del nivel 23.}
  \begin{tabular}{ccccccc}
    \hline
    \textbf{N° Losa} &      \textbf{Lx (m)} &      \textbf{Ly (m)} &      \textbf{e (cm)} &      \textbf{Caso} &      \textbf{SC $(kgf/m^2)$} &      \textbf{Tipo H.}      \bigstrut\\
    \hline
    2301 &      6,05 &      10 &      16 &      6 &      100 &      G20      \bigstrut[t]\\
    2302 &      4,65 &      5,6 &      16 &      6 &      100 &      G20      \\
    2303 &      4,65 &      5,6 &      16 &      6 &      100 &      G20      \\
    2304 &      6,05 &      7,89 &      16 &      6 &      100 &      G20      \\
    2305 &      5,82 &      6,96 &      16 &      6 &      100 &      G20      \\
    2306 &      5 &      5,02 &      16 &      6 &      100 &      G20      \\
    2307 &      2,1 &      4,04 &      16 &      5b &      100 &      G20      \\
    2308 &      1,4 &      11,2 &      16 &      6 &      100 &      G20      \\
    2309 &      2,9 &      6,36 &      16 &      6 &      100 &      G20      \\
    2310 &      5 &      5,02 &      16 &      6 &      100 &      G20      \\
    2311 &      5,82 &      6,49 &      16 &      6 &      100 &      G20      \\
    2312 &      1,51 &      9,1 &      16 &      2a &      100 &      G20      \\
    2313 &      1,51 &      5,6 &      16 &      5b &      100 &      G20      \\
    2314 &      1,51 &      5,6 &      16 &      5b &      100 &      G20      \\
    2315 &      1,51 &      8,93 &      16 &      2a &      100 &      G20      \\
    2316 &      1,04 &      3,83 &      16 &      4 &      100 &      G20      \\
    2317 &      1,04 &      3,83 &      16 &      4 &      100 &      G20      \\
    2318 &      1,34 &      10,69 &      16 &      2a &      100 &      G20      \\
    2319 &      1,34 &      10,86 &      16 &      2a &      100 &      G20      \\
    2320 &      0,74 &      3,83 &      16 &      4 &      100 &      G20      \\
    2321 &      0,7 &      3,83 &      16 &      4 &      100 &      G20      \\
    2322 &      2,32 &      4,3 &      16 &      5b &      100 &      G20      \bigstrut[b]\\
    \hline
  \end{tabular}
  \label{losas23}
\end{table}

\newpage

El siguiente esquema muestra la disposición de las losas en el modelo ETABS:

\insertimage[]{img/losa23}{width=9cm}{Disposición de losas en el nivel 23.}

\newpage

Para el nivel 24 se tendrá:

\begin{table}[H]
  \centering
  \caption{Características de losas del nivel 24.}
  \begin{tabular}{ccccccc}
    \hline
    \textbf{N° Losa} &      \textbf{Lx (m)} &      \textbf{Ly (m)} &      \textbf{e (cm)} &      \textbf{Caso} &      \textbf{SC $(kgf/m^2)$} &      \textbf{Tipo H.}      \bigstrut\\
    \hline
    2401 &      2,1 &      4,04 &      16 &      5a &      100 &      G20      \bigstrut[t]\\
    2402 &      2,32 &      4,3 &      16 &      6 &      100 &      G20      \\
    2403 &      2,9 &      6,36 &      16 &      6 &      100 &      G20      \bigstrut[b]\\
    \hline
  \end{tabular}
  \label{losas24}
\end{table}

El siguiente esquema muestra la disposición de las losas en el modelo ETABS:

\insertimage[]{img/losa24}{width=10cm}{Disposición de losas en el nivel 24.}

\newpage
Para la cubierta se tendrá:

\begin{table}[H]
  \centering
  \caption{Características de losa de cubierta.}
  \begin{tabular}{ccccccc}
    \hline
    \textbf{N° Losa} &      \textbf{Lx (m)} &      \textbf{Ly (m)} &      \textbf{e (cm)} &      \textbf{Caso} &      \textbf{SC $(kgf/m^2)$} &      \textbf{Tipo H.}      \bigstrut\\
    \hline
    CU &      2,2 &      4,04 &      16 &      6 &      100 &      G20      \bigstrut\\
    \hline
  \end{tabular}
  \label{losacu}
\end{table}


El siguiente esquema muestra la disposición de la losa en el modelo ETABS:

\insertimage[]{img/losacubierta}{width=10cm}{Disposición de losa en la cubierta.}

\newpage 
\section{Análisis de losas}

    La metodología de análisis de cada losa es la siguiente:
        \begin{itemize}
            \item Definir dimensiones: lado corto, lado largo y espesor.
            \item Con el parámetro $\epsilon$, si este es menor a 2, determinar los valores de \texttt{mx}, \texttt{my}, \texttt{mex} y \texttt{mey}, los parámetros \texttt{k} y de alternancia de carga $\Delta x$ y $\Delta y$, de las tablas de Czerny y Marcus. Si $\epsilon$ es mayor a 2, se siguen las recomendaciones para franja de losa, mostradas en la figura \ref{franja_losa}.
            \item Una vez determinado el parámetro $\phi$ se puede determinar el espesor mínimo que debiera cumplir la losa, considerando el coeficiente de uso ($\lambda$) correspondiente, y se verifica si se encuentra bajo el escogido previamente. 
            \item Se determinan la sobrecarga según el uso y el peso propio dado por su espesor y peso adicional, y se mayoran para determinar la carga de diseño. Esta se multiplica por el área (se suponen todas las losas rectangulares) para determinar el valor de \texttt{Ku}.
            \item Se determinan los momentos máximos, tanto positivos como negativos, a partir de los valores de las tablas o con las recomendaciones de franja de losa, según sea el caso.
            Se consideran el efecto de la alternancia de carga en los momentos positivos y el aumento de armaduras por torsión.
            \item Realizando un análisis de flexión simple se determina el área de acero requerida. En caso de ser menor al área mínima se usa esta última, si no se escoge la armadura correspondiente al área calculada.
        \end{itemize}
        
    \insertimage[\label{franja_losa} ]{franja_losa}{width=13cm}{Recomendaciones para franjas de losa.}
    
    \newpage
    A continuación se muestran dos ejemplos de cálculo, para una losa rectangular y una franja, respectivamente. 
    
    \begin{table}[H]
      \centering
      \caption{Ejemplo de análisis losa regular: losa N° 0101.}
        \resizebox{.7\textwidth}{!}{
        \begin{tabular}{cccp{8.215em}c}
            \hline
            \multicolumn{2}{c}{\textbf{N° Losa}} &       & \multicolumn{2}{c}{\textbf{0101}} \bigstrut\\
            \hline
            \multicolumn{2}{c}{\textbf{Dimensiones Losa}} &       & \multicolumn{2}{c}{\textbf{Momentos últimos}} \bigstrut\\
            \cline{1-2}\cline{4-5}    \boldmath{}\textbf{$L_x [m]$}\unboldmath{} & 5     &       & \multicolumn{1}{c}{\boldmath{}\textbf{Mx $[kgf \cdot m/m]$}\unboldmath{}} & 968,5 \bigstrut[t]\\
            \boldmath{}\textbf{$L_y [m]$}\unboldmath{} & 5,33  &       & \multicolumn{1}{c}{\boldmath{}\textbf{As $[cm^2/m]$}\unboldmath{}} & 1,86 \\
            \boldmath{}\textbf{$e_{min} [cm]$}\unboldmath{} & 10    &       & \multicolumn{1}{c}{\textbf{a [cm/m]}} & 0,026 \\
            \textbf{Cond. apoyo} & 6     &       & \multicolumn{1}{c}{\boldmath{}\textbf{As $[cm^2/m]$}\unboldmath{}} & 1,68 \bigstrut[b]\\
            \cline{4-5}          &       &       & \textbf{Fs} & \boldmath{}\textbf{$\phi8@16$}\unboldmath{} \bigstrut\\
            \cline{1-2}\cline{4-5}    \multicolumn{2}{c}{\textbf{Parámetros}} &       & \multicolumn{1}{c}{} &  \bigstrut\\
            \cline{1-2}    \boldmath{}\textbf{$\epsilon$}\unboldmath{} & 1,1   &       & \multicolumn{1}{c}{\boldmath{}\textbf{My $[kgf \cdot m/m]$}\unboldmath{}} & 740,6 \bigstrut[t]\\
            \textbf{k} & 0,55  &       & \multicolumn{1}{c}{\boldmath{}\textbf{As $[cm^2/m]$}\unboldmath{}} & 1,42 \\
            \boldmath{}\textbf{$\lambda$}\unboldmath{} & 35    &       & \multicolumn{1}{c}{\textbf{a [cm/m]}} & 0,020 \\
            \textbf{mx} & 1     &       & \multicolumn{1}{c}{\boldmath{}\textbf{As $[cm^2/m]$}\unboldmath{}} & 1,29 \bigstrut[b]\\
            \cline{4-5}    \textbf{my} & 50,7  &       & \textbf{Fi} & \boldmath{}\textbf{$\phi8@16$}\unboldmath{} \bigstrut\\
            \cline{4-5}    \textbf{mex} & 66,3  &       & \multicolumn{1}{c}{} &  \bigstrut[t]\\
            \textbf{mey} & 18,8  &       & \multicolumn{1}{c}{\boldmath{}\textbf{Mex $[kgf \cdot m/m]$}\unboldmath{}} & 2239,7 \\
            \boldmath{}\textbf{$\Delta x$}\unboldmath{} & 20,3  &       & \multicolumn{1}{c}{\boldmath{}\textbf{As $[cm^2/m]$}\unboldmath{}} & 4,31 \\
            \boldmath{}\textbf{$\Delta y$}\unboldmath{} & 1,05  &       & \multicolumn{1}{c}{\textbf{a [cm/m]}} & 0,061 \\
                  &       &       & \multicolumn{1}{c}{\boldmath{}\textbf{As $[cm^2/m]$}\unboldmath{}} & 3,89 \bigstrut[b]\\
            \cline{1-2}\cline{4-5}    \multicolumn{2}{c}{\textbf{Cargas}} &       & \textbf{F'+} & \boldmath{}\textbf{$\phi10@20$}\unboldmath{} \bigstrut\\
            \cline{1-2}\cline{4-5}    \boldmath{}\textbf{SC $[kgf/m^2]$}\unboldmath{} & 500   &       & \multicolumn{1}{c}{} &  \bigstrut[t]\\
            \boldmath{}\textbf{$PP_{losa} [kgf/m^2]$}\unboldmath{} & 425   &       & \multicolumn{1}{c}{\boldmath{}\textbf{Mey $[kgf \cdot m/m]$}\unboldmath{}} & 2074,2 \\
            \boldmath{}\textbf{$PP_t [kgf/m^2]$}\unboldmath{} & 650   &       & \multicolumn{1}{c}{\boldmath{}\textbf{As $[cm^2/m]$}\unboldmath{}} & 3,99 \\
            \boldmath{}\textbf{$q_u [kgf/m^2]$}\unboldmath{} & 1580  &       & \multicolumn{1}{c}{\textbf{a [cm/m]}} & 0,056 \\
            \boldmath{}\textbf{Ku $[kgf \cdot m/m]$}\unboldmath{} & 42107,0 &       & \multicolumn{1}{c}{\boldmath{}\textbf{As $[cm^2/m]$}\unboldmath{}} & 3,6 \bigstrut[b]\\
            \cline{4-5}    \boldmath{}\textbf{$\alpha$}\unboldmath{} & 0,16  &       & \textbf{F'-} & \boldmath{}\textbf{$\phi8@14$}\unboldmath{} \bigstrut\\
            \hline
        \end{tabular}%
        }
        \label{analisis0101}%
    \end{table}%    
    
    \begin{table}[H]
      \centering
      \caption{Ejemplo de análisis franja de losa: losa N° 114.}
        \resizebox{.7\textwidth}{!}{
        \begin{tabular}{cccp{8.215em}c}
            \hline
            \multicolumn{2}{c}{\textbf{N° Losa}} &       & \multicolumn{2}{c}{\textbf{114}} \bigstrut\\
            \hline
            \multicolumn{2}{c}{\textbf{Dimensiones Losa}} &       & \multicolumn{2}{c}{\textbf{Momentos últimos}} \bigstrut\\
            \cline{1-2}\cline{4-5}    \boldmath{}\textbf{$L_x [m]$}\unboldmath{} & 1,4   &       & \multicolumn{1}{c}{\boldmath{}\textbf{Mx $[kgf \cdot m/m]$}\unboldmath{}} & 160,3 \bigstrut[t]\\
            \boldmath{}\textbf{$L_y [m]$}\unboldmath{} & 9,2   &       & \multicolumn{1}{c}{\boldmath{}\textbf{As $[cm^2/m]$}\unboldmath{}} & 0,33 \\
            \boldmath{}\textbf{$e_{min} [cm]$}\unboldmath{} & 4     &       & \multicolumn{1}{c}{\textbf{a [cm/m]}} & 0,005 \\
            \textbf{Cond. apoyo} & 6     &       & \multicolumn{1}{c}{\boldmath{}\textbf{As $[cm^2/m]$}\unboldmath{}} & 0,3 \bigstrut[b]\\
            \cline{4-5}          &       &       & \textbf{Fs} & \boldmath{}\textbf{$\phi8@17$}\unboldmath{} \bigstrut\\
            \cline{1-2}\cline{4-5}    \multicolumn{2}{c}{\textbf{Parámetros}} &       & \multicolumn{1}{c}{} &  \bigstrut\\
            \cline{1-2}    \boldmath{}\textbf{$\epsilon$}\unboldmath{} & 6,6   &       & \multicolumn{1}{c}{\boldmath{}\textbf{My $[kgf \cdot m/m]$}\unboldmath{}} & 0 \bigstrut[t]\\
            \textbf{k} & 0,58  &       & \multicolumn{1}{c}{\boldmath{}\textbf{As $[cm^2/m]$}\unboldmath{}} & 0,00 \\
            \boldmath{}\textbf{$\lambda$}\unboldmath{} & 47    &       & \multicolumn{1}{c}{\textbf{a [cm/m]}} & 0,000 \\
            \textbf{mx} & Franja de losa &       & \multicolumn{1}{c}{\boldmath{}\textbf{As $[cm^2/m]$}\unboldmath{}} & 0 \bigstrut[b]\\
            \cline{4-5}    \textbf{my} & Franja de losa &       & \textbf{Fi} & \boldmath{}\textbf{$\phi8@17$}\unboldmath{} \bigstrut\\
            \cline{4-5}    \textbf{mex} & Franja de losa &       & \multicolumn{1}{c}{} &  \bigstrut[t]\\
            \textbf{mey} & Franja de losa &       & \multicolumn{1}{c}{\boldmath{}\textbf{Mex $[kgf \cdot m/m]$}\unboldmath{}} & 227,0 \\
            \boldmath{}\textbf{$\Delta x$}\unboldmath{} & Franja de losa &       & \multicolumn{1}{c}{\boldmath{}\textbf{As $[cm^2/m]$}\unboldmath{}} & 0,47 \\
            \boldmath{}\textbf{$\Delta y$}\unboldmath{} & Franja de losa &       & \multicolumn{1}{c}{\textbf{a [cm/m]}} & 0,007 \\
                  &       &       & \multicolumn{1}{c}{\boldmath{}\textbf{As $[cm^2/m]$}\unboldmath{}} & 0,43 \bigstrut[b]\\
            \cline{1-2}\cline{4-5}    \multicolumn{2}{c}{\textbf{Cargas}} &       & \textbf{F'+} & \boldmath{}\textbf{$\phi8@17$}\unboldmath{} \bigstrut\\
            \cline{1-2}\cline{4-5}    \boldmath{}\textbf{SC $[kgf/m^2]$}\unboldmath{} & 400   &       & \multicolumn{1}{c}{} &  \bigstrut[t]\\
            \boldmath{}\textbf{$PP_{losa} [kgf/m^2]$}\unboldmath{} & 400   &       & \multicolumn{1}{c}{\boldmath{}\textbf{Mey $[kgf \cdot m/m]$}\unboldmath{}} & 155,7 \\
            \boldmath{}\textbf{$PP_t [kgf/m^2]$}\unboldmath{} & 625   &       & \multicolumn{1}{c}{\boldmath{}\textbf{As $[cm^2/m]$}\unboldmath{}} & 0,32 \\
            \boldmath{}\textbf{$q_u [kgf/m^2]$}\unboldmath{} & 1390  &       & \multicolumn{1}{c}{\textbf{a [cm/m]}} & 0,005 \\
            \boldmath{}\textbf{Ku $[kgf \cdot m/m]$}\unboldmath{} & 17903,2 &       & \multicolumn{1}{c}{\boldmath{}\textbf{As $[cm^2/m]$}\unboldmath{}} & 0,29 \bigstrut[b]\\
            \cline{4-5}    \boldmath{}\textbf{$\alpha$}\unboldmath{} & 0,14  &       & \textbf{F'-} & \boldmath{}\textbf{$\phi8@17$}\unboldmath{} \bigstrut\\
            \hline
        \end{tabular}%
        }
      \label{analisis114}%
    \end{table}%
    
    Para la interacción entre losas se comparan los momentos negativos en el eje correspondiente y se ponderan, con distintos factores según la diferencia entre ellos. Con esto se determina un momento último que usa para diseñar los suples mediante un análisis de flexión simple.
    
    A continuación se muestra un ejemplo del cálculo realizado para la interacción entre las losas 0101 y 0107:
    
    \begin{table}[H]
      \centering
      \caption{Ejemplo de análisis de interacción entre losas 0101 y 0102.}
        \resizebox{.5\textwidth}{!}{
        \begin{tabular}{ccc}
        \hline
        Losas  & \multicolumn{2}{c}{0101-0102} \bigstrut\\
        \hline
        Ejes  & y     & y \bigstrut[t]\\
        Me [kgf*m/m] & 2074,2 & 2171,0 \\
        Dif [\%] & \multicolumn{2}{c}{4,5\%} \\
        Mu $[kgf \cdot m/m]$ & \multicolumn{2}{c}{1910,4} \\
        As $[cm^2/m]$ & \multicolumn{2}{c}{3,7} \\
        a [cm/m] & \multicolumn{2}{c}{0,1} \\
        As $[cm^2/m]$ & \multicolumn{2}{c}{3,3} \bigstrut[b]\\
        \hline
        F'    & \multicolumn{2}{c}{$\phi10@23$} \bigstrut\\
        \hline
        \end{tabular}%
        }
      \label{interaccion0101-0102}%
    \end{table}%
    
\newpage
\section{Resultados}

    A continuación se muestra el resumen de los resultados del diseño de armaduras para las losas del edificio. En estas se muestran tanto armaduras positivas como negativas, siendo estas últimas sólo aplicadas en los bordes de las losas sin interacción.
    
    \begin{table}[H]
      \centering
      \caption{Resumen armaduras nivel -1.}
        \resizebox{\textwidth}{!}{
        \begin{tabular}{|c|c|c|c|c|c|c|c|c|c|c|c|}
        \hline
        \multirow{2}[4]{*}{\textbf{N° Losa}} & \multirow{2}[4]{*}{\textbf{e [cm]}} & \boldmath{}\textbf{$Ag_{min}$}\unboldmath{} & \boldmath{}\textbf{$q_{u}$}\unboldmath{} & \multicolumn{2}{c|}{\boldmath{}\textbf{$F_{s}$}\unboldmath{}} & \multicolumn{2}{c|}{\boldmath{}\textbf{$F_{i}$}\unboldmath{}} & \multicolumn{2}{c|}{\boldmath{}\textbf{$F'_{s}$}\unboldmath{}} & \multicolumn{2}{c|}{\boldmath{}\textbf{$F'_{i}$}\unboldmath{}} \bigstrut\\
        \cline{5-12}          &       & \boldmath{}\textbf{ $(cm^{2}/m)$ }\unboldmath{} & \boldmath{}\textbf{$ (kgf/m^2)$}\unboldmath{} & \multicolumn{2}{c|}{\boldmath{}\textbf{As $(cm^{2}/m)$}\unboldmath{}} & \multicolumn{2}{c|}{\boldmath{}\textbf{As $(cm^{2}/m)$}\unboldmath{}} & \multicolumn{2}{c|}{\boldmath{}\textbf{As $(cm^{2}/m)$}\unboldmath{}} & \multicolumn{2}{c|}{\boldmath{}\textbf{As $(cm^{2}/m)$}\unboldmath{}} \bigstrut\\
        \hline
        \textbf{0101} & \textbf{17} & 3,06  & 1580  & 1,68  & $\phi8@16$ & 1,29  & $\phi8@16$ & 3,89  & $\phi10@20$ & 3,60  & $\phi8@14$ \bigstrut[t]\\
        \textbf{0102} & \textbf{17} & 3,06  & 1580  & 2,69  & $\phi8@16$ & 0,85  & $\phi8@16$ & 4,99  & $\phi8@10$ & 3,77  & $\phi10@21$ \\
        \textbf{0103} & \textbf{17} & 3,06  & 1580  & 2,69  & $\phi8@16$ & 0,85  & $\phi8@16$ & 4,99  & $\phi8@10$ & 3,77  & $\phi10@21$ \\
        \textbf{0104} & \textbf{17} & 3,06  & 1580  & 2,86  & $\phi8@16$ & 0,81  & $\phi8@16$ & 5,28  & $\phi10@15$ & 3,88  & $\phi10@20$ \\
        \textbf{0105} & \textbf{17} & 3,06  & 1580  & 1,52  & $\phi8@16$ & 0,91  & $\phi8@16$ & 3,28  & $\phi8@15$ & 2,84  & $\phi8@16$ \\
        \textbf{0106} & \textbf{17} & 3,06  & 1580  & 1,33  & $\phi8@16$ & 0,51  & $\phi8@16$ & 2,57  & $\phi8@16$ & 2,02  & $\phi8@16$ \\
        \textbf{0107} & \textbf{17} & 3,06  & 1580  & 1,86  & $\phi8@16$ & 1,43  & $\phi8@16$ & 4,31  & $\phi10@18$ & 3,99  & $\phi10@20$ \\
        \textbf{0108} & \textbf{17} & 3,06  & 1580  & 2,87  & $\phi8@16$ & 1,36  & $\phi8@16$ & 5,82  & $\phi12@20$ & 4,78  & $\phi12@24$ \\
        \textbf{0109} & \textbf{17} & 3,06  & 1580  & 2,87  & $\phi8@16$ & 1,36  & $\phi8@16$ & 5,82  & $\phi12@20$ & 4,78  & $\phi12@24$ \\
        \textbf{0110} & \textbf{17} & 3,06  & 1580  & 4,95  & $\phi8@10$ & 0,00  & $\phi8@16$ & 7,02  & $\phi12@16$ & 4,81  & $\phi12@24$ \\
        \textbf{0111} & \textbf{17} & 3,06  & 1580  & 1,45  & $\phi8@16$ & 0,41  & $\phi8@16$ & 2,67  & $\phi8@16$ & 1,96  & $\phi8@16$ \\
        \textbf{0112} & \textbf{17} & 3,06  & 1420  & 3,61  & $\phi8@14$ & 1,02  & $\phi8@16$ & 6,80  & $\phi12@17$ & 5,00  & $\phi8@10$ \\
        \textbf{0113} & \textbf{17} & 3,06  & 1420  & 1,66  & $\phi8@16$ & 0,79  & $\phi8@16$ & 3,42  & $\phi10@23$ & 2,80  & $\phi8@16$ \\
        \textbf{0114} & \textbf{17} & 3,06  & 1580  & 1,87  & $\phi8@16$ & 0,89  & $\phi8@16$ & 3,80  & $\phi8@13$ & 3,12  & $\phi10@25$ \\
        \textbf{0115} & \textbf{17} & 3,06  & 1580  & 4,17  & $\phi8@12$ & 1,18  & $\phi8@16$ & 7,69  & $\phi12@15$ & 5,65  & $\phi12@20$ \\
        \textbf{0116} & \textbf{17} & 3,06  & 1580  & 3,20  & $\phi10@24$ & 0,00  & $\phi8@16$ & 4,53  & $\phi10@17$ & 3,10  & $\phi10@25$ \\
        \textbf{0117} & \textbf{17} & 3,06  & 1580  & 2,09  & $\phi8@16$ & 0,00  & $\phi8@16$ & 2,96  & $\phi8@16$ & 2,03  & $\phi8@16$ \\
        \textbf{0118} & \textbf{17} & 3,06  & 1420  & 5,19  & $\phi10@15$ & 1,47  & $\phi8@16$ & 9,79  & $\phi16@21$ & 7,18  & $\phi10@11$ \\
        \textbf{0119} & \textbf{17} & 3,06  & 1420  & 0,29  & $\phi8@16$ & 0,00  & $\phi8@16$ & 0,41  & $\phi8@16$ & 0,28  & $\phi8@16$ \\
        \textbf{0120} & \textbf{17} & 3,06  & 1420  & 0,61  & $\phi8@16$ & 0,11  & $\phi8@16$ & 1,19  & $\phi8@16$ & 0,85  & $\phi8@16$ \\
        \textbf{0121} & \textbf{17} & 3,06  & 1420  & 1,22  & $\phi8@16$ & 0,00  & $\phi8@16$ & 1,73  & $\phi8@16$ & 1,19  & $\phi8@16$ \\
        \textbf{0122} & \textbf{17} & 3,06  & 1580  & 2,02  & $\phi8@16$ & 0,64  & $\phi8@16$ & 3,75  & $\phi10@21$ & 2,83  & $\phi8@16$ \\
        \textbf{0123} & \textbf{17} & 3,06  & 1580  & 3,17  & $\phi10@24$ & 1,90  & $\phi8@16$ & 6,86  & $\phi12@17$ & 5,93  & $\phi12@19$ \\
        \textbf{0124} & \textbf{17} & 3,06  & 1580  & 10,99 & $\phi16@19$ & 0,00  & $\phi8@16$ & 15,60 & $\phi16@13$ & 10,67 & $\phi16@19$ \\
        \textbf{0125} & \textbf{17} & 3,06  & 1580  & 2,99  & $\phi8@16$ & 0,85  & $\phi8@16$ & 5,52  & $\phi10@14$ & 4,05  & $\phi10@19$ \\
        \textbf{0126} & \textbf{17} & 3,06  & 1580  & 3,39  & $\phi10@23$ & 1,07  & $\phi8@16$ & 6,31  & $\phi12@18$ & 4,76  & $\phi12@24$ \\
        \textbf{0127} & \textbf{17} & 3,06  & 1580  & 5,37  & $\phi12@21$ & 1,70  & $\phi8@16$ & 10,00 & $\phi16@20$ & 7,55  & $\phi12@15$ \\
        \textbf{0128} & \textbf{17} & 3,06  & 1580  & 2,09  & $\phi8@16$ & 0,00  & $\phi8@16$ & 2,96  & $\phi8@16$ & 2,03  & $\phi8@16$ \\
        \textbf{0129} & \textbf{17} & 3,06  & 1420  & 0,61  & $\phi8@16$ & 0,16  & $\phi8@16$ & 1,12  & $\phi8@16$ & 0,79  & $\phi8@16$ \bigstrut[b]\\
        \hline
    \end{tabular}%
        }
        \label{resumenlosas-1}%
    \end{table}%

    \begin{table}[H]
      \centering
      \caption{Resumen armaduras nivel 1.}
        \resizebox{\textwidth}{!}{
        \begin{tabular}{|c|c|c|c|c|c|c|c|c|c|c|c|}
        \hline
        \multirow{2}[4]{*}{\textbf{N° Losa}} & \multirow{2}[4]{*}{\textbf{e [cm]}} & \boldmath{}\textbf{$Ag_{min}$}\unboldmath{} & \boldmath{}\textbf{$q_{u}$}\unboldmath{} & \multicolumn{2}{c|}{\boldmath{}\textbf{$F_{s}$}\unboldmath{}} & \multicolumn{2}{c|}{\boldmath{}\textbf{$F_{i}$}\unboldmath{}} & \multicolumn{2}{c|}{\boldmath{}\textbf{$F'_{s}$}\unboldmath{}} & \multicolumn{2}{c|}{\boldmath{}\textbf{$F'_{i}$}\unboldmath{}} \bigstrut\\
        \cline{5-12}          &       & \boldmath{}\textbf{ $(cm^{2}/m)$ }\unboldmath{} & \boldmath{}\textbf{$ (kgf/m^2)$}\unboldmath{} & \multicolumn{2}{c|}{\boldmath{}\textbf{As $(cm^{2}/m)$}\unboldmath{}} & \multicolumn{2}{c|}{\boldmath{}\textbf{As $(cm^{2}/m)$}\unboldmath{}} & \multicolumn{2}{c|}{\boldmath{}\textbf{As $(cm^{2}/m)$}\unboldmath{}} & \multicolumn{2}{c|}{\boldmath{}\textbf{As $(cm^{2}/m)$}\unboldmath{}} \bigstrut\\
        \hline
        \textbf{101} & \textbf{16} & 2,88  & 1550,00 & 2,06  & $\phi8@17$ & 1,34  & $\phi8@17$ & 4,72  & $\phi12@24$ & 3,94  & $\phi10@20$ \bigstrut[t]\\
        \textbf{102} & \textbf{16} & 2,88  & 1550,00 & 2,83  & $\phi8@17$ & 0,90  & $\phi8@17$ & 5,25  & $\phi10@15$ & 3,96  & $\phi10@20$ \\
        \textbf{103} & \textbf{16} & 2,88  & 1550,00 & 2,83  & $\phi8@17$ & 0,90  & $\phi8@17$ & 5,25  & $\phi10@15$ & 3,96  & $\phi10@20$ \\
        \textbf{104} & \textbf{16} & 2,88  & 1550,00 & 3,02  & $\phi8@17$ & 0,86  & $\phi8@17$ & 5,55  & $\phi10@14$ & 4,08  & $\phi10@19$ \\
        \textbf{105} & \textbf{16} & 2,88  & 1550,00 & 2,48  & $\phi8@17$ & 1,97  & $\phi8@17$ & 5,42  & $\phi12@21$ & 5,05  & $\phi12@22$ \\
        \textbf{106} & \textbf{16} & 2,88  & 1550,00 & 3,03  & $\phi8@17$ & 1,43  & $\phi8@17$ & 6,12  & $\phi10@13$ & 5,03  & $\phi8@10$ \\
        \textbf{107} & \textbf{16} & 2,88  & 1550,00 & 3,03  & $\phi8@17$ & 1,43  & $\phi8@17$ & 6,12  & $\phi10@13$ & 5,03  & $\phi8@10$ \\
        \textbf{108} & \textbf{16} & 2,88  & 1550,00 & 3,44  & $\phi10@23$ & 1,09  & $\phi8@17$ & 6,37  & $\phi12@18$ & 4,81  & $\phi12@24$ \\
        \textbf{109} & \textbf{16} & 2,88  & 1550,00 & 1,92  & $\phi8@17$ & 1,26  & $\phi8@17$ & 4,50  & $\phi8@11$ & 4,11  & $\phi8@12$ \\
        \textbf{110} & \textbf{16} & 2,88  & 1390,00 & 1,74  & $\phi8@17$ & 0,83  & $\phi8@17$ & 3,58  & $\phi8@14$ & 2,94  & $\phi8@17$ \\
        \textbf{111} & \textbf{16} & 2,88  & 1550,00 & 1,98  & $\phi8@17$ & 0,94  & $\phi8@17$ & 3,99  & $\phi10@20$ & 3,28  & $\phi8@15$ \\
        \textbf{112} & \textbf{16} & 2,88  & 1550,00 & 4,40  & $\phi10@18$ & 1,25  & $\phi8@17$ & 8,09  & $\phi12@14$ & 5,94  & $\phi12@19$ \\
        \textbf{113} & \textbf{16} & 2,88  & 1390,00 & 0,52  & $\phi8@17$ & 0,22  & $\phi8@17$ & 1,02  & $\phi8@17$ & 0,79  & $\phi8@17$ \\
        \textbf{114} & \textbf{16} & 2,88  & 1390,00 & 0,30  & $\phi8@17$ & 0,00  & $\phi8@17$ & 0,43  & $\phi8@17$ & 0,29  & $\phi8@17$ \\
        \textbf{115} & \textbf{16} & 2,88  & 1390,00 & 1,28  & $\phi8@17$ & 0,00  & $\phi8@17$ & 1,81  & $\phi8@17$ & 1,24  & $\phi8@17$ \\
        \textbf{116} & \textbf{16} & 2,88  & 1070,00 & 1,08  & $\phi8@17$ & 0,83  & $\phi8@17$ & 2,66  & $\phi8@17$ & 2,46  & $\phi8@17$ \\
        \textbf{117} & \textbf{16} & 2,88  & 1070,00 & 1,76  & $\phi8@17$ & 1,05  & $\phi8@17$ & 4,04  & $\phi10@19$ & 3,50  & $\phi8@14$ \\
        \textbf{118} & \textbf{16} & 2,88  & 1230,00 & 0,35  & $\phi8@17$ & 0,00  & $\phi8@17$ & 0,52  & $\phi8@17$ & 0,37  & $\phi8@17$ \\
        \textbf{119} & \textbf{16} & 2,88  & 1390,00 & 0,65  & $\phi8@17$ & 0,12  & $\phi8@17$ & 1,25  & $\phi8@17$ & 0,89  & $\phi8@17$ \\
        \textbf{120} & \textbf{16} & 2,88  & 1390,00 & 0,64  & $\phi8@17$ & 0,17  & $\phi8@17$ & 1,17  & $\phi8@17$ & 0,82  & $\phi8@17$ \\
        \textbf{121} & \textbf{16} & 2,88  & 1390,00 & 0,39  & $\phi8@17$ & 0,00  & $\phi8@17$ & 0,56  & $\phi8@17$ & 0,38  & $\phi8@17$ \bigstrut[b]\\
        \hline
        \end{tabular}%
        }
      \label{resumenlosas1}%
    \end{table}%
    
    \begin{table}[H]
      \centering
      \caption{Resumen armaduras niveles 2 a 22.}
        \resizebox{\textwidth}{!}{
        \begin{tabular}{|c|c|c|c|c|c|c|c|c|c|c|c|}
        \hline
        \multirow{2}[4]{*}{\textbf{N° Losa}} & \multirow{2}[4]{*}{\textbf{e [cm]}} & \boldmath{}\textbf{$Ag_{min}$}\unboldmath{} & \boldmath{}\textbf{$q_{u}$}\unboldmath{} & \multicolumn{2}{c|}{\boldmath{}\textbf{$F_{s}$}\unboldmath{}} & \multicolumn{2}{c|}{\boldmath{}\textbf{$F_{i}$}\unboldmath{}} & \multicolumn{2}{c|}{\boldmath{}\textbf{$F'_{s}$}\unboldmath{}} & \multicolumn{2}{c|}{\boldmath{}\textbf{$F'_{i}$}\unboldmath{}} \bigstrut\\
        \cline{5-12}          &       & \boldmath{}\textbf{ $(cm^{2}/m)$ }\unboldmath{} & \boldmath{}\textbf{$ (kgf/m^2)$}\unboldmath{} & \multicolumn{2}{c|}{\boldmath{}\textbf{As $(cm^{2}/m)$}\unboldmath{}} & \multicolumn{2}{c|}{\boldmath{}\textbf{As $(cm^{2}/m)$}\unboldmath{}} & \multicolumn{2}{c|}{\boldmath{}\textbf{As $(cm^{2}/m)$}\unboldmath{}} & \multicolumn{2}{c|}{\boldmath{}\textbf{As $(cm^{2}/m)$}\unboldmath{}} \bigstrut\\
        \hline
        (2-22)01 & \textbf{16} & 2,88  & 1070  & 1,27  & $\phi8@17$ & 0,60  & $\phi8@17$ & 0,56  & $\phi8@17$ & 2,26  & $\phi8@17$ \bigstrut[t]\\
        (2-22)02 & \textbf{16} & 2,88  & 1070  & 1,27  & $\phi8@17$ & 0,60  & $\phi8@17$ & 0,56  & $\phi8@17$ & 2,26  & $\phi8@17$ \\
        (2-22)03 & \textbf{16} & 2,88  & 1070  & 2,38  & $\phi8@17$ & 0,91  & $\phi8@17$ & 0,56  & $\phi8@10$ & 3,88  & $\phi10@20$ \\
        (2-22)04 & \textbf{16} & 2,88  & 1070  & 1,88  & $\phi8@17$ & 1,13  & $\phi8@17$ & 0,56  & $\phi10@18$ & 3,75  & $\phi10@21$ \\
        (2-22)05 & \textbf{16} & 2,88  & 1070  & 1,08  & $\phi8@17$ & 0,83  & $\phi8@17$ & 0,56  & $\phi8@17$ & 2,46  & $\phi8@17$ \\
        (2-22)06 & \textbf{16} & 2,88  & 1390  & 0,65  & $\phi8@17$ & 0,12  & $\phi8@17$ & 0,56  & $\phi8@17$ & 0,89  & $\phi8@17$ \\
        (2-22)07 & \textbf{16} & 2,88  & 1390  & 0,30  & $\phi8@17$ & 0,00  & $\phi8@17$ & 0,56  & $\phi8@17$ & 0,29  & $\phi8@17$ \\
        (2-22)08 & \textbf{16} & 2,88  & 1390  & 1,28  & $\phi8@17$ & 0,00  & $\phi8@17$ & 0,56  & $\phi8@17$ & 1,24  & $\phi8@17$ \\
        (2-22)09 & \textbf{16} & 2,88  & 1070  & 1,08  & $\phi8@17$ & 0,83  & $\phi8@17$ & 0,56  & $\phi8@17$ & 2,46  & $\phi8@17$ \\
        (2-22)10 & \textbf{16} & 2,88  & 1070  & 1,76  & $\phi8@17$ & 1,05  & $\phi8@17$ & 0,56  & $\phi10@19$ & 3,50  & $\phi8@14$ \\
        (2-22)11 & \textbf{16} & 2,88  & 1230  & 0,44  & $\phi8@17$ & 0,00  & $\phi8@17$ & 0,56  & $\phi8@17$ & 0,47  & $\phi8@17$ \\
        (2-22)12 & \textbf{16} & 2,88  & 1230  & 0,44  & $\phi8@17$ & 0,00  & $\phi8@17$ & 0,56  & $\phi8@17$ & 0,47  & $\phi8@17$ \\
        (2-22)13 & \textbf{16} & 2,88  & 1230  & 0,44  & $\phi8@17$ & 0,00  & $\phi8@17$ & 0,56  & $\phi8@17$ & 0,47  & $\phi8@17$ \\
        (2-22)14 & \textbf{16} & 2,88  & 1230  & 0,44  & $\phi8@17$ & 0,00  & $\phi8@17$ & 0,56  & $\phi8@17$ & 0,47  & $\phi8@17$ \\
        (2-22)15 & \textbf{16} & 2,88  & 1230  & 0,44  & $\phi8@17$ & 0,00  & $\phi8@17$ & 0,56  & $\phi8@17$ & 0,47  & $\phi8@17$ \\
        (2-22)16 & \textbf{16} & 2,88  & 1230  & 0,21  & $\phi8@17$ & 0,00  & $\phi8@17$ & 0,56  & $\phi8@17$ & 0,22  & $\phi8@17$ \\
        (2-22)17 & \textbf{16} & 2,88  & 1230  & 0,21  & $\phi8@17$ & 0,00  & $\phi8@17$ & 0,56  & $\phi8@17$ & 0,22  & $\phi8@17$ \\
        (2-22)18 & \textbf{16} & 2,88  & 1230  & 0,35  & $\phi8@17$ & 0,00  & $\phi8@17$ & 0,56  & $\phi8@17$ & 0,37  & $\phi8@17$ \\
        (2-22)19 & \textbf{16} & 2,88  & 1230  & 0,35  & $\phi8@17$ & 0,00  & $\phi8@17$ & 0,56  & $\phi8@17$ & 0,37  & $\phi8@17$ \\
        (2-22)20 & \textbf{16} & 2,88  & 1230  & 0,11  & $\phi8@17$ & 0,00  & $\phi8@17$ & 0,56  & $\phi8@17$ & 0,12  & $\phi8@17$ \\
        (2-22)21 & \textbf{16} & 2,88  & 1230  & 0,10  & $\phi8@17$ & 0,00  & $\phi8@17$ & 0,56  & $\phi8@17$ & 0,10  & $\phi8@17$ \\
        (2-22)22 & 16    & 2,88  & 1230  & 0,66  & $\phi8@17$ & 0,17  & $\phi8@17$ & 0,56  & $\phi8@17$ & 1,04  & $\phi8@17$ \bigstrut[b]\\
        \hline
        \end{tabular}%
        }
      \label{resumenlosas2a22}%
    \end{table}%
    
    \begin{table}[H]
      \centering
      \caption{Resumen armaduras nivel 23.}
        \resizebox{\textwidth}{!}{      
        \begin{tabular}{|c|c|c|c|c|c|c|c|c|c|c|c|}
        \hline
        \multirow{2}[4]{*}{\textbf{N° Losa}} & \multirow{2}[4]{*}{\textbf{e [cm]}} & \boldmath{}\textbf{$Ag_{min}$}\unboldmath{} & \boldmath{}\textbf{$q_{u}$}\unboldmath{} & \multicolumn{2}{c|}{\boldmath{}\textbf{$F_{s}$}\unboldmath{}} & \multicolumn{2}{c|}{\boldmath{}\textbf{$F_{i}$}\unboldmath{}} & \multicolumn{2}{c|}{\boldmath{}\textbf{$F'_{s}$}\unboldmath{}} & \multicolumn{2}{c|}{\boldmath{}\textbf{$F'_{i}$}\unboldmath{}} \bigstrut\\
        \cline{5-12}          &       & \boldmath{}\textbf{ $(cm^{2}/m)$ }\unboldmath{} & \boldmath{}\textbf{$ (kgf/m^2)$}\unboldmath{} & \multicolumn{2}{c|}{\boldmath{}\textbf{As $(cm^{2}/m)$}\unboldmath{}} & \multicolumn{2}{c|}{\boldmath{}\textbf{As $(cm^{2}/m)$}\unboldmath{}} & \multicolumn{2}{c|}{\boldmath{}\textbf{As $(cm^{2}/m)$}\unboldmath{}} & \multicolumn{2}{c|}{\boldmath{}\textbf{As $(cm^{2}/m)$}\unboldmath{}} \bigstrut\\
        \hline
        \textbf{2301} & \textbf{16} & 2,88  & 910   & 2,39  & $\phi8@17$ & 0,68  & $\phi8@17$ & 5,01  & $\phi8@10$ & 3,68  & $\phi10@21$ \bigstrut[t]\\
        \textbf{2302} & \textbf{16} & 2,88  & 910   & 1,04  & $\phi8@17$ & 0,49  & $\phi8@17$ & 2,35  & $\phi8@17$ & 1,93  & $\phi8@17$ \\
        \textbf{2303} & \textbf{16} & 2,88  & 910   & 1,04  & $\phi8@17$ & 0,49  & $\phi8@17$ & 2,35  & $\phi8@17$ & 1,93  & $\phi8@17$ \\
        \textbf{2304} & \textbf{16} & 2,88  & 910   & 1,94  & $\phi8@17$ & 0,74  & $\phi8@17$ & 4,22  & $\phi8@12$ & 3,30  & $\phi8@15$ \\
        \textbf{2305} & \textbf{16} & 2,88  & 910   & 1,54  & $\phi8@17$ & 0,92  & $\phi8@17$ & 3,69  & $\phi10@21$ & 3,19  & $\phi10@24$ \\
        \textbf{2306} & \textbf{16} & 2,88  & 910   & 0,89  & $\phi8@17$ & 0,68  & $\phi8@17$ & 2,26  & $\phi8@17$ & 2,10  & $\phi8@17$ \\
        \textbf{2307} & \textbf{16} & 2,88  & 910   & 0,40  & $\phi8@17$ & 0,08  & $\phi8@17$ & 0,82  & $\phi8@17$ & 0,59  & $\phi8@17$ \\
        \textbf{2308} & \textbf{16} & 2,88  & 910   & 0,20  & $\phi8@17$ & 0,00  & $\phi8@17$ & 0,28  & $\phi8@17$ & 0,19  & $\phi8@17$ \\
        \textbf{2309} & \textbf{16} & 2,88  & 910   & 0,84  & $\phi8@17$ & 0,00  & $\phi8@17$ & 1,19  & $\phi8@17$ & 0,82  & $\phi8@17$ \\
        \textbf{2310} & \textbf{16} & 2,88  & 910   & 0,89  & $\phi8@17$ & 0,68  & $\phi8@17$ & 2,26  & $\phi8@17$ & 2,10  & $\phi8@17$ \\
        \textbf{2311} & \textbf{16} & 2,88  & 910   & 1,44  & $\phi8@17$ & 0,86  & $\phi8@17$ & 3,44  & $\phi10@23$ & 2,98  & $\phi8@17$ \\
        \textbf{2312} & \textbf{16} & 2,88  & 910   & 0,33  & $\phi8@17$ & 0,00  & $\phi8@17$ & 0,49  & $\phi8@17$ & 0,35  & $\phi8@17$ \\
        \textbf{2313} & \textbf{16} & 2,88  & 910   & 0,33  & $\phi8@17$ & 0,00  & $\phi8@17$ & 0,49  & $\phi8@17$ & 0,35  & $\phi8@17$ \\
        \textbf{2314} & \textbf{16} & 2,88  & 910   & 0,33  & $\phi8@17$ & 0,00  & $\phi8@17$ & 0,49  & $\phi8@17$ & 0,35  & $\phi8@17$ \\
        \textbf{2315} & \textbf{16} & 2,88  & 910   & 0,33  & $\phi8@17$ & 0,00  & $\phi8@17$ & 0,49  & $\phi8@17$ & 0,35  & $\phi8@17$ \\
        \textbf{2316} & \textbf{16} & 2,88  & 910   & 0,16  & $\phi8@17$ & 0,00  & $\phi8@17$ & 0,23  & $\phi8@17$ & 0,17  & $\phi8@17$ \\
        \textbf{2317} & \textbf{16} & 2,88  & 910   & 0,16  & $\phi8@17$ & 0,00  & $\phi8@17$ & 0,23  & $\phi8@17$ & 0,17  & $\phi8@17$ \\
        \textbf{2318} & \textbf{16} & 2,88  & 910   & 0,26  & $\phi8@17$ & 0,00  & $\phi8@17$ & 0,38  & $\phi8@17$ & 0,27  & $\phi8@17$ \\
        \textbf{2319} & \textbf{16} & 2,88  & 910   & 0,26  & $\phi8@17$ & 0,00  & $\phi8@17$ & 0,38  & $\phi8@17$ & 0,27  & $\phi8@17$ \\
        \textbf{2320} & \textbf{16} & 2,88  & 910   & 0,08  & $\phi8@17$ & 0,00  & $\phi8@17$ & 0,12  & $\phi8@17$ & 0,09  & $\phi8@17$ \\
        \textbf{2321} & \textbf{16} & 2,88  & 910   & 0,07  & $\phi8@17$ & 0,00  & $\phi8@17$ & 0,11  & $\phi8@17$ & 0,08  & $\phi8@17$ \\
        \textbf{2322} & \textbf{16} & 2,88  & 910   & 0,47  & $\phi8@17$ & 0,13  & $\phi8@17$ & 1,01  & $\phi8@17$ & 0,77  & $\phi8@17$ \bigstrut[b]\\
        \hline
        \end{tabular}%
        }
      \label{resumenlosas23}%
    \end{table}%
    
    \begin{table}[H]
      \centering
      \caption{Resumen armaduras nivel 24 y cubierta.}
        \resizebox{\textwidth}{!}{      
        \begin{tabular}{|c|c|c|c|c|c|c|c|c|c|c|c|}
        \hline
        \multirow{2}[4]{*}{\textbf{N° Losa}} & \multirow{2}[4]{*}{\textbf{e [cm]}} & \boldmath{}\textbf{$Ag_{min}$}\unboldmath{} & \boldmath{}\textbf{$q_{u}$}\unboldmath{} & \multicolumn{2}{c|}{\boldmath{}\textbf{$F_{s}$}\unboldmath{}} & \multicolumn{2}{c|}{\boldmath{}\textbf{$F_{i}$}\unboldmath{}} & \multicolumn{2}{c|}{\boldmath{}\textbf{$F'_{s}$}\unboldmath{}} & \multicolumn{2}{c|}{\boldmath{}\textbf{$F'_{i}$}\unboldmath{}} \bigstrut\\
        \cline{5-12}          &       & \boldmath{}\textbf{ $(cm^{2}/m)$ }\unboldmath{} & \boldmath{}\textbf{$ (kgf/m^2)$}\unboldmath{} & \multicolumn{2}{c|}{\boldmath{}\textbf{As $(cm^{2}/m)$}\unboldmath{}} & \multicolumn{2}{c|}{\boldmath{}\textbf{As $(cm^{2}/m)$}\unboldmath{}} & \multicolumn{2}{c|}{\boldmath{}\textbf{As $(cm^{2}/m)$}\unboldmath{}} & \multicolumn{2}{c|}{\boldmath{}\textbf{As $(cm^{2}/m)$}\unboldmath{}} \bigstrut\\
        \hline
        \textbf{2401} & \textbf{16} & 2,88  & 910   & 0,32  & $\phi8@17$ & 0,09  & $\phi8@17$ & 0,61  & $\phi8@17$ & 0,41  & $\phi8@17$ \bigstrut[t]\\
        \textbf{2402} & \textbf{16} & 2,88  & 910   & 0,38  & $\phi8@17$ & 0,10  & $\phi8@17$ & 0,77  & $\phi8@17$ & 0,54  & $\phi8@17$ \\
        \textbf{2403} & \textbf{16} & 2,88  & 910   & 0,84  & $\phi8@17$ & 0,00  & $\phi8@17$ & 1,19  & $\phi8@17$ & 0,82  & $\phi8@17$ \bigstrut[b]\\
        \hline
        \textbf{CU01} & \textbf{16} & 2,88  & 910   & 0,34  & $\phi8@17$ & 0,09  & $\phi8@17$ & 0,69  & $\phi8@17$ & 0,48  & $\phi8@17$ \bigstrut\\
        \hline
        \end{tabular}%
        }
      \label{resumenlosas24ycubierta}%
    \end{table}%
    
    A continuación se muestra el resumen de los resultados del diseño armadura negativa para las interacciones entre losas del edificio. 
    
    \begin{table}[H]
      \centering
      \caption{Resumen armadura negativa nivel -1.}
        \resizebox{0.5\textwidth}{!}{
        \begin{tabular}{cccc}
        \hline
        \textbf{Losas} & \boldmath{}\textbf{Mu $[kgf \cdot m/m]$}\unboldmath{} & \boldmath{}\textbf{As $[cm^2/m]$}\unboldmath{} & \textbf{F'} \bigstrut\\
        \hline
        0101-0102 & 1910,4 & 3,32  & $\phi10@23$ \bigstrut[t]\\
        0101-0107 & 2042,1 & 3,55  & $\phi8@14$ \\
        0102-0103 & 1953,9 & 3,39  & $\phi10@23$ \\
        0102-0108 & 2801,3 & 4,87  & $\phi12@23$ \\
        0103-0104 & 1982,3 & 3,44  & $\phi10@23$ \\
        0103-0109 & 2801,3 & 4,87  & $\phi12@23$ \\
        0104-0110 & 3186,7 & 5,54  & $\phi10@14$ \\
        0104-0105 & 1855,9 & 3,22  & $\phi10@24$ \\
        0105-0106 & 1517,1 & 2,63  & $\phi8@16$ \\
        0105-0110 & 2879,1 & 5,00  & $\phi8@10$ \\
        0106-0111 & 1030,6 & 1,79  & $\phi8@16$ \\
        0107-0108 & 2355,2 & 4,09  & $\phi10@19$ \\
        0107-0112 & 2941,0 & 5,11  & $\phi12@22$ \\
        0108-0112 & 3269,9 & 5,68  & $\phi12@20$ \\
        0108-0109 & 2476,9 & 4,30  & $\phi10@18$ \\
        0108-0113 & 2518,3 & 4,37  & $\phi10@18$ \\
        0109-0110 & 2485,4 & 4,32  & $\phi10@18$ \\
        0109-0114 & 2598,1 & 4,51  & $\phi8@11$ \\
        0109-0115 & 3499,8 & 6,08  & $\phi10@13$ \\
        0110-0111 & 2049,7 & 3,56  & $\phi8@14$ \\
        0110-0115 & 3809,8 & 6,62  & $\phi12@17$ \\
        0110-0116 & 2927,2 & 5,08  & $\phi12@22$ \\
        0111-0117 & 1034,8 & 1,80  & $\phi8@16$ \\
        0112-0118 & 4446,2 & 7,73  & $\phi12@15$ \\
        0112-0113 & 2134,3 & 3,71  & $\phi10@21$ \\
        0112-0119 & 1854,8 & 3,22  & $\phi10@24$ \\
        0113-0114 & 1535,1 & 2,67  & $\phi8@16$ \\
        0113-0119 & 1301,7 & 2,26  & $\phi8@16$ \\
        0114-0115 & 2402,6 & 4,17  & $\phi8@12$ \\
        0114-0119 & 1441,4 & 2,50  & $\phi8@16$ \\
        0115-0122 & 2342,6 & 4,07  & $\phi10@19$ \\
        0115-0123 & 3528,0 & 6,13  & $\phi10@13$ \\
        0115-0116 & 2636,4 & 4,58  & $\phi10@17$ \\
        0116-0117 & 2022,3 & 3,51  & $\phi8@14$ \\
        0116-0123 & 3070,2 & 5,33  & $\phi12@21$ \\
        0116-0127 & 3102,0 & 5,39  & $\phi12@21$ \\
        0117-0128 & 1053,1 & 1,83  & $\phi8@16$ \\
        0118-0124 & 6852,2 & 11,94 & $\phi16@17$ \\
        0118-0119 & 3605,9 & 6,27  & $\phi12@18$ \\
        0118-0120 & 2788,3 & 4,84  & $\phi12@23$ \\
        0118-0129 & 2725,3 & 4,73  & $\phi12@24$ \\
        0119-0120 & 359,5 & 0,63  & $\phi8@16$ \\
        0119-0121 & 472,3 & 0,82  & $\phi8@16$ \\
        0119-0122 & 1089,6 & 1,89  & $\phi8@16$ \\
        0121-0129 & 724,3 & 1,26  & $\phi8@16$ \\
        0121-0122 & 1575,3 & 2,74  & $\phi8@16$ \\
        0121-0124 & 5816,4 & 10,12 & $\phi16@20$ \\
        0122-0125 & 1846,7 & 3,21  & $\phi10@24$ \\
        0122-0123 & 2908,0 & 5,05  & $\phi12@22$ \\
        0123-0126 & 2771,0 & 4,81  & $\phi12@24$ \\
        0123-0127 & 4522,0 & 7,86  & $\phi10@10$ \\
        0123-0127 & 4522,0 & 7,86  & $\phi10@10$ \\
        0124-0125 & 4453,4 & 7,74  & $\phi12@15$ \\
        0124-0129 & 5806,0 & 10,11 & $\phi16@20$ \\
        0125-0126 & 3062,3 & 5,32  & $\phi12@21$ \\
        0126-0127 & 4407,6 & 7,66  & $\phi12@15$ \\
        0127-0128 & 4078,5 & 7,09  & $\phi10@11$ \bigstrut[b]\\
        \hline
        \end{tabular}%
        }
      \label{resumensuples-1}%
    \end{table}%
    
    \begin{table}[H]
      \centering
      \caption{Resumen armadura negativa nivel 1.}
        \resizebox{0.5\textwidth}{!}{
        \begin{tabular}{cccc}
        \hline
        \textbf{Losas} & \boldmath{}\textbf{Mu $[kgf \cdot m/m]$}\unboldmath{} & \boldmath{}\textbf{As $[cm^2/m]$}\unboldmath{} & \textbf{F'} \bigstrut\\
        \hline
        101-102 & 1911,6 & 3,56  & $\phi8@14$ \bigstrut[t]\\
        101-105 & 2449,5 & 4,56  & $\phi10@17$ \\
        102-106 & 2748,1 & 5,12  & $\phi12@22$ \\
        102-103 & 1916,8 & 3,57  & $\phi10@22$ \\
        103-107 & 2748,1 & 5,12  & $\phi12@22$ \\
        103-104 & 1944,6 & 3,62  & $\phi8@14$ \\
        104-108 & 2882,1 & 5,37  & $\phi12@21$ \\
        105-106 & 2436,8 & 4,54  & $\phi10@17$ \\
        106-121 & 2126,7 & 3,96  & $\phi10@20$ \\
        106-110 & 2469,0 & 4,6   & $\phi10@17$ \\
        106-107 & 2429,9 & 4,52  & $\phi12@25$ \\
        107-111 & 2548,8 & 4,74  & $\phi12@24$ \\
        107-108 & 2378,6 & 4,43  & $\phi12@25$ \\
        107-112 & 3433,4 & 6,39  & $\phi12@18$ \\
        108-112 & 3493,3 & 6,51  & $\phi10@12$ \\
        108-109 & 2252,0 & 4,19  & $\phi8@12$ \\
        110-121 & 1075,8 & 2,00  & $\phi8@17$ \\
        110-113 & 1326,9 & 2,47  & $\phi8@17$ \\
        110-114 & 1274,2 & 2,37  & $\phi8@17$ \\
        110-110 & 1422,5 & 2,65  & $\phi8@17$ \\
        111-114 & 1413,9 & 2,63  & $\phi8@17$ \\
        111-112 & 2356,9 & 4,39  & $\phi10@18$ \\
        112-116 & 2989,5 & 5,57  & $\phi12@20$ \\
        112-117 & 3126,4 & 5,82  & $\phi12@20$ \\
        113-121 & 401,3 & 0,75  & $\phi8@17$ \\
        114-119 & 352,0 & 0,66  & $\phi8@17$ \\
        114-115 & 462,3 & 0,86  & $\phi8@17$ \\
        114-116 & 961,3 & 1,79  & $\phi8@17$ \\
        115-120 & 709,0 & 1,32  & $\phi8@17$ \\
        115-116 & 1065,1 & 1,98  & $\phi8@17$ \\
        116-117 & 1491,4 & 2,77  & $\phi8@17$ \bigstrut[b]\\
        \hline
        \end{tabular}%
        }
      \label{resumensuples1}%
    \end{table}%
    
    \begin{table}[H]
      \centering
      \caption{Resumen armadura negativa niveles 2 al 22.}
        \resizebox{0.5\textwidth}{!}{
        \begin{tabular}{cccc}
        \hline
        \multicolumn{1}{c}{\textbf{Losas}} & \boldmath{}\textbf{Mu $[kgf \cdot m/m]$}\unboldmath{} & \boldmath{}\textbf{As $[cm^2/m]$}\unboldmath{} & \multicolumn{1}{c}{\textbf{F'}} \bigstrut\\
        \hline
        201-202 & 1691,0 & 3,15  & $\phi10@25$ \bigstrut[t]\\
        201-205 & 2544,1 & 4,74  & $\phi12@24$ \\
        201-206 & 2264,1 & 4,21  & $\phi8@12$ \\
        202-203 & 1095,0 & 2,04  & $\phi8@17$ \\
        202-208 & 995,0 & 1,85  & $\phi8@17$ \\
        203-208 & 995,0 & 1,85  & $\phi8@17$ \\
        203-204 & 1563,8 & 2,91  & $\phi8@17$ \\
        204-210 & 1950,8 & 3,63  & $\phi8@14$ \\
        204-211 & 2174,9 & 4,05  & $\phi10@19$ \\
        205-206 & 1602,9 & 2,98  & $\phi8@17$ \\
        206-219 & 908,0 & 1,69  & $\phi8@17$ \\
        206-208 & 894,8 & 1,67  & $\phi8@17$ \\
        206-207 & 1046,8 & 1,95  & $\phi8@17$ \\
        206-222 & 1010,6 & 1,88  & $\phi8@17$ \\
        207-208 & 352,0 & 0,66  & $\phi8@17$ \\
        208-209 & 462,3 & 0,86  & $\phi8@17$ \\
        208-210 & 961,3 & 1,79  & $\phi8@17$ \\
        209-222 & 725,9 & 1,35  & $\phi8@17$ \\
        209-210 & 1065,1 & 1,98  & $\phi8@17$ \\
        210-211 & 1491,4 & 2,77  & $\phi8@17$ \bigstrut[b]\\
        \hline
        \end{tabular}%
        }
      \label{resumensuples2a22}%
    \end{table}%
    
    \begin{table}[H]
      \centering
      \caption{Resumen armadura negativa nivel 23.}
        \resizebox{0.5\textwidth}{!}{
        \begin{tabular}{cccc}
        \hline
        \textbf{Losas} & \boldmath{}\textbf{Mu $[kgf \cdot m/m]$}\unboldmath{} & \boldmath{}\textbf{As $[cm^2/m]$}\unboldmath{} & \textbf{F'} \bigstrut\\
        \hline
        2301-2302 & 1438,1 & 2,68  & $\phi8@17$ \bigstrut[t]\\
        2301-2305 & 2163,7 & 4,03  & $\phi10@20$ \\
        2301-2306 & 1925,5 & 3,59  & $\phi8@14$ \\
        2302-2303 & 931,3 & 1,73  & $\phi8@17$ \\
        2302-2308 & 834,2 & 1,55  & $\phi8@17$ \\
        2303-2308 & 834,2 & 1,55  & $\phi8@17$ \\
        2303-2304 & 1329,9 & 2,48  & $\phi8@17$ \\
        2304-2310 & 1659,1 & 3,09  & $\phi10@25$ \\
        2304-2311 & 1849,7 & 3,45  & $\phi10@23$ \\
        2305-2306 & 1363,2 & 2,54  & $\phi8@17$ \\
        2306-2308 & 749,0 & 1,40   & $\phi8@17$ \\
        2306-2307 & 848,9 & 1,58  & $\phi8@17$ \\
        2306-2322 & 840,1 & 1,57  & $\phi8@17$ \\
        2307-2308 & 230,4 & 0,43  & $\phi8@17$ \\
        2308-2309 & 302,7 & 0,57  & $\phi8@17$ \\
        2308-2310 & 805,5 & 1,50   & $\phi8@17$ \\
        2309-2322 & 492,3 & 0,92  & $\phi8@17$ \\
        2309-2310 & 837,2 & 1,56  & $\phi8@17$ \\
        2310-2311 & 1268,4 & 2,36  & $\phi8@17$ \bigstrut[b]\\
        \hline
        \end{tabular}%
        }
      \label{resumensuples23}%
    \end{table}%
    
    \begin{table}[H]
      \centering
      \caption{Resumen armadura negativa nivel 24.}
        \resizebox{0.5\textwidth}{!}{      
        \begin{tabular}{cccc}
        \hline
        \multicolumn{1}{c}{\textbf{Losas}} & \boldmath{}\textbf{Mu $[kgf \cdot m/m]$}\unboldmath{} & \boldmath{}\textbf{As $[cm^2/m]$}\unboldmath{} & \multicolumn{1}{c}{\textbf{F'}} \bigstrut\\
        \hline
        \multicolumn{1}{c}{2401-2402} & 301,3 & 0,56  & $\phi8@17$ \bigstrut[t]\\
        2402-2403 & 464,2 & 0,87  & $\phi8@17$ \bigstrut[b]\\
        \hline
        \end{tabular}%
        }
      \label{resumensuples24}%
    \end{table}%


\newpage
\section{Deformaciones máximas por piso}
Teniendo en consideración lo estipulado por la tabla 9.5 (b) de la Norma ACI 318S-05:

\insertimage[]{img/deformacionesACI}{width=16cm}{Deflexión máxima admisible}

Se presenta la siguiente tabla resumen con las deformaciones máximas por piso, donde la deformación elástica de la losa más desfavorable se considera como la diferencia entre la formación máxima y mínima. La deformación por fluencia lenta, o creep, se considera como el triple de la deformación elástica máxima.

\begin{table}[H]
  \centering
  \caption{Deformaciones máximas por piso en losas más desfavorables.}
  \resizebox{\textwidth}{!}{
  \begin{tabular}{cccccccc}
    \hline
    \textbf{Nivel} &
      \textbf{N° Losa} &
      \multicolumn{1}{p{5.335em}}{\boldmath{}\textbf{$\Delta z_{max} [cm]$}\unboldmath{}} &
      \multicolumn{1}{p{5.335em}}{\boldmath{}\textbf{$\Delta z_{min} [cm]$}\unboldmath{}} &
      \multicolumn{1}{p{5.335em}}{\boldmath{}\textbf{$\Delta_{elast} [cm]$}\unboldmath{}} &
      \multicolumn{1}{p{5.335em}}{\boldmath{}\textbf{$\Delta_{creep} [cm]$}\unboldmath{}} &
      \multicolumn{1}{p{5.335em}}{\boldmath{}\textbf{$\Delta_{adm}  [cm]$}\unboldmath{}} &
      \multicolumn{1}{p{4.055em}}{\textbf{Estado}}
      \bigstrut\\
        \hline
        -1 &      0124  &      -0,5417 &      -0,0175 &      0,52 &      1,57 &      4,58 &      OK      \bigstrut[t]\\
        1 &      108    &      -0,5310 &      -0,0891 &     0,44 &      1,33 &      3,08 &      OK      \\
        2 &      204-215 &      -0,3245 &      -0,0868 &      0,24 &      0,71 &      1,26 &      OK      \\
        3 &      304-315 &      -0,3420 &      -0,1172 &      0,22 &      0,67 &      1,26 &      OK      \\
        4 &      404-415 &      -0,3777 &      -0,1453 &      0,23 &      0,70 &      1,26 &      OK      \\
        5 &      504-515 &      -0,4095 &      -0,1730 &      0,24 &      0,71 &      1,26 &      OK      \\
        6 &      604-615 &      -0,4402 &      -0,1989 &      0,24 &      0,72 &      1,26 &      OK      \\
        7 &      704-715 &      -0,4689 &      -0,2237 &      0,25 &      0,74 &      1,26 &      OK      \\
        8 &      804-815 &      -0,5064 &      -0,2486 &      0,26 &      0,77 &      1,26 &      OK      \\
        9 &      904-915 &      -0,5332 &      -0,2720 &      0,26 &      0,78 &      1,26 &      OK      \\
        10 &      1004-1015 &      -0,5581 &      -0,2937 &      0,26 &      0,79 &      1,26 &      OK      \\
        11 &      1104-1115 &      -0,5810 &      -0,3137 &      0,27 &      0,80 &      1,26 &      OK      \\
        12 &      1204-1215 &      -0,6019 &      -0,3320 &      0,27 &      0,81 &      1,26 &      OK      \\
        13 &      1304-1315 &      -0,6216 &      -0,3487 &      0,27 &      0,82 &      1,26 &      OK      \\
        14 &      1404-1415 &      -0,6549 &      -0,3718 &      0,28 &      0,85 &      1,26 &      OK      \\
        15 &      1504-1515 &      -0,6784 &      -0,3922 &      0,29 &      0,86 &      1,26 &      OK      \\
        16 &      1604-1615 &      -0,6990 &      -0,4103 &      0,29 &      0,87 &      1,26 &      OK      \\
        17 &      1704-1715 &      -0,7169 &      -0,4260 &      0,29 &      0,87 &      1,26 &      OK      \\
        18 &      1804-1815 &      -0,7320 &      -0,4392 &      0,29 &      0,88 &      1,26 &      OK      \\
        19 &      1904-1915 &      -0,7442 &      -0,4501 &      0,29 &      0,88 &      1,26 &      OK      \\
        20 &      2004-2015 &      -0,7537 &      -0,4585 &      0,30 &      0,89 &      1,26 &      OK      \\
        21 &      2104-2115 &      -0,7605 &      -0,4647 &      0,30 &      0,89 &      1,26 &      OK      \\
        22 &      2204-2215 &      -0,7643 &      -0,4680 &      0,30 &      0,89 &      1,26 &      OK      \\
        23 &      2315 &      -0,7698 &      -0,6369 &      0,13 &      0,40 &      0,84 &      OK      \\
        24 &      2404 &      -0,5075 &      -0,4537 &      0,05 &      0,16 &      1,61 &      OK      \\
        CU &      CU01 &      -0,4778 &      -0,4360 &      0,04 &      0,13 &      1,22 &      OK      \bigstrut[b]\\
        \hline
      \end{tabular}
      }
      \label{deflosas}
    \end{table}

\newpage

A continuación se muestran los esquemas de deformaciones verticales de las losas de cada nivel:

\begin{images}[\label{imagenmultiple1}]{Deformaciones piso -1 al 5.}
    \addimage{img/Deformaciones/-1}{width=6.5cm}{nivel -1.}
    \addimage{img/Deformaciones/1}{width=6.5cm}{nivel 1.}
    \addimage{img/Deformaciones/2}{width=6.5cm}{nivel 2.}
    \addimage{img/Deformaciones/3}{width=6.5cm}{nivel 3.}
    \addimage{img/Deformaciones/4}{width=6.5cm}{nivel 4.}
    \addimage{img/Deformaciones/6}{width=6.5cm}{nivel 5.}
\end{images}

\begin{images}[\label{imagenmultiple2}]{Deformaciones piso 6 al 11.}
    \addimage{img/Deformaciones/6}{width=6.5cm}{nivel 6.}
    \addimage{img/Deformaciones/7}{width=6.5cm}{nivel 7.}
    \addimage{img/Deformaciones/8}{width=6.5cm}{nivel 8.}
    \addimage{img/Deformaciones/9}{width=6.5cm}{nivel 9.}
    \addimage{img/Deformaciones/10}{width=6.5cm}{nivel 10.}
    \addimage{img/Deformaciones/11}{width=6.5cm}{nivel 11.}
\end{images}

\begin{images}[\label{imagenmultiple3}]{Deformaciones piso 12 al 17.}
    \addimage{img/Deformaciones/12}{width=6.5cm}{nivel 12.}
    \addimage{img/Deformaciones/13}{width=6.5cm}{nivel 13.}
    \addimage{img/Deformaciones/14}{width=6.5cm}{nivel 14.}
    \addimage{img/Deformaciones/15}{width=6.5cm}{nivel 15.}
    \addimage{img/Deformaciones/16}{width=6.5cm}{nivel 16.}
    \addimage{img/Deformaciones/17}{width=6.5cm}{nivel 17.}
\end{images}

\begin{images}[\label{imagenmultiple4}]{Deformaciones piso 18 al 23.}
    \addimage{img/Deformaciones/18}{width=6.5cm}{nivel 18.}
    \addimage{img/Deformaciones/19}{width=6.5cm}{nivel 19.}
    \addimage{img/Deformaciones/20}{width=6.5cm}{nivel 20.}
    \addimage{img/Deformaciones/21}{width=6.5cm}{nivel 21.}
    \addimage{img/Deformaciones/22}{width=6.5cm}{nivel 22.}
    \addimage{img/Deformaciones/23}{width=6.5cm}{nivel 23.}
\end{images}

\begin{images}[\label{imagenmultiple5}]{Deformaciones piso 24 a cubierta.}
    \addimage{img/Deformaciones/24}{width=6.5cm}{nivel 24.}
    \addimage{img/Deformaciones/Cubierta}{width=6.5cm}{Cubierta.}
\end{images}

\newpage

\section{Comentarios}

\begin{itemize}
    \item Se puede ver que si bien las deformaciones máximas en los niveles inferiores (-1 y 1) son del orden de \texttt{1,5 cm}, estas no exceden al máximo permitido por la norma ya que se dan en losas de estacionamiento, bajo las cuales no existen elementos no estructurales, como tabiques, que puedan resultar dañados. 
    
    A diferencias de lo ocurrido en los niveles superiores, donde la deformación máxima se da en la zona adyacente a los balcones. Aquí podría afectar el correcto desempeño de elementos no estructurales, como las ventanas, por lo que el límite permitido es menor, pero aún así no es excedido.
    
    \item Para la elección de armaduras, una vez determinada es área de acero requerida para resistir la flexión de la losa, se dio prioridad a mantener constante el diámetro de barras, aumentando la separación entre estas. Esto para, en un caso hipotético, facilitar la construcción.
    
    En cuanto a la armadura negativa, en franjas de losas, como el pasillo que une los departamentos, se optó por uniformar el diseño. Para ello se consideraron suples que atraviesan por completo el pasillo por su lado menor, escogiéndose la sección de acero mayor para garantizar la resistencia en ambas interacciones entre losas.
    
    Si bien, para efectos del modelo, las losas de balcones se consideraron separadas de las losas de los departamentos, al no disponer de un elemento que materialice esa división, como un muro o viga, estas en la práctica resultan ser continuas. Es por eso que no se dispusieron suples en dicha interfaz.
    
    Cabe destacar que, se coloca doble malla en los pisos cítricos, es decir, hasta el nivel 2. 
    
    Por último, con respecto a la armadura de refuerzo, se dispone en zonas como dinteles, el perímetro de la caja del ascensor y en los shafts, usando siempre un $F=F'= \phi 2 @ 12$.
    
    \item Las armaduras mínimas están directamente ligadas al espesor de la losa, es por eso que se obtienen valores distintos para el nivel -1 y el resto, en que el espesor se mantiene uniforme en \texttt{16 cm}.
    
    Si bien, al realizarse el cálculo más detallado de estos espesores mínimos se obtuvieron valores menores a los determinados en etapas previas, de pre-diseño, se optó por conservarlos ya que se encuentran dentro del rango usual de espesores acostumbrado en los edificios chilenos.
\end{itemize} 
% FIN DEL DOCUMENTO
\end{document}
