\section{Determinación de cargas de diseño.}

    A continuación se presentan las cargas a considerar en el diseño:
    
    \begin{table}[htbp]
      \centering
      \caption{Cargas de peso propio.}
        \begin{tabular}{cc}
        \hline
        \textbf{Elemento} & \boldmath{}\textbf{Peso Propio $[kgf/m^2]$}\unboldmath{} \bigstrut\\
        \hline
        \boldmath{}\textbf{$PP_{Tabique}$}\unboldmath{} & \textbf{100} \bigstrut[t]\\
        \textit{$\gamma_{yeso} [kgf/m2 \cdot cm]$} & \textit{10} \\
        \textit{$e_{yeso} [cm]$} & \textit{2,5} \\
        \boldmath{}\textbf{$PP_{Yeso}$}\unboldmath{} & \textbf{25} \\
        $\gamma_{sl} [kgf/m2 \cdot cm]$ & 20 \\
        $e_{sl} [cm]$ & 5 \\
        \boldmath{}\textbf{$PP_{Sobrelosa}$}\unboldmath{} & \textbf{100} \bigstrut[b]\\
        \hline
        \boldmath{}\textbf{$PP_{adic}$}\unboldmath{} & \textbf{225} \bigstrut\\
        \hline
        \end{tabular}%
      \label{peso_propio}%
    \end{table}%

    \begin{table}[htbp]
      \centering
      \caption{Sobrecargas.}
        \begin{tabular}{cc}
        \hline
        Ocupación & Sobrecargas $[kgf/m^2]$ \bigstrut\\
        \hline
        Habitacional & 200 \bigstrut[t]\\
        Área común y escalera & 400 \\
        Balcones & 300 \\
        Autos & 500 \\
        Techo & 100 \bigstrut[b]\\
        \hline
        \end{tabular}%
      \label{sobrecarga}%
    \end{table}%
    
    A esto se agrega el peso propio de cada losa y, luego de mayorar las cargas, se obtienen las solicitaciones ($qu$) en cada elemento.