% Template:     Informe/Reporte LaTeX
% Documento:    Archivo principal
% Versión:      6.1.0 (03/11/2018)
% Codificación: UTF-8
%
% Autor: Pablo Pizarro R. @ppizarror
%        Facultad de Ciencias Físicas y Matemáticas
%        Universidad de Chile
%        pablo.pizarro@ing.uchile.cl, ppizarror.com
%
% Manual template: [https://latex.ppizarror.com/Template-Informe/]
% Licencia MIT:    [https://opensource.org/licenses/MIT/]

% CREACIÓN DEL DOCUMENTO
\documentclass[letterpaper,11pt]{article} % Articulo tamaño carta, 11pt
\usepackage[utf8]{inputenc} % Codificación UTF-8

% INFORMACIÓN DEL DOCUMENTO
\def\titulodelinforme {Diseño sísmico de muros y vigas sísmicas}
\def\temaatratar {Proyecto de Hormigón Armado - Entrega N°6}

\def\autordeldocumento {Grupo N°tele1}
\def\nombredelcurso {Proyecto de Hormigón Armado}
\def\codigodelcurso {CI5206-2}

\def\nombreuniversidad {Universidad de Chile}
\def\nombrefacultad {Facultad de Ciencias Físicas y Matemáticas}
\def\departamentouniversidad {Departamento de Ingeniería Civil}
\def\imagendepartamento {dic}
\def\imagendepartamentoescala {0.2}
\def\localizacionuniversidad {Santiago, Chile}

% INTEGRANTES, PROFESORES Y FECHAS
\def\tablaintegrantes {
\begin{tabular}{ll}
    Grupo:
    	& \begin{tabular}[t]{@{}l@{}}
    		N°1
    	\end{tabular} \\
	Integrantes:
	& \begin{tabular}[t]{@{}l@{}}
		Mauricio Leal V. \\
		Pablo Pizarro R. \\
		Ignacio Yáñez G.
	\end{tabular} \\
	Profesor:
	& \begin{tabular}[t]{@{}l@{}}
		Juan Mendoza V.
	\end{tabular} \\
	Auxiliar:
	& \begin{tabular}[t]{@{}l@{}}
		Felipe Andrade T.
	\end{tabular} \\
	& \\
	\multicolumn{2}{l}{Fecha de entrega: 5 de Diciembre de 2018} \\
	\multicolumn{2}{l}{\localizacionuniversidad}
\end{tabular}}{
}

% CONFIGURACIONES
\input{lib/config}

% IMPORTACIÓN DE LIBRERÍAS
\input{lib/env/imports}

% IMPORTACIÓN DE FUNCIONES Y ENTORNOS
\input{lib/cmd/all}

% IMPORTACIÓN DE ESTILOS
\input{lib/style/all}

% CONFIGURACIÓN INICIAL DEL DOCUMENTO
\input{lib/cfg/init}

% INICIO DE LAS PÁGINAS
\begin{document}

% PORTADA
\input{lib/page/portrait} % Se puede borrar

% CONFIGURACIÓN DE PÁGINA Y ENCABEZADOS
\input{lib/cfg/page}


% TABLA DE CONTENIDOS - ÍNDICE
% Template:     Informe/Reporte LaTeX
% Documento:    Índice
% Versión:      6.0.1 (21/10/2018)
% Codificación: UTF-8
%
% Autor: Pablo Pizarro R. @ppizarror
%        Facultad de Ciencias Físicas y Matemáticas
%        Universidad de Chile
%        pablo.pizarro@ing.uchile.cl, ppizarror.com
%
% Manual template: [http://latex.ppizarror.com/Template-Informe/]
% Licencia MIT:    [https://opensource.org/licenses/MIT/]

\ifthenelse{\equal{\showindex}{true}}{
	\newpage
	\begingroup
	\sectionfont{\color{\indextitlecolor} \fontsizetitlei \styletitlei \selectfont}
	\ifthenelse{\equal{\addindextobookmarks}{true}}{
		\belowpdfbookmark{\nomltcont}{contents}}{
	}
	\tocloftpagestyle{fancy}
	\ifthenelse{\equal{\showdotontitles}{true}}{
		\def\cftsecaftersnum {.}
		\def\cftsubsecaftersnum {.}
		\def\cftsubsubsecaftersnum {.}
		\def\cftsubsubsubsecaftersnum {.}
		}{
	}
	\def\cftfigaftersnum {\charafterobjectindex\enspace}
	\def\cftsubfigaftersnum {\charafterobjectindex\enspace}
	\def\cfttabaftersnum {\charafterobjectindex\enspace}
	\def\cftlstlistingaftersnum {\charafterobjectindex\enspace}
	\renewcommand{\cftdot}{\charnumpageindex}
	\ifthenelse{\equal{\showlinenumbers}{true}}{
		\nolinenumbers}{
	}
	\ifthenelse{\equal{\objectindexindent}{true}}{
		\def\cftlstlistingindent {1.495em}
	}{
		\setlength{\cfttabindent}{0in}
		\setlength{\cftfigindent}{0in}
		\setlength{\cftsubfigindent}{0in}
		\setlength{\cftfigindent}{0in}
		\def\cftlstlistingindent {0.01em}
	}
	\ifthenelse{\equal{\equalmarginnumobject}{true}}{
		\ifthenelse{\equal{\showsectioncaption}{none}}{
			\def\cftdefautnumwidth {2.3em}
		}{
		\ifthenelse{\equal{\showsectioncaption}{sec}}{
			\def\cftdefautnumwidth {3.0em}
		}{
		\ifthenelse{\equal{\showsectioncaption}{ssec}}{
			\def\cftdefautnumwidth {3.8em}
		}{
		\ifthenelse{\equal{\showsectioncaption}{sssec}}{
			\def\cftdefautnumwidth {4.3em}
		}{
			\throwbadconfig{Valor configuracion incorrecto}{\showsectioncaption}{none,sec,ssec,sssec}}}}
		}
		\def\cftfignumwidth {\cftdefautnumwidth}
		\def\cftsubfignumwidth {\cftdefautnumwidth}
		\def\cfttabnumwidth {\cftdefautnumwidth}
		\def\cftlstlistingnumwidth {\cftdefautnumwidth}}{
	}
	\ifthenelse{\equal{\showindexofcontents}{true}}{\tableofcontents}{}
	\iftotalfigures
		\ifthenelse{\equal{\showindexoffigures}{true}}{
			\ifthenelse{\equal{\indexforcenewpage}{true}}{\newpage}{}
			\listoffigures
		}{}
	\fi
	\iftotaltables
		\ifthenelse{\equal{\showindexoftables}{true}}{
			\ifthenelse{\equal{\indexforcenewpage}{true}}{\newpage}{}
			\listoftables
		}{}
	\fi
	\iftotallstlistings
		\ifthenelse{\equal{\showindexofcode}{true}}{
			\ifthenelse{\equal{\indexforcenewpage}{true}}{\newpage}{}
			\lstlistoflistings
		}{}
	\fi
	\endgroup
	\ifthenelse{\equal{\addemptypagetwosides}{true}}{
		\vfill
		\checkoddpage
		\ifoddpage
		\else
			\newpage
			\null
			\thispagestyle{empty}
			\newpage
			\addtocounter{page}{-1}
		\fi}{
	}
}{}
 % Se puede borrar

% CONFIGURACIONES FINALES
% Template:     Informe/Reporte LaTeX
% Documento:    Configuraciones finales
% Versión:      6.1.6 (14/12/2018)
% Codificación: UTF-8
%
% Autor: Pablo Pizarro R. @ppizarror
%        Facultad de Ciencias Físicas y Matemáticas
%        Universidad de Chile
%        pablo.pizarro@ing.uchile.cl, ppizarror.com
%
% Manual template: [https://latex.ppizarror.com/Template-Informe/]
% Licencia MIT:    [https://opensource.org/licenses/MIT/]

\markboth{}{}
\newpage
\ifthenelse{\equal{\disablehfrightmark}{false}}{
	\ifthenelse{\equal{\hfstyle}{style1}}{
		\fancyhead[L]{\nouppercase{\leftmark}}}{
	}
	\ifthenelse{\equal{\hfstyle}{style2}}{
		\fancyhead[L]{\nouppercase{\leftmark}}}{
	}
	\ifthenelse{\equal{\hfstyle}{style4}}{
		\fancyhead[L]{\nouppercase{\leftmark}}}{
	}
	\ifthenelse{\equal{\hfstyle}{style5}}{
		\fancyhead[R]{\nouppercase{\leftmark}}}{
	}
	\ifthenelse{\equal{\hfstyle}{style9}}{
		\fancyhead[L]{\nouppercase{\leftmark}}}{
	}
	\ifthenelse{\equal{\hfstyle}{style10}}{
		\fancyhead[L]{\nouppercase{\leftmark}}}{
	}
\ifthenelse{\equal{\hfstyle}{style11}}{
		\fancyhead[L]{\nouppercase{\leftmark}}}{
	}
\ifthenelse{\equal{\hfstyle}{style14}}{
		\fancyhead[L]{\nouppercase{\leftmark}}}{
	}}{
}
\sectionfont{\color{\titlecolor} \fontsizetitle \styletitle \selectfont}
\subsectionfont{\color{\subtitlecolor} \fontsizesubtitle \stylesubtitle \selectfont}
\subsubsectionfont{\color{\subsubtitlecolor} \fontsizesubsubtitle \stylesubsubtitle \selectfont}
\titleformat{\subsubsubsection}{\color{\ssstitlecolor} \normalfont \fontsizessstitle \stylessstitle}{\thesubsubsubsection}{1em}{}
\titlespacing*{\subsubsubsection}{0pt}{3.25ex plus 1ex minus .2ex}{1.5ex plus .2ex}
\ifthenelse{\equal{\showsectioncaption}{none}}{
}{
\ifthenelse{\equal{\showsectioncaption}{sec}}{
	\counterwithin{equation}{section}
	\counterwithin{figure}{section}
	\counterwithin{lstlisting}{section}
	\counterwithin{table}{section}
}{
\ifthenelse{\equal{\showsectioncaption}{ssec}}{
	\counterwithin{equation}{subsection}
	\counterwithin{figure}{subsection}
	\counterwithin{lstlisting}{subsection}
	\counterwithin{table}{subsection}
}{
\ifthenelse{\equal{\showsectioncaption}{sssec}}{
	\counterwithin{equation}{subsubsection}
	\counterwithin{figure}{subsubsection}
	\counterwithin{lstlisting}{subsubsection}
	\counterwithin{table}{subsubsection}
}{
\ifthenelse{\equal{\showsectioncaption}{ssssec}}{
	\counterwithin{equation}{subsubsubsection}
	\counterwithin{figure}{subsubsubsection}
	\counterwithin{lstlisting}{subsubsubsection}
	\counterwithin{table}{subsubsubsection}
}{
	\throwbadconfig{Valor configuracion incorrecto}{\showsectioncaption}{none,sec,ssec,sssec,ssssec}
}}}}}
\ifthenelse{\equal{\predocuseromannumber}{true}}{
	\renewcommand{\thepage}{\arabic{page}}}{
}
\ifthenelse{\equal{\resetpagnumafterindex}{true}}{
	\setcounter{page}{1}}{
}
\setcounter{section}{0}
\setcounter{footnote}{0}
\ifthenelse{\equal{\showlinenumbers}{true}}{
	\linenumbers}{
}


% ======================= INICIO DEL DOCUMENTO =======================
\section{Introducción}
En el presente informe se diseñan los muros del edificio, buscando obtener la resistencia a compresión de cada muro a partir de las fuerzas solicitantes de los mismos. En base a lo anterior y a los diagramas de interacción, determinar las armaduras necesarias en cada muro. Esto se acompaña con los correspondientes esquemas de armaduras.\\

Adicionalmente, se analizarán un par de vigas sísmicas y una estática del edificio, en las cuales, considerando los esfuerzos de diseño y armadura mínima requerida, determinar de forma posterior diagramas de momento y corte que permitan calcular la armadura de corte y flexión y necesaria. Lo anterior se acompañará de los correspondientes esquemas de armaduras de corte y flexión. 
\newpage
\section{Diseño de muros}

\subsection{Metodología de cálculo}

En la presente trabajo se busca calcular la armadura de los muros de cinco ejes del edificio: Eje 6, 11, 13, G entre 3 y 6 y el eje I. \\

Para cada uno de los muros de dichos ejes se obtuvieron las fuerzas de diseño desde ETABS y se calculó la armadura horizontal requerida utilizando dos métodos de resistencia al corte\footnote{Método aproximado y método propuesto por el código de diseño para concreto estructural ACI 318.},  para obtener la armadura vertical y las puntas de muro se utilizaron los diagramas de interacción P-M. Adicionalmente se comprueba la resistencia a compresión y el requisito de esbeltez.

\subsubsection{Obtención de fuerzas solicitantes de cada muro}

Para obtener las fuerzas que actúan sobre cada uno de los muros a diseñar (P, Q, M) se hace uso de la herramienta ETABS. En ella se define un \textit{pier} en cada uno de los muros a analizar, considerando siempre que se requieren dos pier o más si es que el muro presenta discontinuidades en un mismo piso para un eje estudiado. \\

La Figura \ref{piers} ilustra un esquema de asignación de \textit{pier} en los muros modelados en ETABS. A modo de simplificar el análisis numérico se asignó una etiqueta distinta para cada muro, en el caso que no cambie en altura el pier es el mismo para todos los pisos.

\insertimage[\label{piers}]{pier}{width=15cm}{Ejemplo de asignación de pier en ETABS, eje G.}

\newpage
\subsubsection{Comprobación resistencia a compresión y requisito esbeltez}

Si $H$ corresponde a la altura libre de muro y $e$ su ancho se busca que la esbeltez, definida por $\frac{H}{16}$, sea mayor o igual a $e$:

\insertequation{e \geq \frac{H}{16}}

En el caso que lo anterior no se cumpla sólo se puede modificar el ancho del muro $e$, dado que la altura libre es un dato fijo. En cualquier caso todos los muros analizados cumplieron satisfactoriamente con el requisito de esbeltez. \\

Para la resistencia a compresión se comprobó que la carga última $N_u$ fuese menor o igual que la resistencia a la compresión:

\insertequation{N_u \leq 0.35 f'c \cdot A_c}

\subsubsection{Cálculo armadura de corte horizontal}

Para calcular la armadura de corte horizontal requerida se utilizó tanto la fórmula aproximada como el método de ACI para obtener el área mínima de acero requerida en la sección para lograr la resistencia.

\begin{itemize}
    \item \textbf{Fórmula aproximada}: Suponiendo que el acero toma el cien por ciento del corte, obtenido como la suma de los valores absolutos de las fuerzas de corte producto de las cargas de servicio, se busca un $A_e$ área de acero con tal de satisfacer:
    
    \insertgather{Q = \abs{Q_{Peso\ propio}} + \abs{Q_{Sobrecarga}} + \abs{Q_{Sismico}} \\\tau = \frac{Q}{A} \\ A_e = \frac{\tau \cdot 100 \cdot e}{2 \sigma_e}}
    
    En donde $\tau$ es la tensión de corte media en el muro ($kgf$/$cm^2$), $A$ el área de la sección transversal del muro, $A_e$ el área transversal por metro de ancho, considerando dos capas de acero ($cm^2$), $e$ el espesor del muro y $\sigma_e$ la tensión de corte admisible del acero, que para el caso puntual del presente trabajo $\sigma_e = 2800$ ($kgf$/$cm^2$).
    
    \item \textbf{Método ACI}: Para este caso el cálculo de la armadura de corte se basa en una suma de resistencias contribuídas por el hormigón y el acero. En este sentido se considera que la resistencia al corte proporcionada por la sección de hormigón corresponde a:
    
    \insertequation{V_c = 0.17\lambda \sqrt{f'c} h d}
    
    En donde $h$ es el espesor del muro, $d=0.8l_w$, $l_w$ el largo del muro y $\lambda$=1. Una vez conocido $V_c$ se calcula $V_s$ como el necesario para cumplir el requisito de resistencia:
    
    \insertequation{V_s = \frac{V_u}{\phi} - V_c}
    
    En donde $V_u$ se calcula considerando una combinación de las cargas puras de diseño obtenidas mediante ETABS (piers):
    
    \insertequation{V_u = 1.2 Q_{pp} + 1.0 Q_{sc} + 1.4 Q_{sismo}}
    
    De esta manera, se obtiene el diámetro de la enfierradura con tal de cumplir:
    
    \insertequation{V_s = \frac{A_v \cdot f_y \cdot d}{s} = \frac{A_v \cdot f_y \cdot 0.8l_w}{s} \leq 0.66 \sqrt{f'c} b_w d}
    
    Una vez obtenido $V_c$ y $V_s$ se debe verificar que $V_u$ no exceda:
    
    \insertequation{V_n = A_{cv} \cdot \bigg( \alpha_c \lambda \sqrt{f'c}  + \rho_t f_y\bigg)}
    
    En donde $\rho_t = \frac{A_v}{s\cdot e}$, $\alpha_c$ coeficiente que es 0.25 para $h_w/l_w \leq1.5$, 0.15 para $h_w/l_w = 2.0$ y varía linealmente entre 0.25 y 0.17 para $h_w/l_w$ entre 1.5 y 2.0.
\end{itemize}

\subsubsection{Cálculo armadura vertical y puntas de muro}

Para cada uno de los muros se graficó en un diagrama de interacción P-M, de mismo $f'c$ que el muro de análisis, el estado de carga de cada una de las combinaciones. Luego la cuantía $\rho$ de las puntas de muro corresponde a aquella envolvente de todos los puntos, $\rho_w$ la cuantía de la armadura vertical corresponde a la cuantía con la que fué diseñado el diagrama. \\

Los diagramas se obtuvieron digitalizando los gráficos propuestos por el libro \textit{Manual de Cálculo de Hormigón Armado} de GERDAU AZA. La Figura \ref{pm} ilustra un diagrama digitalizado, cada una de las curvas simboliza una cuantía de punta $\rho$ diferente, variada entre 1\% y 8\%.

\insertimage[\label{pm}]{pm}{width=10cm}{Diagrama de interacción P-M, $\rho_w$=0.0025, $f'c$=30MPa}

\subsection{Resultados obtenidos}

\begin{table}[H]
  \centering
  \caption{Resultado armaduras muros eje 6.}
  \itemresize{1.0}{
  \begin{tabular}{P{0.8cm}cP{0.9cm}P{1.5cm}P{0.8cm}P{0.8cm}P{0.8cm}P{0.8cm}P{1.3cm}P{1.1cm}P{1.1cm}P{1.1cm}P{1cm}P{1cm}P{1.3cm}}
    \hline
    \textbf{PISO} & \textbf{MURO} & \textbf{NPIER} & \textbf{EJE SISMO} & \textbf{$f'c$ (MPa)} & \textbf{e (cm)} & \textbf{L (cm)} & \textbf{H (cm)} & \textbf{N (tonf)} & \textbf{PP (tonf)} & \textbf{SC (tonf)} & \textbf{QS (tonf)} & \cellcolor[rgb]{ .949,  .949,  .949}\textbf{M.H} & \cellcolor[rgb]{ .949,  .949,  .949}\textbf{M.V} & \cellcolor[rgb]{ .949,  .949,  .949}\textbf{PUNTA} \bigstrut\\
    \hline
    -1    & EJE 6.C-D & 1     & X     & 35    & 25    & 178   & 230   & 487.81 & 1.78  & 0.51  & 4.05  & \cellcolor[rgb]{ .949,  .949,  .949}$\phi$10@14 & \cellcolor[rgb]{ .949,  .949,  .949}$\phi$8@16 & \cellcolor[rgb]{ .949,  .949,  .949}2$\phi$25 \bigstrut[t]\\
    -1    & EJE 6.C-E & 2     & X     & 35    & 25    & 227   & 230   & 422.11 & 18.19 & 4.55  & 14.89 & \cellcolor[rgb]{ .949,  .949,  .949}$\phi$10@14 & \cellcolor[rgb]{ .949,  .949,  .949}$\phi$8@16 & \cellcolor[rgb]{ .949,  .949,  .949}2$\phi$25 \\
    1     & EJE 6.C-G & 3     & X     & 35    & 25    & 625   & 230   & 1051.10 & 7.51  & 7.46  & 131.33 & \cellcolor[rgb]{ .949,  .949,  .949}$\phi$10@14 & \cellcolor[rgb]{ .949,  .949,  .949}$\phi$8@16 & \cellcolor[rgb]{ .949,  .949,  .949}2$\phi$25 \\
    3     & EJE 6.B-E & 5     & X     & 35    & 25    & 636   & 230   & 963.78 & 37.86 & 7.90  & 61.93 & \cellcolor[rgb]{ .949,  .949,  .949}$\phi$8@16 & \cellcolor[rgb]{ .949,  .949,  .949}$\phi$8@16 & \cellcolor[rgb]{ .949,  .949,  .949}2$\phi$25 \\
    8     & EJE 6.B-E & 5     & X     & 30    & 25    & 636   & 230   & 767.57 & 6.99  & 1.37  & 59.73 & \cellcolor[rgb]{ .949,  .949,  .949}$\phi$8@16 & \cellcolor[rgb]{ .949,  .949,  .949}$\phi$8@16 & \cellcolor[rgb]{ .949,  .949,  .949}2$\phi$25 \\
    15    & EJE 6.B-E & 5     & X     & 20    & 20    & 636   & 230   & 406.09 & 5.70  & 1.24  & 44.63 & \cellcolor[rgb]{ .949,  .949,  .949}$\phi$8@20 & \cellcolor[rgb]{ .949,  .949,  .949}$\phi$8@20 & \cellcolor[rgb]{ .949,  .949,  .949}2$\phi$22 \\
    23    & EJE 6.B-E & 5     & X     & 20    & 20    & 636   & 230   & 33.09 & 3.99  & 0.52  & 23.14 & \cellcolor[rgb]{ .949,  .949,  .949}$\phi$8@20 & \cellcolor[rgb]{ .949,  .949,  .949}$\phi$8@20 & \cellcolor[rgb]{ .949,  .949,  .949}2$\phi$22 \bigstrut[b]\\
    \hline
  \end{tabular}
  }
  \label{tablamuros1}
\end{table}

\begin{table}[H]
  \centering
  \caption{Resultado armaduras muros eje 11.}
  \itemresize{1.0}{
  \begin{tabular}{P{0.8cm}cP{0.9cm}P{1.5cm}P{0.8cm}P{0.8cm}P{0.8cm}P{0.8cm}P{1.3cm}P{1.1cm}P{1.1cm}P{1.1cm}P{1cm}P{1cm}P{1.3cm}}
    \hline
    \textbf{PISO} & \textbf{MURO} & \textbf{NPIER} & \textbf{EJE SISMO} & \textbf{$f'c$ (MPa)} & \textbf{e (cm)} & \textbf{L (cm)} & \textbf{H (cm)} & \textbf{N (tonf)} & \textbf{PP (tonf)} & \textbf{SC (tonf)} & \textbf{QS (tonf)} & \cellcolor[rgb]{ .949,  .949,  .949}\textbf{M.H} & \cellcolor[rgb]{ .949,  .949,  .949}\textbf{M.V} & \cellcolor[rgb]{ .949,  .949,  .949}\textbf{PUNTA} \bigstrut\\
    \hline
    -1    & EJE 11.C-E & 6     & X     & 35    & 25    & 485   & 230   & 1076.19 & 14.05 & 4.38  & 24.04 & \cellcolor[rgb]{ .949,  .949,  .949}$\phi$8@16 & \cellcolor[rgb]{ .949,  .949,  .949}$\phi$8@16 & \cellcolor[rgb]{ .949,  .949,  .949}2$\phi$25 \bigstrut[t]\\
    1     & EJE 11.C-G & 6     & X     & 35    & 25    & 485   & 230   & 1069.67 & 11.60 & 6.96  & 47.22 & \cellcolor[rgb]{ .949,  .949,  .949}$\phi$8@16 & \cellcolor[rgb]{ .949,  .949,  .949}$\phi$8@16 & \cellcolor[rgb]{ .949,  .949,  .949}2$\phi$25 \\
    3     & EJE 11.B-E & 8     & X     & 35    & 25    & 636   & 230   & 999.28 & 35.60 & 7.72  & 62.26 & \cellcolor[rgb]{ .949,  .949,  .949}$\phi$8@16 & \cellcolor[rgb]{ .949,  .949,  .949}$\phi$8@16 & \cellcolor[rgb]{ .949,  .949,  .949}2$\phi$25 \\
    8     & EJE 11.B-E & 8     & X     & 30    & 25    & 636   & 230   & 800.88 & 6.05  & 1.38  & 71.47 & \cellcolor[rgb]{ .949,  .949,  .949}$\phi$8@16 & \cellcolor[rgb]{ .949,  .949,  .949}$\phi$8@16 & \cellcolor[rgb]{ .949,  .949,  .949}2$\phi$25 \\
    15    & EJE 11.B-E & 8     & X     & 20    & 20    & 636   & 230   & 423.45 & 4.79  & 1.17  & 50.25 & \cellcolor[rgb]{ .949,  .949,  .949}$\phi$8@20 & \cellcolor[rgb]{ .949,  .949,  .949}$\phi$8@20 & \cellcolor[rgb]{ .949,  .949,  .949}2$\phi$22 \\
    23    & EJE 11.B-E & 8     & X     & 20    & 20    & 636   & 230   & 30.51 & 8.89  & 2.09  & 15.57 & \cellcolor[rgb]{ .949,  .949,  .949}$\phi$8@20 & \cellcolor[rgb]{ .949,  .949,  .949}$\phi$8@20 & \cellcolor[rgb]{ .949,  .949,  .949}2$\phi$22 \bigstrut[b]\\
    \hline
  \end{tabular}
  }
\end{table}

\begin{table}[H]
  \centering
  \caption{Resultado armaduras muros eje 13.}
  \itemresize{1.0}{
  \begin{tabular}{P{0.8cm}cP{0.9cm}P{1.5cm}P{0.8cm}P{0.8cm}P{0.8cm}P{0.8cm}P{1.3cm}P{1.1cm}P{1.1cm}P{1.1cm}P{1cm}P{1cm}P{1.3cm}}
    \hline
    \textbf{PISO} & \textbf{MURO} & \textbf{NPIER} & \textbf{EJE SISMO} & \textbf{$f'c$ (MPa)} & \textbf{e (cm)} & \textbf{L (cm)} & \textbf{H (cm)} & \textbf{N (tonf)} & \textbf{PP (tonf)} & \textbf{SC (tonf)} & \textbf{QS (tonf)} & \cellcolor[rgb]{ .949,  .949,  .949}\textbf{M.H} & \cellcolor[rgb]{ .949,  .949,  .949}\textbf{M.V} & \cellcolor[rgb]{ .949,  .949,  .949}\textbf{PUNTA} \bigstrut\\
    \hline
    -1    & EJE 13.H-L & 9     & X     & 35    & 25    & 489   & 230   & 663.53 & 26.75 & 2.05  & 21.93 & \cellcolor[rgb]{ .949,  .949,  .949}$\phi$8@16 & \cellcolor[rgb]{ .949,  .949,  .949}$\phi$8@16 & \cellcolor[rgb]{ .949,  .949,  .949}2$\phi$22 \bigstrut[t]\\
    1     & EJE 13.H-K & 10    & X     & 35    & 25    & 345   & 230   & 613.06 & 11.75 & 3.17  & 31.83 & \cellcolor[rgb]{ .949,  .949,  .949}$\phi$8@16 & \cellcolor[rgb]{ .949,  .949,  .949}$\phi$8@16 & \cellcolor[rgb]{ .949,  .949,  .949}2$\phi$22 \\
    3     & EJE 13.B-E & 11    & X     & 35    & 25    & 345   & 230   & 556.23 & 14.33 & 3.24  & 34.75 & \cellcolor[rgb]{ .949,  .949,  .949}$\phi$8@16 & \cellcolor[rgb]{ .949,  .949,  .949}$\phi$8@16 & \cellcolor[rgb]{ .949,  .949,  .949}2$\phi$22 \\
    8     & EJE 13.B-E & 11    & X     & 30    & 25    & 345   & 230   & 409.58 & 4.47  & 1.19  & 16.45 & \cellcolor[rgb]{ .949,  .949,  .949}$\phi$8@16 & \cellcolor[rgb]{ .949,  .949,  .949}$\phi$8@16 & \cellcolor[rgb]{ .949,  .949,  .949}2$\phi$22 \\
    15    & EJE 13.B-E & 11    & X     & 20    & 20    & 345   & 230   & 216.55 & 4.81  & 1.38  & 12.90 & \cellcolor[rgb]{ .949,  .949,  .949}$\phi$8@20 & \cellcolor[rgb]{ .949,  .949,  .949}$\phi$8@20 & \cellcolor[rgb]{ .949,  .949,  .949}2$\phi$16 \\
    23    & EJE 13.B-E & 11    & X     & 20    & 20    & 345   & 230   & 17.46 & 6.21  & 1.30  & 2.35  & \cellcolor[rgb]{ .949,  .949,  .949}$\phi$8@20 & \cellcolor[rgb]{ .949,  .949,  .949}$\phi$8@20 & \cellcolor[rgb]{ .949,  .949,  .949}2$\phi$16 \bigstrut[b]\\
    \hline
  \end{tabular}
  }
\end{table}

\begin{table}[H]
  \centering
  \caption{Resultado armaduras muros eje G entre 3 y 6.}
  \itemresize{1.0}{
  \begin{tabular}{P{0.8cm}cP{0.9cm}P{1.5cm}P{0.8cm}P{0.8cm}P{0.8cm}P{0.8cm}P{1.3cm}P{1.1cm}P{1.1cm}P{1.1cm}P{1cm}P{1cm}P{1.3cm}}
    \hline
    \textbf{PISO} & \textbf{MURO} & \textbf{NPIER} & \textbf{EJE SISMO} & \textbf{$f'c$ (MPa)} & \textbf{e (cm)} & \textbf{L (cm)} & \textbf{H (cm)} & \textbf{N (tonf)} & \textbf{PP (tonf)} & \textbf{SC (tonf)} & \textbf{QS (tonf)} & \cellcolor[rgb]{ .949,  .949,  .949}\textbf{M.H} & \cellcolor[rgb]{ .949,  .949,  .949}\textbf{M.V} & \cellcolor[rgb]{ .949,  .949,  .949}\textbf{PUNTA} \bigstrut\\
    \hline
    -1    & EJE G.3-7 & 12    & Y     & 35    & 30    & 640   & 230   & 734.48 & 32.54 & 6.50  & 41.38 & \cellcolor[rgb]{ .949,  .949,  .949}$\phi$10@13 & \cellcolor[rgb]{ .949,  .949,  .949}$\phi$10@20 & \cellcolor[rgb]{ .949,  .949,  .949}2$\phi$25 \bigstrut[t]\\
    1     & EJE G.3-6 & 20    & Y     & 35    & 30    & 440   & 230   & 656.99 & 3.98  & 2.12  & 68.11 & \cellcolor[rgb]{ .949,  .949,  .949}$\phi$10@13 & \cellcolor[rgb]{ .949,  .949,  .949}$\phi$10@20 & \cellcolor[rgb]{ .949,  .949,  .949}2$\phi$22 \\
    3     & EJE G.3-6 & 21    & Y     & 35    & 25    & 460   & 230   & 637.07 & 28.10 & 7.15  & 77.52 & \cellcolor[rgb]{ .949,  .949,  .949}$\phi$10@13 & \cellcolor[rgb]{ .949,  .949,  .949}$\phi$8@16 & \cellcolor[rgb]{ .949,  .949,  .949}2$\phi$22 \\
    8     & EJE G.3-6 & 21    & Y     & 30    & 25    & 460   & 230   & 459.45 & 3.15  & 0.76  & 67.50 & \cellcolor[rgb]{ .949,  .949,  .949}$\phi$8@16 & \cellcolor[rgb]{ .949,  .949,  .949}$\phi$8@16 & \cellcolor[rgb]{ .949,  .949,  .949}2$\phi$22 \\
    15    & EJE G.3-6 & 21    & Y     & 20    & 20    & 460   & 230   & 282.73 & 3.45  & 0.93  & 54.00 & \cellcolor[rgb]{ .949,  .949,  .949}$\phi$8@17 & \cellcolor[rgb]{ .949,  .949,  .949}$\phi$8@20 & \cellcolor[rgb]{ .949,  .949,  .949}2$\phi$18 \\
    23    & EJE G.3-6 & 21    & Y     & 20    & 20    & 460   & 230   & 37.50 & 1.43  & 2.44  & 12.76 & \cellcolor[rgb]{ .949,  .949,  .949}$\phi$8@20 & \cellcolor[rgb]{ .949,  .949,  .949}$\phi$8@20 & \cellcolor[rgb]{ .949,  .949,  .949}2$\phi$18 \bigstrut[b]\\
    \hline
  \end{tabular}
  }
\end{table}

\begin{table}[H]
  \centering
  \caption{Resultado armaduras muros eje I.}
  \itemresize{1.0}{
  \begin{tabular}{P{0.8cm}cP{0.9cm}P{1.5cm}P{0.8cm}P{0.8cm}P{0.8cm}P{0.8cm}P{1.3cm}P{1.1cm}P{1.1cm}P{1.1cm}P{1cm}P{1cm}P{1.3cm}}
    \hline
    \textbf{PISO} & \textbf{MURO} & \textbf{NPIER} & \textbf{EJE SISMO} & \textbf{$f'c$ (MPa)} & \textbf{e (cm)} & \textbf{L (cm)} & \textbf{H (cm)} & \textbf{N (tonf)} & \textbf{PP (tonf)} & \textbf{SC (tonf)} & \textbf{QS (tonf)} & \cellcolor[rgb]{ .949,  .949,  .949}\textbf{M.H} & \cellcolor[rgb]{ .949,  .949,  .949}\textbf{M.V} & \cellcolor[rgb]{ .949,  .949,  .949}\textbf{PUNTA} \bigstrut\\
    \hline
    -1    & EJE I.7-9 & 23    & X     & 35    & 30    & 230   & 230   & 461.95 & 0.11  & 0.31  & 10.11 & \cellcolor[rgb]{ .949,  .949,  .949}$\phi$8@13 & \cellcolor[rgb]{ .949,  .949,  .949}$\phi$10@20 & \cellcolor[rgb]{ .949,  .949,  .949}2$\phi$16 \bigstrut[t]\\
    1     & EJE I.7-9 & 23    & Y     & 35    & 30    & 230   & 230   & 440.11 & 2.40  & 0.64  & 8.04  & \cellcolor[rgb]{ .949,  .949,  .949}$\phi$8@13 & \cellcolor[rgb]{ .949,  .949,  .949}$\phi$10@20 & \cellcolor[rgb]{ .949,  .949,  .949}2$\phi$16 \\
    3     & EJE I.7-9 & 23    & X     & 35    & 25    & 230   & 230   & 390.08 & 0.67  & 0.69  & 11.85 & \cellcolor[rgb]{ .949,  .949,  .949}$\phi$8@16 & \cellcolor[rgb]{ .949,  .949,  .949}$\phi$8@16 & \cellcolor[rgb]{ .949,  .949,  .949}2$\phi$16 \\
    8     & EJE I.7-9 & 23    & X     & 30    & 25    & 230   & 230   & 269.69 & 1.28  & 1.03  & 5.76  & \cellcolor[rgb]{ .949,  .949,  .949}$\phi$8@16 & \cellcolor[rgb]{ .949,  .949,  .949}$\phi$8@16 & \cellcolor[rgb]{ .949,  .949,  .949}2$\phi$16 \\
    15    & EJE I.7-9 & 23    & Y     & 20    & 20    & 230   & 230   & 150.19 & 1.30  & 1.03  & 5.03  & \cellcolor[rgb]{ .949,  .949,  .949}$\phi$8@20 & \cellcolor[rgb]{ .949,  .949,  .949}$\phi$8@20 & \cellcolor[rgb]{ .949,  .949,  .949}2$\phi$16 \\
    23    & EJE I.7-9 & 23    & Y     & 20    & 20    & 230   & 230   & 45.16 & 4.05  & 1.98  & 5.37  & \cellcolor[rgb]{ .949,  .949,  .949}$\phi$8@20 & \cellcolor[rgb]{ .949,  .949,  .949}$\phi$8@20 & \cellcolor[rgb]{ .949,  .949,  .949}2$\phi$16 \bigstrut[b]\\
    \hline
  \end{tabular}
  }
  \label{tablamuros5}
\end{table}

A continuación se ilustran los diagramas de interacción obtenidos para cada uno de los muros considerando los ditintos $f'c$ de cada piso. Se puede observar que con la mínima cuantía $\rho=1$\% se logra obtener una envolvente para todos los puntos (P,M).

\begin{images}{Diagramas de interacción P-M obtenidos.}
    \addimage{pmfc20}{width=9.8cm}{Diagrama de interacción P-M, $f'c=20$MPa, cuantía vertical $\rho_w$=0.0025.}
    \addimage{pmfc30}{width=9.8cm}{Diagrama de interacción P-M, $f'c=30$MPa, cuantía vertical $\rho_w$=0.0025.}
    \addimage{pmfc35}{width=9.8cm}{Diagrama de interacción P-M, $f'c=35$MPa, cuantía vertical $\rho_w$=0.0025.}
\end{images}

\subsection{Resistencias a compresión de cada muro}

\begin{table}[H]
  \centering
  \caption{Resistencias a compresión de los muros, representado como un factor de utilización para las cargas $N$ solicitantes.}
  \itemresize{0.6}{
  \begin{tabular}{ccccc}
    \hline
    \textbf{EJE} & \textbf{PISO} & \textbf{MURO} & \textbf{NPIER} & \textbf{FU COMPRESIÓN (\%)} \bigstrut\\
    \hline
    \multicolumn{1}{c}{\multirow{7}[2]{*}{\begin{sideways}EJE 6\end{sideways}}} & -1    & EJE 6.C-D & 1     & 88\% \bigstrut[t]\\
          & -1    & EJE 6.C-E & 2     & 60\% \\
          & 1     & EJE 6.C-G & 3     & 54\% \\
          & 3     & EJE 6.B-E & 5     & 49\% \\
          & 8     & EJE 6.B-E & 5     & 45\% \\
          & 15    & EJE 6.B-E & 5     & 45\% \\
          & 23    & EJE 6.B-E & 5     & 4\% \bigstrut[b]\\
    \hline
    \multicolumn{1}{c}{\multirow{6}[2]{*}{\begin{sideways}EJE 11\end{sideways}}} & -1    & EJE 11.C-E & 6     & 71\% \bigstrut[t]\\
          & 1     & EJE 11.C-E & 6     & 71\% \\
          & 3     & EJE 11.B-E & 8     & 50\% \\
          & 8     & EJE 11.B-E & 8     & 47\% \\
          & 15    & EJE 11.B-E & 8     & 47\% \\
          & 23    & EJE 11.B-E & 8     & 3\% \bigstrut[b]\\
    \hline
    \multicolumn{1}{c}{\multirow{6}[2]{*}{\begin{sideways}EJE 13\end{sideways}}} & -1    & EJE 13.H-L & 9     & 43\% \bigstrut[t]\\
          & 1     & EJE 13.H-K & 10    & 57\% \\
          & 3     & EJE 13.B-E & 11    & 52\% \\
          & 8     & EJE 13.B-E & 11    & 44\% \\
          & 15    & EJE 13.B-E & 11    & 44\% \\
          & 23    & EJE 13.B-E & 11    & 4\% \bigstrut[b]\\
    \hline
    \multicolumn{1}{c}{\multirow{6}[2]{*}{\begin{sideways}EJE C\end{sideways}}} & -1    & EJE G.3-7 & 12    & 31\% \bigstrut[t]\\
          & 1     & EJE G.3-6 & 20    & 40\% \\
          & 3     & EJE G.3-6 & 21    & 44\% \\
          & 8     & EJE G.3-6 & 21    & 37\% \\
          & 15    & EJE G.3-6 & 21    & 43\% \\
          & 23    & EJE G.3-6 & 21    & 6\% \bigstrut[b]\\
    \hline
    \multicolumn{1}{c}{\multirow{6}[2]{*}{\begin{sideways}EJE I\end{sideways}}} & -1    & EJE I.7-9 & 23    & 54\% \bigstrut[t]\\
          & 1     & EJE I.7-9 & 23    & 51\% \\
          & 3     & EJE I.7-9 & 23    & 54\% \\
          & 8     & EJE I.7-9 & 23    & 44\% \\
          & 15    & EJE I.7-9 & 23    & 46\% \\
          & 23    & EJE I.7-9 & 23    & 14\% \bigstrut[b]\\
    \hline
  \end{tabular}
  }
  \label{tab:rescompresion}
\end{table}

\subsection{Esquemas de armaduras de muro}

A continuación se ilustran los esquemas de armaduras de muros realizados en AUTOCAD. Cabe destacar que, en los encuentros de dos muros, se utilizó una armadura de punta de 2+2 fierros, usando para ello la armadura mínima obtenida en las Tablas \ref{tablamuros1} a \ref{tablamuros5}; esto se realizó por un tema constructivo, para el amarre de las mallas. Los muros en cuestión que presentan esta característica corresponden a los siguientes:

\begin{itemize}
    \item \textit{Eje 6}: Muro piso -1 eje E, Piso 1 eje G, Pisos 3 al 23 eje E.
    \item \textit{Eje 11}: Muro piso -1 eje E, Piso 1 eje G, Pisos 3 al 8 eje E.
    \item \textit{Eje 13}: No hay encuentro de muros.
    \item \textit{Eje G}: Muro piso -1 eje 7, Piso 1 eje 6, Piso 3 al 23 eje 3.
    \item \textit{Eje I}: Todos los muros en el eje 9.
\end{itemize}

\insertimageboxed[]{arm6}{width=0.99\linewidth}{0.5}{Muros eje 6.}
\insertimageboxed[]{arm11}{width=0.99\linewidth}{0.5}{Muros eje 11.}
\insertimageboxed[]{arm13}{width=0.99\linewidth}{0.5}{Muros eje 13.}
\insertimageboxed[]{armg}{width=0.99\linewidth}{0.5}{Muros eje G entre 3 y 6.}
\insertimageboxed[]{armi}{width=0.99\linewidth}{0.5}{Muros eje I.}
\newpage

\section{Diseño de vigas sísmica}

    Las vigas sísmicas a diseñar corresponden a las ubicadas en los ejes C (entre 11 y 12) y K (entre 10 y 11), en el nivel 2, estas son del tipo \texttt{VI25/168G35} (en adelante Viga C) y \texttt{VI25/180G35} (en adelante viga K), y de largos \texttt{90 cm} y \texttt{184 cm}, respectivamente.
    
    \insertimage[\label{vigas-sismicas}]{vigas-sismicas}{width=0.8\textwidth}{Vigas sísmicas a diseñar.}
    
    \subsection{Armadura mínima}
    
        Las armaduras mínimas, tanto a flexión como a corte, se determinan según lo indicado en la ACI318:
        
        \insertequationcaptioned[\label{a-min-flexion}]{A_s >\frac{\sqrt{f_c}}{4 f_y} > \frac{1,4 \cdot b \cdot d}{f_y}}{Armadura mínima a flexión.}
        
        \insertequationcaptioned[\label{a-min-corte}]{A_s >\frac{b \cdot s}{3 f_y}}{Armadura mínima a corte.}
        
        Con esto se obtienen las siguientes áreas mínimas de refuerzo requeridas:
        
        \begin{table}[H]
          \centering
          \caption{Áreas mínimas de acero de refuerzo para vigas sísmicas.}
            \begin{tabular}{cccc}
            \cline{2-4} & Flexión & Corte & $\rho_{min}$ \bigstrut\\
            \hline
            Viga C & 4,56  & 1,58  & 0,003 \bigstrut[t]\\
            Viga K & 4,9   & 1,68  & 0,003 \bigstrut[b]\\
            \hline
            \end{tabular}%
          \label{area-min-sismica}%
        \end{table}%
        
    \subsection{Esfuerzos de diseño}

        Los esfuerzos de diseño se determinan a través del modelo en ETABS, los resultados se muestran a continuación:
        
        
        \begin{images}[\label{imagenmultiple}]{Cargas aplicadas a Viga C.}
            \addimage{viga-sismica-PP-C}{width=7.5cm}{Esfuerzos de corte y flexión producto de peso propio.}
            \addimage{viga-sismica-SC-C}{width=7.5cm}{Esfuerzos de corte y flexión producto de sobrecarga.}
            \addimage{viga-sismica-sismo-C}{width=7.5cm}{Esfuerzos de corte y flexión producto de cargas de sismo.}
            \addimage{viga-sismica-env-C}{width=7.5cm}{Envolvente de esfuerzos sobre viga sísmica, según combinación LRFD.}
        \end{images}
        
        En la tabla \ref{esfuerzos-viga-C} se muestra el resumen del corte y momento últimos a los que está sometida la Viga C, que serán los que determinan su diseño.
        
        \begin{table}[H]
          \centering
          \caption{Esfuerzos de diseño Viga C.}
            \begin{tabular}{ccccc}
            \hline
            \textbf{Esfuerzo} & \textbf{PP} & \textbf{SC} & \textbf{Sismo} & \boldmath{}\textbf{$1,2  PP + SC+1,4  Sismo$}\unboldmath{} \bigstrut\\
            \hline
            $M [tonf \cdot m] $ & 1,4   & -0,2  & 14,44 & \textbf{21,69} \bigstrut[t]\\
            $V [tonf]$ & 3,26  & 0,17 & 18,20 & \textbf{29,57} \bigstrut[b]\\
            \hline
            \end{tabular}%
          \label{esfuerzos-viga-C}%
        \end{table}%
        
        En análisis para la Viga K es análogo. Se obtienen los siguientes resultados:
        
        \begin{table}[H]
          \centering
          \caption{Esfuerzos de diseño Viga K.}
            \begin{tabular}{ccccc}
            \hline
            \textbf{Esfuerzo} & \textbf{PP} & \textbf{SC} & \textbf{Sismo} & \boldmath{}\textbf{$1,2  PP +  SC+1,4  Sismo$}\unboldmath{} \bigstrut\\
            \hline
            $M [tonf \cdot m] $ & 10,67 & 2,95 & 5,28  & \textbf{23,15} \bigstrut[t]\\
            $V [tonf]$ & 6,70 & 1,54 & 2,43  & \textbf{13,00} \bigstrut[b]\\
            \hline
            \end{tabular}%
          \label{esfuerzos-viga-K}%
        \end{table}%
    
    \subsection{Armadura de corte}
    
        Del análisis se obtienen los siguientes resultados de armadura de corte.
            
        \begin{table}[H]
          \centering
          \caption{Armadura de corte y resistencia al corte Viga C.}
            \begin{tabular}{cc}
            \hline
            \textbf{Estribos} & \boldmath{}\textbf{$\phi Vn [tonf]$}\unboldmath{} \bigstrut\\
            \hline
            $E \phi 12 @ 80$ & 45,09 \bigstrut\\
            \hline
            \end{tabular}%
          \label{arm.corte.c}%
        \end{table}%
        
        \begin{table}[H]
          \centering
          \caption{Armadura de corte y resistencia al corte Viga K.}
            \begin{tabular}{cc}
            \hline
            \textbf{Estribos} & \boldmath{}\textbf{$\phi \cdot Vn [tonf]$}\unboldmath{} \bigstrut\\
            \hline
            $E \phi 12 @ 20$ & 45,09 \bigstrut\\
            \hline
            \end{tabular}%
          \label{arm.corte.k}%
        \end{table}%
        
        
    
    \subsection{Armadura de flexión}
    
        Del análisis se obtienen los siguientes resultados de armadura de flexión.
        
        \begin{table}[H]
          \centering
          \caption{Armadura de flexión y resistencia a flexión Viga C.}
            \begin{tabular}{ccccc}
            \hline
            \boldmath{}\textbf{$\rho_{req}$}\unboldmath{} & \boldmath{}\textbf{$A_{req} [cm^2]$}\unboldmath{} & \textbf{F} & \textbf{F'} & \boldmath{}\textbf{$\phi Mn [tonf \cdot m]$}\unboldmath{} \bigstrut\\
            \hline
            0,003 & 16,66 & $3 \phi 25$ & $3 \phi 25$ & 88,97 \bigstrut\\
            \hline
            \end{tabular}%
          \label{arm.flexion.c}%
    \end{table}%
    
    \begin{table}[H]
      \centering
      \caption{Armadura de flexión y resistencia a flexión Viga K.}
        \begin{tabular}{ccccc}
        \hline
        \boldmath{}\textbf{$\rho_{req}$}\unboldmath{} & \boldmath{}\textbf{$A_{req} [cm^2]$}\unboldmath{} & \textbf{F} & \textbf{F'} & \boldmath{}\textbf{$\phi Mn [tonf \cdot m]$}\unboldmath{} \bigstrut\\
        \hline
        0,003 & 14,66 & $3 \phi 25$ & $3 \phi 25$ & 95,65 \bigstrut\\
        \hline
        \end{tabular}%
      \label{arm.flexion.k}%
    \end{table}%
\newpage
\section{Diseño de viga estática}

    La viga estática a diseñar corresponde a la ubicada entre los ejes A y B, en el nivel -1, esta es del tipo \texttt{V30/50G35}, de largo total \texttt{35,51 m}, con apoyos intermedios como se esquematiza en la figura \ref{diagrama-viga-estatica}, correspondientes a muros.
    
    \insertimage[\label{diagrama-viga-estatica}]{diagrama-viga-estatica}{width=\textwidth}{Diagrama viga estática a diseñar.}
    
    \subsection{Armadura mínima}
        Las armaduras mínimas se calculan nuevamente con las expresiones \eqref{a-min-flexion} y \eqref{a-min-corte}, obteniendose:
        
        \begin{table}[H]
          \centering
          \caption{Armadura mínima para viga estática.}
            \begin{tabular}{ccc}
            \cline{2-3}          & Flexión & Corte \bigstrut\\
            \hline
            Viga estática & 1,537 & 0,476 \bigstrut\\
            \hline
            \end{tabular}%
          \label{area-minima-estatica}%
        \end{table}%
        
    \subsection{Esfuerzos de diseño.}
        
        Las cargas a considerar se obtienen de tributar las sobrecargas y peso propio de la losa en el área tributaria correspondiente a la viga en cuestión. En la figura \ref{tributacion} se ejemplifica la forma en que se tributó el área mostrando una de las secciones consideradas.
        
        \insertimage[\label{tributacion}]{tributacion}{width=0.3\textwidth}{Esquema de tribtación de cargas a viga estática.}
        
        Estas solictaciones se aplican como cargas distribuidas de forma trapezoidal sobre el modelo de la viga, donde además se considera su peso propio. En las figuras \ref{PP-viga-estatica} y \ref{SC-viga-estatica} se muestran las cargas aplicadas en el modelo.
        
        \insertimage[\label{PP-viga-estatica}]{PP-viga-estatica}{width=\textwidth}{Cargas de peso propio aplicadas al modelo de la viga. \textit{SAP2000.}}
        
        \insertimage[\label{SC-viga-estatica}]{SC-viga-estatica}{width=\textwidth}{Sobrecargas aplicadas al modelo de la viga. \textit{SAP2000.}}
        
    \subsection{Diagramas de momento y corte}
        
        Los esfuerzos de diseño, de corte y flexión, se obtienen modelando la viga en SAP2000. Se considera la combinación $1,2PP+1,6SC$.  Se muestran los resultados en las figuras \ref{momentos-viga-estatica} y \ref{corte-viga-estatica}.
        
        \insertimage[\label{momentos-viga-estatica}]{momentos-viga-estatica}{width=\textwidth}{Flexión resultante en viga estática. \textit{SAP2000.}}        
        
        \insertimage[\label{corte-viga-estatica}]{corte-viga-estatica}{width=\textwidth}{Esfuerzo de corte resultante en viga estática. \textit{SAP2000.}} 
        
        Los esfuerzos considerados en el diseño del refuerzo de la viga se muestran en la siguiente tabla:
        
        \begin{table}[H]
          \centering
          \caption{Esfuerzos de diseño viga estática.}
            \begin{tabular}{cc}
            \hline
            \textbf{Esfuerzo} & \boldmath{}\textbf{$1,2 PP+1,6 SC$}\unboldmath{} \bigstrut\\
            \hline
            \multirow{2}[1]{*}{$M [tonf \cdot m] $} & 22,08 \bigstrut[t]\\
                  & \textbf{-30,92} \\
                \hline
            $V [tonf]$ & \textbf{22,36} \bigstrut[b]\\
            \hline
            \end{tabular}%
          \label{esf-diseno-estatica}%
        \end{table}%
        
    \subsection{Armadura de Corte.}
    
        Del análisis se obtienen los siguientes resultados de armadura de corte.
        
        \begin{table}[H]
          \centering
          \caption{Armadura de corte y resistencia al corte Viga estática.}
            \begin{tabular}{cc}
            \hline
            \textbf{Estribos} & \boldmath{}\textbf{$\phi \cdot Vn [tonf]$}\unboldmath{} \bigstrut\\
            \hline
            $E \phi 12 @ 20$ & 26,65 \bigstrut\\
            \hline
            \end{tabular}%
          \label{arm-corte-estatica}%
        \end{table}%
    
    \subsection{Armadura de Flexión.}
    
        Del análisis se obtienen los siguientes resultados de armadura de flexión negativa y positiva.
        
        \begin{table}[htbp]
          \centering
          \caption{Armadura de flexión positiva y resistencia a la flexión positiva Viga estática.}
            \begin{tabular}{ccc}
            \hline
            \boldmath{}\textbf{$A_{req} [cm^2]$}\unboldmath{} & \textbf{F} & \boldmath{}\textbf{$\phi Mn [tonf \cdot m]$}\unboldmath{} \bigstrut\\
            \hline
            13,65 & $3 \phi 25$ & 23,67 \bigstrut\\
            \hline
            \end{tabular}%
          \label{flex+estatica}%
        \end{table}%
        
        \begin{table}[htbp]
          \centering
          \caption{Armadura de flexión negativa y resistencia a la flexión negativa Viga estática.}
            \begin{tabular}{ccc}
            \hline
            \boldmath{}\textbf{$A'_{req} [cm^2]$}\unboldmath{} & \textbf{F'} & \boldmath{}\textbf{$\phi Mn [tonf \cdot m]$}\unboldmath{} \bigstrut\\
            \hline
            19,78 & $4 \phi 28$ & 37,43 \bigstrut\\
            \hline
            \end{tabular}%
          \label{flex-estatica}%
        \end{table}%

    \subsection{Esqema de armadura de corte y flexión}
    
    En las figuras siguinentes se esquematiza la disposición de su armaduras de corte y flexión en la viga estática.
    
    \begin{images}[\label{armadura}]{Esquemas de disposición de armadura en viga estática.}
        \addimage{armadura-viga-estatica}{width=\textwidth}{Esquema longitudinal de uno de los vanos.}
        \addimage{trans-viga-estatica}{width=4cm}{Esquema transversal.}
    \end{images}    
\newpage

\section{Comentarios}

\begin{itemize}
    \item %alguna wea de los muros
    En el diseño de muros se obtuvo, por lo general, resistencias al corte superiores a las solicitantes, por lo que se escogieron armaduras con tal de cumplir el mínimo. En cuanto a la compresión y esbeltez todos los muros resistieron de manera correcta las cargas (Tabla \ref{tab:rescompresion}), sin embargo para los pisos superiores la resistencia a compresión es muy grande (factores de utilización del orden del 5\%).
    
    \item En cuanto a la resistencia a flexo-compresión todos los muros escogidos cumplieron con la armadura mínima, con lo cual se escogieron armaduras de acuerdo al espesor de cada muro.
    
    \item En vigas sísmicas se obtuvo resistencias, tanto a flexión como a corte, muy superior a las requerida, entre $50\%$ y $300\%$ de resistencia adicional. Esto debido a los requerimientos de cuantías mínimas de acero de refuerzo a flexión en ambas vigas. Esto sumado a que las sección de armadura escogida excede al área requerida, que se acentúa por la gran distancia, $d$, de esta al borde en compresión, resultó en secciones con elevadas resistencias a la flexión.\\ 
    
    Por otro lado, para el caso del corte en la viga K, al tener secciones tan grandes se da el caso en que el hormigón por si solo es capaz de resistir más del doble del esfuerzo de corte, por lo que no requiriría refuerzo adicional según la verificación establecida en la ACI, sin embargo se considera de todas formas la armadura mínima. 
    Si bien la sección de hormigón de la viga C es de dimensiones similares a la viga K, el esfuerzo de corte al que se ve sometida es mayor, por lo que sí requiere armadura de corte, sin embargo esta es la mínima.
    
    \item En la viga estática el excedente de resistencia fue menor, debido a su menor sección, con la que requería una cuantía mayor a la mínima para resistir los esfuerzos de flexión. Con eso se obtuvo entre un $7\%$ y $21\%$ de excedente de resistencia. Similar a lo ocurrido con la resistencia al corte, donde el excedente fue de $19\%$.
\end{itemize}

% FIN DEL DOCUMENTO
\end{document}
