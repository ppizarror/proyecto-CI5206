\newpage

\section{Comentarios}

\begin{itemize}
    \item %alguna wea de los muros
    En el diseño de muros se obtuvo, por lo general, resistencias al corte superiores a las solicitantes, por lo que se escogieron armaduras con tal de cumplir el mínimo. En cuanto a la compresión y esbeltez todos los muros resistieron de manera correcta las cargas (Tabla \ref{tab:rescompresion}), sin embargo para los pisos superiores la resistencia a compresión es muy grande (factores de utilización del orden del 5\%).
    
    \item En cuanto a la resistencia a flexo-compresión todos los muros escogidos cumplieron con la armadura mínima, con lo cual se escogieron armaduras de acuerdo al espesor de cada muro.
    
    \item En vigas sísmicas se obtuvo resistencias, tanto a flexión como a corte, muy superior a las requerida, entre $50\%$ y $300\%$ de resistencia adicional. Esto debido a los requerimientos de cuantías mínimas de acero de refuerzo a flexión en ambas vigas. Esto sumado a que las sección de armadura escogida excede al área requerida, que se acentúa por la gran distancia, $d$, de esta al borde en compresión, resultó en secciones con elevadas resistencias a la flexión.\\ 
    
    Por otro lado, para el caso del corte en la viga K, al tener secciones tan grandes se da el caso en que el hormigón por si solo es capaz de resistir más del doble del esfuerzo de corte, por lo que no requiriría refuerzo adicional según la verificación establecida en la ACI, sin embargo se considera de todas formas la armadura mínima. 
    Si bien la sección de hormigón de la viga C es de dimensiones similares a la viga K, el esfuerzo de corte al que se ve sometida es mayor, por lo que sí requiere armadura de corte, sin embargo esta es la mínima.
    
    \item En la viga estática el excedente de resistencia fue menor, debido a su menor sección, con la que requería una cuantía mayor a la mínima para resistir los esfuerzos de flexión. Con eso se obtuvo entre un $7\%$ y $21\%$ de excedente de resistencia. Similar a lo ocurrido con la resistencia al corte, donde el excedente fue de $19\%$.
\end{itemize}