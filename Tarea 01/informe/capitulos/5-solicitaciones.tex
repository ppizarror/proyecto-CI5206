\newpage
\section{Solicitaciones}

\subsection{Cargas muertas}

Además de considerarse como cargas muertas el peso de las estructuras, techumbres, pisos, muros y paneles, plataformas, equipo permanente, materiales normalmente almacenados, etc., las presiones laterales y verticales de líquidos, gases y materiales fluidos (granulares o similares) serán también tratadas como cargas muertas.
    
Se consideran como cargas muertas los siguientes conjuntos:

\begin{enumerate}[label=\alph*) ]
    \item Peso del Edificio que considera los elementos estructurales pesados: vigas, losas, techumbre, columnas, muros, tabiques, plataformas, etc., y los no estructurales pesados: peso de las terminaciones (cielos falsos, estuco, terminaciones de piso, etc.).
    
    \item Carga suspendida. Todas las cargas colgantes permanentes tales como puentes de cañerías, bandejas de soportes de cables eléctricos, luminarias, etc.

    \item Cargas fijas. Todo el equipo menor que no se muestra en el diagrama de cargas, tales como pequeñas bombas, motores, agitadores, etc., incluyendo el cojinete de hormigón.

    \item Cargas introducidas por el equipo. Estas cargas son proporcionadas por los proveedores o indicadas en el Diagrama de Cargas. (Peso del equipo, carga de operación, carga de exceso - overflow).

    \item Empuje de tierras sobre muros de sostenimiento. Presiones laterales y verticales de líquidos, gases y materiales capaces de fluir. Materiales almacenados.

    % \item La acumulación de polvo en la estructura será considerada como carga permanente, con un peso específico de 1400 kg/m3 y un talud de 1:1,5 (V:H).
\end{enumerate}

\subsection{Cargas de uso}

Como carga de uso se considera lo establecido en la NCh 1537. A continuación se listan las cargas aplicables al caso:

\begin{table}[H]
  \centering
  \caption{Cargas de uso aplicables al edificio}
    \itemresize{1.0}{
        \begin{tabular}{|l|l|c|}
        \hline
        \multicolumn{1}{|c|}{\textbf{Tipo de edificio}} & \multicolumn{1}{c|}{\textbf{Descripción de uso}} & \multicolumn{1}{p{5.355em}|}{\textbf{Carga de uso [kPa]}} \bigstrut\\
        \hline
        Estacionamiento & Vehículos livianos & 3 \bigstrut\\
        \hline
        \multirow{4}[2]{*}{Viviendas} & Áreas de uso general & 2 \bigstrut[t]\\
              & Dormitorios y buhardillas habitables & 2 \\
              & Balcones que no excedan $10m^2$ & 3 \\
              & Entretecho con almacenaje & 1,5 \bigstrut[b]\\
        \hline
        \multicolumn{1}{|l|}{\multirow{3}[2]{*}{Lugares especiales de uso público}} & Corredores/lugares de uso público & 5 \bigstrut[t]\\
              & Balcones exteriores & 5 \\
              & Escaleras y vías de evacuación & 5 \bigstrut[b]\\
        \hline
        Techos & Con acceso peatonal (uso público) & 5 \bigstrut\\
        \hline
        \end{tabular}%
    }
  \label{carga-uso}%
\end{table}%

\subsection{Carga de nieve}

No se considera carga de nieve, debido a la ubicación geográfica del edificio.

\subsection{Carga de viento}

Para el cálculo de fuerzas debidas a la acción del viento se aplicará lo especificado en la norma NCh 432.
Para ello se suponen los casos W+ y W-, con viento en sentido positivo y negativo de la dirección X, respectivamente, que se aplicarán de forma independiente en el modelo. 

\insertimage[\label{viento}]{bases-de-calculo/geometriaviento}{width=11cm}{Coeficiente de forma a considerar para cargas de viento.}