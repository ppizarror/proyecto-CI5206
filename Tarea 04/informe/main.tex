% Template:     Informe/Reporte LaTeX
% Documento:    Archivo principal
% Versión:      6.0.6 (30/10/2018)
% Codificación: UTF-8
%
% Autor: Pablo Pizarro R. @ppizarror
%        Facultad de Ciencias Físicas y Matemáticas
%        Universidad de Chile
%        pablo.pizarro@ing.uchile.cl, ppizarror.com
%
% Manual template: [https://latex.ppizarror.com/Template-Informe/]
% Licencia MIT:    [https://opensource.org/licenses/MIT/]

% CREACIÓN DEL DOCUMENTO
\documentclass[letterpaper,11pt]{article} % Articulo tamaño carta, 11pt
\usepackage[utf8]{inputenc} % Codificación UTF-8

% INFORMACIÓN DEL DOCUMENTO
\def\titulodelinforme {Modelo sísmico}
\def\temaatratar {Proyecto de Hormigón Armado - Entrega N°4}

\def\autordeldocumento {Grupo N1}
\def\nombredelcurso {Proyecto de Hormigón Armado}
\def\codigodelcurso {CI5206-2}

\def\nombreuniversidad {Universidad de Chile}
\def\nombrefacultad {Facultad de Ciencias Físicas y Matemáticas}
\def\departamentouniversidad {Departamento de Ingeniería Civil}
\def\imagendepartamento {dic}
\def\imagendepartamentoescala {0.2}
\def\localizacionuniversidad {Santiago, Chile}

% INTEGRANTES, PROFESORES Y FECHAS
\def\tablaintegrantes {
\begin{tabular}{ll}
	Integrantes:
	& \begin{tabular}[t]{@{}l@{}}
		Mauricio Leal V. \\
		Pablo Pizarro R. \\
		Ignacio Yáñez G.
	\end{tabular} \\
	Profesor:
	& \begin{tabular}[t]{@{}l@{}}
		Juan Mendoza V.
	\end{tabular} \\
	Auxiliar:
	& \begin{tabular}[t]{@{}l@{}}
		Felipe Andrade T.
	\end{tabular} \\
	& \\
	\multicolumn{2}{l}{Fecha de entrega: 07 de Noviembre de 2018} \\
	\multicolumn{2}{l}{\localizacionuniversidad}
\end{tabular}}{
}

% CONFIGURACIONES
\input{lib/config}

% IMPORTACIÓN DE LIBRERÍAS
\input{lib/env/imports}

% IMPORTACIÓN DE FUNCIONES Y ENTORNOS
\input{lib/cmd/all}

% IMPORTACIÓN DE ESTILOS
\input{lib/style/all}

% CONFIGURACIÓN INICIAL DEL DOCUMENTO
\input{lib/cfg/init}

% INICIO DE LAS PÁGINAS
\begin{document}

% PORTADA
\input{lib/page/portrait} % Se puede borrar

% CONFIGURACIÓN DE PÁGINA Y ENCABEZADOS
\input{lib/cfg/page}

% TABLA DE CONTENIDOS - ÍNDICE
% Template:     Informe/Reporte LaTeX
% Documento:    Índice
% Versión:      6.0.1 (21/10/2018)
% Codificación: UTF-8
%
% Autor: Pablo Pizarro R. @ppizarror
%        Facultad de Ciencias Físicas y Matemáticas
%        Universidad de Chile
%        pablo.pizarro@ing.uchile.cl, ppizarror.com
%
% Manual template: [http://latex.ppizarror.com/Template-Informe/]
% Licencia MIT:    [https://opensource.org/licenses/MIT/]

\ifthenelse{\equal{\showindex}{true}}{
	\newpage
	\begingroup
	\sectionfont{\color{\indextitlecolor} \fontsizetitlei \styletitlei \selectfont}
	\ifthenelse{\equal{\addindextobookmarks}{true}}{
		\belowpdfbookmark{\nomltcont}{contents}}{
	}
	\tocloftpagestyle{fancy}
	\ifthenelse{\equal{\showdotontitles}{true}}{
		\def\cftsecaftersnum {.}
		\def\cftsubsecaftersnum {.}
		\def\cftsubsubsecaftersnum {.}
		\def\cftsubsubsubsecaftersnum {.}
		}{
	}
	\def\cftfigaftersnum {\charafterobjectindex\enspace}
	\def\cftsubfigaftersnum {\charafterobjectindex\enspace}
	\def\cfttabaftersnum {\charafterobjectindex\enspace}
	\def\cftlstlistingaftersnum {\charafterobjectindex\enspace}
	\renewcommand{\cftdot}{\charnumpageindex}
	\ifthenelse{\equal{\showlinenumbers}{true}}{
		\nolinenumbers}{
	}
	\ifthenelse{\equal{\objectindexindent}{true}}{
		\def\cftlstlistingindent {1.495em}
	}{
		\setlength{\cfttabindent}{0in}
		\setlength{\cftfigindent}{0in}
		\setlength{\cftsubfigindent}{0in}
		\setlength{\cftfigindent}{0in}
		\def\cftlstlistingindent {0.01em}
	}
	\ifthenelse{\equal{\equalmarginnumobject}{true}}{
		\ifthenelse{\equal{\showsectioncaption}{none}}{
			\def\cftdefautnumwidth {2.3em}
		}{
		\ifthenelse{\equal{\showsectioncaption}{sec}}{
			\def\cftdefautnumwidth {3.0em}
		}{
		\ifthenelse{\equal{\showsectioncaption}{ssec}}{
			\def\cftdefautnumwidth {3.8em}
		}{
		\ifthenelse{\equal{\showsectioncaption}{sssec}}{
			\def\cftdefautnumwidth {4.3em}
		}{
			\throwbadconfig{Valor configuracion incorrecto}{\showsectioncaption}{none,sec,ssec,sssec}}}}
		}
		\def\cftfignumwidth {\cftdefautnumwidth}
		\def\cftsubfignumwidth {\cftdefautnumwidth}
		\def\cfttabnumwidth {\cftdefautnumwidth}
		\def\cftlstlistingnumwidth {\cftdefautnumwidth}}{
	}
	\ifthenelse{\equal{\showindexofcontents}{true}}{\tableofcontents}{}
	\iftotalfigures
		\ifthenelse{\equal{\showindexoffigures}{true}}{
			\ifthenelse{\equal{\indexforcenewpage}{true}}{\newpage}{}
			\listoffigures
		}{}
	\fi
	\iftotaltables
		\ifthenelse{\equal{\showindexoftables}{true}}{
			\ifthenelse{\equal{\indexforcenewpage}{true}}{\newpage}{}
			\listoftables
		}{}
	\fi
	\iftotallstlistings
		\ifthenelse{\equal{\showindexofcode}{true}}{
			\ifthenelse{\equal{\indexforcenewpage}{true}}{\newpage}{}
			\lstlistoflistings
		}{}
	\fi
	\endgroup
	\ifthenelse{\equal{\addemptypagetwosides}{true}}{
		\vfill
		\checkoddpage
		\ifoddpage
		\else
			\newpage
			\null
			\thispagestyle{empty}
			\newpage
			\addtocounter{page}{-1}
		\fi}{
	}
}{}
 % Se puede borrar

% CONFIGURACIONES FINALES
% Template:     Informe/Reporte LaTeX
% Documento:    Configuraciones finales
% Versión:      6.1.6 (14/12/2018)
% Codificación: UTF-8
%
% Autor: Pablo Pizarro R. @ppizarror
%        Facultad de Ciencias Físicas y Matemáticas
%        Universidad de Chile
%        pablo.pizarro@ing.uchile.cl, ppizarror.com
%
% Manual template: [https://latex.ppizarror.com/Template-Informe/]
% Licencia MIT:    [https://opensource.org/licenses/MIT/]

\markboth{}{}
\newpage
\ifthenelse{\equal{\disablehfrightmark}{false}}{
	\ifthenelse{\equal{\hfstyle}{style1}}{
		\fancyhead[L]{\nouppercase{\leftmark}}}{
	}
	\ifthenelse{\equal{\hfstyle}{style2}}{
		\fancyhead[L]{\nouppercase{\leftmark}}}{
	}
	\ifthenelse{\equal{\hfstyle}{style4}}{
		\fancyhead[L]{\nouppercase{\leftmark}}}{
	}
	\ifthenelse{\equal{\hfstyle}{style5}}{
		\fancyhead[R]{\nouppercase{\leftmark}}}{
	}
	\ifthenelse{\equal{\hfstyle}{style9}}{
		\fancyhead[L]{\nouppercase{\leftmark}}}{
	}
	\ifthenelse{\equal{\hfstyle}{style10}}{
		\fancyhead[L]{\nouppercase{\leftmark}}}{
	}
\ifthenelse{\equal{\hfstyle}{style11}}{
		\fancyhead[L]{\nouppercase{\leftmark}}}{
	}
\ifthenelse{\equal{\hfstyle}{style14}}{
		\fancyhead[L]{\nouppercase{\leftmark}}}{
	}}{
}
\sectionfont{\color{\titlecolor} \fontsizetitle \styletitle \selectfont}
\subsectionfont{\color{\subtitlecolor} \fontsizesubtitle \stylesubtitle \selectfont}
\subsubsectionfont{\color{\subsubtitlecolor} \fontsizesubsubtitle \stylesubsubtitle \selectfont}
\titleformat{\subsubsubsection}{\color{\ssstitlecolor} \normalfont \fontsizessstitle \stylessstitle}{\thesubsubsubsection}{1em}{}
\titlespacing*{\subsubsubsection}{0pt}{3.25ex plus 1ex minus .2ex}{1.5ex plus .2ex}
\ifthenelse{\equal{\showsectioncaption}{none}}{
}{
\ifthenelse{\equal{\showsectioncaption}{sec}}{
	\counterwithin{equation}{section}
	\counterwithin{figure}{section}
	\counterwithin{lstlisting}{section}
	\counterwithin{table}{section}
}{
\ifthenelse{\equal{\showsectioncaption}{ssec}}{
	\counterwithin{equation}{subsection}
	\counterwithin{figure}{subsection}
	\counterwithin{lstlisting}{subsection}
	\counterwithin{table}{subsection}
}{
\ifthenelse{\equal{\showsectioncaption}{sssec}}{
	\counterwithin{equation}{subsubsection}
	\counterwithin{figure}{subsubsection}
	\counterwithin{lstlisting}{subsubsection}
	\counterwithin{table}{subsubsection}
}{
\ifthenelse{\equal{\showsectioncaption}{ssssec}}{
	\counterwithin{equation}{subsubsubsection}
	\counterwithin{figure}{subsubsubsection}
	\counterwithin{lstlisting}{subsubsubsection}
	\counterwithin{table}{subsubsubsection}
}{
	\throwbadconfig{Valor configuracion incorrecto}{\showsectioncaption}{none,sec,ssec,sssec,ssssec}
}}}}}
\ifthenelse{\equal{\predocuseromannumber}{true}}{
	\renewcommand{\thepage}{\arabic{page}}}{
}
\ifthenelse{\equal{\resetpagnumafterindex}{true}}{
	\setcounter{page}{1}}{
}
\setcounter{section}{0}
\setcounter{footnote}{0}
\ifthenelse{\equal{\showlinenumbers}{true}}{
	\linenumbers}{
}


% ======================= INICIO DEL DOCUMENTO =======================

% 1 Información general edificio: zona, masa, peso, suelo, Periodos fundamentales de la estructura, aceleración. [Mauro] [OK]
\section{Información General del Edificio}
% 1 Información general edificio: zona, masa, peso, suelo, Periodos fundamentales de la estructura, aceleración. [Mauro]
    A continuación se presentan los principales parámetros usados para el análisis sísimco del edificio:
    
    \begin{table}[H]
      \centering
      \caption{Información general del edificio.}
        \begin{tabular}{cccc}
            \toprule
            \textbf{Zona} & \multicolumn{1}{c}{\textbf{Suelo}} & \textbf{Masa [tonf]} & \textbf{Peso Sísmico [Tonf]} \\
            \midrule
            3     & A     & 11434,2 & 11763,3 \\
            \bottomrule
        \end{tabular}%
      \label{info_general}%
    \end{table}%
    
    \begin{table}[H]
      \centering
      \caption{Períodos fundamentales.}
        \begin{tabular}{ccccc}
            \toprule
            \textbf{Modo} & \textbf{Tipo} & \textbf{Período (s)} & \textbf{\% De masa} & \multicolumn{1}{c}{\textbf{${S_a}_{max} [cm/s^2$]}} \\
            \midrule
                1   & Torsión z     & 1,345     & 45,52\%   & - \\
                2   & Traslación x  & 1,112     & 55,52\%   & 2,17 \\
                3   & Traslación y  & 0,744     & 56,05\%   & 3,22 \\
            \bottomrule
        \end{tabular}%
      \label{periodos}%
    \end{table}%

% 2 Espectro Nch433 para Edificio 
% 3 Factor R y R0, materiales 
\newpage

% 2 Espectro Nch433 para Edificio
% 3 Factor R y R0, materiales
\section{Parámetros de Diseño}

\subsection{Espectro de diseño NCh 433}

El espectro de diseño elástico se construyó a partir de los datos del suelo (tipo A), de la ubicación del edificio (Antofagasta, zona sísmica 3) y del uso (Residencial, categoría II). \\

La Figura \ref{fig-espectro-elastico} ilustra el espectro elástico utilizado.

\insertimage[\label{fig-espectro-elastico}]{espectro}{width=10cm}{Espectro elástico.}

Dado que el suelo de fundación de la estructura es de buena calidad (roca, tipo A) se puede considerar que no existe un contraste de impedancias alto con la roca basal, ello implica que el factor de amplificación del desplazamiento en superficie es similar a 1. Así, el espectro funciona como una envolvente de pseudoaceleraciones de los posibles sismos.

\subsection{Estructuración}

El tipo de estructuración es del tipo muro y losa de Hormigón Armado (H.A.); En este sentido el muro tiene la función de transmitir cargas tipo gravitacionales (compresión) a las fundaciones y resistir cargas cortantes, tracciones y compresiones por flexión en en caso de un sismo. Las losas por otra parte tributan las cargas al sistema de muros. \\

De acuerdo a lo estipulado en la NCh433 Tabla 5.1 y Tabla 6.1 se tienen los siguientes parámetros de modificación de acuerdo al tipo de estructuración, en donde $R$ y $R_o$ corresponden a factores de reducción e $I$ es el nivel de importancia de la estructura.

\begin{table}[H]
  \centering
  \caption{Parámetros de diseño sísmico dado la estructuración.}
    \begin{tabular}{|c|c|}
    \hline
    \textbf{Parámetro} & \textbf{Valor} \bigstrut\\
    \hline
    R     & 7 \bigstrut\\
    \hline
    $R_o$ & 11 \bigstrut\\
    \hline
    I     & 1 \bigstrut\\
    \hline
    \end{tabular}%
\end{table}%

El espesor de muros se calculó en base a una resistencia de corte admisible promedio de $\tau$=7 $kgf/cm2$, corregida por efectos de deformación sísmica considerando el espectro elástico; en la Tabla \ref{tabla-densidades-muro} se detalla la densidad de muros obtenida en cada eje.

\begin{table}[H]
  \centering
  \caption{Densidad de muros.}
  \begin{tabular}{ccccc}
    \hline
    \textbf{N° piso} & \textbf{ex [m]} & \textbf{ey [m]} & \textbf{Densidad muro x} & \textbf{Densidad muro y} \bigstrut\\
    \hline
    23    & 0.2   & 0.2   & 0.028 & 0.030 \bigstrut[t]\\
    22    & 0.2   & 0.2   & 0.028 & 0.030 \\
    21    & 0.2   & 0.2   & 0.028 & 0.030 \\
    20    & 0.2   & 0.2   & 0.028 & 0.030 \\
    19    & 0.2   & 0.2   & 0.028 & 0.030 \\
    18    & 0.2   & 0.2   & 0.028 & 0.030 \\
    17    & 0.2   & 0.2   & 0.028 & 0.030 \\
    16    & 0.2   & 0.2   & 0.028 & 0.030 \\
    15    & 0.2   & 0.2   & 0.028 & 0.030 \\
    14    & 0.2   & 0.2   & 0.034 & 0.037 \\
    13    & 0.25  & 0.25  & 0.034 & 0.037 \\
    12    & 0.25  & 0.25  & 0.034 & 0.037 \\
    11    & 0.25  & 0.25  & 0.034 & 0.037 \\
    10    & 0.25  & 0.25  & 0.034 & 0.037 \\
    9     & 0.25  & 0.25  & 0.034 & 0.037 \\
    8     & 0.25  & 0.25  & 0.034 & 0.037 \\
    7     & 0.25  & 0.25  & 0.034 & 0.037 \\
    6     & 0.25  & 0.25  & 0.034 & 0.037 \\
    5     & 0.25  & 0.25  & 0.034 & 0.037 \\
    4     & 0.25  & 0.25  & 0.034 & 0.037 \\
    3     & 0.25  & 0.25  & 0.037 & 0.037 \\
    2     & 0.25  & 0.25  & 0.041 & 0.038 \\
    1     & 0.25  & 0.3   & 0.043 & 0.041 \\
    -1    & 0.25  & 0.3   & 0.029 & 0.027 \bigstrut[b]\\
    \hline
  \end{tabular}
  \label{tabla-densidades-muro}
\end{table}

En los últimos años la densidad de muros se ha concentrado entre 0.02 y 0.035, obteniendo un buen desempeño en la estructura, por tanto, como se puede observar en la tabla anterior, en todos los pisos se obtuvo una densidad de muros superior al 0.02 y muy cercana a los 0.035 sugeridos.

\newpage
\subsection{Materiales}

\begin{itemize}
    \item \textbf{Hormigón}
    
    Se hará uso de tres tipos de hormigones para el edificio, H-20, H-30 y H-35. Esto se eligió en base a un requerimiento mínimo de resistencia a fuerzas axiales (peso), considerando la siguiente relación:
    
    \insertequationanum{N_u \geq 0.35 \cdot f'_c \cdot A_g}
    
    En donde $N_u$ corresponde a la resistencia última a compresión, $f'c$ la resistencia a compresión del hormigón y $A_g$ el área gruesa de la sección de muros resistente. Dado lo anterior se construyó la Tabla \ref{tab-g-horm-piso} para verificar el tipo de hormigón. En promedio se obtuvo un 70\% de factor de utilización.
    
\begin{table}[H]
  \centering
  \caption{Definición del tipo de hormigón para cada piso.}
  \begin{tabular}{|c|cc|c|c|c|}
    \hline
    \textbf{N° piso} & \textbf{Peso} & \textbf{Peso acum} & \boldmath{}\textbf{$f'c$}\unboldmath{} & \boldmath{}\textbf{$N_u$}\unboldmath{} & \boldmath{}\textbf{$N_u$}\unboldmath{} \bigstrut\\
    \hline
    23    & 530.45 & 530.45 & \cellcolor[rgb]{ .816,  .808,  .808}20 & 4760.3 & 11\% \bigstrut[t]\\
    22    & 435.97 & 966.41 & \cellcolor[rgb]{ .816,  .808,  .808}20 & 4760.3 & 20\% \\
    21    & 435.97 & 1402.38 & \cellcolor[rgb]{ .816,  .808,  .808}20 & 4760.3 & 29\% \\
    20    & 435.97 & 1838.35 & \cellcolor[rgb]{ .816,  .808,  .808}20 & 4760.3 & 39\% \\
    19    & 435.97 & 2274.31 & \cellcolor[rgb]{ .816,  .808,  .808}20 & 4760.3 & 48\% \\
    18    & 435.97 & 2710.28 & \cellcolor[rgb]{ .816,  .808,  .808}20 & 4760.3 & 57\% \\
    17    & 435.97 & 3146.25 & \cellcolor[rgb]{ .816,  .808,  .808}20 & 4760.3 & 66\% \\
    16    & 435.97 & 3582.21 & \cellcolor[rgb]{ .816,  .808,  .808}20 & 4760.3 & 75\% \\
    15    & 435.97 & 4018.18 & \cellcolor[rgb]{ .816,  .808,  .808}20 & 4760.3 & 84\% \\
    14    & 452.51 & 4470.69 & \cellcolor[rgb]{ .816,  .808,  .808}20 & 4760.3 & 94\% \\
    13    & 469.05 & 4939.74 & \cellcolor[rgb]{ .651,  .651,  .651}30 & 8925.5 & 55\% \\
    12    & 469.05 & 5408.79 & \cellcolor[rgb]{ .651,  .651,  .651}30 & 8925.5 & 61\% \\
    11    & 469.05 & 5877.84 & \cellcolor[rgb]{ .651,  .651,  .651}30 & 8925.5 & 66\% \\
    10    & 469.05 & 6346.89 & \cellcolor[rgb]{ .651,  .651,  .651}30 & 8925.5 & 71\% \\
    9     & 469.05 & 6815.94 & \cellcolor[rgb]{ .651,  .651,  .651}30 & 8925.5 & 76\% \\
    8     & 469.05 & 7284.99 & \cellcolor[rgb]{ .651,  .651,  .651}30 & 8925.5 & 82\% \\
    7     & 469.05 & 7754.04 & \cellcolor[rgb]{ .502,  .502,  .502}35 & 10413.1 & 74\% \\
    6     & 469.05 & 8223.09 & \cellcolor[rgb]{ .502,  .502,  .502}35 & 10413.1 & 79\% \\
    5     & 469.05 & 8692.14 & \cellcolor[rgb]{ .502,  .502,  .502}35 & 10413.1 & 83\% \\
    4     & 469.05 & 9161.19 & \cellcolor[rgb]{ .502,  .502,  .502}35 & 10413.1 & 88\% \\
    3     & 471.81 & 9633.00 & \cellcolor[rgb]{ .502,  .502,  .502}35 & 10413.1 & 93\% \\
    2     & 480.38 & 10113.38 & \cellcolor[rgb]{ .502,  .502,  .502}35 & 10765.9 & 94\% \\
    1     & 475.35 & 10588.73 & \cellcolor[rgb]{ .502,  .502,  .502}35 & 11569.5 & 92\% \\
    -1    & 670.51 & 11259.24 & \cellcolor[rgb]{ .502,  .502,  .502}35 & 12491.0 & 90\% \bigstrut[b]\\
    \hline
  \end{tabular}
  \label{tab-g-horm-piso}
\end{table}

\begin{table}[H]
  \centering
  \caption{Tensiones admisibles y características de hormigones a utilizar.}
  \begin{tabular}{cccccc}
    \hline
    \textbf{\textbf{Grado}} & \boldmath{}\textbf{\textbf{$\rho\  [tonf/m^3]$}}\unboldmath{} & \boldmath{}\textbf{\textbf{$f'c\  [Mpa]$}}\unboldmath{} & \boldmath{}\textbf{\textbf{$E\  [Mpa]$}}\unboldmath{} & \boldmath{}\textbf{\textbf{$\nu\  [-]$}}\unboldmath{} & \boldmath{}\textbf{\textbf{$G\  [Mpa]$}}\unboldmath{} \bigstrut\\
    \hline
    H20   & 2.5   & 20    & 21019.04 & 0.2   & 8757.93 \bigstrut[t]\\
    H30   & 2.5   & 30    & 25742.96 & 0.2   & 10726.23 \\
    H35   & 2.5   & 35    & 27805.57 & 0.2   & 11585.66 \bigstrut[b]\\
    \hline
  \end{tabular}
  \label{tab:addlabel}
\end{table}

\item \textbf{Acero}\\
    Se considera un acero ASTM-A36 como acero estructural y de refuerzo, el cual posee las siguientes características:
    
    \begin{table}[H]
      \centering
      \caption{Tensiones admisibles y características del acero a utilizar.}
      \resizebox{\textwidth}{!}{%
        \begin{tabular}{|c|c|c|c|c|c|}
        \hline
        \textbf{Grado} &
          \boldmath{}\textbf{$F_{y} \ [tonf/cm^2]$} &
          \boldmath{}\textbf{$F_{u} \ [tonf/cm^2]$} &
          \boldmath{}\textbf{$E \ [tonf/cm^2]$} &
          \boldmath{}\textbf{$\nu \ [-]$} &
          \boldmath{}\textbf{$G \ [tonf/cm^2]$}
          \bigstrut\\
        \hline
        A36 &
          2,53 &
          4,08 &
          2100 &
          0,29 &
          787,44
          \bigstrut\\
        \hline
        \end{tabular}}%
    \end{table}% 

\end{itemize} % [Pablo] [OK]

% 4 Calculo Factor corrección R*, para sismo X e Y [Pablo]
% 5 Chequear Masa equivalente mayor 90% en las dos direcciones. Imprimir tabla respectiva. [Pablo]
% TABLA MAGICA
% 6 Imprimir tablas: [Pablo]
% - Cortes basales
% - Periodos y participación de masas
% - Deformaciones del centro de masa
% 7 Coeficientes Sísmicos Max y Min. [Pablo]
% 8 Coeficientes Sísmicos para X e Y. [Pablo]
\newpage
\section{Resultados del Análisis Estructural}
% 4 Calculo Factor corrección R*, para sismo X e Y [Pablo]
% 5 Chequear Masa equivalente mayor 90% en las dos direcciones. Imprimir tabla respectiva. [Pablo]
% TABLA MAGICA
% 6 Imprimir tablas: [Pablo]
% - Cortes basales
% - Periodos y participación de masas
% - Deformaciones del centro de masa
% 7 Coeficientes Sísmicos Max y Min. [Pablo]
% 8 Coeficientes Sísmicos para X e Y. [Pablo]

\subsection{Cálculo factor corrección R*}

Al calcular los períodos fundamentales de la estructura en los ejes traslacionales $x$ e $y$ se obtuvo los factores de reducción $R*_x$ y $R*_y$ del pseudo-espectro de aceleración mostrados en la Tabla \ref{tab:factores-reduccion-r}. Posteriormente al obtener el corte basal en ambos ejes se calcularon los factores $f_x$ e $f_y$ para mayorar los $R*$ de cada eje.

\begin{table}[H]
  \centering
  \caption{Factores de corrección $R*$.}
  \begin{tabular}{ccccc}
    \hline
    \textbf{Eje} & \textbf{R*} & \textbf{1/R* [g] (scale factor)} & \textbf{f} & \textbf{R* mayorado} \bigstrut\\
    \hline
    x     & 10.579 & 0.927 & 3.076 & 2.851 \bigstrut[t]\\
    y     & 10.003 & 0.980 & 2.435 & 2.387 \bigstrut[b]\\
    \hline
  \end{tabular}
  \label{tab:factores-reduccion-r}
\end{table}

Los factores $R*$ mayorados fueron usados para definir los casos sísmicos en ETABS.

\begin{images}{Factores de mayoración del espectro en ETABS.}
    \addimage{sisx}{width=7cm}{Sismo en x.}
    \addimage{sisy}{width=7cm}{Sismo en y.}
\end{images}

\newpage
\subsection{Resultados valores por dirección x/y}

A partir del método modal espectral se obtuvieron los siguientes valores por cada eje X e Y :

\begin{table}[H]
  \centering
  \caption{Valores por dirección X del sismo.}
  \begin{tabular}{lrl}
    \hline
    \textbf{Sismo X} & \multicolumn{1}{c}{\textbf{Valor}} & \multicolumn{1}{c}{\textbf{Unidad}} \bigstrut\\
    \hline
    Peso sísmico & 11434.22 & [Tonf] \bigstrut[t]\\
    Corte Basal Mínimo & 705.80 & [Tonf] \\
    Corte Basal Máximo & 1482.18 & [Tonf] \\
    Período Predominante & 1.112 & [s] \\
    $R^*$ & 10.579 & [-] \\
    Factor de Mayoración & 3.076 & [-] \\
    Factor de Minoración & 1.000 & [-] \\
    Corte Basal Efectivo & 705.61 & [Tonf] \\
    Momento Volcante & 20305.03 & [Tonf-m] \\
    Brazo de Palanca & 28.78 & [m] \\
    Desplazamiento Último ($\delta ux=1.3Sd$) & 0.09  & [cm] \bigstrut[b]\\
    \hline
  \end{tabular}
\end{table}

\begin{table}[H]
  \centering
  \caption{Valores por dirección Y del sismo.}
  \begin{tabular}{lrl}
    \hline
    \textbf{Sismo Y} & \multicolumn{1}{c}{\textbf{Valor}} & \multicolumn{1}{c}{\textbf{Unidad}} \bigstrut\\
    \hline
    Peso sísmico & 11434.22 & [Tonf] \bigstrut[t]\\
    Corte Basal Mínimo & 705.80 & [Tonf] \\
    Corte Basal Máximo & 1482.18 & [Tonf] \\
    Período Predominante & 0.744 & [s] \\
    $R^*$ & 10.003 & [-] \\
    Factor de Mayoración & 2.435 & [-] \\
    Factor de Minoración & 1.000 & [-] \\
    Corte Basal Efectivo & 705.90 & [Tonf] \\
    Momento Volcante & 24152.00 & [Tonf-m] \\
    Brazo de Palanca & 34.21 & [m] \\
    Desplazamiento Último ($\delta ux=1.3Sd$) & 0.06  & [cm] \bigstrut[b]\\
    \hline
  \end{tabular}
\end{table}

\newpage
\subsection{Tablas de resultados}

\subsubsection{Periodos y participación de masas}

En la Tabla \ref{tabla-periodo-participacion} se detallan los períodos obtenidos para los modos en los cuales se alcanzó un 90\% de masa en las direcciones de análisis.

\begin{longtable}{cccccccc}
\caption{Períodos y participación de masas.}\label{tabla-periodo-participacion}\\
\hline
\multicolumn{1}{|c}{\textbf{Modo}} & \textbf{T [s]} & \textbf{\% Mx} & \textbf{\% My} & \textbf{\% Rz} & \textbf{\% \boldmath{}\textbf{$\sum$}\unboldmath{}Mx} & \textbf{\% \boldmath{}\textbf{$\sum$}\unboldmath{}My} & \textbf{\% \boldmath{}\textbf{$\sum$}\unboldmath{}Rz} \bigstrut\\
\hline
\endfirsthead
\hline
\multicolumn{1}{|c}{\textbf{Modo}} & \textbf{T (s)} & \textbf{\% Mx} & \textbf{\% My} & \textbf{\% Rz} & \textbf{\% \boldmath{}\textbf{$\sum$}\unboldmath{}Mx} & \textbf{\% \boldmath{}\textbf{$\sum$}\unboldmath{}My} & \textbf{\% \boldmath{}\textbf{$\sum$}\unboldmath{}Rz} \bigstrut\\
\hline
\endhead
\hline
\endfoot
\hline
\endlastfoot
1     & 1.315 & 4.54  & 5.42  & 45.52 & 4.54  & 5.42  & 45.52 \\
    2     & 1.112 & 55.52 & 0.78  & 1.95  & 60.06 & 6.20  & 47.47 \\
    3     & 0.744 & 0.03  & 56.05 & 6.66  & 60.08 & 62.24 & 54.13 \\
    4     & 0.382 & 0.45  & 1.06  & 8.75  & 60.54 & 63.30 & 62.89 \\
    5     & 0.272 & 15.40 & 0.13  & 0.06  & 75.94 & 63.43 & 62.95 \\
    6     & 0.216 & 0.08  & 12.12 & 1.33  & 76.02 & 75.55 & 64.28 \\
    7     & 0.185 & 0.01  & 0.70  & 3.62  & 76.03 & 76.25 & 67.90 \\
    8     & 0.124 & 5.66  & 0.11  & 0.04  & 81.69 & 76.36 & 67.94 \\
    9     & 0.111 & 0.16  & 0.00  & 2.55  & 81.85 & 76.36 & 70.50 \\
    10    & 0.107 & 0.11  & 5.11  & 0.08  & 81.96 & 81.48 & 70.58 \\
    11    & 0.077 & 1.33  & 0.00  & 1.06  & 83.29 & 81.48 & 71.64 \\
    12    & 0.073 & 1.94  & 0.32  & 0.71  & 85.23 & 81.79 & 72.35 \\
    13    & 0.07  & 0.00  & 0.00  & 0.02  & 85.23 & 81.80 & 72.37 \\
    14    & 0.069 & 0.25  & 2.83  & 0.14  & 85.48 & 84.63 & 72.50 \\
    15    & 0.067 & 0.00  & 0.00  & 0.01  & 85.49 & 84.63 & 72.52 \\
    16    & 0.056 & 0.05  & 0.04  & 1.39  & 85.53 & 84.67 & 73.91 \\
    17    & 0.052 & 0.00  & 0.02  & 0.00  & 85.54 & 84.68 & 73.91 \\
    18    & 0.052 & 2.04  & 0.33  & 0.05  & 87.57 & 85.01 & 73.96 \\
    19    & 0.049 & 0.00  & 0.01  & 0.01  & 87.58 & 85.02 & 73.97 \\
    20    & 0.049 & 0.41  & 1.75  & 0.08  & 87.99 & 86.76 & 74.05 \\
    21    & 0.048 & 0.00  & 0.00  & 0.03  & 87.99 & 86.77 & 74.08 \\
    22    & 0.046 & 0.00  & 0.00  & 0.00  & 87.99 & 86.77 & 74.08 \\
    23    & 0.043 & 0.01  & 0.07  & 1.39  & 88.00 & 86.84 & 75.47 \\
    24    & 0.04  & 0.15  & 0.01  & 0.00  & 88.15 & 86.85 & 75.47 \\
    25    & 0.039 & 1.03  & 0.47  & 0.00  & 89.18 & 87.32 & 75.48 \\
    26    & 0.038 & 0.68  & 1.04  & 0.03  & 89.86 & 88.37 & 75.50 \\
    27    & 0.035 & 0.09  & 0.08  & 1.47  & 89.95 & 88.45 & 76.97 \\
    28    & 0.035 & 0.03  & 0.00  & 0.02  & 89.99 & 88.45 & 76.99 \\
    29    & 0.033 & 0.04  & 0.00  & 0.02  & 90.02 & 88.46 & 77.01 \\
    30    & 0.032 & 0.15  & 0.85  & 0.00  & 90.18 & 89.30 & 77.01 \\
    31    & 0.031 & 0.39  & 0.21  & 0.07  & 90.57 & 89.52 & 77.07 \\
    32    & 0.031 & 0.64  & 0.05  & 0.02  & 91.21 & 89.57 & 77.09 \\
    33    & 0.03  & 0.23  & 0.00  & 0.38  & 91.43 & 89.57 & 77.47 \\
    34    & 0.029 & 0.22  & 0.08  & 1.04  & 91.65 & 89.65 & 78.50 \\
    35    & 0.029 & 0.00  & 0.00  & 0.00  & 91.65 & 89.65 & 78.50 \\
    36    & 0.028 & 0.00  & 1.08  & 0.01  & 91.65 & 90.73 & 78.52 \\
    37    & 0.027 & 0.07  & 0.01  & 0.06  & 91.72 & 90.73 & 78.57 \\
    38    & 0.026 & 0.90  & 0.00  & 0.02  & 92.62 & 90.74 & 78.59 \\
    39    & 0.026 & 0.03  & 0.06  & 0.16  & 92.65 & 90.80 & 78.75 \\
    40    & 0.025 & 0.26  & 0.10  & 1.11  & 92.91 & 90.90 & 79.85 \\
    41    & 0.025 & 0.00  & 0.00  & 0.00  & 92.91 & 90.90 & 79.86 \\
    42    & 0.025 & 0.02  & 0.34  & 0.00  & 92.93 & 91.24 & 79.86 \\
    43    & 0.024 & 0.00  & 0.01  & 0.01  & 92.93 & 91.25 & 79.86 \\
    44    & 0.024 & 0.01  & 0.73  & 0.02  & 92.94 & 91.99 & 79.89 \\
    45    & 0.022 & 0.00  & 0.00  & 0.02  & 92.94 & 91.99 & 79.91 \\
    46    & 0.022 & 0.00  & 0.22  & 1.15  & 92.94 & 92.20 & 81.06 \\
    47    & 0.022 & 0.87  & 0.15  & 0.27  & 93.81 & 92.35 & 81.33 \\
    48    & 0.022 & 0.27  & 0.06  & 0.04  & 94.08 & 92.42 & 81.37 \\
    49    & 0.021 & 0.04  & 1.30  & 0.12  & 94.12 & 93.71 & 81.48 \\
    50    & 0.021 & 0.00  & 0.02  & 0.00  & 94.12 & 93.73 & 81.49 \\
    51    & 0.02  & 0.00  & 0.00  & 0.01  & 94.12 & 93.74 & 81.50 \\
    52    & 0.02  & 0.02  & 0.32  & 1.33  & 94.14 & 94.06 & 82.83 \\
    53    & 0.02  & 0.01  & 0.01  & 0.13  & 94.15 & 94.06 & 82.95 \\
    54    & 0.02  & 0.87  & 0.01  & 0.04  & 95.02 & 94.07 & 83.00 \\
    55    & 0.019 & 0.03  & 1.48  & 0.32  & 95.05 & 95.55 & 83.31 \\
    56    & 0.018 & 0.00  & 0.01  & 0.42  & 95.05 & 95.56 & 83.73 \\
    57    & 0.018 & 0.00  & 0.00  & 0.00  & 95.05 & 95.56 & 83.73 \\
    58    & 0.018 & 0.07  & 0.28  & 1.23  & 95.11 & 95.84 & 84.97 \\
    59    & 0.017 & 0.80  & 0.01  & 0.00  & 95.92 & 95.84 & 84.97 \\
    60    & 0.017 & 0.02  & 1.00  & 0.28  & 95.94 & 96.85 & 85.25 \\
    61    & 0.016 & 0.01  & 0.00  & 1.24  & 95.95 & 96.85 & 86.49 \\
    62    & 0.016 & 0.79  & 0.00  & 0.07  & 96.75 & 96.85 & 86.56 \\
    63    & 0.016 & 0.00  & 0.44  & 0.37  & 96.75 & 97.29 & 86.92 \\
    64    & 0.015 & 0.00  & 0.01  & 0.41  & 96.75 & 97.30 & 87.34 \\
    65    & 0.015 & 0.26  & 0.02  & 0.56  & 97.01 & 97.32 & 87.90 \\
    66    & 0.015 & 0.63  & 0.00  & 0.39  & 97.63 & 97.32 & 88.29 \\
    67    & 0.014 & 0.00  & 0.16  & 0.24  & 97.64 & 97.48 & 88.53 \\
    68    & 0.014 & 0.44  & 0.10  & 0.43  & 98.08 & 97.58 & 88.96 \\
    69    & 0.014 & 0.31  & 0.11  & 0.84  & 98.38 & 97.69 & 89.80 \\
    70    & 0.013 & 0.01  & 0.00  & 0.24  & 98.39 & 97.69 & 90.04 \\
\hline
\end{longtable}

\newpage
\subsubsection{Cortes basales}

La tabla de cortes basales (Tabla \ref{tabla-cortes-basales}) fue obtenida a partir del resultado \textit{Story forces} de ETABS considerando un espectro sísmico inelástico \footnote{Reducido por el factor $1/R*$.}.

\begin{table}[H]
  \centering
  \caption{Cortes basales.}
  \resizebox{1\textwidth}{!}{
  
  \begin{tabular}{ccccccc}
    \hline
    \textbf{Combinación} & \textbf{P [tonf]} & \cellcolor[rgb]{ .949,  .949,  .949}\textbf{Vx [tonf]} & \cellcolor[rgb]{ .949,  .949,  .949}\textbf{Vy [tonf]} & \textbf{T [tonf-m]} & \textbf{Mx [tonf-m]} & \textbf{My [tonf-m]} \bigstrut\\
    \hline
    PP    & 11041.9 & \cellcolor[rgb]{ .949,  .949,  .949}0.0 & \cellcolor[rgb]{ .949,  .949,  .949}0.0 & 0.0   & 185690.8 & -175463.3 \bigstrut[t]\\
    SC    & 2885.7 & \cellcolor[rgb]{ .949,  .949,  .949}0.0 & \cellcolor[rgb]{ .949,  .949,  .949}0.0 & 0.0   & 50485.4 & -44740.7 \\
    SX Max & 0.0   & \cellcolor[rgb]{ .949,  .949,  .949}705.6 & \cellcolor[rgb]{ .949,  .949,  .949}85.4 & 12791.9 & 2422.1 & 20305.0 \\
    SY Max & 0.0   & \cellcolor[rgb]{ .949,  .949,  .949}71.5 & \cellcolor[rgb]{ .949,  .949,  .949}705.9 & 14354.7 & 24152.0 & 2140.1 \\
    PPSC  & 13927.6 & \cellcolor[rgb]{ .949,  .949,  .949}0.0 & \cellcolor[rgb]{ .949,  .949,  .949}0.0 & 0.0   & 236176.2 & -220204.0 \\
    C1    & 15458.7 & \cellcolor[rgb]{ .949,  .949,  .949}0.0 & \cellcolor[rgb]{ .949,  .949,  .949}0.0 & 0.0   & 259967.1 & -245648.6 \\
    C2    & 17867.3 & \cellcolor[rgb]{ .949,  .949,  .949}0.0 & \cellcolor[rgb]{ .949,  .949,  .949}0.0 & 0.0   & 303605.6 & -282141.1 \\
    C3 Max & 9937.7 & \cellcolor[rgb]{ .949,  .949,  .949}987.9 & \cellcolor[rgb]{ .949,  .949,  .949}119.6 & 17908.7 & 170512.6 & -129489.9 \\
    C4 Max & 9937.7 & \cellcolor[rgb]{ .949,  .949,  .949}987.9 & \cellcolor[rgb]{ .949,  .949,  .949}119.6 & 17908.7 & 170512.6 & -129489.9 \\
    C5 Max & 9937.7 & \cellcolor[rgb]{ .949,  .949,  .949}100.1 & \cellcolor[rgb]{ .949,  .949,  .949}988.3 & 20096.5 & 200934.5 & -154920.9 \\
    C6 Max & 9937.7 & \cellcolor[rgb]{ .949,  .949,  .949}100.1 & \cellcolor[rgb]{ .949,  .949,  .949}988.3 & 20096.5 & 200934.5 & -154920.9 \\
    C7 Max & 16136.0 & \cellcolor[rgb]{ .949,  .949,  .949}987.9 & \cellcolor[rgb]{ .949,  .949,  .949}119.6 & 17908.7 & 276705.2 & -226869.6 \\
    C8 Max & 16136.0 & \cellcolor[rgb]{ .949,  .949,  .949}987.9 & \cellcolor[rgb]{ .949,  .949,  .949}119.6 & 17908.7 & 276705.2 & -226869.6 \\
    C9 Max & 16136.0 & \cellcolor[rgb]{ .949,  .949,  .949}100.1 & \cellcolor[rgb]{ .949,  .949,  .949}988.3 & 20096.5 & 307127.1 & -252300.5 \\
    C10 Max & 16136.0 & \cellcolor[rgb]{ .949,  .949,  .949}100.1 & \cellcolor[rgb]{ .949,  .949,  .949}988.3 & 20096.5 & 307127.1 & -252300.5 \\
    ENVC Max & 17867.3 & \cellcolor[rgb]{ .949,  .949,  .949}987.9 & \cellcolor[rgb]{ .949,  .949,  .949}988.3 & 20096.5 & 307127.1 & -129489.9 \bigstrut[b]\\
    \hline
  \end{tabular}
  }
  \label{tabla-cortes-basales}
\end{table}


% 9 Cortes basales Max y Min [Mauro] [OK]
% 10 Cortes basales X e Y [Mauro] [OK]
% 11 Control de deformaciones en las dos direcciones drift piso [Nacho] [OK]
% Tabla deformación máxima --> 1.3Sd [Pablo]
\newpage
\section{Cortes Basales y Control de Deformaciones}

    \subsection{Cortes basales}
    
        De acuerdo a lo indicado en el punto 6.2.3 de la NCh433, se determinan los cortes basales máximos y mínimos mostrados a continuación:
        
        \begin{table}[H]
            \centering
            \caption{Cortes basales máximo y mínimo.}
                \begin{tabular}{cc}
                    \hline
                          & \textbf{Cortes basal norma [tonf]} \bigstrut\\
                    \hline
                    $Q_{min}$ & 705,800 \bigstrut[t]\\
                    $Q_{max}$ & 1482,180 \bigstrut[b]\\
                    \hline
                \end{tabular}%
            \label{cortex_max_min}%
        \end{table}%
        
        Del modelo se determina el corte basal efectivo. En la primera iteración este resultó ser menor al mínimo por lo que se corrigió el espectro, resultando finalmente lo mostrado en la siguiente tabla:
        
        \begin{table}[htbp]
          \centering
          \caption{Cortes basales efectivos.}
            \begin{tabular}{cc}
            \toprule
                  & \textbf{Corte efectivo [tonf]} \\
            \midrule
            $Q_x$ & 705,613 \\
            $Q_y$ & 705,897 \\
            \bottomrule
            \end{tabular}%
          \label{cortes}%
        \end{table}%

    \newpage
    \subsection{Control de deformaciones}
    A continuación, se muestran los desplazamientos entre pisos, sin considerar el efecto del factor de reducción, considerando la losa como un elemento completamente rígido. Se debe tener en consideración que este valor no puede superar $0.001 \cdot H_{entrepiso}$.
    
    \begin{table}[H]
  \centering
  \caption{Verificación de drifts utlizando centro de masa, sismo y desplazamiento en eje X. }
  \resizebox{\textwidth}{!}{%
  \begin{tabular}{cccccccc}
    \hline
    \multicolumn{1}{c}{\textbf{Story}} &
      \multicolumn{1}{c}{\textbf{UX [m]}} &
      \multicolumn{1}{c}{\textbf{X [m]}} &
      \multicolumn{1}{c}{\textbf{Y [m]}} &
      \multicolumn{1}{c}{\textbf{Z [m]}} &
      \multicolumn{1}{c}{\textbf{Drift [-]}} &
      \multicolumn{1}{c}{\textbf{Admisible [-]}} &
      \multicolumn{1}{c}{\textbf{Porcentaje [\%]}}
      \bigstrut\\
    \hline
    Cubierta &
      0,123 &
      18,610 &
      15,900 &
      63,38 &
      - &
      - &
      -
      \bigstrut[t]\\
    24 &
      0,118 &
      19,870 &
      16,364 &
      61,33 &
      0,002 &
      0,0021 &
      121,6
      \\
    23 &
      0,112 &
      16,314 &
      16,339 &
      58,88 &
      0,002 &
      0,0025 &
      91,4
      \\
    22 &
      0,107 &
      16,053 &
      16,409 &
      56,43 &
      0,002 &
      0,0025 &
      93,3
      \\
    21 &
      0,101 &
      16,053 &
      16,409 &
      53,98 &
      0,002 &
      0,0025 &
      93,5
      \\
    20 &
      0,095 &
      16,053 &
      16,409 &
      51,53 &
      0,002 &
      0,0025 &
      94,8
      \\
    19 &
      0,090 &
      16,053 &
      16,409 &
      49,08 &
      0,002 &
      0,0025 &
      95,9
      \\
    18 &
      0,084 &
      16,053 &
      16,409 &
      46,63 &
      0,002 &
      0,0025 &
      96,7
      \\
    17 &
      0,078 &
      16,053 &
      16,409 &
      44,18 &
      0,002 &
      0,0025 &
      97,1
      \\
    16 &
      0,072 &
      16,053 &
      16,409 &
      41,73 &
      0,002 &
      0,0025 &
      97,2
      \\
    15 &
      0,066 &
      16,053 &
      16,409 &
      39,28 &
      0,002 &
      0,0025 &
      96,9
      \\
    14 &
      0,061 &
      16,053 &
      16,409 &
      36,83 &
      0,002 &
      0,0025 &
      96,2
      \\
    13 &
      0,055 &
      16,053 &
      16,409 &
      34,38 &
      0,002 &
      0,0025 &
      95,2
      \\
    12 &
      0,049 &
      16,053 &
      16,409 &
      31,93 &
      0,002 &
      0,0025 &
      93,1
      \\
    11 &
      0,044 &
      16,053 &
      16,409 &
      29,48 &
      0,002 &
      0,0025 &
      91,5
      \\
    10 &
      0,038 &
      16,053 &
      16,409 &
      27,03 &
      0,002 &
      0,0025 &
      89,5
      \\
    9 &
      0,033 &
      16,053 &
      16,409 &
      24,58 &
      0,002 &
      0,0025 &
      87,2
      \\
    8 &
      0,028 &
      16,053 &
      16,409 &
      22,13 &
      0,002 &
      0,0025 &
      84,2
      \\
    7 &
      0,023 &
      16,062 &
      16,420 &
      19,68 &
      0,002 &
      0,0025 &
      80,4
      \\
    6 &
      0,019 &
      16,070 &
      16,430 &
      17,23 &
      0,002 &
      0,0025 &
      74,9
      \\
    5 &
      0,015 &
      16,070 &
      16,430 &
      14,78 &
      0,002 &
      0,0025 &
      70,6
      \\
    4 &
      0,011 &
      16,072 &
      16,437 &
      12,33 &
      0,002 &
      0,0025 &
      65,6
      \\
    3 &
      0,007 &
      16,070 &
      16,430 &
      9,88 &
      0,001 &
      0,0025 &
      59,4
      \\
    2 &
      0,004 &
      16,176 &
      16,550 &
      7,43 &
      0,001 &
      0,0025 &
      51,8
      \\
    1 &
      0,002 &
      11,555 &
      19,840 &
      4,98 &
      0,001 &
      0,0025 &
      39,3
      \\
    -1 &
      0,000 &
      15,476 &
      19,695 &
      2,5 &
      0,001 &
      0,0025 &
      21,4
      \bigstrut[b]\\
    \hline
  \end{tabular}}
  \label{sxCMD}
\end{table}

\begin{table}[H]
  \centering
  \caption{Verificación de drifts utlizando centro de masa, sismo y desplazamiento en eje Y.}
  \resizebox{\textwidth}{!}{%
  \begin{tabular}{cccccccc}
    \hline
    \multicolumn{1}{c}{\textbf{Story}} &
      \multicolumn{1}{c}{\textbf{UY [m]}} &
      \multicolumn{1}{c}{\textbf{X [m]}} &
      \multicolumn{1}{c}{\textbf{Y [m]}} &
      \multicolumn{1}{c}{\textbf{Z [m]}} &
      \multicolumn{1}{c}{\textbf{Drift [-]}} &
      \multicolumn{1}{c}{\textbf{Admisible [-]}} &
      \multicolumn{1}{c}{\textbf{Porcentaje [\%]}}
      \bigstrut\\
    \hline
    Cubierta &
      0,0754 &
      18,610 &
      15,900 &
      63,38 &
      - &
      - &
      -
      \bigstrut[t]\\
    24 &
      0,0778 &
      19,870 &
      16,364 &
      61,33 &
      -0,001 &
      0,00205 &
      57,1
      \\
    23 &
      0,0709 &
      16,314 &
      16,339 &
      58,88 &
      0,003 &
      0,00245 &
      115,5
      \\
    22 &
      0,0684 &
      16,053 &
      16,409 &
      56,43 &
      0,001 &
      0,00245 &
      41,3
      \\
    21 &
      0,0657 &
      16,053 &
      16,409 &
      53,98 &
      0,001 &
      0,00245 &
      45,1
      \\
    20 &
      0,0628 &
      16,053 &
      16,409 &
      51,53 &
      0,001 &
      0,00245 &
      47,8
      \\
    19 &
      0,0598 &
      16,053 &
      16,409 &
      49,08 &
      0,001 &
      0,00245 &
      50,4
      \\
    18 &
      0,0566 &
      16,053 &
      16,409 &
      46,63 &
      0,001 &
      0,00245 &
      52,9
      \\
    17 &
      0,0533 &
      16,053 &
      16,409 &
      44,18 &
      0,001 &
      0,00245 &
      55,2
      \\
    16 &
      0,0499 &
      16,053 &
      16,409 &
      41,73 &
      0,001 &
      0,00245 &
      57,3
      \\
    15 &
      0,0463 &
      16,053 &
      16,409 &
      39,28 &
      0,001 &
      0,00245 &
      59,1
      \\
    14 &
      0,0427 &
      16,053 &
      16,409 &
      36,83 &
      0,001 &
      0,00245 &
      60,6
      \\
    13 &
      0,0390 &
      16,053 &
      16,409 &
      34,38 &
      0,002 &
      0,00245 &
      61,9
      \\
    12 &
      0,0353 &
      16,053 &
      16,409 &
      31,93 &
      0,002 &
      0,00245 &
      61,5
      \\
    11 &
      0,0316 &
      16,053 &
      16,409 &
      29,48 &
      0,002 &
      0,00245 &
      61,9
      \\
    10 &
      0,0278 &
      16,053 &
      16,409 &
      27,03 &
      0,002 &
      0,00245 &
      62,1
      \\
    9 &
      0,0241 &
      16,053 &
      16,409 &
      24,58 &
      0,002 &
      0,00245 &
      61,7
      \\
    8 &
      0,0205 &
      16,053 &
      16,409 &
      22,13 &
      0,001 &
      0,00245 &
      60,8
      \\
    7 &
      0,0169 &
      16,062 &
      16,420 &
      19,68 &
      0,001 &
      0,00245 &
      59,0
      \\
    6 &
      0,0137 &
      16,070 &
      16,430 &
      17,23 &
      0,001 &
      0,00245 &
      54,8
      \\
    5 &
      0,0105 &
      16,070 &
      16,430 &
      14,78 &
      0,001 &
      0,00245 &
      52,2
      \\
    4 &
      0,0076 &
      16,072 &
      16,437 &
      12,33 &
      0,001 &
      0,00245 &
      48,8
      \\
    3 &
      0,0049 &
      16,070 &
      16,430 &
      9,88 &
      0,001 &
      0,00245 &
      44,1
      \\
    2 &
      0,0027 &
      16,176 &
      16,550 &
      7,43 &
      0,001 &
      0,00245 &
      37,6
      \\
    1 &
      0,0009 &
      11,555 &
      19,840 &
      4,98 &
      0,001 &
      0,00245 &
      30,2
      \\
    -1 &
      0,0002 &
      15,476 &
      19,695 &
      2,5 &
      0,000 &
      0,00248 &
      11,2
      \bigstrut[b]\\
    \hline
  \end{tabular}}
  \label{syCMD}
\end{table}

%poner texto
Por otro lado, utilizando la información directa de los Story Drifts obtenidos del modelo, los cuales no usan el supuesto de losa rigida, se comparan los drifts con su deformación admisible considerando este último como $0.002 \cdot H_{entrepiso}$.
\begin{table}[H]
  \centering
  \caption{Verificación de drifts utlizando deformaciones máximas y sismo en eje X.}
  \resizebox{\textwidth}{!}{%
  \begin{tabular}{ccccccc}
    \hline
    \multicolumn{1}{c}{\textbf{Story}} &
      \multicolumn{1}{c}{\textbf{X [m]}} &
      \multicolumn{1}{c}{\textbf{Y [m]}} &
      \multicolumn{1}{c}{\textbf{Z [m]}} &
      \multicolumn{1}{c}{\textbf{Drift [-]}} &
      \multicolumn{1}{c}{\textbf{Admisible [-]}} &
      \multicolumn{1}{c}{\textbf{Porcentaje [\%]}}
      \bigstrut\\
    \hline
    Cubierta &
      20,63 &
      17 &
      63,38 &
      0,002 &
      - &
      -
      \bigstrut[t]\\
    24 &
      16,59 &
      20,65 &
      61,33 &
      0,002 &
      0,0041 &
      60,3
      \\
    23 &
      17,21 &
      31,83 &
      58,88 &
      0,003 &
      0,0049 &
      69,4
      \\
    22 &
      11,14 &
      32,43 &
      56,43 &
      0,004 &
      0,0049 &
      73,6
      \\
    21 &
      11,14 &
      32,43 &
      53,98 &
      0,004 &
      0,0049 &
      77,0
      \\
    20 &
      11,14 &
      32,43 &
      51,53 &
      0,004 &
      0,0049 &
      80,0
      \\
    19 &
      11,14 &
      32,43 &
      49,08 &
      0,004 &
      0,0049 &
      82,5
      \\
    18 &
      11,14 &
      32,43 &
      46,63 &
      0,004 &
      0,0049 &
      84,4
      \\
    17 &
      11,14 &
      32,43 &
      44,18 &
      0,004 &
      0,0049 &
      85,5
      \\
    16 &
      11,14 &
      32,43 &
      41,73 &
      0,004 &
      0,0049 &
      86,0
      \\
    15 &
      11,14 &
      32,43 &
      39,28 &
      0,004 &
      0,0049 &
      85,8
      \\
    14 &
      11,14 &
      32,43 &
      36,83 &
      0,004 &
      0,0049 &
      84,8
      \\
    13 &
      11,14 &
      32,43 &
      34,38 &
      0,004 &
      0,0049 &
      82,2
      \\
    12 &
      11,14 &
      32,43 &
      31,93 &
      0,004 &
      0,0049 &
      81,1
      \\
    11 &
      11,14 &
      32,43 &
      29,48 &
      0,004 &
      0,0049 &
      79,9
      \\
    10 &
      11,14 &
      32,43 &
      27,03 &
      0,004 &
      0,0049 &
      78,4
      \\
    9 &
      11,14 &
      32,43 &
      24,58 &
      0,004 &
      0,0049 &
      76,4
      \\
    8 &
      11,14 &
      32,43 &
      22,13 &
      0,004 &
      0,0049 &
      73,5
      \\
    7 &
      11,14 &
      32,43 &
      19,68 &
      0,003 &
      0,0049 &
      69,5
      \\
    6 &
      11,14 &
      32,43 &
      17,23 &
      0,003 &
      0,0049 &
      64,9
      \\
    5 &
      11,14 &
      32,43 &
      14,78 &
      0,003 &
      0,0049 &
      58,8
      \\
    4 &
      11,14 &
      32,43 &
      12,33 &
      0,002 &
      0,0049 &
      50,7
      \\
    3 &
      20,3 &
      0 &
      9,88 &
      0,002 &
      0,0049 &
      39,9
      \\
    2 &
      15,79 &
      1,36 &
      7,43 &
      0,002 &
      0,0049 &
      31,1
      \\
    1 &
      22,95 &
      7,7 &
      4,98 &
      0,001 &
      0,0049 &
      19,5
      \\
    -1 &
      21,04 &
      39,44 &
      2,5 &
      0,000 &
      0,0050 &
      2,5
      \bigstrut[b]\\
    \hline
  \end{tabular}}
  \label{sxDD}
\end{table}


\begin{table}[H]
  \centering
  \caption{Verificación de drifts utlizando deformaciones máximas y sismo en eje Y.}
  \resizebox{\textwidth}{!}{%
  \begin{tabular}{ccccccc}
    \hline
    \multicolumn{1}{c}{\textbf{Story}} &
      \multicolumn{1}{c}{\textbf{X [m]}} &
      \multicolumn{1}{c}{\textbf{Y [m]}} &
      \multicolumn{1}{c}{\textbf{Z [m]}} &
      \multicolumn{1}{c}{\textbf{Drift [-]}} &
      \multicolumn{1}{c}{\textbf{Admisible [-]}} &
      \multicolumn{1}{c}{\textbf{Porcentaje [\%]}}
      \bigstrut\\
    \hline
    Cubierta &
      16,59 &
      17 &
      63,38 &
      0,001 &
      - &
      -
      \bigstrut[t]\\
    24 &
      16,59 &
      20,65 &
      61,33 &
      0,001 &
      0,0041 &
      24,4
      \\
    23 &
      9,03 &
      21,9 &
      58,88 &
      0,001 &
      0,0049 &
      26,2
      \\
    22 &
      9,03 &
      21,9 &
      56,43 &
      0,001 &
      0,0049 &
      27,9
      \\
    21 &
      22,95 &
      17 &
      53,98 &
      0,001 &
      0,0049 &
      30,0
      \\
    20 &
      22,95 &
      12,7 &
      51,53 &
      0,002 &
      0,0049 &
      32,6
      \\
    19 &
      22,95 &
      17 &
      49,08 &
      0,002 &
      0,0049 &
      34,8
      \\
    18 &
      22,95 &
      17 &
      46,63 &
      0,002 &
      0,0049 &
      36,7
      \\
    17 &
      22,95 &
      17 &
      44,18 &
      0,002 &
      0,0049 &
      38,2
      \\
    16 &
      22,95 &
      17 &
      41,73 &
      0,002 &
      0,0049 &
      39,4
      \\
    15 &
      22,95 &
      17 &
      39,28 &
      0,002 &
      0,0049 &
      40,1
      \\
    14 &
      22,95 &
      17 &
      36,83 &
      0,002 &
      0,0049 &
      40,2
      \\
    13 &
      22,95 &
      17 &
      34,38 &
      0,002 &
      0,0049 &
      38,4
      \\
    12 &
      22,95 &
      17 &
      31,93 &
      0,002 &
      0,0049 &
      38,3
      \\
    11 &
      22,95 &
      12,7 &
      29,48 &
      0,002 &
      0,0049 &
      38,2
      \\
    10 &
      22,95 &
      17 &
      27,03 &
      0,002 &
      0,0049 &
      38,1
      \\
    9 &
      22,95 &
      17 &
      24,58 &
      0,002 &
      0,0049 &
      37,7
      \\
    8 &
      22,95 &
      17 &
      22,13 &
      0,002 &
      0,0049 &
      37,0
      \\
    7 &
      22,95 &
      17 &
      19,68 &
      0,002 &
      0,0049 &
      35,7
      \\
    6 &
      22,95 &
      17 &
      17,23 &
      0,002 &
      0,0049 &
      34,3
      \\
    5 &
      22,95 &
      17 &
      14,78 &
      0,002 &
      0,0049 &
      32,4
      \\
    4 &
      22,95 &
      17 &
      12,33 &
      0,001 &
      0,0049 &
      29,6
      \\
    3 &
      22,95 &
      17 &
      9,88 &
      0,001 &
      0,0049 &
      25,3
      \\
    2 &
      17,21 &
      0,7 &
      7,43 &
      0,001 &
      0,0049 &
      22,3
      \\
    1 &
      22,95 &
      9,62 &
      4,98 &
      0,001 &
      0,0049 &
      14,5
      \\
    -1 &
      31,19 &
      1,36 &
      2,5 &
      0,000 &
      0,00496 &
      1,9
      \bigstrut[b]\\
    \hline
  \end{tabular}}
  \label{syDD}
\end{table}

\newpage
\subsection{Gráficos de respuesta}

\subsubsection{Corte por piso}

\begin{images}[\label{corte-piso}]{Cortes por piso.}
\addimage{corte-sismox}{width=7cm}{Sismo x.}
\addimage{corte-sismoy}{width=7cm}{Sismo y.}
\end{images}

\subsubsection{Momento volcante}

\begin{images}{Momento volcante por piso.}
\addimage{momento-sismox}{width=7cm}{Sismo x.}
\addimage{momento-sismoy}{width=7cm}{Sismo y.}
\end{images}

\newpage
\subsubsection{Desplazamiento entre piso}

\begin{images}{Desplazamiento entre piso.}
\addimage{despl-sismox}{width=7cm}{Sismo x.}
\addimage{despl-sismoy}{width=7cm}{Sismo y.}
\end{images}

\subsubsection{Drift entre piso}

\begin{images}{Drift entre piso.}
\addimage{drift-sismox}{width=7cm}{Sismo x.}
\addimage{drift-sismoy}{width=7cm}{Sismo y.}
\end{images}

% 12 Verificación de Rigidez H/T [Mauro] [OK]
% 13 Indicador acoplamiento MATRIZ [Nacho] [OK]
\newpage
\section{Indicadores de Rigidez y Acoplamiento}

    \subsection{Indicador de Rigidez}
        De la verificación de rigidez se obtiene el siguiente resultado:
        \insertequationanum{H/T= \frac{58,88}{1,345} = 43,8}
        
        \begin{table}[H]
          \centering
          \caption{Verificación de rigidez del edificio.}
            \begin{tabular}{cc}
            \toprule
            \textbf{H/T}   & \textbf{Nivel de Rigidez} \\
            \midrule
            20-40 & Flexible \\
            40-70 & Normal \\
            70-150 & Rígido \\
            < 20  & Muy Flexible \\
            > 150 & Muy Rígido \\
            \bottomrule
            \end{tabular}%
          \label{rigidez}%
        \end{table}%
        
        De acuerdo a la tabla \ref{rigidez} se determina que el edificio es de rigidez \textit{Normal}.
        
    \subsection{Indicador de Acoplamiento}
    De la verificación de Acoplamiento se obtiene el siguiente resultado:
    
    \begin{table}[H]
  \centering
  \caption{Indicadores de acoplamiento.}
  \begin{tabular}{cc}
    \hline
    \textbf{Acoplamiento} &
      \textbf{Valor (-)}
      \bigstrut\\
    \hline
    T1/T2 &
      1,8
      \bigstrut[t]\\
    T1/T3 &
      1,2
      \\
    T3/T2 &
      1,5
      \bigstrut[b]\\
    \hline
  \end{tabular}
  \label{IndAcpl}
\end{table}

Como se puede apreciar en la tabla anterior, todos los factores obtenidos son mayores o iguales que 1.2 como lo establece la norma.

% 14 Entrega modelo sísmico funcionando. [OK]
% 15 Conclusiones y Comentarios para al menos: Deformaciones, Rigidez, Acoplamiento, Cortes basales
% - concluir drift piso < 0.002
% - periodo torsional, concluir FT suelo roca, poca amplificacion, respuesta periodo similar
% - 
\newpage
\section{Comentarios y Conclusiones}

En cuanto al período predominante se obtuvo un primer modo torsional, si bien esto no es lo deseable hay que destacar que el período obtenido está lejos del período en que el suelo presenta mayores aceleraciones (el cual es cercano a los 0.2s), por tanto el contenido de frecuencias de un sismo no proporcionará energía suficiente a la estructura para excitar un modo torsional. Por otro lado en cuanto al período en $X$ e $Y$ se obtuvieron igualmente períodos altos en comparación a la zona de amplificación, lo cual es deseable dado que la mayor masa del edificio se tributará en periodos para los cuales la estructura recibe menos energía (aceleración).
    
\insertimage{periodos}{width=12cm}{Períodos de los primeros períodos junto al espectro de aceleraciones reducido.}

Para lograr el 90\% de masa trasladada en cada dirección de análisis se necesitó de un total de 70 modos, esto conversa muy bien con la hipótesis de diseño inicial la cual consideró un diafragma rígido, requiriendo así de un total de tres grados de libertad por cada piso para representar los desplazamientos de la estructura ($u_x$, $u_y$ desplazamientos horizontales en planta y $\theta$ ángulo de giro) sin mayores diferencias con la realidad.  Así, teóricamente con un total de $3 \cdot 23 = 69$ modos se logra un buen nivel de confianza, resultado muy cercano al obtenido en la práctica. \\

Con respecto a los desplazamientos obtenidos por piso, se puede apreciar como estos son mayores por efecto del sismo en el eje $X$. Lo anterior se debe a la baja inercia que posee la estructura en este eje en comparación con el eje $Y$. Sin embargo, es posible apreciar que estos desplazamientos son menores que los admisibles o solicitados por norma.

Los casos puntuales de los pisos más superiores donde esto no se cumple es prácticamente por la condición de apoyo libre que existe a nivel de la cubierta. \\
    
Dado que los factores de acoplamiento obtenidos en la Tabla \ref{IndAcpl} son mayores o iguales que lo indicado en la norma, se puede concluir que no habrá efecto de acoplamiento entre los modos obtenidos. \\

Al analizar los gráficos de distribución del cortante por piso (Figuras \ref{corte-piso}.a y \ref{corte-piso}.b) es posible observar que en cuanto al sismo en el eje $Y$ la distribución es relativamente homogénea, ello significa que el corte se tributa de igual manera en cada piso, por tanto en dicho eje la rigidez (dado el hormigón y espesor de muros elegidos) fue la adecuada. \\

Distinto es en el caso del sismo en x, en donde en los pisos intermedios de la estructura (concretamente entre el 17 y el 12) toman poco corte, tributándolo a los pisos inferiores de la estructura. Esto indica que dichos pisos en ese eje resultaron muy rígidos, disminuyendo por tanto la distorsión angular (Mayor módulo de rigidez, mayor módulo de corte, menor distorsión angular a un mismo desplazamiento), tomando así menos corte dado la relación $\tau = G \cdot \gamma$. Una posible solución es disminuir el espesor de los muros, verificando que dicho cambio no afecte al drift entre pisos.

% FIN DEL DOCUMENTO
\end{document}
