% Template:     Informe/Reporte LaTeX
% Documento:    Archivo principal
% Versión:      6.0.6 (30/10/2018)
% Codificación: UTF-8
%
% Autor: Pablo Pizarro R. @ppizarror
%        Facultad de Ciencias Físicas y Matemáticas
%        Universidad de Chile
%        pablo.pizarro@ing.uchile.cl, ppizarror.com
%
% Manual template: [https://latex.ppizarror.com/Template-Informe/]
% Licencia MIT:    [https://opensource.org/licenses/MIT/]

% CREACIÓN DEL DOCUMENTO
\documentclass[letterpaper,11pt]{article} % Articulo tamaño carta, 11pt
\usepackage[utf8]{inputenc} % Codificación UTF-8

% INFORMACIÓN DEL DOCUMENTO
\def\titulodelinforme {Modelo sísmico}
\def\temaatratar {Proyecto de Hormigón Armado - Entrega N°4}

\def\autordeldocumento {Grupo N1}
\def\nombredelcurso {Proyecto de Hormigón Armado}
\def\codigodelcurso {CI5206-2}

\def\nombreuniversidad {Universidad de Chile}
\def\nombrefacultad {Facultad de Ciencias Físicas y Matemáticas}
\def\departamentouniversidad {Departamento de Ingeniería Civil}
\def\imagendepartamento {dic}
\def\imagendepartamentoescala {0.2}
\def\localizacionuniversidad {Santiago, Chile}

% INTEGRANTES, PROFESORES Y FECHAS
\def\tablaintegrantes {
\begin{tabular}{ll}
	Integrantes:
	& \begin{tabular}[t]{@{}l@{}}
		Mauricio Leal V. \\
		Pablo Pizarro R. \\
		Ignacio Yáñez G.
	\end{tabular} \\
	Profesor:
	& \begin{tabular}[t]{@{}l@{}}
		Juan Mendoza V.
	\end{tabular} \\
	Auxiliar:
	& \begin{tabular}[t]{@{}l@{}}
		Felipe Andrade T.
	\end{tabular} \\
	& \\
	\multicolumn{2}{l}{Fecha de entrega: 07 de Noviembre de 2018} \\
	\multicolumn{2}{l}{\localizacionuniversidad}
\end{tabular}}{
}

% CONFIGURACIONES
% Template:     Informe/Reporte LaTeX
% Documento:    Configuraciones del template
% Versión:      6.0.0 (13/10/2018)
% Codificación: UTF-8
%
% Autor: Pablo Pizarro R. @ppizarror
%        Facultad de Ciencias Físicas y Matemáticas
%        Universidad de Chile
%        pablo.pizarro@ing.uchile.cl, ppizarror.com
%
% Manual template: [http://latex.ppizarror.com/Template-Informe/]
% Licencia MIT:    [https://opensource.org/licenses/MIT/]

% CONFIGURACIONES GENERALES
\def\addemptypagetwosides {false}  % Añade pag en blanco al imprimir a 2 caras
\def\columnhspace {-0.4}           % Margen horizontal entre obj. \createcolumn
\def\columnsepwidth {2.1}          % Separación entre columnas [em]
\def\defaultinterline {1.0}        % Interlineado por defecto [pt]
\def\defaultnewlinesize {11}       % Tamaño del salto de línea [pt]
\def\documentlang {es-CL}          % Define el idioma del documento
\def\fontdocument {lmodern}        % Tipografía base, ver soportadas en manual
\def\fonttypewriter {tmodern}      % Tipografía de \texttt, ver manual
\def\importtikz {false}            % Utilizar la librería tikz
\def\numberedequation {true}       % Ecuaciones con \insert... numeradas
\def\pointdecimal {false}          % N° decimales con punto en vez de coma
\def\romanpageuppercase {false}    % Páginas en número romano en mayúsculas
\def\showlinenumbers {false}       % Muestra los n° de línea del documento
\def\tablepaddingh {1.0}           % Espaciado horizontal de celda de las tablas
\def\tablepaddingv {1.0}           % Espaciado vertical de celda de las tablas

% ESTILO PORTADA Y HEADER-FOOTER
\def\hfstyle {style1}              % Estilo header-footer (14 estilos disp.)
\def\portraitstyle {style1}        % Estilo portada (18 estilos disp.)

% MÁRGENES DE PÁGINA
\def\firstpagemargintop {3.8}      % Margen superior página portada [cm]
\def\pagemarginbottom {2.7}        % Margen inferior página [cm]
\def\pagemarginleft {2.54}         % Margen izquierdo página [cm]
\def\pagemarginright {2.54}        % Margen derecho página [cm]
\def\pagemargintop {3.0}           % Margen superior página [cm]

% POSICIÓN POR DEFECTO DE OBJETOS
\def\imagedefaultplacement {H}     % Posición por defecto de las imágenes
\def\tabledefaultplacement {H}     % Posición por defecto de las tablas
\def\tikzdefaultplacement {H}      % Posición por defecto de las figuras tikz

% CONFIGURACIÓN DE LAS LEYENDAS - CAPTION
\def\captionalignment {justified}  % Pos. ley. (centered,justified,left,right)
\def\captionlessmarginimage {0.1}  % Marg. sup/inf de fig. si no hay ley. [cm]
\def\captionlrmargin {2.0}         % Márgenes izq/der de la leyenda [cm]
\def\captionmarginmultimg {0.45}   % Margen izq/der leyendas múltiple img [cm]
\def\captiontbmarginfigure {9.35}  % Margen sup/inf de la leyenda en fig. [pt]
\def\captiontbmargintable {7.0}    % Margen sup/inf de la leyenda en tab. [pt]
\def\captiontextbold {false}       % Etiqueta (código,figura,tabla) en negrita
\def\codecaptiontop {true}         % Leyenda arriba del código fuente
\def\figurecaptiontop {false}      % Leyenda arriba de las imágenes
\def\showsectioncaption {none}     % N° sec. en objeto (none,sec,ssec,sssec)
\def\tablecaptiontop {true}        % Leyenda arriba de las tablas

% CONFIGURACIÓN DEL ÍNDICE
\def\charafterobjectindex {.}      % Carácter después de n° figura,tabla,código
\def\charnumpageindex {.}          % Carácter número de página en índice
\def\equalmarginnumobject {true}   % Iguala el margen de los números de objetos
\def\indexdepth {4}                % Profundidad máxima del índice
\def\indexforcenewpage {false}     % Fuerza cada índice en una página nueva
\def\indextitlemargin {7.0}        % Margen título índice \insertindextitle [pt]
\def\objectindexindent {true}      % Indenta la lista de objetos
\def\showindex {true}              % Muestra el índice
\def\showindexofcode {true}        % Muestra la lista de códigos fuente
\def\showindexofcontents {true}    % Muestra la lista de contenidos
\def\showindexoffigures {true}     % Muestra la lista de figuras
\def\showindexoftables {true}      % Muestra la lista de tablas

% CONFIGURACIÓN DE LOS COLORES DEL DOCUMENTO
\def\captioncolor {black}          % Color de la ley. (código,figura,tabla)
\def\captiontextcolor {black}      % Color de la leyenda
\def\colorpage {white}             % Color de la página
\def\highlightcolor {yellow}       % Color del subrayado con \hl
\def\indextitlecolor {black}       % Color de los títulos del índice
\def\linenumbercolor {dkgray}      % Color del n° de línea (\showlinenumbers)
\def\linkcolor {black}             % Color de los links del doc.
\def\maintextcolor {black}         % Color principal del texto
\def\numcitecolor {black}          % Color del n° de las referencias o citas
\def\portraittitlecolor {black}    % Color de los títulos de la portada
\def\showborderonlinks {false}     % Color de un link por un recuadro de color
\def\sourcecodebgcolor {lgray}     % Color de fondo del código fuente
\def\ssstitlecolor {black}         % Color de los sub-sub-subtítulos
\def\subsubtitlecolor {black}      % Color de los sub-subtítulos
\def\subtitlecolor {black}         % Color de los subtítulos
\def\tablelinecolor {black}        % Color de las líneas de las tablas
\def\titlecolor {black}            % Color de los títulos
\def\urlcolor {magenta}            % Color de los enlaces web (\url,\href)

% CONFIGURACIÓN DE FIGURAS
\def\defaultimagefolder {img/}     % Carpeta raíz de las imágenes
\def\marginbottomimages {-0.2}     % Margen inferior figura [cm]
\def\marginfloatimages {-13.0}     % Mrg. sup. fig \insertimageleft/right [pt]
\def\marginrightmultimage {0.1}    % Margen derecho imágenes múltiples [cm]
\def\margintopimages {0.0}         % Margen superior figura [cm]

% ANEXO, CITAS, REFERENCIAS
\def\apaciterefsep {9}             % Separación entre refs. {apacite} [pt]
\def\appendixindepobjnum {true}    % Anexo usa n° objetos independientes
\def\bibtexrefsep {9}              % Separación entre refs. {bibtex} [pt]
\def\donumrefsection {false}       % Sección de referencias numerada
\def\natbibrefsep {5}              % Separación entre referencia {natbib} [pt]
\def\sectionappendixlastchar {.}   % Carácter entre n° de sec. anexo y título
\def\showappendixsecindex {false}  % Título de la sec. de anexos en el índice
\def\showappendixsectitle {false}  % Título de la sec. de anexo en el informe
\def\stylecitereferences {bibtex}  % Estilo cita/ref. (bibtex,apacite,natbib)
\def\twocolumnreferences {false}   % Referencias en dos columnas

% CONFIGURACIÓN DE LOS TÍTULOS
\def\anumsecaddtocounter {false}   % Insertar títulos 'anum' aumenta n° de sec
\def\fontsizessstitle{\normalsize} % Tamaño sub-sub-subtítulos
\def\fontsizesubsubtitle {\large}  % Tamaño sub-subtítulos
\def\fontsizesubtitle {\Large}     % Tamaño subtítulos
\def\fontsizetitle {\huge}         % Tamaño títulos
\def\fontsizetitlei {\huge}        % Tamaño títulos en el índice
\def\showdotontitles {true}        % Punto al final de n° (ti/sub/subsub)título
\def\stylessstitle {\bfseries}     % Estilo sub-sub-subtítulos
\def\stylesubsubtitle {\bfseries}  % Estilo sub-subtítulos
\def\stylesubtitle {\bfseries}     % Estilo subtítulos
\def\styletitle {\bfseries}        % Estilo títulos
\def\styletitlei {\bfseries}       % Estilo títulos en el índice

% OPCIONES DEL PDF COMPILADO
\def\addindextobookmarks {false}   % Añade el índice a los marcadores del pdf
\def\cfgbookmarksopenlevel {1}     % Nivel marcadores en pdf (1:secciones)
\def\cfgpdfbookmarkopen {true}     % Expande marcadores del nivel configurado
\def\cfgpdfcenterwindow {true}     % Centra ventana del lector al abrir el pdf
\def\cfgpdfcopyright {}            % Establece el copyright del documento
\def\cfgpdfdisplaydoctitle {true}  % Muestra título del informe en visor
\def\cfgpdffitwindow {false}       % Ajusta la ventana del lector tamaño pdf
\def\cfgpdfmenubar {true}          % Muestra el menú del lector
\def\cfgpdfpagemode {OneColumn}    % Modo de página (OneColumn,SinglePage)
\def\cfgpdfpageview {FitH}         % Fit,FitH,FitV,FitR,FitB,FitBH,FitBV
\def\cfgpdfsecnumbookmarks {true}  % Número de la sec. en marcadores del pdf
\def\cfgpdftoolbar {true}          % Muestra barra de herramientas lector pdf
\def\cfgshowbookmarkmenu {false}   % Muestra menú marcadores al abrir el pdf
\def\pdfcompileversion {7}         % Versión mínima del pdf compilado (1.x)

% NOMBRE DE OBJETOS
\def\nameabstract {Resumen}           % Nombre del resumen-abstract
\def\nameappendixsection {Anexos}     % Nombre de la sec. de anexos/apéndices
\def\nameportraitpage {Portada}       % Etiqueta página de la portada
\def\namereferences {Referencias}     % Nombre de la sección de referencias
\def\nomchapter {Capítulo}            % Nombre de los capítulos
\def\nomltappendixsection {Anexo}     % Etiqueta sección en anexo/apéndices
\def\nomltcont {Índice de Contenidos} % Nombre del índice de contenidos
\def\nomltfigure {Lista de Figuras}   % Nombre del índice de lista de figuras
\def\nomltsrc {Lista de Códigos}      % Nombre del índice de lista de código
\def\nomlttable {Lista de Tablas}     % Nombre del índice de lista de tablas
\def\nomltwfigure {Figura}            % Etiqueta leyenda de las figuras
\def\nomltwsrc {Código}               % Etiqueta leyenda del código fuente
\def\nomltwtable {Tabla}              % Etiqueta leyenda de las tablas
\def\nomnpageof { de }                % Etiqueta página # de #


% IMPORTACIÓN DE LIBRERÍAS
% Template:     Informe/Reporte LaTeX
% Documento:    Importación de librerías
% Versión:      6.0.0 (13/10/2018)
% Codificación: UTF-8
%
% Autor: Pablo Pizarro R. @ppizarror
%        Facultad de Ciencias Físicas y Matemáticas
%        Universidad de Chile
%        pablo.pizarro@ing.uchile.cl, ppizarror.com
%
% Manual template: [http://latex.ppizarror.com/Template-Informe/]
% Licencia MIT:    [https://opensource.org/licenses/MIT/]

\newcommand{\throwbadconfig}[3]{
	\errmessage{LaTeX Warning: #1 \noexpand #2=#2. Valores esperados: #3}
	\stop
}
\usepackage[spanish,es-nosectiondot,es-lcroman,es-noquoting]{babel}
\usepackage{ifthen}
\let\counterwithout\relax
\let\counterwithin\relax
\ifthenelse{\equal{\fontdocument}{lmodern}}{
	\usepackage{lmodern}
}{
\ifthenelse{\equal{\fontdocument}{arial}}{
	\usepackage{helvet}
	\renewcommand{\familydefault}{\sfdefault}
}{
\ifthenelse{\equal{\fontdocument}{accantis}}{
	\usepackage{accanthis}
}{
\ifthenelse{\equal{\fontdocument}{alegreya}}{
	\usepackage{Alegreya}
	\renewcommand*\oldstylenums[1]{{\AlegreyaOsF #1}}
}{
\ifthenelse{\equal{\fontdocument}{alegreyasans}}{
	\usepackage[sfdefault]{AlegreyaSans}
	\renewcommand*\oldstylenums[1]{{\AlegreyaSansOsF #1}}
}{
\ifthenelse{\equal{\fontdocument}{algolrevived}}{
	\usepackage{algolrevived}
}{
\ifthenelse{\equal{\fontdocument}{antiqua}}{
	\usepackage{antiqua}
}{
\ifthenelse{\equal{\fontdocument}{antpolt}}{
	\usepackage{antpolt}
}{
\ifthenelse{\equal{\fontdocument}{antpoltlight}}{
	\usepackage[light]{antpolt}
}{
\ifthenelse{\equal{\fontdocument}{anttor}}{
	\usepackage[math]{anttor}
}{
\ifthenelse{\equal{\fontdocument}{anttorcondensed}}{
	\usepackage[condensed,math]{anttor}
}{
\ifthenelse{\equal{\fontdocument}{anttorlight}}{
	\usepackage[light,math]{anttor}
}{
\ifthenelse{\equal{\fontdocument}{anttorlightcondensed}}{
	\usepackage[light,condensed,math]{anttor}
}{
\ifthenelse{\equal{\fontdocument}{arev}}{
	\usepackage{arev}
}{
\ifthenelse{\equal{\fontdocument}{arimo}}{
	\usepackage[sfdefault]{arimo}
	\renewcommand*\familydefault{\sfdefault}
}{
\ifthenelse{\equal{\fontdocument}{aurical}}{
	\usepackage{aurical}
}{
\ifthenelse{\equal{\fontdocument}{avant}}{
	\usepackage{avant}
}{
\ifthenelse{\equal{\fontdocument}{baskervald}}{
	\usepackage{baskervald}
}{
\ifthenelse{\equal{\fontdocument}{berasans}}{
	\usepackage[scaled]{berasans}
	\renewcommand*\familydefault{\sfdefault}
}{
\ifthenelse{\equal{\fontdocument}{beraserif}}{
	\usepackage{bera}
}{
\ifthenelse{\equal{\fontdocument}{biolinum}}{
	\usepackage{libertine}
	\renewcommand*\familydefault{\sfdefault}
}{
\ifthenelse{\equal{\fontdocument}{cabin}}{
	\usepackage[sfdefault]{cabin}
	\renewcommand*\familydefault{\sfdefault}
}{
\ifthenelse{\equal{\fontdocument}{cabincondensed}}{
	\usepackage[sfdefault,condensed]{cabin}
	\renewcommand*\familydefault{\sfdefault}
}{
\ifthenelse{\equal{\fontdocument}{cantarell}}{
	\usepackage[default]{cantarell}
}{
\ifthenelse{\equal{\fontdocument}{caladea}}{
	\usepackage{caladea}
}{
\ifthenelse{\equal{\fontdocument}{carlito}}{
	\usepackage[sfdefault]{carlito}
	\renewcommand*\familydefault{\sfdefault}
}{
\ifthenelse{\equal{\fontdocument}{chivolight}}{
	\usepackage[familydefault,light]{Chivo}
}{
\ifthenelse{\equal{\fontdocument}{chivoregular}}{
	\usepackage[familydefault,regular]{Chivo}
}{
\ifthenelse{\equal{\fontdocument}{clearsans}}{
	\usepackage[sfdefault]{ClearSans}
	\renewcommand*\familydefault{\sfdefault}
}{
\ifthenelse{\equal{\fontdocument}{comfortaa}}{
	\usepackage[default]{comfortaa}
}{
\ifthenelse{\equal{\fontdocument}{comicneue}}{
	\usepackage[default]{comicneue}
}{
\ifthenelse{\equal{\fontdocument}{comicneueangular}}{
	\usepackage[default,angular]{comicneue}
}{
\ifthenelse{\equal{\fontdocument}{crimson}}{
	\usepackage{crimson}
}{
\ifthenelse{\equal{\fontdocument}{cyklop}}{
	\usepackage{cyklop}
}{
\ifthenelse{\equal{\fontdocument}{dejavusans}}{
	\usepackage{DejaVuSans}
	\renewcommand*\familydefault{\sfdefault}
}{
\ifthenelse{\equal{\fontdocument}{dejavusanscondensed}}{
	\usepackage{DejaVuSansCondensed}
	\renewcommand*\familydefault{\sfdefault}
}{
\ifthenelse{\equal{\fontdocument}{droidsans}}{
	\usepackage[defaultsans]{droidsans}
	\renewcommand*\familydefault{\sfdefault}
}{
\ifthenelse{\equal{\fontdocument}{fetamont}}{
	\usepackage{fetamont}
	\renewcommand*\familydefault{\sfdefault}
}{
\ifthenelse{\equal{\fontdocument}{firasans}}{
	\usepackage[sfdefault]{FiraSans}
	\renewcommand*\familydefault{\sfdefault}
}{
\ifthenelse{\equal{\fontdocument}{iwona}}{
	\usepackage[math]{iwona}
}{
\ifthenelse{\equal{\fontdocument}{iwonacondensed}}{
	\usepackage[math]{iwona}
}{
\ifthenelse{\equal{\fontdocument}{iwonalight}}{
	\usepackage[light,math]{iwona}
}{
\ifthenelse{\equal{\fontdocument}{iwonalightcondensed}}{
	\usepackage[light,condensed,math]{iwona}
}{
\ifthenelse{\equal{\fontdocument}{kurier}}{
	\usepackage[math]{kurier}
}{
\ifthenelse{\equal{\fontdocument}{kuriercondensed}}{
	\usepackage[condensed,math]{kurier}
}{
\ifthenelse{\equal{\fontdocument}{kurierlight}}{
	\usepackage[light,math]{kurier}
}{
\ifthenelse{\equal{\fontdocument}{kurierlightcondensed}}{
	\usepackage[light,condensed,math]{kurier}
}{
\ifthenelse{\equal{\fontdocument}{lato}}{
	\usepackage[default]{lato}
}{
\ifthenelse{\equal{\fontdocument}{libris}}{
	\usepackage{libris}
	\renewcommand*\familydefault{\sfdefault}
}{
\ifthenelse{\equal{\fontdocument}{lxfonts}}{
	\usepackage{lxfonts}
}{
\ifthenelse{\equal{\fontdocument}{merriweather}}{
	\usepackage[sfdefault]{merriweather}
}{
\ifthenelse{\equal{\fontdocument}{merriweatherlight}}{
	\usepackage[sfdefault,light]{merriweather}
}{
\ifthenelse{\equal{\fontdocument}{mintspirit}}{
	\usepackage[default]{mintspirit}
}{
\ifthenelse{\equal{\fontdocument}{montserratalternatesextralight}}{
	\usepackage[defaultfam,extralight,tabular,lining,alternates]{montserrat}
	\renewcommand*\oldstylenums[1]{{\fontfamily{Montserrat-TOsF}\selectfont #1}}
}{
\ifthenelse{\equal{\fontdocument}{montserratalternatesregular}}{
	\usepackage[defaultfam,tabular,lining,alternates]{montserrat}
	\renewcommand*\oldstylenums[1]{{\fontfamily{Montserrat-TOsF}\selectfont #1}}
}{
\ifthenelse{\equal{\fontdocument}{montserratalternatesthin}}{
	\usepackage[defaultfam,thin,tabular,lining,alternates]{montserrat}
	\renewcommand*\oldstylenums[1]{{\fontfamily{Montserrat-TOsF}\selectfont #1}}
}{
\ifthenelse{\equal{\fontdocument}{montserratextralight}}{
	\usepackage[defaultfam,extralight,tabular,lining]{montserrat}
	\renewcommand*\oldstylenums[1]{{\fontfamily{Montserrat-TOsF}\selectfont #1}}
}{
\ifthenelse{\equal{\fontdocument}{montserratlight}}{
	\usepackage[defaultfam,light,tabular,lining]{montserrat}
	\renewcommand*\oldstylenums[1]{{\fontfamily{Montserrat-TOsF}\selectfont #1}}
}{
\ifthenelse{\equal{\fontdocument}{montserratregular}}{
	\usepackage[defaultfam,tabular,lining]{montserrat}
	\renewcommand*\oldstylenums[1]{{\fontfamily{Montserrat-TOsF}\selectfont #1}}
}{
\ifthenelse{\equal{\fontdocument}{montserratthin}}{
	\usepackage[defaultfam,thin,tabular,lining]{montserrat}
	\renewcommand*\oldstylenums[1]{{\fontfamily{Montserrat-TOsF}\selectfont #1}}
}{
\ifthenelse{\equal{\fontdocument}{nimbussans}}{
	\usepackage{nimbussans}
	\renewcommand*\familydefault{\sfdefault}
}{
\ifthenelse{\equal{\fontdocument}{noto}}{
	\usepackage[sfdefault]{noto}
	\renewcommand*\familydefault{\sfdefault}
}{
\ifthenelse{\equal{\fontdocument}{opensans}}{
	\usepackage[default,osfigures,scale=0.95]{opensans}
}{
\ifthenelse{\equal{\fontdocument}{overlock}}{
	\usepackage[sfdefault]{overlock}
	\renewcommand*\familydefault{\sfdefault}
}{
\ifthenelse{\equal{\fontdocument}{paratype}}{
	\usepackage{paratype}
	\renewcommand*\familydefault{\sfdefault}
}{
\ifthenelse{\equal{\fontdocument}{paratypesanscaption}}{
	\usepackage{PTSansCaption}
	\renewcommand*\familydefault{\sfdefault}
}{
\ifthenelse{\equal{\fontdocument}{paratypesansnarrow}}{
	\usepackage{PTSansNarrow}
	\renewcommand*\familydefault{\sfdefault}
}{
\ifthenelse{\equal{\fontdocument}{quattrocento}}{
	\usepackage[sfdefault]{quattrocento}
}{
\ifthenelse{\equal{\fontdocument}{raleway}}{
	\usepackage[default]{raleway}
}{
\ifthenelse{\equal{\fontdocument}{roboto}}{
	\usepackage[sfdefault]{roboto}
}{
\ifthenelse{\equal{\fontdocument}{robotocondensed}}{
	\usepackage[sfdefault,condensed]{roboto}
}{
\ifthenelse{\equal{\fontdocument}{robotolight}}{
	\usepackage[sfdefault,light]{roboto}
}{
\ifthenelse{\equal{\fontdocument}{robotolightcondensed}}{
	\usepackage[sfdefault,light,condensed]{roboto}
}{
\ifthenelse{\equal{\fontdocument}{robotothin}}{
	\usepackage[sfdefault,thin]{roboto}
}{
\ifthenelse{\equal{\fontdocument}{rosario}}{
	\usepackage[familydefault]{Rosario}
}{
\ifthenelse{\equal{\fontdocument}{sourcesanspro}}{
	\usepackage[default]{sourcesanspro}
}{
\ifthenelse{\equal{\fontdocument}{uarial}}{
	\usepackage{uarial}
	\renewcommand*\familydefault{\sfdefault}
}{
\ifthenelse{\equal{\fontdocument}{ugq}}{
	\renewcommand*\sfdefault{ugq}
	\renewcommand*\familydefault{\sfdefault}
}{
\ifthenelse{\equal{\fontdocument}{universalis}}{
	\usepackage[sfdefault]{universalis}
}{
\ifthenelse{\equal{\fontdocument}{universaliscondensed}}{
	\usepackage[condensed,sfdefault]{universalis}
}{
\ifthenelse{\equal{\fontdocument}{venturis}}{
	\usepackage[lf]{venturis}
	\renewcommand*\familydefault{\sfdefault}
}{
	\throwbadconfig{Fuente desconocida}{\fontdocument}{lmodern,arial,helvet,
	accantis,alegreya,alegreyasans,algolrevived,antiqua,antpolt,antpoltlight,
	anttor,anttorcondensed,anttorlight,anttorlightcondensed,arev,arimo,aurical,
	avant,baskervald,berasans,beraserif,biolinum,cabin,cabincondensed,cantarell,
	caladea,carlito,chivolight,chivoregular,clearsans,comfortaa,comicneue,
	comicneueangular,crimson,cyklop,dejavusans,dejavusanscondensed,droidsans,
	firasans,iwona,iwonacondensed,iwonalight,iwonalightcondensed,kurier,
	kuriercondensed,kurierlight,kurierlightcondensed,lato,libris,lxfonts,
	merriweather,merriweatherlight,mintspirit,montserratalternatesextralight,
	montserratalternatesregular,montserratalternatesthin,montserratextralight,
	montserratlight,montserratregular,montserratthin,nimbussans,noto,opensans,
	overlock,paratype,paratypesanscaption,paratypesansnarrow,quattrocento,
	raleway,roboto,robotolight,robotolightcondensed,robotothin,rosario,
	sourcesanspro,uarial,ugq,universalis,universaliscondensed,venturis}
	}}}}}}}}}}}}}}}}}}}}}}}}}}}}}}}}}}}}}}}}}}}}}}}}}}}}}}}}}}}}}}}}}}}}}}}}}}}}}}}}
}
\ifthenelse{\equal{\fonttypewriter}{tmodern}}{
	\renewcommand*\ttdefault{lmvtt}
}{
\ifthenelse{\equal{\fonttypewriter}{anonymouspro}}{
	\usepackage[ttdefault=true]{AnonymousPro}
}{
\ifthenelse{\equal{\fonttypewriter}{ascii}}{
	\usepackage{ascii}
	\let\SI\relax
}{
\ifthenelse{\equal{\fonttypewriter}{beramono}}{
	\usepackage[scaled]{beramono}
}{
\ifthenelse{\equal{\fonttypewriter}{cmpica}}{
	\usepackage{addfont}
	\addfont{OT1}{cmpica}{\pica}
	\addfont{OT1}{cmpicab}{\picab}
	\addfont{OT1}{cmpicati}{\picati}
	\renewcommand*\ttdefault{pica}
}{
\ifthenelse{\equal{\fonttypewriter}{courier}}{
	\usepackage{courier}
}{
\ifthenelse{\equal{\fonttypewriter}{dejavusansmono}}{
	\usepackage[scaled]{DejaVuSansMono}
}{
\ifthenelse{\equal{\fonttypewriter}{firamono}}{
	\usepackage[scale=0.85]{FiraMono}
}{
\ifthenelse{\equal{\fonttypewriter}{gomono}}{
	\usepackage[scale=0.85]{GoMono}
}{
\ifthenelse{\equal{\fonttypewriter}{inconsolata}}{
	\usepackage{inconsolata}
}{
\ifthenelse{\equal{\fonttypewriter}{nimbusmono}}{
	\usepackage{nimbusmono}
}{
\ifthenelse{\equal{\fonttypewriter}{newtxtt}}{
	\usepackage[zerostyle=d]{newtxtt}
}{
\ifthenelse{\equal{\fonttypewriter}{nimbusmono}}{
	\usepackage{nimbusmono}
}{
\ifthenelse{\equal{\fonttypewriter}{nimbusmononarrow}}{
	\usepackage{nimbusmononarrow}
}{
\ifthenelse{\equal{\fonttypewriter}{lcmtt}}{
	\renewcommand*\ttdefault{lcmtt}
}{
\ifthenelse{\equal{\fonttypewriter}{sourcecodepro}}{
	\usepackage[ttdefault=true,scale=0.85]{sourcecodepro}
}{
\ifthenelse{\equal{\fonttypewriter}{texgyrecursor}}{
	\usepackage{tgcursor}
}{
	\throwbadconfig{Fuente desconocida}{\fonttypewriter}{anonymouspro,ascii,beramono,cmpica,courier,dejavusansmono,firamono,gomono,inconsolata,kpmonospaced,lcmtt,newtxtt,nimbusmono,nimbusmononarrow,texgyrecursor,tmodern}
	}}}}}}}}}}}}}}}}
}
\usepackage[T1]{fontenc}
\ifthenelse{\equal{\showlinenumbers}{true}}{
	\usepackage[switch,columnwise,running]{lineno}}{
}
\usepackage{amsmath}
\usepackage{amssymb}
\usepackage{array}
\usepackage{bigstrut}
\usepackage{bm}
\usepackage{booktabs}
\usepackage{caption}
\usepackage{changepage}
\usepackage{chngcntr}
\usepackage{color}
\usepackage{colortbl}
\usepackage{csquotes}
\usepackage{datetime}
\usepackage{floatpag}
\usepackage{floatrow}
\usepackage{framed}
\usepackage{gensymb}
\usepackage{geometry}
\usepackage{graphicx}
\usepackage{lipsum}
\usepackage{listings}
\usepackage{listingsutf8}
\usepackage{longtable}
\usepackage{mathtools}
\usepackage{multicol}
\usepackage{needspace}
\usepackage{pdflscape}
\usepackage{pdfpages}
\usepackage{physics}
\usepackage{rotating}
\usepackage{sectsty}
\usepackage{selinput}
\usepackage{setspace}
\usepackage{siunitx}
\usepackage{soul}
\usepackage{subfig}
\usepackage{textcomp}
\usepackage{url}
\usepackage{wasysym}
\usepackage{wrapfig}
\usepackage{xspace}
\usepackage[makeroom]{cancel}
\usepackage[inline]{enumitem}
\usepackage[bottom,norule,hang]{footmisc}
\usepackage[subfigure,titles]{tocloft}
\usepackage[pdfencoding=auto,psdextra]{hyperref}
\usepackage[figure,table,lstlisting]{totalcount}
\usepackage[normalem]{ulem}
\usepackage[usenames,dvipsnames]{xcolor}
\ifthenelse{\equal{\showdotontitles}{true}}{
	\usepackage{secdot}
	\sectiondot{subsection}
	\sectiondot{subsubsection}}{
}
\ifthenelse{\equal{\stylecitereferences}{natbib}}{
	\usepackage{natbib}
}{
\ifthenelse{\equal{\stylecitereferences}{apacite}}{
		\usepackage{apacite}
	}{
\ifthenelse{\equal{\stylecitereferences}{bibtex}}{
		}{}
	}
}
\ifthenelse{\equal{\showappendixsecindex}{true}}{
	\usepackage[toc]{appendix}
}{
	\usepackage{appendix}
}
\ifthenelse{\equal{\importtikz}{true}}{
	\usepackage{tikz}}{
}
\ifthenelse{\equal{\hfstyle}{style11}}{
	\usepackage{lastpage}}{}
\ifthenelse{\equal{\hfstyle}{style12}}{
	\usepackage{lastpage}}{}
\ifthenelse{\equal{\hfstyle}{style13}}{
	\usepackage{lastpage}}{}
\ifthenelse{\equal{\hfstyle}{style14}}{
	\usepackage{lastpage}}{
}
\usepackage{bookmark}
\usepackage{fancyhdr}
\usepackage{float}
\usepackage{hyperxmp}
\usepackage{multirow}
\usepackage{notoccite}
\usepackage{titlesec}


% IMPORTACIÓN DE FUNCIONES Y ENTORNOS
% Template:     Informe/Reporte LaTeX
% Documento:    Carga las funciones del template
% Versión:      5.5.5 (27/09/2018)
% Codificación: UTF-8
%
% Autor: Pablo Pizarro R. @ppizarror
%        Facultad de Ciencias Físicas y Matemáticas
%        Universidad de Chile
%        pablo.pizarro@ing.uchile.cl, ppizarror.com
%
% Manual template: [http://latex.ppizarror.com/Template-Informe/]
% Licencia MIT:    [https://opensource.org/licenses/MIT/]

% Template:     Informe/Reporte LaTeX
% Documento:    Funciones del núcleo del template
% Versión:      6.1.6 (14/12/2018)
% Codificación: UTF-8
%
% Autor: Pablo Pizarro R. @ppizarror
%        Facultad de Ciencias Físicas y Matemáticas
%        Universidad de Chile
%        pablo.pizarro@ing.uchile.cl, ppizarror.com
%
% Manual template: [https://latex.ppizarror.com/Template-Informe/]
% Licencia MIT:    [https://opensource.org/licenses/MIT/]

\def\GLOBALcoretikzimported {false}
\def\GLOBALenvimageadded {false}
\def\GLOBALenvimageinitialized {false}
\def\GLOBALenvimagenewlinemarg {0.25}
\def\GLOBALsectionalph {false}
\def\GLOBALsectionanumenabled {false}
\def\GLOBALsubsectionanumenabled {false}
\newcommand{\throwerror}[2]{
	\errmessage{LaTeX Error: \noexpand#1 #2 (linea \the\inputlineno)}
	\stop
}
\newcommand{\throwwarning}[1]{
	\errmessage{LaTeX Warning: #1 (linea \the\inputlineno)}
}
\newcommand{\throwbadconfigondoc}[3]{
	\errmessage{#1 \noexpand #2=#2. Valores esperados: #3}
	\stop
}
\newcommand{\checkvardefined}[1]{
	\ifthenelse{\isundefined{#1}}{
		\errmessage{LaTeX Warning: Variable \noexpand#1 no definida}
		\stop}{
	}
}
\newcommand{\checkextravarexist}[2]{
	\ifthenelse{\isundefined{#1}}{
		\errmessage{LaTeX Warning: Variable \noexpand#1 no definida}
		\ifx\hfuzz#2\hfuzz
			\errmessage{LaTeX Warning: Defina la variable en el bloque de INFORMACION DEL DOCUMENTO al comienzo del archivo principal del Template}
		\else
			\errmessage{LaTeX Warning: #2}
		\fi}{
	}
}
\newcommand{\emptyvarerr}[3]{
	\ifx\hfuzz#2\hfuzz
		\errmessage{LaTeX Warning: \noexpand#1 #3 (linea \the\inputlineno)}
	\fi
}
\newcommand{\setcaptionmargincm}[1]{
	\captionsetup{margin=#1cm}
}
\newcommand{\setpagemargincm}[4]{
	\newgeometry{left=#1cm, top=#2cm, right=#3cm, bottom=#4cm}
}
\newcommand{\changemargin}[2]{
	\emptyvarerr{\changemargin}{#1}{Margen izquierdo no definido}
	\emptyvarerr{\changemargin}{#2}{Margen derecho no definido}
	\list{}{\rightmargin#2\leftmargin#1}\item[]
}
\let\endchangemargin=\endlist
\newcommand{\coreimporttikz}{
	\ifthenelse{\equal{\GLOBALcoretikzimported}{false}}{
		\ifthenelse{\equal{\importtikz}{false}}{
			\usepackage{tikz}}{
		}
		\def\GLOBALcoretikzimported{true}
		}{
	}
}
\newcommand{\bgtemplatetestimg}{
	\definecolor{c040302}{RGB}{4, 3, 2}
	\definecolor{cb26929}{RGB}{178,105, 41}
	\definecolor{cfa943c}{RGB}{250, 148, 60}
	\definecolor{cfdc594}{RGB}{253, 197, 148}
	\begin{tikzpicture}[y=1.0pt, x=1.0pt, yscale=-1, xscale=1, inner sep=0pt, outer sep=0pt, opacity=0.1]
	\path[fill=cfdc594](46.3187,318.2550)..controls(46.0737,317.5650)and
	(45.7487,314.0750)..(45.5987,310.5)--(45.3247,304)--(37.8247,303.5)--(30.3247,303)--(29.8247,295.5)-- (29.3247,288)--(21.8247,287.5)--(14.3247,287)--(13.8247,271.5)--(13.3247,256)--(5.8247,255.5)-- (-1.6753,255)--(-1.6753,206.5)--(-1.6753,158)--(5.8247,157.5)--(13.3247,157)--(13.8247,141.5)-- (14.3247,126)--(21.8247,125.5)--(29.3247,125)--(29.8247,109.5)--(30.3247,94)--(37.8247,93.5)-- (45.3247,93)--(45.8247,69.5)--(46.3247,46)--(53.8247,45.5)--(61.3247,45)--(61.8247,29.5)-- (62.3247,14)--(69.8247,13.5)--(77.3247,13)--(77.8247,5.5)--(78.3247,-2)--(94.8247,-2)-- (111.3247,-2)--(111.8247,5.5)--(112.3247,13)--(119.8247,13.5)--(127.3247,14)--(127.8247,21.5)-- (128.3247,29)--(166.8247,29)--(205.3247,29)--(205.8247,21.5)--(206.3247,14)--(213.8247,13.5)-- (221.3247,13)--(221.8247,5.5)--(222.3247,-2)--(230.8247,-2)--(239.3247,-2)--(239.8247,5.5)-- (240.3247,13)--(247.8247,13.5)--(255.3247,14)--(255.8247,21.5)--(256.3247,29)--(263.8247,29.5)-- (271.3247,30)--(271.8247,45.5)--(272.3247,61)--(279.8247,61.5)--(287.3247,62)--(287.8247,85.5)-- (288.3247,109)--(295.8247,109.5)--(303.3247,110)--(303.8247,133.5)--(304.3247,157)--(311.8247,157.5)-- (319.3247,158)--(319.3247,198.5)--(319.3247,239)--(311.8247,239.5)--(304.3247,240)--(303.8247,255.5)-- (303.3247,271)--(295.8247,271.5)--(288.3247,272)--(287.8247,279.5)--(287.3247,287)--(279.8247,287.5)-- (272.3247,288)--(271.8247,295.5)--(271.3247,303)--(263.8247,303.5)--(256.3247,304)--(255.8247,311.5)-- (255.3247,319)--(151.0447,319.2550)..controls(68.1357,319.4570)and (46.6737,319.2530)..(46.3187,318.2550)--cycle(173.8247,142.5)-- (173.8247,127.5)--(166.8247,127.5)--(159.8247,127.5)-- (159.8247,142.5)--(159.8247,157.5)--(166.8247,157.5)-- (173.8247,157.5)--cycle(269.8247,126.5)--(269.8247,111.5)-- (262.8247,111.5)--(255.8247,111.5)--(255.8247,126.5)-- (255.8247,141.5)--(262.8247,141.5)--(269.8247,141.5)--cycle; \path[fill=cfa943c](47.3187,317.2550)..controls(47.0737,316.5650)and (46.7487,313.0750)..(46.5987,309.5)--(46.3247,303)--(38.8247,302.5)--(31.3247,302)--(30.8247,294.5)-- (30.3247,287)--(22.8247,286.5)--(15.3247,286)--(14.8247,270.5)--(14.3247,255)--(6.8247,254.5)-- (-0.6753,254)--(-0.6753,206.5)--(-0.6753,159)--(6.8247,158.5)--(14.3247,158)--(14.8247,142.5)-- (15.3247,127)--(22.8247,126.5)--(30.3247,126)--(30.8247,110.5)--(31.3247,95)--(38.8247,94.5)-- (46.3247,94)--(46.8247,70.5)--(47.3247,47)--(54.8247,46.5)--(62.3247,46)--(62.8247,30.5)-- (63.3247,15)--(70.8247,14.5)--(78.3247,14)--(78.8247,6.5)--(79.3247,-1)--(94.8247,-1)-- (110.3247,-1)--(110.8247,6.5)--(111.3247,14)--(118.8247,14.5)--(126.3247,15)--(126.8247,22.5)-- (127.3247,30)--(166.8247,30)--(206.3247,30)--(206.8247,22.5)--(207.3247,15)--(214.8247,14.5)-- (222.3247,14)--(222.8247,6.5)--(223.3247,-1)--(230.8247,-1)--(238.3247,-1)--(238.8247,6.5)-- (239.3247,14)--(246.8247,14.5)--(254.3247,15)--(254.8247,22.5)--(255.3247,30)--(262.8247,30.5)-- (270.3247,31)--(270.8247,46.5)--(271.3247,62)--(278.8247,62.5)--(286.3247,63)--(286.8247,86.5)-- (287.3247,110)--(294.8247,110.5)--(302.3247,111)--(302.8247,134.5)--(303.3247,158)--(310.8247,158.5)-- (318.3247,159)--(318.3247,198.5)--(318.3247,238)--(310.8247,238.5)--(303.3247,239)--(302.8247,254.5)-- (302.3247,270)--(294.8247,270.5)--(287.3247,271)--(286.8247,278.5)--(286.3247,286)--(278.8247,286.5)-- (271.3247,287)--(270.8247,294.5)--(270.3247,302)--(262.8247,302.5)--(255.3247,303)--(254.8247,310.5)-- (254.3247,318)--(151.0447,318.2550)..controls(68.9357,318.4570)and (47.6737,318.2520)..(47.3187,317.2550)--cycle(253.8247,294.70).. controls(253.8247,290.7440)and(254.3267,287.3980)..(255.0247,286.70).. controls(255.7227,286.0020)and(259.0687,285.5)..(263.0247,285.5)-- (269.8247,285.5)--(269.8247,278.70)..controls(269.8247,274.7440)and (270.3267,271.3980)..(271.0247,270.70)..controls(271.7227,270.0020)and (275.0687,269.5)..(279.0247,269.5)--(285.8247,269.5)-- (285.8247,254.70)..controls(285.8247,244.5220)and(286.1997,239.5250).. (287.0247,238.70)..controls(287.7227,238.0020)and(291.0687,237.5).. (295.0247,237.5)--(301.8247,237.5)--(301.8247,206)-- (301.8247,174.5)--(294.8247,174.5)--(287.8247,174.5)-- (287.8247,189.80)..controls(287.8247,200.3670)and(287.4537,205.4710).. (286.6247,206.30)..controls(285.9267,206.9980)and(282.5807,207.5).. (278.6247,207.5)--(271.8247,207.5)--(271.8247,214.30)..controls (271.8247,218.2560)and(271.3227,221.6020)..(270.6247,222.30)..controls (269.9267,222.9980)and(266.5807,223.5)..(262.6247,223.5)-- (255.8247,223.5)--(255.8247,230.3780)..controls(255.8247,234.9380)and (255.3687,237.6340)..(254.4727,238.3780)..controls(252.4847,240.0280)and (176.6787,239.9540)..(175.0247,238.30)..controls(174.3337,237.6090)and (173.8247,234.30)..(173.8247,230.5)..controls(173.8247,226.70)and (174.3337,223.3910)..(175.0247,222.70)..controls(175.9167,221.8080)and (186.1807,221.5)..(215.0247,221.5)--(253.8247,221.5)-- (253.8247,214.5)--(253.8247,207.5)--(249.0747,207.3720)..controls (246.4627,207.3020)and(235.9997,206.9640)..(225.8247,206.6220)-- (207.3247,206)--(206.8247,198.5)--(206.3247,191)--(190.8247,190.5)--(175.3247,190)--(174.8247,182.5)-- (174.3247,175)--(134.8247,175)--(95.3247,175)--(94.8247,182.5)--(94.3247,190)--(86.8247,190.5)-- (79.3247,191)--(79.3247,222.5)--(79.3247,254)--(86.8247,254.5)--(94.3247,255)--(94.8247,270.5)-- (95.3247,286)--(102.8247,286.5)--(110.3247,287)-- (110.6187,294.2500)--(110.9127,301.5)--(182.3687,301.5)-- (253.8247,301.5)--cycle(174.6247,158.30)..controls (176.4747,156.4500)and(176.2867,128.1290)..(174.4147,126.5750)..controls (173.5527,125.8600)and(170.3467,125.5210)..(166.1647,125.7030)-- (159.3247,126)--(159.0477,141.4580)..controls(158.8957,149.9600)and (158.9937,157.4970)..(159.2667,158.2080)..controls(159.9187,159.9070)and (172.9397,159.9850)..(174.6257,158.30)--cycle(270.6247,142.30).. controls(272.4747,140.4500)and(272.2867,112.1290)..(270.4147,110.5750).. controls(269.5527,109.8600)and(266.3467,109.5210)..(262.1647,109.7030)-- (255.3247,110)--(255.0477,125.4580)..controls(254.8957,133.9600)and (254.9937,141.4970)..(255.2667,142.2080)..controls(255.9187,143.9070)and (268.9397,143.9850)..(270.6257,142.30)--cycle(78.8247,86.5)-- (79.3247,79)--(86.8247,78.5)--(94.3247,78)--(94.8247,70.5)--(95.3247,63)--(102.8247,62.5)-- (110.3247,62)--(110.8247,54.5)--(111.3247,47)-- (118.5747,46.7060)--(125.8247,46.4120)--(125.8247,38.9560)-- (125.8247,31.5)--(119.0247,31.5)..controls(115.0687,31.5)and (111.7227,30.9980)..(111.0247,30.30)..controls(110.3267,29.6020)and (109.8247,26.2560)..(109.8247,22.30)--(109.8247,15.5)-- (94.8247,15.5)--(79.8247,15.5)--(79.8247,30.30)..controls (79.8247,40.4780)and(79.4497,45.4750)..(78.6247,46.30)..controls (77.9267,46.9980)and(74.5807,47.5)..(70.6247,47.5)-- (63.8247,47.5)--(63.8247,71.0440)--(63.8247,94.5880)-- (71.0747,94.2940)--(78.3247,94)--cycle(253.8247,39)-- (253.8247,31.5)--(247.0247,31.5)..controls(243.0687,31.5)and (239.7227,30.9980)..(239.0247,30.30)..controls(238.3267,29.6020)and (237.8247,26.2560)..(237.8247,22.30)--(237.8247,15.5)-- (230.8247,15.5)--(223.8247,15.5)--(223.8247,22.30)..controls (223.8247,26.2560)and(223.3227,29.6020)..(222.6247,30.30)..controls (221.9267,30.9980)and(218.5807,31.5)..(214.6247,31.5)-- (207.8247,31.5)--(207.8247,39)--(207.8247,46.5)-- (230.8247,46.5)--(253.8247,46.5)--cycle; \path[fill=cb26929](47.3187,317.2550)..controls(47.0737,316.5650)and (46.7487,313.0750)..(46.5987,309.5)--(46.3247,303)--(38.8247,302.5)--(31.3247,302)--(30.8247,294.5)-- (30.3247,287)--(22.8247,286.5)--(15.3247,286)--(14.8247,270.5)--(14.3247,255)--(6.8247,254.5)-- (-0.6753,254)--(-0.6753,206.5)--(-0.6753,159)--(6.8247,158.5)--(14.3247,158)--(14.8247,142.5)-- (15.3247,127)--(22.8247,126.5)--(30.3247,126)--(30.8247,110.5)--(31.3247,95)--(38.8247,94.5)-- (46.3247,94)--(46.8247,70.5)--(47.3247,47)--(54.8247,46.5)--(62.3247,46)--(62.8247,30.5)-- (63.3247,15)--(70.8247,14.5)--(78.3247,14)--(78.8247,6.5)--(79.3247,-1)--(94.8247,-1)-- (110.3247,-1)--(110.8247,6.5)--(111.3247,14)--(118.8247,14.5)--(126.3247,15)--(126.8247,22.5)-- (127.3247,30)--(166.8247,30)--(206.3247,30)--(206.8247,22.5)--(207.3247,15)--(214.8247,14.5)-- (222.3247,14)--(222.8247,6.5)--(223.3247,-1)--(230.8247,-1)--(238.3247,-1)--(238.8247,6.5)-- (239.3247,14)--(246.8247,14.5)--(254.3247,15)--(254.8247,22.5)--(255.3247,30)--(262.8247,30.5)-- (270.3247,31)--(270.8247,46.5)--(271.3247,62)--(278.8247,62.5)--(286.3247,63)--(286.8247,86.5)-- (287.3247,110)--(294.8247,110.5)--(302.3247,111)--(302.8247,134.5)--(303.3247,158)--(310.8247,158.5)-- (318.3247,159)--(318.3247,198.5)--(318.3247,238)--(310.8247,238.5)--(303.3247,239)--(302.8247,254.5)-- (302.3247,270)--(294.8247,270.5)--(287.3247,271)--(286.8247,278.5)--(286.3247,286)--(278.8247,286.5)-- (271.3247,287)--(270.8247,294.5)--(270.3247,302)--(262.8247,302.5)--(255.3247,303)--(254.8247,310.5)-- (254.3247,318)--(151.0447,318.2550)..controls(68.9357,318.4570)and (47.6737,318.2520)..(47.3187,317.2550)--cycle(253.8247,294.70).. controls(253.8247,290.7440)and(254.3267,287.3980)..(255.0247,286.70).. controls(255.7227,286.0020)and(259.0687,285.5)..(263.0247,285.5)-- (269.8247,285.5)--(269.8247,278.70)..controls(269.8247,274.7440)and (270.3267,271.3980)..(271.0247,270.70)..controls(271.7227,270.0020)and (275.0687,269.5)..(279.0247,269.5)--(285.8247,269.5)-- (285.8247,254.70)..controls(285.8247,244.5220)and(286.1997,239.5250).. (287.0247,238.70)..controls(287.7227,238.0020)and(291.0717,237.5).. (295.0367,237.5)--(301.8487,237.5)--(301.5867,198.7500)-- (301.3247,160)--(294.8247,160)--(288.3247,160)-- (288.0547,182.4210)..controls(287.8807,196.9060)and(287.3947,205.3120).. (286.6827,206.1710)..controls(285.9607,207.0410)and(283.2017,207.5).. (278.7027,207.5)--(271.8247,207.5)--(271.8247,214.30)..controls (271.8247,218.2560)and(271.3227,221.6020)..(270.6247,222.30)..controls (269.9267,222.9980)and(266.5807,223.5)..(262.6247,223.5)-- (255.8247,223.5)--(255.8247,230.3780)..controls(255.8247,234.9380)and (255.3687,237.6340)..(254.4727,238.3780)..controls(252.4847,240.0280)and (176.6787,239.9540)..(175.0247,238.30)..controls(174.3337,237.6090)and (173.8247,234.30)..(173.8247,230.5)..controls(173.8247,226.70)and (174.3337,223.3910)..(175.0247,222.70)..controls(175.9167,221.8080)and (186.1807,221.5)..(215.0247,221.5)--(253.8247,221.5)-- (253.8247,214.5)--(253.8247,207.5)--(247.0747,207.4850)..controls (243.3627,207.4770)and(239.7747,207.1200)..(239.1027,206.6930)..controls (238.2507,206.1520)and(237.7967,201.3880)..(237.6027,190.9580)-- (237.3247,176)--(229.8247,175.5)--(222.3247,175)--(222.3247,166.5)--(222.3247,158)--(253.8247,157.5)-- (285.3247,157)--(285.3247,134.5)--(285.3247,112)--(277.8247,111.5)--(270.3247,111)--(269.8247,87.5)-- (269.3247,64)--(261.8247,63.5)--(254.3247,63)-- (254.0497,47.2500)--(253.7737,31.5)--(246.9987,31.5)..controls (243.0637,31.5)and(239.7217,30.9970)..(239.0247,30.30)..controls (238.3267,29.6020)and(237.8247,26.2560)..(237.8247,22.30)-- (237.8247,15.5)--(230.8247,15.5)--(223.8247,15.5)-- (223.8247,22.30)..controls(223.8247,26.2560)and(223.3227,29.6020).. (222.6247,30.30)..controls(221.9277,30.9970)and(218.5897,31.5).. (214.6667,31.5)--(207.9087,31.5)--(207.6167,39.2500)-- (207.3247,47)--(166.8247,47)--(126.3247,47)-- (126.0327,39.2500)--(125.7407,31.5)--(118.9827,31.5)..controls (115.0597,31.5)and(111.7217,30.9970)..(111.0247,30.30)..controls (110.3267,29.6020)and(109.8247,26.2560)..(109.8247,22.30)-- (109.8247,15.5)--(94.8247,15.5)--(79.8247,15.5)-- (79.8247,30.30)..controls(79.8247,40.4780)and(79.4497,45.4750).. (78.6247,46.30)..controls(77.9277,46.9970)and(74.5847,47.5).. (70.6427,47.5)--(63.8607,47.5)--(63.5927,71.2500)-- (63.3247,95)--(55.8247,95.5)--(48.3247,96)--(47.8247,111.5)--(47.3247,127)--(39.8247,127.5)-- (32.3247,128)--(31.8247,143.5)--(31.3247,159)--(23.8247,159.5)--(16.3247,160)--(16.3247,175)-- (16.3247,190)--(23.3247,190.5)--(30.3247,191)--(30.8247,198.5)--(31.3247,206)--(38.8247,206.5)-- (46.3247,207)--(46.8247,214.5)--(47.3247,222)--(62.8247,222.5)--(78.3247,223)--(78.8247,238.5)-- (79.3247,254)--(86.8247,254.5)--(94.3247,255)--(94.8247,270.5)--(95.3247,286)--(102.8247,286.5)-- (110.3247,287)--(110.6187,294.2500)--(110.9127,301.5)-- (182.3687,301.5)--(253.8247,301.5)--cycle(126.2657,158.2080).. controls(125.9927,157.4970)and(125.8947,149.9600)..(126.0467,141.4580)-- (126.3247,126)--(142.5747,125.7250)--(158.8247,125.4500)-- (158.8247,142.4750)--(158.8247,159.5)--(142.7937,159.5)..controls (130.6137,159.5)and(126.6437,159.1900)..(126.2667,158.2080)-- cycle(222.2657,142.2080)..controls(221.9927,141.4970)and (221.8947,133.9600)..(222.0467,125.4580)--(222.3247,110)-- (238.5747,109.7250)--(254.8247,109.4500)--(254.8247,126.4750)-- (254.8247,143.5)--(238.7937,143.5)..controls(226.6137,143.5)and (222.6437,143.1900)..(222.2667,142.2080)--cycle; \path[fill=c040302](46.3247,303)--(79.0747,303.0080)..controls (95.9877,303.0120)and(142.2457,303.1250)..(181.8707,303.2580)-- (253.9167,303.5)--(253.6207,310.2500)--(253.3247,317)-- (151.0447,317.2550)..controls(65.3137,317.4680)and(48.6807,317.2880).. (48.2407,316.1430)..controls(46.9629,307.2110)and(46.3247,303).. (46.3247,303)--cycle(32.2567,300.1840)..controls(31.9597,299.4100) and(31.8537,296.1260)..(32.0207,292.8880)--(32.3247,287)-- (39.3247,287)--(46.3247,287)--(46.3247,294)-- (46.3247,301)--(39.5607,301.2960)..controls(34.4617,301.5190)and (32.6637,301.2460)..(32.2567,300.1840)--cycle(255.8247,294.5460)-- (255.8247,287.5)--(262.8707,287.5)--(269.9167,287.5)-- (269.6207,294.2500)--(269.3247,301)--(262.5747,301.2960)-- (255.8247,301.5920)--cycle(16.2817,284.2480)..controls(15.9977,283.5090) and(15.8917,276.6260)..(16.0457,268.9520)--(16.3247,255)-- (23.3247,255)--(30.3247,255)--(30.3247,270)-- (30.3247,285)--(23.5607,285.2960)..controls(18.6047,285.5130)and (16.6587,285.2330)..(16.2807,284.2480)--cycle(271.8247,278.5460)-- (271.8247,271.5)--(278.8707,271.5)--(285.9167,271.5)-- (285.6207,278.2500)--(285.3247,285)--(278.5747,285.2960)-- (271.8247,285.5920)--cycle(287.8247,254.5460)--(287.8247,239.5)-- (294.8517,239.5)--(301.8787,239.5)--(301.6017,254.2500)-- (301.3247,269)--(294.5747,269.2960)--(287.8247,269.5920)-- cycle(0.3007,252.2960)..controls(0.0267,251.5830)and(-0.0793,230.5250).. (0.0637,205.5)--(0.3247,160)--(6.9587,159.7070)-- (13.5927,159.4140)--(14.2807,190.7070)..controls(14.6597,207.9180)and (14.8247,228.9750)..(14.6467,237.5)--(14.3247,253)-- (7.5607,253.2960)..controls(2.7217,253.5080)and(0.6557,253.2230).. (0.2997,252.2960)--cycle(175.8247,230.5)--(175.8247,223.5)-- (214.8247,223.5)--(253.8247,223.5)--(253.8247,230.5)-- (253.8247,237.5)--(214.8247,237.5)--(175.8247,237.5)-- cycle(303.8247,198.5)--(303.8247,159.4090)--(310.5747,159.7050)-- (317.3247,160.0010)--(317.3247,198.5010)--(317.3247,237.0010)-- (310.5747,237.2970)--(303.8247,237.5930)--cycle(255.8247,214.70).. controls(255.8247,210.7440)and(255.3227,207.3980)..(254.6247,206.70).. controls(253.9267,206.0020)and(250.5807,205.5)..(246.6247,205.5)-- (239.8247,205.5)--(239.8247,190.70)..controls(239.8247,180.5220)and (239.4497,175.5250)..(238.6247,174.70)..controls(237.9267,174.0020)and (234.5807,173.5)..(230.6247,173.5)--(223.8247,173.5)-- (223.8247,166.5)--(223.8247,159.5)--(254.8247,159.5)-- (285.8247,159.5)--(285.8247,182.5)--(285.8247,205.5)-- (279.0247,205.5)..controls(275.0687,205.5)and(271.7227,206.0020).. (271.0247,206.70)..controls(270.3267,207.3980)and(269.8247,210.7440).. (269.8247,214.70)--(269.8247,221.5)--(262.8247,221.5)-- (255.8247,221.5)--cycle(16.0477,142.7500)--(16.3247,128)-- (23.0747,127.7040)--(29.8247,127.4090)--(29.8247,142.4540)-- (29.8247,157.5)--(22.7977,157.5)--(15.7717,157.5)-- cycle(127.8247,142.5)--(127.8247,127.5)--(142.8247,127.5)-- (157.8247,127.5)--(157.8247,142.5)--(157.8247,157.5)-- (142.8247,157.5)--(127.8247,157.5)--cycle(287.8247,134.4540)-- (287.8247,111.4080)--(294.5747,111.7040)--(301.3247,112)-- (301.5937,134.7500)--(301.8627,157.5)--(294.8437,157.5)-- (287.8247,157.5)--cycle(223.8247,126.5)--(223.8247,111.5)-- (238.8247,111.5)--(253.8247,111.5)--(253.8247,126.5)-- (253.8247,141.5)--(238.8247,141.5)--(223.8247,141.5)-- cycle(32.0477,110.7500)--(32.3247,96)--(39.0747,95.7040)-- (45.8247,95.4090)--(45.8247,110.4540)--(45.8247,125.5)-- (38.7977,125.5)--(31.7717,125.5)--cycle(271.8247,86.4540)-- (271.8247,63.4090)--(278.5747,63.7050)--(285.3247,64.0010)-- (285.5937,86.7510)--(285.8627,109.5010)--(278.8437,109.5010)-- (271.8247,109.5010)--cycle(48.0557,70.7500)--(48.3247,48)-- (55.0747,47.7040)--(61.8247,47.4090)--(61.8247,70.4540)-- (61.8247,93.5)--(54.8057,93.5)--(47.7867,93.5)-- cycle(255.8247,46.4540)--(255.8247,31.4090)--(262.5747,31.7050)-- (269.3247,32.0010)--(269.6017,46.7510)--(269.8787,61.5010)-- (262.8517,61.5010)--(255.8247,61.5010)--cycle(64.0477,30.7500)-- (64.3247,16)--(71.0747,15.7040)--(77.8247,15.4090)--(77.8247,30.4540)--(77.8247,45.5)--(70.7977,45.5)-- (63.7707,45.5)--cycle(127.8247,38.5)--(127.8247,31.5)--(166.8247,31.5)--(205.8247,31.5)--(205.8247,38.5)-- (205.8247,45.5)--(166.8247,45.5)--(127.8247,45.5)-- cycle(111.8247,22.4540)--(111.8247,15.4080)--(118.5747,15.7040)-- (125.3247,16)--(125.6207,22.7500)--(125.9167,29.5)-- (118.8707,29.5)--(111.8247,29.5)--cycle(208.0287,22.7500)-- (208.3247,16)--(215.0747,15.7040)--(221.8247,15.4090)-- (221.8247,22.4540)--(221.8247,29.5)--(214.7787,29.5)-- (207.7327,29.5)--cycle(239.8247,22.4540)--(239.8247,15.4080)-- (246.5747,15.7040)--(253.3247,16)--(253.6207,22.7500)-- (253.9167,29.5)--(246.8707,29.5)--(239.8247,29.5)--cycle(80.0287,6.7500)--(80.3247,0)--(94.8247,0)-- (109.3247,0)--(109.6207,6.7500)--(109.9167,13.5)--(94.8247,13.5)--(79.7337,13.5)--cycle(224.0287,6.7500)-- (224.3247,0)--(230.8247,0)--(237.3247,0)--(237.6207,6.7500)--(237.9167,13.5)--(230.8247,13.5)-- (223.7337,13.5)--cycle;
	\end{tikzpicture}
}
\def\bgtemplatetestcode {d0g3}
\newcommand{\checkonlyonenvimage}{\ifthenelse{\equal{\GLOBALenvimageinitialized}{true}}{}{\throwwarning{Funciones \noexpand\addimage o \noexpand\addimageboxed no pueden usarse fuera del entorno \noexpand\images}\stop}}
\newcommand{\checkoutsideenvimage}{
	\ifthenelse{\equal{\GLOBALenvimageinitialized}{true}}{
		\throwwarning{Esta funcion solo puede usarse fuera del entorno \noexpand\images}
		\stop}{
	}
}
% Template:     Informe/Reporte LaTeX
% Documento:    Funciones matemáticas
% Versión:      6.0.1 (21/10/2018)
% Codificación: UTF-8
%
% Autor: Pablo Pizarro R. @ppizarror
%        Facultad de Ciencias Físicas y Matemáticas
%        Universidad de Chile
%        pablo.pizarro@ing.uchile.cl, ppizarror.com
%
% Manual template: [http://latex.ppizarror.com/Template-Informe/]
% Licencia MIT:    [https://opensource.org/licenses/MIT/]

\newcommand{\lpow}[2]{
	\ensuremath{{#1}_{#2}}
}
\newcommand{\pow}[2]{
	\ensuremath{{#1}^{#2}}
}
\newcommand{\aasin}[1][]{
	\ifx\hfuzz#1\hfuzz
		\ensuremath{\sin^{-1}#1}
	\else
		\ensuremath{{\sin}^{-1}}
	\fi
}
\newcommand{\aacos}[1][]{
	\ifx\hfuzz#1\hfuzz
		\ensuremath{\cos^{-1}#1}
	\else
		\ensuremath{\cos^{-1}}
	\fi
}
\newcommand{\aatan}[1][]{
	\ifx\hfuzz#1\hfuzz
		\ensuremath{\tan^{-1}#1}
	\else
		\ensuremath{\tan^{-1}}
	\fi
}
\newcommand{\aacsc}[1][]{
	\ifx\hfuzz#1\hfuzz
		\ensuremath{\csc^{-1}#1}
	\else
		\ensuremath{\csc^{-1}}
	\fi
}
\newcommand{\aasec}[1][]{
	\ifx\hfuzz#1\hfuzz
		\ensuremath{\sec^{-1}#1}
	\else
		\ensuremath{\sec^{-1}}
	\fi
}
\newcommand{\aacot}[1][]{
	\ifx\hfuzz#1\hfuzz
		\ensuremath{\cot^{-1}#1}
	\else
		\ensuremath{\cot^{-1}}
	\fi
}
\newcommand{\fracpartial}[2]{
	\ensuremath{\pdv{#1}{#2}}
}
\newcommand{\fracdpartial}[2]{
	\ensuremath{\pdv[2]{#1}{#2}}
}
\newcommand{\fracnpartial}[3]{
	\ensuremath{\pdv[#3]{#1}{#2}}
}
\newcommand{\fracderivat}[2]{
	\ensuremath{\dv{#1}{#2}}
}
\newcommand{\fracdderivat}[2]{
	\ensuremath{\dv[2]{#1}{#2}}
}
\newcommand{\fracnderivat}[3]{
	\ensuremath{\dv[#3]{#1}{#2}}
}
\newcommand{\topequal}[2]{
	\ensuremath{\overbrace{#1}^{\mathclap{#2}}}
}
\newcommand{\underequal}[2]{
	\ensuremath{\underbrace{#1}_{\mathclap{#2}}}
}
\newcommand{\topsequal}[2]{
	\ensuremath{\overbracket{#1}^{\mathclap{#2}}}
}
\newcommand{\undersequal}[2]{
	\ensuremath{\underbracket{#1}_{\mathclap{#2}}}
}
% Template:     Informe/Reporte LaTeX
% Documento:    Funciones para insertar ecuaciones
% Versión:      6.0.0 (13/10/2018)
% Codificación: UTF-8
%
% Autor: Pablo Pizarro R. @ppizarror
%        Facultad de Ciencias Físicas y Matemáticas
%        Universidad de Chile
%        pablo.pizarro@ing.uchile.cl, ppizarror.com
%
% Manual template: [http://latex.ppizarror.com/Template-Informe/]
% Licencia MIT:    [https://opensource.org/licenses/MIT/]

\newcommand{\equationresize}[2]{
	\emptyvarerr{\equationresize}{#1}{Dimension no definida}
	\emptyvarerr{\equationresize}{#2}{Ecuacion a redimensionar no definida}
	\resizebox{#1\textwidth}{!}{$#2$}
}
\newcommand{\insertequation}[2][]{
	\emptyvarerr{\insertequation}{#2}{Ecuacion no definida}
	\ifthenelse{\equal{\numberedequation}{true}}{
		\vspace{-0.1cm}
		\begin{equation}
			\text{#1} #2
		\end{equation}
		\vspace{-0.26cm}
	}{
		\ifx\hfuzz#1\hfuzz
		\else
			\throwwarning{Label invalido en ecuacion sin numero}
		\fi
		\insertequationanum{#2}
	}
}
\newcommand{\insertequationanum}[1]{
	\emptyvarerr{\insertequationanum}{#1}{Ecuacion no definida}
	\vspace{-0.1cm}
	\begin{equation*}
		\ensuremath{#1}
	\end{equation*}
	\vspace{-0.26cm}
}
\newcommand{\insertequationcaptioned}[3][]{
	\emptyvarerr{\insertequationcaptioned}{#2}{Ecuacion no definida}
	\ifx\hfuzz#3\hfuzz
		\insertequation[#1]{#2}
	\else
		\ifthenelse{\equal{\numberedequation}{true}}{
			\vspace{-0.10cm}
			\begin{equation}
				\text{#1} #2
			\end{equation}
			\vspace{-0.65cm}
			\begin{changemargin}{\captionlrmargin cm}{\captionlrmargin cm}
				\centering \textcolor{\captiontextcolor}{#3}
				\vspace{0.05cm}
			\end{changemargin}
			\vspace{0.05cm}
		}{
			\ifx\hfuzz#1\hfuzz
			\else
				\throwwarning{Label invalido en ecuacion sin numero}
			\fi
			\insertequationcaptionedanum{#2}{#3}
		}
	\fi
}
\newcommand{\insertequationcaptionedanum}[2]{
	\emptyvarerr{\insertequationcaptionedanum}{#1}{Ecuacion no definida}
	\ifx\hfuzz#2\hfuzz
		\insertequationanum{#1}
	\else
		\vspace{-0.10cm}
		\begin{equation*}
			\ensuremath{#1}
		\end{equation*}
		\vspace{-0.65cm}
		\begin{changemargin}{\captionlrmargin cm}{\captionlrmargin cm}
			\centering \textcolor{\captiontextcolor}{#2}
			\vspace{0.05cm}
		\end{changemargin}
		\vspace{0.05cm}
	\fi
}
\newcommand{\insertgather}[1]{
	\emptyvarerr{\insertgather}{#1}{Ecuacion no definida}
	\ifthenelse{\equal{\numberedequation}{true}}{
		\vspace{-0.4cm}
		\begin{gather}
			\ensuremath{#1}
		\end{gather}
		\vspace{-0.4cm}
	}{
		\insertgatheranum{#1}
	}
}
\newcommand{\insertgatheranum}[1]{
	\emptyvarerr{\insertgatheranum}{#1}{Ecuacion no definida}
	\vspace{-0.4cm}
	\begin{gather*}
		\ensuremath{#1}
	\end{gather*}
	\vspace{-0.4cm}
}
\newcommand{\insertgathercaptioned}[2]{
	\emptyvarerr{\insertgathercaptioned}{#1}{Ecuacion no definida}
	\ifx\hfuzz#2\hfuzz
		\insertgather{#1}
	\else
		\ifthenelse{\equal{\numberedequation}{true}}{
			\vspace{-0.45cm}
			\begin{gather}
				\ensuremath{#1}
			\end{gather}
			\vspace{-0.77cm}
			\begin{changemargin}{\captionlrmargin cm}{\captionlrmargin cm}
				\centering \textcolor{\captiontextcolor}{#2}
				\vspace{0.05cm}
			\end{changemargin}
			\vspace{0cm}
		}{
			\insertgathercaptionedanum{#1}{#2}
		}
	\fi
}
\newcommand{\insertgathercaptionedanum}[2]{
	\emptyvarerr{\insertgathercaptionedanum}{#1}{Ecuacion no definida}
	\ifx\hfuzz#2\hfuzz
		\insertgatheranum{#1}
	\else
		\vspace{-0.45cm}
		\begin{gather*}
			\ensuremath{#1}
		\end{gather*}
		\vspace{-0.77cm}
		\begin{changemargin}{\captionlrmargin cm}{\captionlrmargin cm}
			\centering \textcolor{\captiontextcolor}{#2}
			\vspace{0.05cm}
		\end{changemargin}
		\vspace{0cm}
	\fi
}
\newcommand{\insertgathered}[2][]{
	\emptyvarerr{\insertgathered}{#2}{Ecuacion no definida}
	\ifthenelse{\equal{\numberedequation}{true}}{
		\vspace{-0.1cm}
		\begin{equation}
			\begin{gathered}
				\text{#1} \ensuremath{#2}
			\end{gathered}
		\end{equation}
		\vspace{-0.05cm}
	}{
		\ifx\hfuzz#1\hfuzz
		\else
			\throwwarning{Label invalido en ecuacion (gathered) sin numero}
		\fi
		\insertgatheredanum{#2}
	}
}
\newcommand{\insertgatheredanum}[1]{
	\emptyvarerr{\insertgatheredanum}{#1}{Ecuacion no definida}
	\vspace{-0.4cm}
	\begin{gather*}
		\ensuremath{#1}
	\end{gather*}
	\vspace{-0.4cm}
}
\newcommand{\insertgatheredcaptioned}[3][]{
	\emptyvarerr{\insertgatheredcaptioned}{#2}{Ecuacion no definida}
	\ifx\hfuzz#3\hfuzz
		\insertgathered[#1]{#2}
	\else
		\ifthenelse{\equal{\numberedequation}{true}}{
			\vspace{0cm}
			\begin{equation}
				\begin{gathered}
					\text{#1} \ensuremath{#2}
				\end{gathered}
			\end{equation}
			\vspace{-0.65cm}
			\begin{changemargin}{\captionlrmargin cm}{\captionlrmargin cm}
				\centering \textcolor{\captiontextcolor}{#3}
				\vspace{0.05cm}
			\end{changemargin}
			\vspace{0cm}
		}{
			\ifx\hfuzz#1\hfuzz
			\else
				\throwwarning{Label invalido en ecuacion (gathered) sin numero}
			\fi
			\insertgatheredcaptionedanum{#2}{#3}
		}
		\fi
}
\newcommand{\insertgatheredcaptionedanum}[2]{
	\emptyvarerr{\insertgatheredcaptionedanum}{#1}{Ecuacion no definida}
	\ifx\hfuzz#2\hfuzz
		\insertgatheredanum{#1}
	\else
		\vspace{-0.45cm}
		\begin{gather*}
			\ensuremath{#1}
		\end{gather*}
		\vspace{-0.7cm}
		\begin{changemargin}{\captionlrmargin cm}{\captionlrmargin cm}
			\centering \textcolor{\captiontextcolor}{#2}
			\vspace{0.05cm}
		\end{changemargin}
		\vspace{0cm}
	\fi
}
\newcommand{\insertalign}[1]{
	\emptyvarerr{\insertalign}{#1}{Ecuacion no definida}
	\ifthenelse{\equal{\numberedequation}{true}}{
		\vspace{-0.45cm}
		\begin{align}
			\ensuremath{#1}
		\end{align}
		\vspace{-0.4cm}
	}{
		\insertalignanum{#1}
	}
}
\newcommand{\insertalignanum}[1]{
	\emptyvarerr{\insertalignanum}{#1}{Ecuacion no definida}
	\vspace{-0.45cm}
	\begin{align*}
		\ensuremath{#1}
	\end{align*}
	\vspace{-0.4cm}
}
\newcommand{\insertaligncaptioned}[2]{
	\emptyvarerr{\insertaligncaptioned}{#1}{Ecuacion no definida}
	\ifx\hfuzz#2\hfuzz
		\insertalign{#1}
	\else
		\ifthenelse{\equal{\numberedequation}{true}}{
			\vspace{-0.45cm}
			\begin{align}
				\ensuremath{#1}
			\end{align}
			\vspace{-0.77cm}
			\begin{changemargin}{\captionlrmargin cm}{\captionlrmargin cm}
				\centering \textcolor{\captiontextcolor}{#2}
				\vspace{0.05cm}
			\end{changemargin}
			\vspace{0cm}
		}{
			\insertaligncaptionedanum{#1}{#2}
		}
	\fi
}
\newcommand{\insertaligncaptionedanum}[2]{
	\emptyvarerr{\insertaligncaptioned}{#1}{Ecuacion no definida}
	\ifx\hfuzz#2\hfuzz
		\insertalignanum{#1}
	\else
		\vspace{-0.45cm}
		\begin{align*}
			\ensuremath{#1}
		\end{align*}
		\vspace{-0.77cm}
		\begin{changemargin}{\captionlrmargin cm}{\captionlrmargin cm}
			\centering \textcolor{\captiontextcolor}{#2}
			\vspace{0.05cm}
		\end{changemargin}
		\vspace{0cm}
	\fi
}
\newcommand{\insertaligned}[2][]{
	\emptyvarerr{\insertaligned}{#2}{Ecuacion no definida}
	\ifthenelse{\equal{\numberedequation}{true}}{
		\vspace{-0.1cm}
		\begin{equation}
			\begin{aligned}
				\text{#1} \ensuremath{#2}
			\end{aligned}
		\end{equation}
		\vspace{-0.05cm}
	}{
		\ifx\hfuzz#1\hfuzz
		\else
			\throwwarning{Label invalido en ecuacion (aligned) sin numero}
		\fi
		\insertalignedanum{#2}
	}
}
\newcommand{\insertalignedanum}[1]{
	\emptyvarerr{\insertalignedanum}{#1}{Ecuacion no definida}
	\vspace{-0.45cm}
	\begin{align*}
		\ensuremath{#1}
	\end{align*}
	\vspace{-0.4cm}
}
\newcommand{\insertalignedcaptioned}[3][]{
	\emptyvarerr{\insertalignedcaptioned}{#2}{Ecuacion no definida}
	\ifx\hfuzz#3\hfuzz
		\insertaligned[#1]{#2}
	\else
		\ifthenelse{\equal{\numberedequation}{true}}{
			\vspace{0cm}
			\begin{equation}
				\begin{aligned}
					\text{#1} \ensuremath{#2}
				\end{aligned}
			\end{equation}
			\vspace{-0.65cm}
			\begin{changemargin}{\captionlrmargin cm}{\captionlrmargin cm}
				\centering \textcolor{\captiontextcolor}{#3}
				\vspace{0.05cm}
			\end{changemargin}
			\vspace{0cm}
		}{
			\ifx\hfuzz#1\hfuzz
			\else
				\throwwarning{Label invalido en ecuacion (aligned) sin numero}
			\fi
			\insertalignedcaptionedanum{#2}{#3}
		}
	\fi
}
\newcommand{\insertalignedcaptionedanum}[2]{
	\emptyvarerr{\insertalignedcaptioned}{#1}{Ecuacion no definida}
	\ifx\hfuzz#2\hfuzz
		\insertalignedanum{#1}
	\else
		\vspace{0cm}
		\begin{equation}
			\begin{aligned}
				\ensuremath{#1}
			\end{aligned}
		\end{equation}
		\vspace{-0.65cm}
		\begin{changemargin}{\captionlrmargin cm}{\captionlrmargin cm}
			\centering \textcolor{\captiontextcolor}{#2}
			\vspace{0.05cm}
		\end{changemargin}
		\vspace{0cm}
	\fi
}
% Template:     Informe/Reporte LaTeX
% Documento:    Funciones para insertar imágenes
% Versión:      5.5.5 (27/09/2018)
% Codificación: UTF-8
%
% Autor: Pablo Pizarro R. @ppizarror
%        Facultad de Ciencias Físicas y Matemáticas
%        Universidad de Chile
%        pablo.pizarro@ing.uchile.cl, ppizarror.com
%
% Manual template: [http://latex.ppizarror.com/Template-Informe/]
% Licencia MIT:    [https://opensource.org/licenses/MIT/]

\newcommand{\addimage}[3]{\addimageboxed{#1}{#2}{0}{#3}}\newcommand{\addimageboxed}[4]{\checkonlyonenvimage\begingroup\setlength{\fboxsep}{0 pt}\setlength{\fboxrule}{#3 pt}\hspace{\marginrightmultimage cm}\subfloat[#4]{\fbox{\includegraphics[#2]{#1}}}\endgroup}\newcommand{\insertimage}[4][]{\insertimageboxed[#1]{#2}{#3}{0}{#4}}\newcommand{\insertimageboxed}[5][]{\emptyvarerr{\insertimageboxed}{#2}{Direccion de la imagen no definida}\emptyvarerr{\insertimageboxed}{#3}{Parametros de la imagen no definidos}\emptyvarerr{\insertimageboxed}{#4}{Ancho de la linea no definido}\checkoutsideenvimage\vspace{\margintopimages cm}\begin{figure}[H]\begingroup\setlength{\fboxsep}{0 pt}\setlength{\fboxrule}{#4 pt}\centering\fbox{\includegraphics[#3]{#2}}\endgroup\ifx\hfuzz#5\hfuzz\vspace{\captionlessmarginimage cm}\else\hspace{0cm}\caption{#5 #1}\fi\end{figure}\vspace{\marginbottomimages cm}}\newcommand{\insertdoubleimage}[8][]{\emptyvarerr{\insertdoubleimage}{#2}{Direccion de la imagen 1 no definida}\emptyvarerr{\insertdoubleimage}{#3}{Parametros de la imagen 1 no definidos}\emptyvarerr{\insertdoubleimage}{#5}{Direccion de la imagen 2 no definida}\emptyvarerr{\insertdoubleimage}{#6}{Parametros de la imagen 2 no definidos}\checkoutsideenvimage\alertdeprecatedcmdimage{\insertdoubleimage}\vspace{\margintopimages cm}\captionsetup{margin=\captionmarginmultimg cm}\begin{figure}[H] \centering\subfloat[#4]{\includegraphics[#3]{#2}}\hspace{\marginrightmultimage cm}\subfloat[#7]{\includegraphics[#6]{#5}}\setcaptionmargincm{\captionlrmargin}\ifx\hfuzz#8\hfuzz\vspace{\captionlessmarginimage cm}\else\caption{#8 #1}\fi\end{figure}\setcaptionmargincm{\captionlrmargin}\vspace{\marginbottomimages cm}}\newcommand{\insertdoubleeqimage}[7][]{\insertdoubleimage[#1]{#2}{#6}{#3}{#4}{#6}{#5}{#7}}\newcommand{\inserttripleimage}[8][]{\emptyvarerr{\inserttripleimage}{#2}{Direccion de la imagen 1 no definida}\emptyvarerr{\inserttripleimage}{#3}{Parametros de la imagen 1 no definidos}\emptyvarerr{\inserttripleimage}{#4}{Direccion de la imagen 2 no definida}\emptyvarerr{\inserttripleimage}{#5}{Parametros de la imagen 2 no definidos}\emptyvarerr{\inserttripleimage}{#6}{Direccion de la imagen 3 no definida}\emptyvarerr{\inserttripleimage}{#7}{Parametros de la imagen 3 no definidos}\checkoutsideenvimage\alertdeprecatedcmdimage{\inserttripleimage}\vspace{\margintopimages cm}\captionsetup{margin=\captionmarginmultimg cm}\begin{figure}[H] \centering\subfloat[]{\includegraphics[#3]{#2}}\hspace{\marginrightmultimage cm}\subfloat[]{\includegraphics[#5]{#4}}\hspace{\marginrightmultimage cm}\subfloat[]{\includegraphics[#7]{#6}}\setcaptionmargincm{\captionlrmargin}\ifx\hfuzz#8\hfuzz\vspace{\captionlessmarginimage cm}\else\caption{#8 #1}\fi\end{figure}\setcaptionmargincm{\captionlrmargin}\vspace{\marginbottomimages cm}}\newcommand{\inserttripleeqimage}[6][]{\inserttripleimage[#1]{#2}{#5}{#3}{#5}{#4}{#5}{#6}}\newcommand{\insertquadimage}[7][]{\emptyvarerr{\insertquadimage}{#2}{Direccion de la imagen 1 no definida}\emptyvarerr{\insertquadimage}{#3}{Direccion de la imagen 2 no definida}\emptyvarerr{\insertquadimage}{#4}{Direccion de la imagen 3 no definida}\emptyvarerr{\insertquadimage}{#5}{Direccion de la imagen 4 no definida}\emptyvarerr{\insertquadimage}{#6}{Propiedades de las imagenes no definidos}\checkoutsideenvimage\alertdeprecatedcmdimage{\insertquadimage}\vspace{\margintopimages cm}\captionsetup{margin=\captionmarginmultimg cm cm}\begin{figure}[H] \centering\subfloat[]{\includegraphics[#6]{#2}}\hspace{\marginrightmultimage cm}\subfloat[]{\includegraphics[#6]{#3}}\hspace{\marginrightmultimage cm}\subfloat[]{\includegraphics[#6]{#4}}\hspace{\marginrightmultimage cm}\subfloat[]{\includegraphics[#6]{#5}}\setcaptionmargincm{\captionlrmargin}\ifx\hfuzz#7\hfuzz\vspace{\captionlessmarginimage cm}\else\caption{#7 #1}\fi\end{figure}\setcaptionmargincm{\captionlrmargin}\vspace{\marginbottomimages cm}}\newcommand{\insertpentaimage}[8][]{\emptyvarerr{\insertpentaimage}{#2}{Direccion de la imagen 1 no definida}\emptyvarerr{\insertpentaimage}{#3}{Direccion de la imagen 2 no definida}\emptyvarerr{\insertpentaimage}{#4}{Direccion de la imagen 3 no definida}\emptyvarerr{\insertpentaimage}{#5}{Direccion de la imagen 4 no definida}\emptyvarerr{\insertpentaimage}{#6}{Direccion de la imagen 5 no definida}\emptyvarerr{\insertpentaimage}{#7}{Propiedades de las imagenes no definidas}\checkoutsideenvimage\alertdeprecatedcmdimage{\insertpentaimage}\vspace{\margintopimages cm}\captionsetup{margin=\captionmarginmultimg cm cm}\begin{figure}[H] \centering\subfloat[]{\includegraphics[#7]{#2}}\hspace{\marginrightmultimage cm}\subfloat[]{\includegraphics[#7]{#3}}\hspace{\marginrightmultimage cm}\subfloat[]{\includegraphics[#7]{#4}}\hspace{\marginrightmultimage cm}\subfloat[]{\includegraphics[#7]{#5}}\hspace{\marginrightmultimage cm}\subfloat[]{\includegraphics[#7]{#6}}\setcaptionmargincm{\captionlrmargin}\ifx\hfuzz#8\hfuzz\vspace{\captionlessmarginimage cm}\else\caption{#8 #1}\fi\end{figure}\setcaptionmargincm{\captionlrmargin}\vspace{\marginbottomimages cm}}\newcommand{\inserthexaimage}[9][]{\emptyvarerr{\inserthexaimage}{#2}{Direccion de la imagen 1 no definida}\emptyvarerr{\inserthexaimage}{#3}{Direccion de la imagen 2 no definida}\emptyvarerr{\inserthexaimage}{#4}{Direccion de la imagen 3 no definida}\emptyvarerr{\inserthexaimage}{#5}{Direccion de la imagen 4 no definida}\emptyvarerr{\inserthexaimage}{#6}{Direccion de la imagen 5 no definida}\emptyvarerr{\inserthexaimage}{#7}{Direccion de la imagen 6 no definida}\emptyvarerr{\inserthexaimage}{#8}{Propiedades de las imagenes no definidas}\checkoutsideenvimage\alertdeprecatedcmdimage{\inserthexaimage}\vspace{\margintopimages cm}\captionsetup{margin=\captionmarginmultimg cm}\begin{figure}[H] \centering\subfloat[]{\includegraphics[#8]{#2}}\hspace{\marginrightmultimage cm}\subfloat[]{\includegraphics[#8]{#3}}\hspace{\marginrightmultimage cm}\subfloat[]{\includegraphics[#8]{#4}}\hspace{\marginrightmultimage cm}\subfloat[]{\includegraphics[#8]{#5}}\hspace{\marginrightmultimage cm}\subfloat[]{\includegraphics[#8]{#6}}\hspace{\marginrightmultimage cm}\subfloat[]{\includegraphics[#8]{#7}}\setcaptionmargincm{\captionlrmargin}\ifx\hfuzz#9\hfuzz\vspace{\captionlessmarginimage cm}\else\caption{#9 #1}\fi\end{figure}\setcaptionmargincm{\captionlrmargin}\vspace{\marginbottomimages cm}}\newcommand{\insertimageleft}[4][]{\insertimageleftboxed[#1]{#2}{#3}{0}{#4}}\newcommand{\insertimageleftboxed}[5][]{\emptyvarerr{\insertimageleftboxed}{#2}{Direccion de la imagen no definida}\emptyvarerr{\insertimageleftboxed}{#3}{Ancho de la imagen no definido}\emptyvarerr{\insertimageleftboxed}{#4}{Ancho de la linea no definido}\checkoutsideenvimage~\vspace{-\baselineskip}\par\begin{wrapfigure}{l}{#3\textwidth}\setcaptionmargincm{0}\ifthenelse{\equal{\figurecaptiontop}{true}}{}{\vspace{\marginfloatimages pt}}\begingroup\setlength{\fboxsep}{0 pt}\setlength{\fboxrule}{#4 pt}\centering\fbox{\includegraphics[width=\linewidth]{#2}}\endgroup\ifx\hfuzz#5\hfuzz\vspace{\captionlessmarginimage cm}\else\caption{#5 #1}\fi\end{wrapfigure}\setcaptionmargincm{\captionlrmargin}}\newcommand{\insertimageleftline}[5][]{\insertimageleftlineboxed[#1]{#2}{#3}{0}{#4}{#5}}\newcommand{\insertimageleftlineboxed}[6][]{\emptyvarerr{\insertimageleftlineboxed}{#2}{Direccion de la imagen no definida}\emptyvarerr{\insertimageleftlineboxed}{#3}{Ancho de la imagen no definido}\emptyvarerr{\insertimageleftlineboxed}{#4}{Ancho de la linea no definido}\emptyvarerr{\insertimageleftlineboxed}{#5}{Altura en lineas de la imagen flotante izquierda no definida}\checkoutsideenvimage~\vspace{-\baselineskip}\par\begin{wrapfigure}[#5]{l}{#3\textwidth}\setcaptionmargincm{0}\ifthenelse{\equal{\figurecaptiontop}{true}}{}{\vspace{\marginfloatimages pt}}\begingroup\setlength{\fboxsep}{0 pt}\setlength{\fboxrule}{#4 pt}\centering\fbox{\includegraphics[width=\linewidth]{#2}}\endgroup\ifx\hfuzz#6\hfuzz\vspace{\captionlessmarginimage cm}\else\caption{#6 #1}\fi\end{wrapfigure}\setcaptionmargincm{\captionlrmargin}}\newcommand{\insertimageright}[4][]{\insertimagerightboxed[#1]{#2}{#3}{0}{#4}}\newcommand{\insertimagerightboxed}[5][]{\emptyvarerr{\insertimagerightboxed}{#2}{Direccion de la imagen no definida}\emptyvarerr{\insertimagerightboxed}{#3}{Ancho de la imagen no defindo}\emptyvarerr{\insertimagerightboxed}{#4}{Ancho de la linea no definido}\checkoutsideenvimage~\vspace{-\baselineskip}\par\begin{wrapfigure}{r}{#3\textwidth}\setcaptionmargincm{0}\ifthenelse{\equal{\figurecaptiontop}{true}}{}{\vspace{\marginfloatimages pt}}\begingroup\setlength{\fboxsep}{0 pt}\setlength{\fboxrule}{#4 pt}\centering\fbox{\includegraphics[width=\linewidth]{#2}}\endgroup\ifx\hfuzz#5\hfuzz\vspace{\captionlessmarginimage cm}\else\caption{#5 #1}\fi\end{wrapfigure}\setcaptionmargincm{\captionlrmargin}}\newcommand{\insertimagerightline}[5][]{\insertimagerightlineboxed[#1]{#2}{#3}{0}{#4}{#5}}\newcommand{\insertimagerightlineboxed}[6][]{\emptyvarerr{\insertimagerightlineboxed}{#2}{Direccion de la imagen no definida}\emptyvarerr{\insertimagerightlineboxed}{#3}{Ancho de la imagen no defindo}\emptyvarerr{\insertimagerightlineboxed}{#4}{Ancho de la linea no definido}\emptyvarerr{\insertimagerightlineboxed}{#5}{Altura en lineas de la imagen flotante derecha no definida}\checkoutsideenvimage~\vspace{-\baselineskip}\par\begin{wrapfigure}[#5]{r}{#3\textwidth}\setcaptionmargincm{0}\ifthenelse{\equal{\figurecaptiontop}{true}}{}{\vspace{\marginfloatimages pt}}\begingroup\setlength{\fboxsep}{0 pt}\setlength{\fboxrule}{#4 pt}\centering\fbox{\includegraphics[width=\linewidth]{#2}}\endgroup\ifx\hfuzz#6\hfuzz\vspace{\captionlessmarginimage cm}\else\caption{#6 #1}\fi\end{wrapfigure}\setcaptionmargincm{\captionlrmargin}}\newcommand{\insertimageleftp}[5][]{\xspace~\\\vspace{-2\baselineskip}\par\insertimageleftboxedp[#1]{#2}{#3}{#4}{0}{#5}}\newcommand{\insertimageleftboxedp}[6][]{\emptyvarerr{\insertimageleftboxedp}{#2}{Direccion de la imagen no definida}\emptyvarerr{\insertimageleftboxedp}{#3}{Ancho del objeto no definido}\emptyvarerr{\insertimageleftboxedp}{#4}{Propiedades de la imagen no defindos}\emptyvarerr{\insertimageleftboxedp}{#5}{Ancho de la linea no definido}\checkoutsideenvimage~\vspace{-\baselineskip}\par\begin{wrapfigure}{l}{#3}\setcaptionmargincm{0}\ifthenelse{\equal{\figurecaptiontop}{true}}{}{\vspace{\marginfloatimages pt}}\begingroup\setlength{\fboxsep}{0 pt}\setlength{\fboxrule}{#5 pt}\centering\fbox{\includegraphics[#4]{#2}}\endgroup\ifx\hfuzz#6\hfuzz\vspace{\captionlessmarginimage cm}\else\caption{#6 #1}\fi\end{wrapfigure}\setcaptionmargincm{\captionlrmargin}}\newcommand{\insertimageleftlinep}[6][]{\insertimageleftlineboxedp[#1]{#2}{#3}{#4}{0}{#5}{#6}}\newcommand{\insertimageleftlineboxedp}[7][]{\emptyvarerr{\insertimageleftlineboxedp}{#2}{Direccion de la imagen no definida}\emptyvarerr{\insertimageleftlineboxedp}{#3}{Ancho del objeto no definido}\emptyvarerr{\insertimageleftlineboxedp}{#4}{Propiedades de la imagen no definidos}\emptyvarerr{\insertimageleftlineboxedp}{#5}{Ancho de la linea no definido}\emptyvarerr{\insertimageleftlineboxedp}{#6}{Altura en lineas de la imagen flotante izquierda no definida}\checkoutsideenvimage~\vspace{-\baselineskip}\par\begin{wrapfigure}[#6]{l}{#3}\setcaptionmargincm{0}\ifthenelse{\equal{\figurecaptiontop}{true}}{}{\vspace{\marginfloatimages pt}}\begingroup\setlength{\fboxsep}{0 pt}\setlength{\fboxrule}{#5 pt}\centering\fbox{\includegraphics[#4]{#2}}\endgroup\ifx\hfuzz#7\hfuzz\vspace{\captionlessmarginimage cm}\else\caption{#7 #1}\fi\end{wrapfigure}\setcaptionmargincm{\captionlrmargin}}\newcommand{\insertimagerightp}[5][]{\xspace~\\\vspace{-2\baselineskip}\par\insertimagerightboxedp[#1]{#2}{#3}{#4}{0}{#5}}\newcommand{\insertimagerightboxedp}[6][]{\emptyvarerr{\insertimagerightboxedp}{#2}{Direccion de la imagen no definida}\emptyvarerr{\insertimagerightboxedp}{#3}{Ancho del objeto no definido}\emptyvarerr{\insertimagerightboxedp}{#4}{Propiedades de la imagen no definidos}\emptyvarerr{\insertimagerightboxedp}{#5}{Ancho de la linea no definido}\checkoutsideenvimage~\vspace{-\baselineskip}\par\begin{wrapfigure}{r}{#3}\setcaptionmargincm{0}\ifthenelse{\equal{\figurecaptiontop}{true}}{}{\vspace{\marginfloatimages pt}}\begingroup\setlength{\fboxsep}{0 pt}\setlength{\fboxrule}{#5 pt}\centering\fbox{\includegraphics[#4]{#2}}\endgroup\ifx\hfuzz#6\hfuzz\vspace{\captionlessmarginimage cm}\else\caption{#6 #1}\fi\end{wrapfigure}\setcaptionmargincm{\captionlrmargin}}\newcommand{\insertimagerightlinep}[6][]{\insertimagerightlineboxedp[#1]{#2}{#3}{#4}{0}{#5}{#6}}\newcommand{\insertimagerightlineboxedp}[7][]{\emptyvarerr{\insertimagerightlineboxedp}{#2}{Direccion de la imagen no definida}\emptyvarerr{\insertimagerightlineboxedp}{#3}{Ancho del objeto no definido}\emptyvarerr{\insertimagerightlineboxedp}{#4}{Propiedades de la imagen no definidos}\emptyvarerr{\insertimagerightlineboxedp}{#5}{Ancho de la linea no definido}\emptyvarerr{\insertimagerightlineboxedp}{#6}{Altura en lineas de la imagen flotante derecha no definida}\checkoutsideenvimage~\vspace{-\baselineskip}\par\begin{wrapfigure}[#6]{r}{#3}\setcaptionmargincm{0}\ifthenelse{\equal{\figurecaptiontop}{true}}{}{\vspace{\marginfloatimages pt}}\begingroup\setlength{\fboxsep}{0 pt}\setlength{\fboxrule}{#5 pt}\centering\fbox{\includegraphics[#4]{#2}}\endgroup\ifx\hfuzz#7\hfuzz\vspace{\captionlessmarginimage cm}\else\caption{#7 #1}\fi\end{wrapfigure}\setcaptionmargincm{\captionlrmargin}}
% Template:     Informe/Reporte LaTeX
% Documento:    Funciones para insertar títulos
% Versión:      6.0.0 (13/10/2018)
% Codificación: UTF-8
%
% Autor: Pablo Pizarro R. @ppizarror
%        Facultad de Ciencias Físicas y Matemáticas
%        Universidad de Chile
%        pablo.pizarro@ing.uchile.cl, ppizarror.com
%
% Manual template: [http://latex.ppizarror.com/Template-Informe/]
% Licencia MIT:    [https://opensource.org/licenses/MIT/]

\pretocmd{\section}{
	\ifthenelse{\equal{\GLOBALsectionalph}{true}}{
		\renewcommand{\thesubsection}{\Alph{section}.\arabic{subsection}}
	}{
		\renewcommand{\thesubsection}{\arabic{section}.\arabic{subsection}}
	}
	\def\GLOBALsectionanumenabled{false}
	\def\GLOBALsubsectionanumenabled{false}
}{}{}
\pretocmd{\subsection}{
	\ifthenelse{\equal{\GLOBALsectionanumenabled}{true}}{
		\renewcommand{\thesubsubsection}{\arabic{subsection}.\arabic{subsubsection}}
	}{
		\ifthenelse{\equal{\GLOBALsectionalph}{true}}{
			\renewcommand{\thesubsubsection}{\Alph{section}.\arabic{subsection}.\arabic{subsubsection}}
		}{
			\renewcommand{\thesubsubsection}{\arabic{section}.\arabic{subsection}.\arabic{subsubsection}}
		}
	}
	\def\GLOBALsubsectionanumenabled{false}
}{}{}
\pretocmd{\subsubsection}{
	\ifthenelse{\equal{\GLOBALsubsectionanumenabled}{true}}{
		\renewcommand{\thesubsubsubsection}{\arabic{subsubsection}.\arabic{subsubsubsection}}
	}{
		\ifthenelse{\equal{\GLOBALsectionanumenabled}{true}}{
			\ifthenelse{\equal{\showdotontitles}{true}}{
				\renewcommand{\thesubsubsubsection}{\arabic{subsection}.\arabic{subsubsection}.\arabic{subsubsubsection}.}
			}{
				\renewcommand{\thesubsubsubsection}{\arabic{subsection}.\arabic{subsubsection}.\arabic{subsubsubsection}}
			}
		}{
			\ifthenelse{\equal{\showdotontitles}{true}}{
				\ifthenelse{\equal{\GLOBALsectionalph}{true}}{
					\renewcommand{\thesubsubsubsection}{\Alph{section}.\arabic{subsection}.\arabic{subsubsection}.\arabic{subsubsubsection}.}	
				}{
					\renewcommand{\thesubsubsubsection}{\arabic{section}.\arabic{subsection}.\arabic{subsubsection}.\arabic{subsubsubsection}.}
				}
			}{
				\ifthenelse{\equal{\GLOBALsectionalph}{true}}{
					\renewcommand{\thesubsubsubsection}{\Alph{section}.\arabic{subsection}.\arabic{subsubsection}.\arabic{subsubsubsection}}
				}{
					\renewcommand{\thesubsubsubsection}{\arabic{section}.\arabic{subsection}.\arabic{subsubsection}.\arabic{subsubsubsection}}
				}
			}
		}
	}
}{}{}
\newcommand{\sectionanum}[1]{
	\emptyvarerr{\sectionanum}{#1}{Titulo no definido}
	\phantomsection
	\needspace{3\baselineskip}
	\section*{#1}
	\addcontentsline{toc}{section}{#1}
	\ifthenelse{\equal{\anumsecaddtocounter}{true}}{\stepcounter{section}}{}
	\changeheadertitle{#1}
	\setcounter{subsection}{0}
	\renewcommand{\thesubsection}{\arabic{subsection}}
	\def\GLOBALsectionanumenabled{true}
}
\newcommand{\sectionanumnoi}[1]{
	\emptyvarerr{\sectionanumnoi}{#1}{Titulo no definido}
	\phantomsection
	\needspace{3\baselineskip}
	\section*{#1}
	\ifthenelse{\equal{\anumsecaddtocounter}{true}}{\stepcounter{section}}{}
	\changeheadertitle{#1}
	\setcounter{subsection}{0}
	\renewcommand{\thesubsection}{\arabic{subsection}}
	\def\GLOBALsectionanumenabled{true}
}
\newcommand{\sectionanumheadless}[1]{
	\emptyvarerr{\sectionanumnoheadless}{#1}{Titulo no definido}
	\section*{#1}
	\addcontentsline{toc}{section}{#1}
	\ifthenelse{\equal{\anumsecaddtocounter}{true}}{\stepcounter{section}}{}
	\setcounter{subsection}{0}
	\renewcommand{\thesubsection}{\arabic{subsection}}
	\def\GLOBALsectionanumenabled{true}
}
\newcommand{\sectionanumnoiheadless}[1]{
	\emptyvarerr{\sectionanumnoi}{#1}{Titulo no definido}
	\section*{#1}
	\ifthenelse{\equal{\anumsecaddtocounter}{true}}{\stepcounter{section}}{}
	\setcounter{subsection}{0}
	\renewcommand{\thesubsection}{\arabic{subsection}}
	\def\GLOBALsectionanumenabled{true}
}
\newcommand{\subsectionanum}[1]{
	\emptyvarerr{\subsectionanum}{#1}{Subtitulo no definido}
	\subsection*{#1}
	\addcontentsline{toc}{subsection}{#1}
	\ifthenelse{\equal{\anumsecaddtocounter}{true}}{\stepcounter{subsection}}{}
	\setcounter{subsubsection}{0}
	\renewcommand{\thesubsubsection}{\arabic{subsubsection}}
	\def\GLOBALsubsectionanumenabled{true}
}
\newcommand{\subsectionanumnoi}[1]{
	\emptyvarerr{\subsectionanumnoi}{#1}{Subtitulo no definido}
	\subsection*{#1}
	\ifthenelse{\equal{\anumsecaddtocounter}{true}}{\stepcounter{subsection}}{}
	\setcounter{subsubsection}{0}
	\renewcommand{\thesubsubsection}{\arabic{subsubsection}}
	\def\GLOBALsubsectionanumenabled{true}
}
\newcommand{\subsubsectionanum}[1]{
	\emptyvarerr{\subsubsectionanum}{#1}{Sub-subtitulo no definido}
	\subsubsection*{#1}
	\addcontentsline{toc}{subsubsection}{#1}
	\ifthenelse{\equal{\anumsecaddtocounter}{true}}{\stepcounter{subsubsection}}{}
	\setcounter{subsubsubsection}{0}
	\ifthenelse{\equal{\showdotontitles}{true}}{
		\renewcommand{\thesubsubsubsection}{\arabic{subsubsubsection}.}
	}{
		\renewcommand{\thesubsubsubsection}{\arabic{subsubsubsection}}
	}
}
\newcommand{\subsubsectionanumnoi}[1]{
	\emptyvarerr{\subsubsectionanumnoi}{#1}{Sub-subtitulo no definido}
	\subsubsection*{#1}
	\ifthenelse{\equal{\anumsecaddtocounter}{true}}{\stepcounter{subsubsection}}{}
	\setcounter{subsubsubsection}{0}
	\ifthenelse{\equal{\showdotontitles}{true}}{
		\renewcommand{\thesubsubsubsection}{\arabic{subsubsubsection}.}
	}{
		\renewcommand{\thesubsubsubsection}{\arabic{subsubsubsection}}
	}
}
\newcommand{\subsubsubsectionanum}[1]{
	\emptyvarerr{\subsubsubsectionanum}{#1}{Sub-sub-subtitulo no definido}
	\subsubsubsection*{#1}
	\addcontentsline{toc}{subsubsubsection}{#1}
	\ifthenelse{\equal{\anumsecaddtocounter}{true}}{\stepcounter{subsubsubsection}}{}
}
\newcommand{\subsubsubsectionanumnoi}[1]{
	\emptyvarerr{\subsubsubsectionanumnoi}{#1}{Sub-sub-subtitulo no definido}
	\subsubsection*{#1}
	\ifthenelse{\equal{\anumsecaddtocounter}{true}}{\stepcounter{subsubsubsection}}{}
}
\newcommand{\changeheadertitle}[1]{
	\emptyvarerr{\changeheadertitle}{#1}{Titulo no definido}
	\markboth{#1}{}
}
\newcommand{\insertindextitle}[2]{
	\emptyvarerr{\insertindextitle}{#1}{Titulo no definido}
	\ifx\hfuzz#2\hfuzz
		\addtocontents{toc}{\protect\addvspace{\indextitlemargin pt}}
	\else
		\addtocontents{toc}{\protect\addvspace{#2 pt}}
	\fi
	\addtocontents{toc}{\noindent\hyperref[swpn]{\textbf{#1}}}
}
\newcommand{\newchapter}[1]{
	\emptyvarerr{\newchapter}{#1}{Titulo no definido}
	\newpage
	\stepcounter{section}
	\phantomsection
	\needspace{3\baselineskip}
	\vspace* {3cm}
	\noindent {\huge{\textbf{\nomchapter\ \thesection}}} \\
	\vspace* {0.5cm} \\
	\noindent {\Huge{\textbf{#1}}} \\
	\vspace {0.5cm} \\
	\addcontentsline{toc}{section}{\protect\numberline{\thesection}#1}
	\markboth{#1}{}
}
% Template:     Informe/Reporte LaTeX
% Documento:    Otros estilos
% Versión:      6.1.0 (03/11/2018)
% Codificación: UTF-8
%
% Autor: Pablo Pizarro R. @ppizarror
%        Facultad de Ciencias Físicas y Matemáticas
%        Universidad de Chile
%        pablo.pizarro@ing.uchile.cl, ppizarror.com
%
% Manual template: [https://latex.ppizarror.com/Template-Informe/]
% Licencia MIT:    [https://opensource.org/licenses/MIT/]

\RequirePackage{enumitem}
\makeatletter
\def\greek#1{\expandafter\@greek\csname c@#1\endcsname}
\def\Greek#1{\expandafter\@Greek\csname c@#1\endcsname}
\def\@greek#1{
	\ifcase#1
		\or $\alpha$
		\or $\beta$
		\or $\gamma$
		\or $\delta$
		\or $\epsilon$
		\or $\zeta$
		\or $\eta$
		\or $\theta$
		\or $\iota$
		\or $\kappa$
		\or $\lambda$
		\or $\mu$
		\or $\nu$
		\or $\xi$
		\or $o$
		\or $\pi$
		\or $\rho$
		\or $\sigma$
		\or $\tau$
		\or $\upsilon$
		\or $\phi$
		\or $\chi$
		\or $\psi$
		\or $\omega$
	\fi
}
\def\@Greek#1{
	\ifcase#1
		\or $\mathrm{A}$
		\or $\mathrm{B}$
		\or $\Gamma$
		\or $\Delta$
		\or $\mathrm{E}$
		\or $\mathrm{Z}$
		\or $\mathrm{H}$
		\or $\Theta$
		\or $\mathrm{I}$
		\or $\mathrm{K}$
		\or $\Lambda$
		\or $\mathrm{M}$
		\or $\mathrm{N}$
		\or $\Xi$
		\or $\mathrm{O}$
		\or $\Pi$
		\or $\mathrm{P}$
		\or $\Sigma$
		\or $\mathrm{T}$
		\or $\mathrm{Y}$
		\or $\Phi$
		\or $\mathrm{X}$
		\or $\Psi$
		\or $\Omega$
	\fi
}
\makeatother
\AddEnumerateCounter{\greek}{\@greek}{24}
\AddEnumerateCounter{\Greek}{\@Greek}{12}
\newcolumntype{P}[1]{
	>{\centering\arraybackslash}p{#1}
}
% Template:     Informe/Reporte LaTeX
% Documento:    Funciones para crear columnas con contenido
% Versión:      6.0.6 (30/10/2018)
% Codificación: UTF-8
%
% Autor: Pablo Pizarro R. @ppizarror
%        Facultad de Ciencias Físicas y Matemáticas
%        Universidad de Chile
%        pablo.pizarro@ing.uchile.cl, ppizarror.com
%
% Manual template: [https://latex.ppizarror.com/Template-Informe/]
% Licencia MIT:    [https://opensource.org/licenses/MIT/]

\newcommand{\createtwocolumn}[5]{
	\setcaptionmargincm{0}
	\begin{flushleft}
		\hspace{0cm}
		\begin{tabular}{c}
			\hspace{-0.34cm}
			\begin{minipage}[t]{#1\linewidth}
				\vspace{-2em}\nobreak~ #4
			\end{minipage}
			\hspace{\columnhspace cm}
			\hspace{#3}
			\begin{minipage}[t]{#2\linewidth}
				\vspace{-2em}\nobreak~ #5
			\end{minipage}
			\\
		\end{tabular}
		~
	\end{flushleft}
	\setcaptionmargincm{\captionlrmargin}
}
\newcommand{\createtwocolumnl}[6]{
	\createthreecolumn{0.001}{#1}{#2}{#3}{#4}{~}{#5}{#6}
}
\newcommand{\createhalfcolumn}[2]{
	\createtwocolumn{0.5}{0.5}{0cm}{#1}{#2}
}
\newcommand{\createtwocolumnc}[5]{
	\createtwocolumn{#1}{#2}{#3}{\begin{center}#4\end{center}}{\begin{center}#5\end{center}}
}
\newcommand{\createtwocolumncl}[6]{
	\createtwocolumnl{#1}{#2}{#3}{#4}{\begin{center}#5\end{center}}{\begin{center}#6\end{center}}
}
\newcommand{\createhalfcolumnc}[2]{
	\createtwocolumnc{0.493}{0.493}{0cm}{#1}{#2}
}
\newcommand{\createthreecolumn}[8]{
	\setcaptionmargincm{0}
	\begin{flushleft}
		\hspace{0cm}
		\begin{tabular}{l}
			\hspace{-0.34cm}
			\begin{minipage}[t]{#1\linewidth}
				\vspace{-2em}\nobreak~ #6
			\end{minipage}
			\hspace{\columnhspace cm}
			\hspace{#4}
			\begin{minipage}[t]{#2\linewidth}
				\vspace{-2em}\nobreak~ #7
			\end{minipage}
			\hspace{\columnhspace cm}
			\hspace{#5}
			\begin{minipage}[t]{#3\linewidth}
				\vspace{-2em}\nobreak~ #8
			\end{minipage}
			\\
		\end{tabular}
		~
	\end{flushleft}
	\setcaptionmargincm{\captionlrmargin}
}
\newcommand{\createthirdcolumn}[3]{
	\createthreecolumn{0.3333}{0.3333}{0.333}{0cm}{0cm}{#1}{#2}{#3}
}
\newcommand{\createthreecolumnc}[8]{
	\createthreecolumn{#1}{#2}{#3}{#4}{#5}{\begin{center}#6\end{center}}{\begin{center}#7\end{center}}{\begin{center}#8\end{center}}
}
\newcommand{\createthirdcolumnc}[3]{
	\createthreecolumnc{0.3333}{0.3333}{0.3333}{0cm}{0cm}{#1}{#2}{#3}
}
% Template:     Informe/Reporte LaTeX
% Documento:    Definición de entornos
% Versión:      6.0.1 (21/10/2018)
% Codificación: UTF-8
%
% Autor: Pablo Pizarro R. @ppizarror
%        Facultad de Ciencias Físicas y Matemáticas
%        Universidad de Chile
%        pablo.pizarro@ing.uchile.cl, ppizarror.com
%
% Manual template: [http://latex.ppizarror.com/Template-Informe/]
% Licencia MIT:    [https://opensource.org/licenses/MIT/]

\newenvironment{references}{
\ifthenelse{\equal{\stylecitereferences}{bibtex}}{
	}{
		\throwerror{\references}{Solo se puede usar entorno references con estilo citas \noexpand\stylecitereferences=bibtex}
	}
	\begingroup
\ifthenelse{\equal{\donumrefsection}{true}}{
		\section{\namereferences}
	}{
		\sectionanum{\namereferences}
	}
	\renewcommand{\section}[2]{}
\begin{thebibliography}{99}
	}
	{
	\end{thebibliography}
\endgroup
}
\newenvironment{anexo}{
\begingroup
	\clearpage
	\phantomsection
	\ifthenelse{\equal{\showappendixsectitle}{true}}{
		\appendixpage}{
	}
\def\GLOBALsectionalph{true}
\appendixtitleon
	\appendicestocpagenum
	\appendixtitletocon
	\bookmarksetup{
		numbered,
		openlevel=0
	}
\begin{appendices}
		\bookmarksetupnext{level=part}
		\ifthenelse{\equal{\showappendixsecindex}{true}}{}{
			\belowpdfbookmark{\nameappendixsection}{contents}
		}
		\setcounter{secnumdepth}{4}
		\setcounter{tocdepth}{4}
		\ifthenelse{\equal{\appendixindepobjnum}{true}}{
			\counterwithin{equation}{section}
			\counterwithin{figure}{section}
			\counterwithin{lstlisting}{section}
			\counterwithin{table}{section}}{
		}
	}{
	\end{appendices}
\def\GLOBALsectionalph{false}
\bookmarksetupnext{level=0}
	\endgroup
}
\newcommand{\coreinitsourcecodep}[4]{
	\emptyvarerr{sourcecodep}{#2}{Estilo no definido}
	\checkvalidsourcecodestyle{#2}
	\ifthenelse{\equal{\showlinenumbers}{true}}{
		\rightlinenumbers}{
	}
	\lstset{
		backgroundcolor=\color{\sourcecodebgcolor}
	}
	\ifthenelse{\equal{\codecaptiontop}{true}}{
		\ifx\hfuzz#4\hfuzz
			\ifx\hfuzz#3\hfuzz
				\lstset{
					style=#2
				}
			\else
				\lstset{
					style=#2,
					#3
				}
			\fi
		\else
			\ifx\hfuzz#3\hfuzz
				\lstset{
					caption={#4 #1},
					captionpos=t,
					style=#2
				}
			\else
				\lstset{
					caption={#4 #1},
					captionpos=t,
					style=#2,
					#3
				}
			\fi
		\fi
	}{
		\ifx\hfuzz#4\hfuzz
			\ifx\hfuzz#3\hfuzz
				\lstset{
					style=#2
				}
			\else
				\lstset{
					style=#2,
					#3
				}
			\fi
		\else
			\ifx\hfuzz#3\hfuzz
				\lstset{
					caption={#4 #1},
					captionpos=b,
					style=#2
				}
			\else
				\lstset{
					caption={#4 #1},
					captionpos=b,
					style=#2,
					#3
				}
			\fi
		\fi	
	}
}
\lstnewenvironment{sourcecodep}[4][]{
	\coreinitsourcecodep{#1}{#2}{#3}{#4}
}{
	\ifthenelse{\equal{\showlinenumbers}{true}}{
		\leftlinenumbers}{
	}
}
\newcommand{\importsourcecodep}[5][]{
	\coreinitsourcecodep{#1}{#2}{#3}{#5}
	\inputlisting{#4}
	\ifthenelse{\equal{\showlinenumbers}{true}}{
		\leftlinenumbers}{
	}
}
\newcommand{\coreinitsourcecode}[3]{
	\emptyvarerr{\equationresize}{#2}{Estilo no definido}
	\checkvalidsourcecodestyle{#2}
	\ifthenelse{\equal{\showlinenumbers}{true}}{
		\rightlinenumbers}{
	}
	\lstset{
		backgroundcolor=\color{\sourcecodebgcolor}
	}
	\ifthenelse{\equal{\codecaptiontop}{true}}{
		\ifx\hfuzz#3\hfuzz
			\lstset{
				style=#2
			}
		\else
			\lstset{
				style=#2,
				caption={#3 #1},
				captionpos=t
			}
		\fi
	}{
		\ifx\hfuzz#3\hfuzz
			\lstset{
				style=#2
			}
		\else
			\lstset{
				style=#2,
				caption={#3 #1},
				captionpos=b
			}
		\fi
	}
}
\lstnewenvironment{sourcecode}[3][]{
	\coreinitsourcecode{#1}{#2}{#3}
}{
	\ifthenelse{\equal{\showlinenumbers}{true}}{
		\leftlinenumbers}{
	}
}
\newcommand{\importsourcecode}[4][]{
	\coreinitsourcecode{#1}{#2}{#4}
	\lstinputlisting{#3}
	\ifthenelse{\equal{\showlinenumbers}{true}}{
		\leftlinenumbers}{
	}
}
\newenvironment{itemizebf}[1][]{
	\begin{itemize}[font=\bfseries,#1]
	}{
	\end{itemize}
}
\newenvironment{enumeratebf}[1][]{
	\begin{enumerate}[font=\bfseries,#1]
	}{
	\end{enumerate}
}
\newenvironment{resumen}{
	\sectionfont{\color{\titlecolor} \fontsizetitle \styletitle \selectfont}
	\sectionanumnoiheadless{\nameabstract}}{
	\ifthenelse{\equal{\addemptypagetwosides}{true}}{
		\checkoddpage
		\ifoddpage
			\newpage
			\null
			\thispagestyle{empty}
			\newpage
			\addtocounter{page}{-1}
		\else
		\fi}{
	}
}
\newenvironment{images}[2][]{
	\def\envimageslabelvar {#1}
	\def\envimagescaptionvar {#2}
	\def\GLOBALenvimageintialized {true}
	\vspace{\margintopimages cm}
	\captionsetup{margin=\captionmarginmultimg cm}
	\begin{figure}[H] \centering
		\vspace{-\marginrightmultimage cm}
		\vspace{-\marginrightmultimage cm}
		\vspace{-\marginrightmultimage cm}
		}{
		\setcaptionmargincm{\captionlrmargin}
		\ifx\hfuzz\envimagescaptionvar\hfuzz
			\vspace{\captionlessmarginimage cm}
		\else
			\caption{\envimagescaptionvar\envimageslabelvar}
		\fi
	\end{figure}
	\setcaptionmargincm{\captionlrmargin}
	\vspace{\marginbottomimages cm}
	\def\GLOBALenvimageintialized {false}
}



% IMPORTACIÓN DE ESTILOS
% Template:     Informe/Reporte LaTeX
% Documento:    Carga las funciones del template
% Versión:      5.5.5 (27/09/2018)
% Codificación: UTF-8
%
% Autor: Pablo Pizarro R. @ppizarror
%        Facultad de Ciencias Físicas y Matemáticas
%        Universidad de Chile
%        pablo.pizarro@ing.uchile.cl, ppizarror.com
%
% Manual template: [http://latex.ppizarror.com/Template-Informe/]
% Licencia MIT:    [https://opensource.org/licenses/MIT/]

% Template:     Informe/Reporte LaTeX
% Documento:    Funciones del núcleo del template
% Versión:      6.1.6 (14/12/2018)
% Codificación: UTF-8
%
% Autor: Pablo Pizarro R. @ppizarror
%        Facultad de Ciencias Físicas y Matemáticas
%        Universidad de Chile
%        pablo.pizarro@ing.uchile.cl, ppizarror.com
%
% Manual template: [https://latex.ppizarror.com/Template-Informe/]
% Licencia MIT:    [https://opensource.org/licenses/MIT/]

\def\GLOBALcoretikzimported {false}
\def\GLOBALenvimageadded {false}
\def\GLOBALenvimageinitialized {false}
\def\GLOBALenvimagenewlinemarg {0.25}
\def\GLOBALsectionalph {false}
\def\GLOBALsectionanumenabled {false}
\def\GLOBALsubsectionanumenabled {false}
\newcommand{\throwerror}[2]{
	\errmessage{LaTeX Error: \noexpand#1 #2 (linea \the\inputlineno)}
	\stop
}
\newcommand{\throwwarning}[1]{
	\errmessage{LaTeX Warning: #1 (linea \the\inputlineno)}
}
\newcommand{\throwbadconfigondoc}[3]{
	\errmessage{#1 \noexpand #2=#2. Valores esperados: #3}
	\stop
}
\newcommand{\checkvardefined}[1]{
	\ifthenelse{\isundefined{#1}}{
		\errmessage{LaTeX Warning: Variable \noexpand#1 no definida}
		\stop}{
	}
}
\newcommand{\checkextravarexist}[2]{
	\ifthenelse{\isundefined{#1}}{
		\errmessage{LaTeX Warning: Variable \noexpand#1 no definida}
		\ifx\hfuzz#2\hfuzz
			\errmessage{LaTeX Warning: Defina la variable en el bloque de INFORMACION DEL DOCUMENTO al comienzo del archivo principal del Template}
		\else
			\errmessage{LaTeX Warning: #2}
		\fi}{
	}
}
\newcommand{\emptyvarerr}[3]{
	\ifx\hfuzz#2\hfuzz
		\errmessage{LaTeX Warning: \noexpand#1 #3 (linea \the\inputlineno)}
	\fi
}
\newcommand{\setcaptionmargincm}[1]{
	\captionsetup{margin=#1cm}
}
\newcommand{\setpagemargincm}[4]{
	\newgeometry{left=#1cm, top=#2cm, right=#3cm, bottom=#4cm}
}
\newcommand{\changemargin}[2]{
	\emptyvarerr{\changemargin}{#1}{Margen izquierdo no definido}
	\emptyvarerr{\changemargin}{#2}{Margen derecho no definido}
	\list{}{\rightmargin#2\leftmargin#1}\item[]
}
\let\endchangemargin=\endlist
\newcommand{\coreimporttikz}{
	\ifthenelse{\equal{\GLOBALcoretikzimported}{false}}{
		\ifthenelse{\equal{\importtikz}{false}}{
			\usepackage{tikz}}{
		}
		\def\GLOBALcoretikzimported{true}
		}{
	}
}
\newcommand{\bgtemplatetestimg}{
	\definecolor{c040302}{RGB}{4, 3, 2}
	\definecolor{cb26929}{RGB}{178,105, 41}
	\definecolor{cfa943c}{RGB}{250, 148, 60}
	\definecolor{cfdc594}{RGB}{253, 197, 148}
	\begin{tikzpicture}[y=1.0pt, x=1.0pt, yscale=-1, xscale=1, inner sep=0pt, outer sep=0pt, opacity=0.1]
	\path[fill=cfdc594](46.3187,318.2550)..controls(46.0737,317.5650)and
	(45.7487,314.0750)..(45.5987,310.5)--(45.3247,304)--(37.8247,303.5)--(30.3247,303)--(29.8247,295.5)-- (29.3247,288)--(21.8247,287.5)--(14.3247,287)--(13.8247,271.5)--(13.3247,256)--(5.8247,255.5)-- (-1.6753,255)--(-1.6753,206.5)--(-1.6753,158)--(5.8247,157.5)--(13.3247,157)--(13.8247,141.5)-- (14.3247,126)--(21.8247,125.5)--(29.3247,125)--(29.8247,109.5)--(30.3247,94)--(37.8247,93.5)-- (45.3247,93)--(45.8247,69.5)--(46.3247,46)--(53.8247,45.5)--(61.3247,45)--(61.8247,29.5)-- (62.3247,14)--(69.8247,13.5)--(77.3247,13)--(77.8247,5.5)--(78.3247,-2)--(94.8247,-2)-- (111.3247,-2)--(111.8247,5.5)--(112.3247,13)--(119.8247,13.5)--(127.3247,14)--(127.8247,21.5)-- (128.3247,29)--(166.8247,29)--(205.3247,29)--(205.8247,21.5)--(206.3247,14)--(213.8247,13.5)-- (221.3247,13)--(221.8247,5.5)--(222.3247,-2)--(230.8247,-2)--(239.3247,-2)--(239.8247,5.5)-- (240.3247,13)--(247.8247,13.5)--(255.3247,14)--(255.8247,21.5)--(256.3247,29)--(263.8247,29.5)-- (271.3247,30)--(271.8247,45.5)--(272.3247,61)--(279.8247,61.5)--(287.3247,62)--(287.8247,85.5)-- (288.3247,109)--(295.8247,109.5)--(303.3247,110)--(303.8247,133.5)--(304.3247,157)--(311.8247,157.5)-- (319.3247,158)--(319.3247,198.5)--(319.3247,239)--(311.8247,239.5)--(304.3247,240)--(303.8247,255.5)-- (303.3247,271)--(295.8247,271.5)--(288.3247,272)--(287.8247,279.5)--(287.3247,287)--(279.8247,287.5)-- (272.3247,288)--(271.8247,295.5)--(271.3247,303)--(263.8247,303.5)--(256.3247,304)--(255.8247,311.5)-- (255.3247,319)--(151.0447,319.2550)..controls(68.1357,319.4570)and (46.6737,319.2530)..(46.3187,318.2550)--cycle(173.8247,142.5)-- (173.8247,127.5)--(166.8247,127.5)--(159.8247,127.5)-- (159.8247,142.5)--(159.8247,157.5)--(166.8247,157.5)-- (173.8247,157.5)--cycle(269.8247,126.5)--(269.8247,111.5)-- (262.8247,111.5)--(255.8247,111.5)--(255.8247,126.5)-- (255.8247,141.5)--(262.8247,141.5)--(269.8247,141.5)--cycle; \path[fill=cfa943c](47.3187,317.2550)..controls(47.0737,316.5650)and (46.7487,313.0750)..(46.5987,309.5)--(46.3247,303)--(38.8247,302.5)--(31.3247,302)--(30.8247,294.5)-- (30.3247,287)--(22.8247,286.5)--(15.3247,286)--(14.8247,270.5)--(14.3247,255)--(6.8247,254.5)-- (-0.6753,254)--(-0.6753,206.5)--(-0.6753,159)--(6.8247,158.5)--(14.3247,158)--(14.8247,142.5)-- (15.3247,127)--(22.8247,126.5)--(30.3247,126)--(30.8247,110.5)--(31.3247,95)--(38.8247,94.5)-- (46.3247,94)--(46.8247,70.5)--(47.3247,47)--(54.8247,46.5)--(62.3247,46)--(62.8247,30.5)-- (63.3247,15)--(70.8247,14.5)--(78.3247,14)--(78.8247,6.5)--(79.3247,-1)--(94.8247,-1)-- (110.3247,-1)--(110.8247,6.5)--(111.3247,14)--(118.8247,14.5)--(126.3247,15)--(126.8247,22.5)-- (127.3247,30)--(166.8247,30)--(206.3247,30)--(206.8247,22.5)--(207.3247,15)--(214.8247,14.5)-- (222.3247,14)--(222.8247,6.5)--(223.3247,-1)--(230.8247,-1)--(238.3247,-1)--(238.8247,6.5)-- (239.3247,14)--(246.8247,14.5)--(254.3247,15)--(254.8247,22.5)--(255.3247,30)--(262.8247,30.5)-- (270.3247,31)--(270.8247,46.5)--(271.3247,62)--(278.8247,62.5)--(286.3247,63)--(286.8247,86.5)-- (287.3247,110)--(294.8247,110.5)--(302.3247,111)--(302.8247,134.5)--(303.3247,158)--(310.8247,158.5)-- (318.3247,159)--(318.3247,198.5)--(318.3247,238)--(310.8247,238.5)--(303.3247,239)--(302.8247,254.5)-- (302.3247,270)--(294.8247,270.5)--(287.3247,271)--(286.8247,278.5)--(286.3247,286)--(278.8247,286.5)-- (271.3247,287)--(270.8247,294.5)--(270.3247,302)--(262.8247,302.5)--(255.3247,303)--(254.8247,310.5)-- (254.3247,318)--(151.0447,318.2550)..controls(68.9357,318.4570)and (47.6737,318.2520)..(47.3187,317.2550)--cycle(253.8247,294.70).. controls(253.8247,290.7440)and(254.3267,287.3980)..(255.0247,286.70).. controls(255.7227,286.0020)and(259.0687,285.5)..(263.0247,285.5)-- (269.8247,285.5)--(269.8247,278.70)..controls(269.8247,274.7440)and (270.3267,271.3980)..(271.0247,270.70)..controls(271.7227,270.0020)and (275.0687,269.5)..(279.0247,269.5)--(285.8247,269.5)-- (285.8247,254.70)..controls(285.8247,244.5220)and(286.1997,239.5250).. (287.0247,238.70)..controls(287.7227,238.0020)and(291.0687,237.5).. (295.0247,237.5)--(301.8247,237.5)--(301.8247,206)-- (301.8247,174.5)--(294.8247,174.5)--(287.8247,174.5)-- (287.8247,189.80)..controls(287.8247,200.3670)and(287.4537,205.4710).. (286.6247,206.30)..controls(285.9267,206.9980)and(282.5807,207.5).. (278.6247,207.5)--(271.8247,207.5)--(271.8247,214.30)..controls (271.8247,218.2560)and(271.3227,221.6020)..(270.6247,222.30)..controls (269.9267,222.9980)and(266.5807,223.5)..(262.6247,223.5)-- (255.8247,223.5)--(255.8247,230.3780)..controls(255.8247,234.9380)and (255.3687,237.6340)..(254.4727,238.3780)..controls(252.4847,240.0280)and (176.6787,239.9540)..(175.0247,238.30)..controls(174.3337,237.6090)and (173.8247,234.30)..(173.8247,230.5)..controls(173.8247,226.70)and (174.3337,223.3910)..(175.0247,222.70)..controls(175.9167,221.8080)and (186.1807,221.5)..(215.0247,221.5)--(253.8247,221.5)-- (253.8247,214.5)--(253.8247,207.5)--(249.0747,207.3720)..controls (246.4627,207.3020)and(235.9997,206.9640)..(225.8247,206.6220)-- (207.3247,206)--(206.8247,198.5)--(206.3247,191)--(190.8247,190.5)--(175.3247,190)--(174.8247,182.5)-- (174.3247,175)--(134.8247,175)--(95.3247,175)--(94.8247,182.5)--(94.3247,190)--(86.8247,190.5)-- (79.3247,191)--(79.3247,222.5)--(79.3247,254)--(86.8247,254.5)--(94.3247,255)--(94.8247,270.5)-- (95.3247,286)--(102.8247,286.5)--(110.3247,287)-- (110.6187,294.2500)--(110.9127,301.5)--(182.3687,301.5)-- (253.8247,301.5)--cycle(174.6247,158.30)..controls (176.4747,156.4500)and(176.2867,128.1290)..(174.4147,126.5750)..controls (173.5527,125.8600)and(170.3467,125.5210)..(166.1647,125.7030)-- (159.3247,126)--(159.0477,141.4580)..controls(158.8957,149.9600)and (158.9937,157.4970)..(159.2667,158.2080)..controls(159.9187,159.9070)and (172.9397,159.9850)..(174.6257,158.30)--cycle(270.6247,142.30).. controls(272.4747,140.4500)and(272.2867,112.1290)..(270.4147,110.5750).. controls(269.5527,109.8600)and(266.3467,109.5210)..(262.1647,109.7030)-- (255.3247,110)--(255.0477,125.4580)..controls(254.8957,133.9600)and (254.9937,141.4970)..(255.2667,142.2080)..controls(255.9187,143.9070)and (268.9397,143.9850)..(270.6257,142.30)--cycle(78.8247,86.5)-- (79.3247,79)--(86.8247,78.5)--(94.3247,78)--(94.8247,70.5)--(95.3247,63)--(102.8247,62.5)-- (110.3247,62)--(110.8247,54.5)--(111.3247,47)-- (118.5747,46.7060)--(125.8247,46.4120)--(125.8247,38.9560)-- (125.8247,31.5)--(119.0247,31.5)..controls(115.0687,31.5)and (111.7227,30.9980)..(111.0247,30.30)..controls(110.3267,29.6020)and (109.8247,26.2560)..(109.8247,22.30)--(109.8247,15.5)-- (94.8247,15.5)--(79.8247,15.5)--(79.8247,30.30)..controls (79.8247,40.4780)and(79.4497,45.4750)..(78.6247,46.30)..controls (77.9267,46.9980)and(74.5807,47.5)..(70.6247,47.5)-- (63.8247,47.5)--(63.8247,71.0440)--(63.8247,94.5880)-- (71.0747,94.2940)--(78.3247,94)--cycle(253.8247,39)-- (253.8247,31.5)--(247.0247,31.5)..controls(243.0687,31.5)and (239.7227,30.9980)..(239.0247,30.30)..controls(238.3267,29.6020)and (237.8247,26.2560)..(237.8247,22.30)--(237.8247,15.5)-- (230.8247,15.5)--(223.8247,15.5)--(223.8247,22.30)..controls (223.8247,26.2560)and(223.3227,29.6020)..(222.6247,30.30)..controls (221.9267,30.9980)and(218.5807,31.5)..(214.6247,31.5)-- (207.8247,31.5)--(207.8247,39)--(207.8247,46.5)-- (230.8247,46.5)--(253.8247,46.5)--cycle; \path[fill=cb26929](47.3187,317.2550)..controls(47.0737,316.5650)and (46.7487,313.0750)..(46.5987,309.5)--(46.3247,303)--(38.8247,302.5)--(31.3247,302)--(30.8247,294.5)-- (30.3247,287)--(22.8247,286.5)--(15.3247,286)--(14.8247,270.5)--(14.3247,255)--(6.8247,254.5)-- (-0.6753,254)--(-0.6753,206.5)--(-0.6753,159)--(6.8247,158.5)--(14.3247,158)--(14.8247,142.5)-- (15.3247,127)--(22.8247,126.5)--(30.3247,126)--(30.8247,110.5)--(31.3247,95)--(38.8247,94.5)-- (46.3247,94)--(46.8247,70.5)--(47.3247,47)--(54.8247,46.5)--(62.3247,46)--(62.8247,30.5)-- (63.3247,15)--(70.8247,14.5)--(78.3247,14)--(78.8247,6.5)--(79.3247,-1)--(94.8247,-1)-- (110.3247,-1)--(110.8247,6.5)--(111.3247,14)--(118.8247,14.5)--(126.3247,15)--(126.8247,22.5)-- (127.3247,30)--(166.8247,30)--(206.3247,30)--(206.8247,22.5)--(207.3247,15)--(214.8247,14.5)-- (222.3247,14)--(222.8247,6.5)--(223.3247,-1)--(230.8247,-1)--(238.3247,-1)--(238.8247,6.5)-- (239.3247,14)--(246.8247,14.5)--(254.3247,15)--(254.8247,22.5)--(255.3247,30)--(262.8247,30.5)-- (270.3247,31)--(270.8247,46.5)--(271.3247,62)--(278.8247,62.5)--(286.3247,63)--(286.8247,86.5)-- (287.3247,110)--(294.8247,110.5)--(302.3247,111)--(302.8247,134.5)--(303.3247,158)--(310.8247,158.5)-- (318.3247,159)--(318.3247,198.5)--(318.3247,238)--(310.8247,238.5)--(303.3247,239)--(302.8247,254.5)-- (302.3247,270)--(294.8247,270.5)--(287.3247,271)--(286.8247,278.5)--(286.3247,286)--(278.8247,286.5)-- (271.3247,287)--(270.8247,294.5)--(270.3247,302)--(262.8247,302.5)--(255.3247,303)--(254.8247,310.5)-- (254.3247,318)--(151.0447,318.2550)..controls(68.9357,318.4570)and (47.6737,318.2520)..(47.3187,317.2550)--cycle(253.8247,294.70).. controls(253.8247,290.7440)and(254.3267,287.3980)..(255.0247,286.70).. controls(255.7227,286.0020)and(259.0687,285.5)..(263.0247,285.5)-- (269.8247,285.5)--(269.8247,278.70)..controls(269.8247,274.7440)and (270.3267,271.3980)..(271.0247,270.70)..controls(271.7227,270.0020)and (275.0687,269.5)..(279.0247,269.5)--(285.8247,269.5)-- (285.8247,254.70)..controls(285.8247,244.5220)and(286.1997,239.5250).. (287.0247,238.70)..controls(287.7227,238.0020)and(291.0717,237.5).. (295.0367,237.5)--(301.8487,237.5)--(301.5867,198.7500)-- (301.3247,160)--(294.8247,160)--(288.3247,160)-- (288.0547,182.4210)..controls(287.8807,196.9060)and(287.3947,205.3120).. (286.6827,206.1710)..controls(285.9607,207.0410)and(283.2017,207.5).. (278.7027,207.5)--(271.8247,207.5)--(271.8247,214.30)..controls (271.8247,218.2560)and(271.3227,221.6020)..(270.6247,222.30)..controls (269.9267,222.9980)and(266.5807,223.5)..(262.6247,223.5)-- (255.8247,223.5)--(255.8247,230.3780)..controls(255.8247,234.9380)and (255.3687,237.6340)..(254.4727,238.3780)..controls(252.4847,240.0280)and (176.6787,239.9540)..(175.0247,238.30)..controls(174.3337,237.6090)and (173.8247,234.30)..(173.8247,230.5)..controls(173.8247,226.70)and (174.3337,223.3910)..(175.0247,222.70)..controls(175.9167,221.8080)and (186.1807,221.5)..(215.0247,221.5)--(253.8247,221.5)-- (253.8247,214.5)--(253.8247,207.5)--(247.0747,207.4850)..controls (243.3627,207.4770)and(239.7747,207.1200)..(239.1027,206.6930)..controls (238.2507,206.1520)and(237.7967,201.3880)..(237.6027,190.9580)-- (237.3247,176)--(229.8247,175.5)--(222.3247,175)--(222.3247,166.5)--(222.3247,158)--(253.8247,157.5)-- (285.3247,157)--(285.3247,134.5)--(285.3247,112)--(277.8247,111.5)--(270.3247,111)--(269.8247,87.5)-- (269.3247,64)--(261.8247,63.5)--(254.3247,63)-- (254.0497,47.2500)--(253.7737,31.5)--(246.9987,31.5)..controls (243.0637,31.5)and(239.7217,30.9970)..(239.0247,30.30)..controls (238.3267,29.6020)and(237.8247,26.2560)..(237.8247,22.30)-- (237.8247,15.5)--(230.8247,15.5)--(223.8247,15.5)-- (223.8247,22.30)..controls(223.8247,26.2560)and(223.3227,29.6020).. (222.6247,30.30)..controls(221.9277,30.9970)and(218.5897,31.5).. (214.6667,31.5)--(207.9087,31.5)--(207.6167,39.2500)-- (207.3247,47)--(166.8247,47)--(126.3247,47)-- (126.0327,39.2500)--(125.7407,31.5)--(118.9827,31.5)..controls (115.0597,31.5)and(111.7217,30.9970)..(111.0247,30.30)..controls (110.3267,29.6020)and(109.8247,26.2560)..(109.8247,22.30)-- (109.8247,15.5)--(94.8247,15.5)--(79.8247,15.5)-- (79.8247,30.30)..controls(79.8247,40.4780)and(79.4497,45.4750).. (78.6247,46.30)..controls(77.9277,46.9970)and(74.5847,47.5).. (70.6427,47.5)--(63.8607,47.5)--(63.5927,71.2500)-- (63.3247,95)--(55.8247,95.5)--(48.3247,96)--(47.8247,111.5)--(47.3247,127)--(39.8247,127.5)-- (32.3247,128)--(31.8247,143.5)--(31.3247,159)--(23.8247,159.5)--(16.3247,160)--(16.3247,175)-- (16.3247,190)--(23.3247,190.5)--(30.3247,191)--(30.8247,198.5)--(31.3247,206)--(38.8247,206.5)-- (46.3247,207)--(46.8247,214.5)--(47.3247,222)--(62.8247,222.5)--(78.3247,223)--(78.8247,238.5)-- (79.3247,254)--(86.8247,254.5)--(94.3247,255)--(94.8247,270.5)--(95.3247,286)--(102.8247,286.5)-- (110.3247,287)--(110.6187,294.2500)--(110.9127,301.5)-- (182.3687,301.5)--(253.8247,301.5)--cycle(126.2657,158.2080).. controls(125.9927,157.4970)and(125.8947,149.9600)..(126.0467,141.4580)-- (126.3247,126)--(142.5747,125.7250)--(158.8247,125.4500)-- (158.8247,142.4750)--(158.8247,159.5)--(142.7937,159.5)..controls (130.6137,159.5)and(126.6437,159.1900)..(126.2667,158.2080)-- cycle(222.2657,142.2080)..controls(221.9927,141.4970)and (221.8947,133.9600)..(222.0467,125.4580)--(222.3247,110)-- (238.5747,109.7250)--(254.8247,109.4500)--(254.8247,126.4750)-- (254.8247,143.5)--(238.7937,143.5)..controls(226.6137,143.5)and (222.6437,143.1900)..(222.2667,142.2080)--cycle; \path[fill=c040302](46.3247,303)--(79.0747,303.0080)..controls (95.9877,303.0120)and(142.2457,303.1250)..(181.8707,303.2580)-- (253.9167,303.5)--(253.6207,310.2500)--(253.3247,317)-- (151.0447,317.2550)..controls(65.3137,317.4680)and(48.6807,317.2880).. (48.2407,316.1430)..controls(46.9629,307.2110)and(46.3247,303).. (46.3247,303)--cycle(32.2567,300.1840)..controls(31.9597,299.4100) and(31.8537,296.1260)..(32.0207,292.8880)--(32.3247,287)-- (39.3247,287)--(46.3247,287)--(46.3247,294)-- (46.3247,301)--(39.5607,301.2960)..controls(34.4617,301.5190)and (32.6637,301.2460)..(32.2567,300.1840)--cycle(255.8247,294.5460)-- (255.8247,287.5)--(262.8707,287.5)--(269.9167,287.5)-- (269.6207,294.2500)--(269.3247,301)--(262.5747,301.2960)-- (255.8247,301.5920)--cycle(16.2817,284.2480)..controls(15.9977,283.5090) and(15.8917,276.6260)..(16.0457,268.9520)--(16.3247,255)-- (23.3247,255)--(30.3247,255)--(30.3247,270)-- (30.3247,285)--(23.5607,285.2960)..controls(18.6047,285.5130)and (16.6587,285.2330)..(16.2807,284.2480)--cycle(271.8247,278.5460)-- (271.8247,271.5)--(278.8707,271.5)--(285.9167,271.5)-- (285.6207,278.2500)--(285.3247,285)--(278.5747,285.2960)-- (271.8247,285.5920)--cycle(287.8247,254.5460)--(287.8247,239.5)-- (294.8517,239.5)--(301.8787,239.5)--(301.6017,254.2500)-- (301.3247,269)--(294.5747,269.2960)--(287.8247,269.5920)-- cycle(0.3007,252.2960)..controls(0.0267,251.5830)and(-0.0793,230.5250).. (0.0637,205.5)--(0.3247,160)--(6.9587,159.7070)-- (13.5927,159.4140)--(14.2807,190.7070)..controls(14.6597,207.9180)and (14.8247,228.9750)..(14.6467,237.5)--(14.3247,253)-- (7.5607,253.2960)..controls(2.7217,253.5080)and(0.6557,253.2230).. (0.2997,252.2960)--cycle(175.8247,230.5)--(175.8247,223.5)-- (214.8247,223.5)--(253.8247,223.5)--(253.8247,230.5)-- (253.8247,237.5)--(214.8247,237.5)--(175.8247,237.5)-- cycle(303.8247,198.5)--(303.8247,159.4090)--(310.5747,159.7050)-- (317.3247,160.0010)--(317.3247,198.5010)--(317.3247,237.0010)-- (310.5747,237.2970)--(303.8247,237.5930)--cycle(255.8247,214.70).. controls(255.8247,210.7440)and(255.3227,207.3980)..(254.6247,206.70).. controls(253.9267,206.0020)and(250.5807,205.5)..(246.6247,205.5)-- (239.8247,205.5)--(239.8247,190.70)..controls(239.8247,180.5220)and (239.4497,175.5250)..(238.6247,174.70)..controls(237.9267,174.0020)and (234.5807,173.5)..(230.6247,173.5)--(223.8247,173.5)-- (223.8247,166.5)--(223.8247,159.5)--(254.8247,159.5)-- (285.8247,159.5)--(285.8247,182.5)--(285.8247,205.5)-- (279.0247,205.5)..controls(275.0687,205.5)and(271.7227,206.0020).. (271.0247,206.70)..controls(270.3267,207.3980)and(269.8247,210.7440).. (269.8247,214.70)--(269.8247,221.5)--(262.8247,221.5)-- (255.8247,221.5)--cycle(16.0477,142.7500)--(16.3247,128)-- (23.0747,127.7040)--(29.8247,127.4090)--(29.8247,142.4540)-- (29.8247,157.5)--(22.7977,157.5)--(15.7717,157.5)-- cycle(127.8247,142.5)--(127.8247,127.5)--(142.8247,127.5)-- (157.8247,127.5)--(157.8247,142.5)--(157.8247,157.5)-- (142.8247,157.5)--(127.8247,157.5)--cycle(287.8247,134.4540)-- (287.8247,111.4080)--(294.5747,111.7040)--(301.3247,112)-- (301.5937,134.7500)--(301.8627,157.5)--(294.8437,157.5)-- (287.8247,157.5)--cycle(223.8247,126.5)--(223.8247,111.5)-- (238.8247,111.5)--(253.8247,111.5)--(253.8247,126.5)-- (253.8247,141.5)--(238.8247,141.5)--(223.8247,141.5)-- cycle(32.0477,110.7500)--(32.3247,96)--(39.0747,95.7040)-- (45.8247,95.4090)--(45.8247,110.4540)--(45.8247,125.5)-- (38.7977,125.5)--(31.7717,125.5)--cycle(271.8247,86.4540)-- (271.8247,63.4090)--(278.5747,63.7050)--(285.3247,64.0010)-- (285.5937,86.7510)--(285.8627,109.5010)--(278.8437,109.5010)-- (271.8247,109.5010)--cycle(48.0557,70.7500)--(48.3247,48)-- (55.0747,47.7040)--(61.8247,47.4090)--(61.8247,70.4540)-- (61.8247,93.5)--(54.8057,93.5)--(47.7867,93.5)-- cycle(255.8247,46.4540)--(255.8247,31.4090)--(262.5747,31.7050)-- (269.3247,32.0010)--(269.6017,46.7510)--(269.8787,61.5010)-- (262.8517,61.5010)--(255.8247,61.5010)--cycle(64.0477,30.7500)-- (64.3247,16)--(71.0747,15.7040)--(77.8247,15.4090)--(77.8247,30.4540)--(77.8247,45.5)--(70.7977,45.5)-- (63.7707,45.5)--cycle(127.8247,38.5)--(127.8247,31.5)--(166.8247,31.5)--(205.8247,31.5)--(205.8247,38.5)-- (205.8247,45.5)--(166.8247,45.5)--(127.8247,45.5)-- cycle(111.8247,22.4540)--(111.8247,15.4080)--(118.5747,15.7040)-- (125.3247,16)--(125.6207,22.7500)--(125.9167,29.5)-- (118.8707,29.5)--(111.8247,29.5)--cycle(208.0287,22.7500)-- (208.3247,16)--(215.0747,15.7040)--(221.8247,15.4090)-- (221.8247,22.4540)--(221.8247,29.5)--(214.7787,29.5)-- (207.7327,29.5)--cycle(239.8247,22.4540)--(239.8247,15.4080)-- (246.5747,15.7040)--(253.3247,16)--(253.6207,22.7500)-- (253.9167,29.5)--(246.8707,29.5)--(239.8247,29.5)--cycle(80.0287,6.7500)--(80.3247,0)--(94.8247,0)-- (109.3247,0)--(109.6207,6.7500)--(109.9167,13.5)--(94.8247,13.5)--(79.7337,13.5)--cycle(224.0287,6.7500)-- (224.3247,0)--(230.8247,0)--(237.3247,0)--(237.6207,6.7500)--(237.9167,13.5)--(230.8247,13.5)-- (223.7337,13.5)--cycle;
	\end{tikzpicture}
}
\def\bgtemplatetestcode {d0g3}
\newcommand{\checkonlyonenvimage}{\ifthenelse{\equal{\GLOBALenvimageinitialized}{true}}{}{\throwwarning{Funciones \noexpand\addimage o \noexpand\addimageboxed no pueden usarse fuera del entorno \noexpand\images}\stop}}
\newcommand{\checkoutsideenvimage}{
	\ifthenelse{\equal{\GLOBALenvimageinitialized}{true}}{
		\throwwarning{Esta funcion solo puede usarse fuera del entorno \noexpand\images}
		\stop}{
	}
}
% Template:     Informe/Reporte LaTeX
% Documento:    Funciones matemáticas
% Versión:      6.0.1 (21/10/2018)
% Codificación: UTF-8
%
% Autor: Pablo Pizarro R. @ppizarror
%        Facultad de Ciencias Físicas y Matemáticas
%        Universidad de Chile
%        pablo.pizarro@ing.uchile.cl, ppizarror.com
%
% Manual template: [http://latex.ppizarror.com/Template-Informe/]
% Licencia MIT:    [https://opensource.org/licenses/MIT/]

\newcommand{\lpow}[2]{
	\ensuremath{{#1}_{#2}}
}
\newcommand{\pow}[2]{
	\ensuremath{{#1}^{#2}}
}
\newcommand{\aasin}[1][]{
	\ifx\hfuzz#1\hfuzz
		\ensuremath{\sin^{-1}#1}
	\else
		\ensuremath{{\sin}^{-1}}
	\fi
}
\newcommand{\aacos}[1][]{
	\ifx\hfuzz#1\hfuzz
		\ensuremath{\cos^{-1}#1}
	\else
		\ensuremath{\cos^{-1}}
	\fi
}
\newcommand{\aatan}[1][]{
	\ifx\hfuzz#1\hfuzz
		\ensuremath{\tan^{-1}#1}
	\else
		\ensuremath{\tan^{-1}}
	\fi
}
\newcommand{\aacsc}[1][]{
	\ifx\hfuzz#1\hfuzz
		\ensuremath{\csc^{-1}#1}
	\else
		\ensuremath{\csc^{-1}}
	\fi
}
\newcommand{\aasec}[1][]{
	\ifx\hfuzz#1\hfuzz
		\ensuremath{\sec^{-1}#1}
	\else
		\ensuremath{\sec^{-1}}
	\fi
}
\newcommand{\aacot}[1][]{
	\ifx\hfuzz#1\hfuzz
		\ensuremath{\cot^{-1}#1}
	\else
		\ensuremath{\cot^{-1}}
	\fi
}
\newcommand{\fracpartial}[2]{
	\ensuremath{\pdv{#1}{#2}}
}
\newcommand{\fracdpartial}[2]{
	\ensuremath{\pdv[2]{#1}{#2}}
}
\newcommand{\fracnpartial}[3]{
	\ensuremath{\pdv[#3]{#1}{#2}}
}
\newcommand{\fracderivat}[2]{
	\ensuremath{\dv{#1}{#2}}
}
\newcommand{\fracdderivat}[2]{
	\ensuremath{\dv[2]{#1}{#2}}
}
\newcommand{\fracnderivat}[3]{
	\ensuremath{\dv[#3]{#1}{#2}}
}
\newcommand{\topequal}[2]{
	\ensuremath{\overbrace{#1}^{\mathclap{#2}}}
}
\newcommand{\underequal}[2]{
	\ensuremath{\underbrace{#1}_{\mathclap{#2}}}
}
\newcommand{\topsequal}[2]{
	\ensuremath{\overbracket{#1}^{\mathclap{#2}}}
}
\newcommand{\undersequal}[2]{
	\ensuremath{\underbracket{#1}_{\mathclap{#2}}}
}
% Template:     Informe/Reporte LaTeX
% Documento:    Funciones para insertar ecuaciones
% Versión:      6.0.0 (13/10/2018)
% Codificación: UTF-8
%
% Autor: Pablo Pizarro R. @ppizarror
%        Facultad de Ciencias Físicas y Matemáticas
%        Universidad de Chile
%        pablo.pizarro@ing.uchile.cl, ppizarror.com
%
% Manual template: [http://latex.ppizarror.com/Template-Informe/]
% Licencia MIT:    [https://opensource.org/licenses/MIT/]

\newcommand{\equationresize}[2]{
	\emptyvarerr{\equationresize}{#1}{Dimension no definida}
	\emptyvarerr{\equationresize}{#2}{Ecuacion a redimensionar no definida}
	\resizebox{#1\textwidth}{!}{$#2$}
}
\newcommand{\insertequation}[2][]{
	\emptyvarerr{\insertequation}{#2}{Ecuacion no definida}
	\ifthenelse{\equal{\numberedequation}{true}}{
		\vspace{-0.1cm}
		\begin{equation}
			\text{#1} #2
		\end{equation}
		\vspace{-0.26cm}
	}{
		\ifx\hfuzz#1\hfuzz
		\else
			\throwwarning{Label invalido en ecuacion sin numero}
		\fi
		\insertequationanum{#2}
	}
}
\newcommand{\insertequationanum}[1]{
	\emptyvarerr{\insertequationanum}{#1}{Ecuacion no definida}
	\vspace{-0.1cm}
	\begin{equation*}
		\ensuremath{#1}
	\end{equation*}
	\vspace{-0.26cm}
}
\newcommand{\insertequationcaptioned}[3][]{
	\emptyvarerr{\insertequationcaptioned}{#2}{Ecuacion no definida}
	\ifx\hfuzz#3\hfuzz
		\insertequation[#1]{#2}
	\else
		\ifthenelse{\equal{\numberedequation}{true}}{
			\vspace{-0.10cm}
			\begin{equation}
				\text{#1} #2
			\end{equation}
			\vspace{-0.65cm}
			\begin{changemargin}{\captionlrmargin cm}{\captionlrmargin cm}
				\centering \textcolor{\captiontextcolor}{#3}
				\vspace{0.05cm}
			\end{changemargin}
			\vspace{0.05cm}
		}{
			\ifx\hfuzz#1\hfuzz
			\else
				\throwwarning{Label invalido en ecuacion sin numero}
			\fi
			\insertequationcaptionedanum{#2}{#3}
		}
	\fi
}
\newcommand{\insertequationcaptionedanum}[2]{
	\emptyvarerr{\insertequationcaptionedanum}{#1}{Ecuacion no definida}
	\ifx\hfuzz#2\hfuzz
		\insertequationanum{#1}
	\else
		\vspace{-0.10cm}
		\begin{equation*}
			\ensuremath{#1}
		\end{equation*}
		\vspace{-0.65cm}
		\begin{changemargin}{\captionlrmargin cm}{\captionlrmargin cm}
			\centering \textcolor{\captiontextcolor}{#2}
			\vspace{0.05cm}
		\end{changemargin}
		\vspace{0.05cm}
	\fi
}
\newcommand{\insertgather}[1]{
	\emptyvarerr{\insertgather}{#1}{Ecuacion no definida}
	\ifthenelse{\equal{\numberedequation}{true}}{
		\vspace{-0.4cm}
		\begin{gather}
			\ensuremath{#1}
		\end{gather}
		\vspace{-0.4cm}
	}{
		\insertgatheranum{#1}
	}
}
\newcommand{\insertgatheranum}[1]{
	\emptyvarerr{\insertgatheranum}{#1}{Ecuacion no definida}
	\vspace{-0.4cm}
	\begin{gather*}
		\ensuremath{#1}
	\end{gather*}
	\vspace{-0.4cm}
}
\newcommand{\insertgathercaptioned}[2]{
	\emptyvarerr{\insertgathercaptioned}{#1}{Ecuacion no definida}
	\ifx\hfuzz#2\hfuzz
		\insertgather{#1}
	\else
		\ifthenelse{\equal{\numberedequation}{true}}{
			\vspace{-0.45cm}
			\begin{gather}
				\ensuremath{#1}
			\end{gather}
			\vspace{-0.77cm}
			\begin{changemargin}{\captionlrmargin cm}{\captionlrmargin cm}
				\centering \textcolor{\captiontextcolor}{#2}
				\vspace{0.05cm}
			\end{changemargin}
			\vspace{0cm}
		}{
			\insertgathercaptionedanum{#1}{#2}
		}
	\fi
}
\newcommand{\insertgathercaptionedanum}[2]{
	\emptyvarerr{\insertgathercaptionedanum}{#1}{Ecuacion no definida}
	\ifx\hfuzz#2\hfuzz
		\insertgatheranum{#1}
	\else
		\vspace{-0.45cm}
		\begin{gather*}
			\ensuremath{#1}
		\end{gather*}
		\vspace{-0.77cm}
		\begin{changemargin}{\captionlrmargin cm}{\captionlrmargin cm}
			\centering \textcolor{\captiontextcolor}{#2}
			\vspace{0.05cm}
		\end{changemargin}
		\vspace{0cm}
	\fi
}
\newcommand{\insertgathered}[2][]{
	\emptyvarerr{\insertgathered}{#2}{Ecuacion no definida}
	\ifthenelse{\equal{\numberedequation}{true}}{
		\vspace{-0.1cm}
		\begin{equation}
			\begin{gathered}
				\text{#1} \ensuremath{#2}
			\end{gathered}
		\end{equation}
		\vspace{-0.05cm}
	}{
		\ifx\hfuzz#1\hfuzz
		\else
			\throwwarning{Label invalido en ecuacion (gathered) sin numero}
		\fi
		\insertgatheredanum{#2}
	}
}
\newcommand{\insertgatheredanum}[1]{
	\emptyvarerr{\insertgatheredanum}{#1}{Ecuacion no definida}
	\vspace{-0.4cm}
	\begin{gather*}
		\ensuremath{#1}
	\end{gather*}
	\vspace{-0.4cm}
}
\newcommand{\insertgatheredcaptioned}[3][]{
	\emptyvarerr{\insertgatheredcaptioned}{#2}{Ecuacion no definida}
	\ifx\hfuzz#3\hfuzz
		\insertgathered[#1]{#2}
	\else
		\ifthenelse{\equal{\numberedequation}{true}}{
			\vspace{0cm}
			\begin{equation}
				\begin{gathered}
					\text{#1} \ensuremath{#2}
				\end{gathered}
			\end{equation}
			\vspace{-0.65cm}
			\begin{changemargin}{\captionlrmargin cm}{\captionlrmargin cm}
				\centering \textcolor{\captiontextcolor}{#3}
				\vspace{0.05cm}
			\end{changemargin}
			\vspace{0cm}
		}{
			\ifx\hfuzz#1\hfuzz
			\else
				\throwwarning{Label invalido en ecuacion (gathered) sin numero}
			\fi
			\insertgatheredcaptionedanum{#2}{#3}
		}
		\fi
}
\newcommand{\insertgatheredcaptionedanum}[2]{
	\emptyvarerr{\insertgatheredcaptionedanum}{#1}{Ecuacion no definida}
	\ifx\hfuzz#2\hfuzz
		\insertgatheredanum{#1}
	\else
		\vspace{-0.45cm}
		\begin{gather*}
			\ensuremath{#1}
		\end{gather*}
		\vspace{-0.7cm}
		\begin{changemargin}{\captionlrmargin cm}{\captionlrmargin cm}
			\centering \textcolor{\captiontextcolor}{#2}
			\vspace{0.05cm}
		\end{changemargin}
		\vspace{0cm}
	\fi
}
\newcommand{\insertalign}[1]{
	\emptyvarerr{\insertalign}{#1}{Ecuacion no definida}
	\ifthenelse{\equal{\numberedequation}{true}}{
		\vspace{-0.45cm}
		\begin{align}
			\ensuremath{#1}
		\end{align}
		\vspace{-0.4cm}
	}{
		\insertalignanum{#1}
	}
}
\newcommand{\insertalignanum}[1]{
	\emptyvarerr{\insertalignanum}{#1}{Ecuacion no definida}
	\vspace{-0.45cm}
	\begin{align*}
		\ensuremath{#1}
	\end{align*}
	\vspace{-0.4cm}
}
\newcommand{\insertaligncaptioned}[2]{
	\emptyvarerr{\insertaligncaptioned}{#1}{Ecuacion no definida}
	\ifx\hfuzz#2\hfuzz
		\insertalign{#1}
	\else
		\ifthenelse{\equal{\numberedequation}{true}}{
			\vspace{-0.45cm}
			\begin{align}
				\ensuremath{#1}
			\end{align}
			\vspace{-0.77cm}
			\begin{changemargin}{\captionlrmargin cm}{\captionlrmargin cm}
				\centering \textcolor{\captiontextcolor}{#2}
				\vspace{0.05cm}
			\end{changemargin}
			\vspace{0cm}
		}{
			\insertaligncaptionedanum{#1}{#2}
		}
	\fi
}
\newcommand{\insertaligncaptionedanum}[2]{
	\emptyvarerr{\insertaligncaptioned}{#1}{Ecuacion no definida}
	\ifx\hfuzz#2\hfuzz
		\insertalignanum{#1}
	\else
		\vspace{-0.45cm}
		\begin{align*}
			\ensuremath{#1}
		\end{align*}
		\vspace{-0.77cm}
		\begin{changemargin}{\captionlrmargin cm}{\captionlrmargin cm}
			\centering \textcolor{\captiontextcolor}{#2}
			\vspace{0.05cm}
		\end{changemargin}
		\vspace{0cm}
	\fi
}
\newcommand{\insertaligned}[2][]{
	\emptyvarerr{\insertaligned}{#2}{Ecuacion no definida}
	\ifthenelse{\equal{\numberedequation}{true}}{
		\vspace{-0.1cm}
		\begin{equation}
			\begin{aligned}
				\text{#1} \ensuremath{#2}
			\end{aligned}
		\end{equation}
		\vspace{-0.05cm}
	}{
		\ifx\hfuzz#1\hfuzz
		\else
			\throwwarning{Label invalido en ecuacion (aligned) sin numero}
		\fi
		\insertalignedanum{#2}
	}
}
\newcommand{\insertalignedanum}[1]{
	\emptyvarerr{\insertalignedanum}{#1}{Ecuacion no definida}
	\vspace{-0.45cm}
	\begin{align*}
		\ensuremath{#1}
	\end{align*}
	\vspace{-0.4cm}
}
\newcommand{\insertalignedcaptioned}[3][]{
	\emptyvarerr{\insertalignedcaptioned}{#2}{Ecuacion no definida}
	\ifx\hfuzz#3\hfuzz
		\insertaligned[#1]{#2}
	\else
		\ifthenelse{\equal{\numberedequation}{true}}{
			\vspace{0cm}
			\begin{equation}
				\begin{aligned}
					\text{#1} \ensuremath{#2}
				\end{aligned}
			\end{equation}
			\vspace{-0.65cm}
			\begin{changemargin}{\captionlrmargin cm}{\captionlrmargin cm}
				\centering \textcolor{\captiontextcolor}{#3}
				\vspace{0.05cm}
			\end{changemargin}
			\vspace{0cm}
		}{
			\ifx\hfuzz#1\hfuzz
			\else
				\throwwarning{Label invalido en ecuacion (aligned) sin numero}
			\fi
			\insertalignedcaptionedanum{#2}{#3}
		}
	\fi
}
\newcommand{\insertalignedcaptionedanum}[2]{
	\emptyvarerr{\insertalignedcaptioned}{#1}{Ecuacion no definida}
	\ifx\hfuzz#2\hfuzz
		\insertalignedanum{#1}
	\else
		\vspace{0cm}
		\begin{equation}
			\begin{aligned}
				\ensuremath{#1}
			\end{aligned}
		\end{equation}
		\vspace{-0.65cm}
		\begin{changemargin}{\captionlrmargin cm}{\captionlrmargin cm}
			\centering \textcolor{\captiontextcolor}{#2}
			\vspace{0.05cm}
		\end{changemargin}
		\vspace{0cm}
	\fi
}
% Template:     Informe/Reporte LaTeX
% Documento:    Funciones para insertar imágenes
% Versión:      5.5.5 (27/09/2018)
% Codificación: UTF-8
%
% Autor: Pablo Pizarro R. @ppizarror
%        Facultad de Ciencias Físicas y Matemáticas
%        Universidad de Chile
%        pablo.pizarro@ing.uchile.cl, ppizarror.com
%
% Manual template: [http://latex.ppizarror.com/Template-Informe/]
% Licencia MIT:    [https://opensource.org/licenses/MIT/]

\newcommand{\addimage}[3]{\addimageboxed{#1}{#2}{0}{#3}}\newcommand{\addimageboxed}[4]{\checkonlyonenvimage\begingroup\setlength{\fboxsep}{0 pt}\setlength{\fboxrule}{#3 pt}\hspace{\marginrightmultimage cm}\subfloat[#4]{\fbox{\includegraphics[#2]{#1}}}\endgroup}\newcommand{\insertimage}[4][]{\insertimageboxed[#1]{#2}{#3}{0}{#4}}\newcommand{\insertimageboxed}[5][]{\emptyvarerr{\insertimageboxed}{#2}{Direccion de la imagen no definida}\emptyvarerr{\insertimageboxed}{#3}{Parametros de la imagen no definidos}\emptyvarerr{\insertimageboxed}{#4}{Ancho de la linea no definido}\checkoutsideenvimage\vspace{\margintopimages cm}\begin{figure}[H]\begingroup\setlength{\fboxsep}{0 pt}\setlength{\fboxrule}{#4 pt}\centering\fbox{\includegraphics[#3]{#2}}\endgroup\ifx\hfuzz#5\hfuzz\vspace{\captionlessmarginimage cm}\else\hspace{0cm}\caption{#5 #1}\fi\end{figure}\vspace{\marginbottomimages cm}}\newcommand{\insertdoubleimage}[8][]{\emptyvarerr{\insertdoubleimage}{#2}{Direccion de la imagen 1 no definida}\emptyvarerr{\insertdoubleimage}{#3}{Parametros de la imagen 1 no definidos}\emptyvarerr{\insertdoubleimage}{#5}{Direccion de la imagen 2 no definida}\emptyvarerr{\insertdoubleimage}{#6}{Parametros de la imagen 2 no definidos}\checkoutsideenvimage\alertdeprecatedcmdimage{\insertdoubleimage}\vspace{\margintopimages cm}\captionsetup{margin=\captionmarginmultimg cm}\begin{figure}[H] \centering\subfloat[#4]{\includegraphics[#3]{#2}}\hspace{\marginrightmultimage cm}\subfloat[#7]{\includegraphics[#6]{#5}}\setcaptionmargincm{\captionlrmargin}\ifx\hfuzz#8\hfuzz\vspace{\captionlessmarginimage cm}\else\caption{#8 #1}\fi\end{figure}\setcaptionmargincm{\captionlrmargin}\vspace{\marginbottomimages cm}}\newcommand{\insertdoubleeqimage}[7][]{\insertdoubleimage[#1]{#2}{#6}{#3}{#4}{#6}{#5}{#7}}\newcommand{\inserttripleimage}[8][]{\emptyvarerr{\inserttripleimage}{#2}{Direccion de la imagen 1 no definida}\emptyvarerr{\inserttripleimage}{#3}{Parametros de la imagen 1 no definidos}\emptyvarerr{\inserttripleimage}{#4}{Direccion de la imagen 2 no definida}\emptyvarerr{\inserttripleimage}{#5}{Parametros de la imagen 2 no definidos}\emptyvarerr{\inserttripleimage}{#6}{Direccion de la imagen 3 no definida}\emptyvarerr{\inserttripleimage}{#7}{Parametros de la imagen 3 no definidos}\checkoutsideenvimage\alertdeprecatedcmdimage{\inserttripleimage}\vspace{\margintopimages cm}\captionsetup{margin=\captionmarginmultimg cm}\begin{figure}[H] \centering\subfloat[]{\includegraphics[#3]{#2}}\hspace{\marginrightmultimage cm}\subfloat[]{\includegraphics[#5]{#4}}\hspace{\marginrightmultimage cm}\subfloat[]{\includegraphics[#7]{#6}}\setcaptionmargincm{\captionlrmargin}\ifx\hfuzz#8\hfuzz\vspace{\captionlessmarginimage cm}\else\caption{#8 #1}\fi\end{figure}\setcaptionmargincm{\captionlrmargin}\vspace{\marginbottomimages cm}}\newcommand{\inserttripleeqimage}[6][]{\inserttripleimage[#1]{#2}{#5}{#3}{#5}{#4}{#5}{#6}}\newcommand{\insertquadimage}[7][]{\emptyvarerr{\insertquadimage}{#2}{Direccion de la imagen 1 no definida}\emptyvarerr{\insertquadimage}{#3}{Direccion de la imagen 2 no definida}\emptyvarerr{\insertquadimage}{#4}{Direccion de la imagen 3 no definida}\emptyvarerr{\insertquadimage}{#5}{Direccion de la imagen 4 no definida}\emptyvarerr{\insertquadimage}{#6}{Propiedades de las imagenes no definidos}\checkoutsideenvimage\alertdeprecatedcmdimage{\insertquadimage}\vspace{\margintopimages cm}\captionsetup{margin=\captionmarginmultimg cm cm}\begin{figure}[H] \centering\subfloat[]{\includegraphics[#6]{#2}}\hspace{\marginrightmultimage cm}\subfloat[]{\includegraphics[#6]{#3}}\hspace{\marginrightmultimage cm}\subfloat[]{\includegraphics[#6]{#4}}\hspace{\marginrightmultimage cm}\subfloat[]{\includegraphics[#6]{#5}}\setcaptionmargincm{\captionlrmargin}\ifx\hfuzz#7\hfuzz\vspace{\captionlessmarginimage cm}\else\caption{#7 #1}\fi\end{figure}\setcaptionmargincm{\captionlrmargin}\vspace{\marginbottomimages cm}}\newcommand{\insertpentaimage}[8][]{\emptyvarerr{\insertpentaimage}{#2}{Direccion de la imagen 1 no definida}\emptyvarerr{\insertpentaimage}{#3}{Direccion de la imagen 2 no definida}\emptyvarerr{\insertpentaimage}{#4}{Direccion de la imagen 3 no definida}\emptyvarerr{\insertpentaimage}{#5}{Direccion de la imagen 4 no definida}\emptyvarerr{\insertpentaimage}{#6}{Direccion de la imagen 5 no definida}\emptyvarerr{\insertpentaimage}{#7}{Propiedades de las imagenes no definidas}\checkoutsideenvimage\alertdeprecatedcmdimage{\insertpentaimage}\vspace{\margintopimages cm}\captionsetup{margin=\captionmarginmultimg cm cm}\begin{figure}[H] \centering\subfloat[]{\includegraphics[#7]{#2}}\hspace{\marginrightmultimage cm}\subfloat[]{\includegraphics[#7]{#3}}\hspace{\marginrightmultimage cm}\subfloat[]{\includegraphics[#7]{#4}}\hspace{\marginrightmultimage cm}\subfloat[]{\includegraphics[#7]{#5}}\hspace{\marginrightmultimage cm}\subfloat[]{\includegraphics[#7]{#6}}\setcaptionmargincm{\captionlrmargin}\ifx\hfuzz#8\hfuzz\vspace{\captionlessmarginimage cm}\else\caption{#8 #1}\fi\end{figure}\setcaptionmargincm{\captionlrmargin}\vspace{\marginbottomimages cm}}\newcommand{\inserthexaimage}[9][]{\emptyvarerr{\inserthexaimage}{#2}{Direccion de la imagen 1 no definida}\emptyvarerr{\inserthexaimage}{#3}{Direccion de la imagen 2 no definida}\emptyvarerr{\inserthexaimage}{#4}{Direccion de la imagen 3 no definida}\emptyvarerr{\inserthexaimage}{#5}{Direccion de la imagen 4 no definida}\emptyvarerr{\inserthexaimage}{#6}{Direccion de la imagen 5 no definida}\emptyvarerr{\inserthexaimage}{#7}{Direccion de la imagen 6 no definida}\emptyvarerr{\inserthexaimage}{#8}{Propiedades de las imagenes no definidas}\checkoutsideenvimage\alertdeprecatedcmdimage{\inserthexaimage}\vspace{\margintopimages cm}\captionsetup{margin=\captionmarginmultimg cm}\begin{figure}[H] \centering\subfloat[]{\includegraphics[#8]{#2}}\hspace{\marginrightmultimage cm}\subfloat[]{\includegraphics[#8]{#3}}\hspace{\marginrightmultimage cm}\subfloat[]{\includegraphics[#8]{#4}}\hspace{\marginrightmultimage cm}\subfloat[]{\includegraphics[#8]{#5}}\hspace{\marginrightmultimage cm}\subfloat[]{\includegraphics[#8]{#6}}\hspace{\marginrightmultimage cm}\subfloat[]{\includegraphics[#8]{#7}}\setcaptionmargincm{\captionlrmargin}\ifx\hfuzz#9\hfuzz\vspace{\captionlessmarginimage cm}\else\caption{#9 #1}\fi\end{figure}\setcaptionmargincm{\captionlrmargin}\vspace{\marginbottomimages cm}}\newcommand{\insertimageleft}[4][]{\insertimageleftboxed[#1]{#2}{#3}{0}{#4}}\newcommand{\insertimageleftboxed}[5][]{\emptyvarerr{\insertimageleftboxed}{#2}{Direccion de la imagen no definida}\emptyvarerr{\insertimageleftboxed}{#3}{Ancho de la imagen no definido}\emptyvarerr{\insertimageleftboxed}{#4}{Ancho de la linea no definido}\checkoutsideenvimage~\vspace{-\baselineskip}\par\begin{wrapfigure}{l}{#3\textwidth}\setcaptionmargincm{0}\ifthenelse{\equal{\figurecaptiontop}{true}}{}{\vspace{\marginfloatimages pt}}\begingroup\setlength{\fboxsep}{0 pt}\setlength{\fboxrule}{#4 pt}\centering\fbox{\includegraphics[width=\linewidth]{#2}}\endgroup\ifx\hfuzz#5\hfuzz\vspace{\captionlessmarginimage cm}\else\caption{#5 #1}\fi\end{wrapfigure}\setcaptionmargincm{\captionlrmargin}}\newcommand{\insertimageleftline}[5][]{\insertimageleftlineboxed[#1]{#2}{#3}{0}{#4}{#5}}\newcommand{\insertimageleftlineboxed}[6][]{\emptyvarerr{\insertimageleftlineboxed}{#2}{Direccion de la imagen no definida}\emptyvarerr{\insertimageleftlineboxed}{#3}{Ancho de la imagen no definido}\emptyvarerr{\insertimageleftlineboxed}{#4}{Ancho de la linea no definido}\emptyvarerr{\insertimageleftlineboxed}{#5}{Altura en lineas de la imagen flotante izquierda no definida}\checkoutsideenvimage~\vspace{-\baselineskip}\par\begin{wrapfigure}[#5]{l}{#3\textwidth}\setcaptionmargincm{0}\ifthenelse{\equal{\figurecaptiontop}{true}}{}{\vspace{\marginfloatimages pt}}\begingroup\setlength{\fboxsep}{0 pt}\setlength{\fboxrule}{#4 pt}\centering\fbox{\includegraphics[width=\linewidth]{#2}}\endgroup\ifx\hfuzz#6\hfuzz\vspace{\captionlessmarginimage cm}\else\caption{#6 #1}\fi\end{wrapfigure}\setcaptionmargincm{\captionlrmargin}}\newcommand{\insertimageright}[4][]{\insertimagerightboxed[#1]{#2}{#3}{0}{#4}}\newcommand{\insertimagerightboxed}[5][]{\emptyvarerr{\insertimagerightboxed}{#2}{Direccion de la imagen no definida}\emptyvarerr{\insertimagerightboxed}{#3}{Ancho de la imagen no defindo}\emptyvarerr{\insertimagerightboxed}{#4}{Ancho de la linea no definido}\checkoutsideenvimage~\vspace{-\baselineskip}\par\begin{wrapfigure}{r}{#3\textwidth}\setcaptionmargincm{0}\ifthenelse{\equal{\figurecaptiontop}{true}}{}{\vspace{\marginfloatimages pt}}\begingroup\setlength{\fboxsep}{0 pt}\setlength{\fboxrule}{#4 pt}\centering\fbox{\includegraphics[width=\linewidth]{#2}}\endgroup\ifx\hfuzz#5\hfuzz\vspace{\captionlessmarginimage cm}\else\caption{#5 #1}\fi\end{wrapfigure}\setcaptionmargincm{\captionlrmargin}}\newcommand{\insertimagerightline}[5][]{\insertimagerightlineboxed[#1]{#2}{#3}{0}{#4}{#5}}\newcommand{\insertimagerightlineboxed}[6][]{\emptyvarerr{\insertimagerightlineboxed}{#2}{Direccion de la imagen no definida}\emptyvarerr{\insertimagerightlineboxed}{#3}{Ancho de la imagen no defindo}\emptyvarerr{\insertimagerightlineboxed}{#4}{Ancho de la linea no definido}\emptyvarerr{\insertimagerightlineboxed}{#5}{Altura en lineas de la imagen flotante derecha no definida}\checkoutsideenvimage~\vspace{-\baselineskip}\par\begin{wrapfigure}[#5]{r}{#3\textwidth}\setcaptionmargincm{0}\ifthenelse{\equal{\figurecaptiontop}{true}}{}{\vspace{\marginfloatimages pt}}\begingroup\setlength{\fboxsep}{0 pt}\setlength{\fboxrule}{#4 pt}\centering\fbox{\includegraphics[width=\linewidth]{#2}}\endgroup\ifx\hfuzz#6\hfuzz\vspace{\captionlessmarginimage cm}\else\caption{#6 #1}\fi\end{wrapfigure}\setcaptionmargincm{\captionlrmargin}}\newcommand{\insertimageleftp}[5][]{\xspace~\\\vspace{-2\baselineskip}\par\insertimageleftboxedp[#1]{#2}{#3}{#4}{0}{#5}}\newcommand{\insertimageleftboxedp}[6][]{\emptyvarerr{\insertimageleftboxedp}{#2}{Direccion de la imagen no definida}\emptyvarerr{\insertimageleftboxedp}{#3}{Ancho del objeto no definido}\emptyvarerr{\insertimageleftboxedp}{#4}{Propiedades de la imagen no defindos}\emptyvarerr{\insertimageleftboxedp}{#5}{Ancho de la linea no definido}\checkoutsideenvimage~\vspace{-\baselineskip}\par\begin{wrapfigure}{l}{#3}\setcaptionmargincm{0}\ifthenelse{\equal{\figurecaptiontop}{true}}{}{\vspace{\marginfloatimages pt}}\begingroup\setlength{\fboxsep}{0 pt}\setlength{\fboxrule}{#5 pt}\centering\fbox{\includegraphics[#4]{#2}}\endgroup\ifx\hfuzz#6\hfuzz\vspace{\captionlessmarginimage cm}\else\caption{#6 #1}\fi\end{wrapfigure}\setcaptionmargincm{\captionlrmargin}}\newcommand{\insertimageleftlinep}[6][]{\insertimageleftlineboxedp[#1]{#2}{#3}{#4}{0}{#5}{#6}}\newcommand{\insertimageleftlineboxedp}[7][]{\emptyvarerr{\insertimageleftlineboxedp}{#2}{Direccion de la imagen no definida}\emptyvarerr{\insertimageleftlineboxedp}{#3}{Ancho del objeto no definido}\emptyvarerr{\insertimageleftlineboxedp}{#4}{Propiedades de la imagen no definidos}\emptyvarerr{\insertimageleftlineboxedp}{#5}{Ancho de la linea no definido}\emptyvarerr{\insertimageleftlineboxedp}{#6}{Altura en lineas de la imagen flotante izquierda no definida}\checkoutsideenvimage~\vspace{-\baselineskip}\par\begin{wrapfigure}[#6]{l}{#3}\setcaptionmargincm{0}\ifthenelse{\equal{\figurecaptiontop}{true}}{}{\vspace{\marginfloatimages pt}}\begingroup\setlength{\fboxsep}{0 pt}\setlength{\fboxrule}{#5 pt}\centering\fbox{\includegraphics[#4]{#2}}\endgroup\ifx\hfuzz#7\hfuzz\vspace{\captionlessmarginimage cm}\else\caption{#7 #1}\fi\end{wrapfigure}\setcaptionmargincm{\captionlrmargin}}\newcommand{\insertimagerightp}[5][]{\xspace~\\\vspace{-2\baselineskip}\par\insertimagerightboxedp[#1]{#2}{#3}{#4}{0}{#5}}\newcommand{\insertimagerightboxedp}[6][]{\emptyvarerr{\insertimagerightboxedp}{#2}{Direccion de la imagen no definida}\emptyvarerr{\insertimagerightboxedp}{#3}{Ancho del objeto no definido}\emptyvarerr{\insertimagerightboxedp}{#4}{Propiedades de la imagen no definidos}\emptyvarerr{\insertimagerightboxedp}{#5}{Ancho de la linea no definido}\checkoutsideenvimage~\vspace{-\baselineskip}\par\begin{wrapfigure}{r}{#3}\setcaptionmargincm{0}\ifthenelse{\equal{\figurecaptiontop}{true}}{}{\vspace{\marginfloatimages pt}}\begingroup\setlength{\fboxsep}{0 pt}\setlength{\fboxrule}{#5 pt}\centering\fbox{\includegraphics[#4]{#2}}\endgroup\ifx\hfuzz#6\hfuzz\vspace{\captionlessmarginimage cm}\else\caption{#6 #1}\fi\end{wrapfigure}\setcaptionmargincm{\captionlrmargin}}\newcommand{\insertimagerightlinep}[6][]{\insertimagerightlineboxedp[#1]{#2}{#3}{#4}{0}{#5}{#6}}\newcommand{\insertimagerightlineboxedp}[7][]{\emptyvarerr{\insertimagerightlineboxedp}{#2}{Direccion de la imagen no definida}\emptyvarerr{\insertimagerightlineboxedp}{#3}{Ancho del objeto no definido}\emptyvarerr{\insertimagerightlineboxedp}{#4}{Propiedades de la imagen no definidos}\emptyvarerr{\insertimagerightlineboxedp}{#5}{Ancho de la linea no definido}\emptyvarerr{\insertimagerightlineboxedp}{#6}{Altura en lineas de la imagen flotante derecha no definida}\checkoutsideenvimage~\vspace{-\baselineskip}\par\begin{wrapfigure}[#6]{r}{#3}\setcaptionmargincm{0}\ifthenelse{\equal{\figurecaptiontop}{true}}{}{\vspace{\marginfloatimages pt}}\begingroup\setlength{\fboxsep}{0 pt}\setlength{\fboxrule}{#5 pt}\centering\fbox{\includegraphics[#4]{#2}}\endgroup\ifx\hfuzz#7\hfuzz\vspace{\captionlessmarginimage cm}\else\caption{#7 #1}\fi\end{wrapfigure}\setcaptionmargincm{\captionlrmargin}}
% Template:     Informe/Reporte LaTeX
% Documento:    Funciones para insertar títulos
% Versión:      6.0.0 (13/10/2018)
% Codificación: UTF-8
%
% Autor: Pablo Pizarro R. @ppizarror
%        Facultad de Ciencias Físicas y Matemáticas
%        Universidad de Chile
%        pablo.pizarro@ing.uchile.cl, ppizarror.com
%
% Manual template: [http://latex.ppizarror.com/Template-Informe/]
% Licencia MIT:    [https://opensource.org/licenses/MIT/]

\pretocmd{\section}{
	\ifthenelse{\equal{\GLOBALsectionalph}{true}}{
		\renewcommand{\thesubsection}{\Alph{section}.\arabic{subsection}}
	}{
		\renewcommand{\thesubsection}{\arabic{section}.\arabic{subsection}}
	}
	\def\GLOBALsectionanumenabled{false}
	\def\GLOBALsubsectionanumenabled{false}
}{}{}
\pretocmd{\subsection}{
	\ifthenelse{\equal{\GLOBALsectionanumenabled}{true}}{
		\renewcommand{\thesubsubsection}{\arabic{subsection}.\arabic{subsubsection}}
	}{
		\ifthenelse{\equal{\GLOBALsectionalph}{true}}{
			\renewcommand{\thesubsubsection}{\Alph{section}.\arabic{subsection}.\arabic{subsubsection}}
		}{
			\renewcommand{\thesubsubsection}{\arabic{section}.\arabic{subsection}.\arabic{subsubsection}}
		}
	}
	\def\GLOBALsubsectionanumenabled{false}
}{}{}
\pretocmd{\subsubsection}{
	\ifthenelse{\equal{\GLOBALsubsectionanumenabled}{true}}{
		\renewcommand{\thesubsubsubsection}{\arabic{subsubsection}.\arabic{subsubsubsection}}
	}{
		\ifthenelse{\equal{\GLOBALsectionanumenabled}{true}}{
			\ifthenelse{\equal{\showdotontitles}{true}}{
				\renewcommand{\thesubsubsubsection}{\arabic{subsection}.\arabic{subsubsection}.\arabic{subsubsubsection}.}
			}{
				\renewcommand{\thesubsubsubsection}{\arabic{subsection}.\arabic{subsubsection}.\arabic{subsubsubsection}}
			}
		}{
			\ifthenelse{\equal{\showdotontitles}{true}}{
				\ifthenelse{\equal{\GLOBALsectionalph}{true}}{
					\renewcommand{\thesubsubsubsection}{\Alph{section}.\arabic{subsection}.\arabic{subsubsection}.\arabic{subsubsubsection}.}	
				}{
					\renewcommand{\thesubsubsubsection}{\arabic{section}.\arabic{subsection}.\arabic{subsubsection}.\arabic{subsubsubsection}.}
				}
			}{
				\ifthenelse{\equal{\GLOBALsectionalph}{true}}{
					\renewcommand{\thesubsubsubsection}{\Alph{section}.\arabic{subsection}.\arabic{subsubsection}.\arabic{subsubsubsection}}
				}{
					\renewcommand{\thesubsubsubsection}{\arabic{section}.\arabic{subsection}.\arabic{subsubsection}.\arabic{subsubsubsection}}
				}
			}
		}
	}
}{}{}
\newcommand{\sectionanum}[1]{
	\emptyvarerr{\sectionanum}{#1}{Titulo no definido}
	\phantomsection
	\needspace{3\baselineskip}
	\section*{#1}
	\addcontentsline{toc}{section}{#1}
	\ifthenelse{\equal{\anumsecaddtocounter}{true}}{\stepcounter{section}}{}
	\changeheadertitle{#1}
	\setcounter{subsection}{0}
	\renewcommand{\thesubsection}{\arabic{subsection}}
	\def\GLOBALsectionanumenabled{true}
}
\newcommand{\sectionanumnoi}[1]{
	\emptyvarerr{\sectionanumnoi}{#1}{Titulo no definido}
	\phantomsection
	\needspace{3\baselineskip}
	\section*{#1}
	\ifthenelse{\equal{\anumsecaddtocounter}{true}}{\stepcounter{section}}{}
	\changeheadertitle{#1}
	\setcounter{subsection}{0}
	\renewcommand{\thesubsection}{\arabic{subsection}}
	\def\GLOBALsectionanumenabled{true}
}
\newcommand{\sectionanumheadless}[1]{
	\emptyvarerr{\sectionanumnoheadless}{#1}{Titulo no definido}
	\section*{#1}
	\addcontentsline{toc}{section}{#1}
	\ifthenelse{\equal{\anumsecaddtocounter}{true}}{\stepcounter{section}}{}
	\setcounter{subsection}{0}
	\renewcommand{\thesubsection}{\arabic{subsection}}
	\def\GLOBALsectionanumenabled{true}
}
\newcommand{\sectionanumnoiheadless}[1]{
	\emptyvarerr{\sectionanumnoi}{#1}{Titulo no definido}
	\section*{#1}
	\ifthenelse{\equal{\anumsecaddtocounter}{true}}{\stepcounter{section}}{}
	\setcounter{subsection}{0}
	\renewcommand{\thesubsection}{\arabic{subsection}}
	\def\GLOBALsectionanumenabled{true}
}
\newcommand{\subsectionanum}[1]{
	\emptyvarerr{\subsectionanum}{#1}{Subtitulo no definido}
	\subsection*{#1}
	\addcontentsline{toc}{subsection}{#1}
	\ifthenelse{\equal{\anumsecaddtocounter}{true}}{\stepcounter{subsection}}{}
	\setcounter{subsubsection}{0}
	\renewcommand{\thesubsubsection}{\arabic{subsubsection}}
	\def\GLOBALsubsectionanumenabled{true}
}
\newcommand{\subsectionanumnoi}[1]{
	\emptyvarerr{\subsectionanumnoi}{#1}{Subtitulo no definido}
	\subsection*{#1}
	\ifthenelse{\equal{\anumsecaddtocounter}{true}}{\stepcounter{subsection}}{}
	\setcounter{subsubsection}{0}
	\renewcommand{\thesubsubsection}{\arabic{subsubsection}}
	\def\GLOBALsubsectionanumenabled{true}
}
\newcommand{\subsubsectionanum}[1]{
	\emptyvarerr{\subsubsectionanum}{#1}{Sub-subtitulo no definido}
	\subsubsection*{#1}
	\addcontentsline{toc}{subsubsection}{#1}
	\ifthenelse{\equal{\anumsecaddtocounter}{true}}{\stepcounter{subsubsection}}{}
	\setcounter{subsubsubsection}{0}
	\ifthenelse{\equal{\showdotontitles}{true}}{
		\renewcommand{\thesubsubsubsection}{\arabic{subsubsubsection}.}
	}{
		\renewcommand{\thesubsubsubsection}{\arabic{subsubsubsection}}
	}
}
\newcommand{\subsubsectionanumnoi}[1]{
	\emptyvarerr{\subsubsectionanumnoi}{#1}{Sub-subtitulo no definido}
	\subsubsection*{#1}
	\ifthenelse{\equal{\anumsecaddtocounter}{true}}{\stepcounter{subsubsection}}{}
	\setcounter{subsubsubsection}{0}
	\ifthenelse{\equal{\showdotontitles}{true}}{
		\renewcommand{\thesubsubsubsection}{\arabic{subsubsubsection}.}
	}{
		\renewcommand{\thesubsubsubsection}{\arabic{subsubsubsection}}
	}
}
\newcommand{\subsubsubsectionanum}[1]{
	\emptyvarerr{\subsubsubsectionanum}{#1}{Sub-sub-subtitulo no definido}
	\subsubsubsection*{#1}
	\addcontentsline{toc}{subsubsubsection}{#1}
	\ifthenelse{\equal{\anumsecaddtocounter}{true}}{\stepcounter{subsubsubsection}}{}
}
\newcommand{\subsubsubsectionanumnoi}[1]{
	\emptyvarerr{\subsubsubsectionanumnoi}{#1}{Sub-sub-subtitulo no definido}
	\subsubsection*{#1}
	\ifthenelse{\equal{\anumsecaddtocounter}{true}}{\stepcounter{subsubsubsection}}{}
}
\newcommand{\changeheadertitle}[1]{
	\emptyvarerr{\changeheadertitle}{#1}{Titulo no definido}
	\markboth{#1}{}
}
\newcommand{\insertindextitle}[2]{
	\emptyvarerr{\insertindextitle}{#1}{Titulo no definido}
	\ifx\hfuzz#2\hfuzz
		\addtocontents{toc}{\protect\addvspace{\indextitlemargin pt}}
	\else
		\addtocontents{toc}{\protect\addvspace{#2 pt}}
	\fi
	\addtocontents{toc}{\noindent\hyperref[swpn]{\textbf{#1}}}
}
\newcommand{\newchapter}[1]{
	\emptyvarerr{\newchapter}{#1}{Titulo no definido}
	\newpage
	\stepcounter{section}
	\phantomsection
	\needspace{3\baselineskip}
	\vspace* {3cm}
	\noindent {\huge{\textbf{\nomchapter\ \thesection}}} \\
	\vspace* {0.5cm} \\
	\noindent {\Huge{\textbf{#1}}} \\
	\vspace {0.5cm} \\
	\addcontentsline{toc}{section}{\protect\numberline{\thesection}#1}
	\markboth{#1}{}
}
% Template:     Informe/Reporte LaTeX
% Documento:    Otros estilos
% Versión:      6.1.0 (03/11/2018)
% Codificación: UTF-8
%
% Autor: Pablo Pizarro R. @ppizarror
%        Facultad de Ciencias Físicas y Matemáticas
%        Universidad de Chile
%        pablo.pizarro@ing.uchile.cl, ppizarror.com
%
% Manual template: [https://latex.ppizarror.com/Template-Informe/]
% Licencia MIT:    [https://opensource.org/licenses/MIT/]

\RequirePackage{enumitem}
\makeatletter
\def\greek#1{\expandafter\@greek\csname c@#1\endcsname}
\def\Greek#1{\expandafter\@Greek\csname c@#1\endcsname}
\def\@greek#1{
	\ifcase#1
		\or $\alpha$
		\or $\beta$
		\or $\gamma$
		\or $\delta$
		\or $\epsilon$
		\or $\zeta$
		\or $\eta$
		\or $\theta$
		\or $\iota$
		\or $\kappa$
		\or $\lambda$
		\or $\mu$
		\or $\nu$
		\or $\xi$
		\or $o$
		\or $\pi$
		\or $\rho$
		\or $\sigma$
		\or $\tau$
		\or $\upsilon$
		\or $\phi$
		\or $\chi$
		\or $\psi$
		\or $\omega$
	\fi
}
\def\@Greek#1{
	\ifcase#1
		\or $\mathrm{A}$
		\or $\mathrm{B}$
		\or $\Gamma$
		\or $\Delta$
		\or $\mathrm{E}$
		\or $\mathrm{Z}$
		\or $\mathrm{H}$
		\or $\Theta$
		\or $\mathrm{I}$
		\or $\mathrm{K}$
		\or $\Lambda$
		\or $\mathrm{M}$
		\or $\mathrm{N}$
		\or $\Xi$
		\or $\mathrm{O}$
		\or $\Pi$
		\or $\mathrm{P}$
		\or $\Sigma$
		\or $\mathrm{T}$
		\or $\mathrm{Y}$
		\or $\Phi$
		\or $\mathrm{X}$
		\or $\Psi$
		\or $\Omega$
	\fi
}
\makeatother
\AddEnumerateCounter{\greek}{\@greek}{24}
\AddEnumerateCounter{\Greek}{\@Greek}{12}
\newcolumntype{P}[1]{
	>{\centering\arraybackslash}p{#1}
}
% Template:     Informe/Reporte LaTeX
% Documento:    Funciones para crear columnas con contenido
% Versión:      6.0.6 (30/10/2018)
% Codificación: UTF-8
%
% Autor: Pablo Pizarro R. @ppizarror
%        Facultad de Ciencias Físicas y Matemáticas
%        Universidad de Chile
%        pablo.pizarro@ing.uchile.cl, ppizarror.com
%
% Manual template: [https://latex.ppizarror.com/Template-Informe/]
% Licencia MIT:    [https://opensource.org/licenses/MIT/]

\newcommand{\createtwocolumn}[5]{
	\setcaptionmargincm{0}
	\begin{flushleft}
		\hspace{0cm}
		\begin{tabular}{c}
			\hspace{-0.34cm}
			\begin{minipage}[t]{#1\linewidth}
				\vspace{-2em}\nobreak~ #4
			\end{minipage}
			\hspace{\columnhspace cm}
			\hspace{#3}
			\begin{minipage}[t]{#2\linewidth}
				\vspace{-2em}\nobreak~ #5
			\end{minipage}
			\\
		\end{tabular}
		~
	\end{flushleft}
	\setcaptionmargincm{\captionlrmargin}
}
\newcommand{\createtwocolumnl}[6]{
	\createthreecolumn{0.001}{#1}{#2}{#3}{#4}{~}{#5}{#6}
}
\newcommand{\createhalfcolumn}[2]{
	\createtwocolumn{0.5}{0.5}{0cm}{#1}{#2}
}
\newcommand{\createtwocolumnc}[5]{
	\createtwocolumn{#1}{#2}{#3}{\begin{center}#4\end{center}}{\begin{center}#5\end{center}}
}
\newcommand{\createtwocolumncl}[6]{
	\createtwocolumnl{#1}{#2}{#3}{#4}{\begin{center}#5\end{center}}{\begin{center}#6\end{center}}
}
\newcommand{\createhalfcolumnc}[2]{
	\createtwocolumnc{0.493}{0.493}{0cm}{#1}{#2}
}
\newcommand{\createthreecolumn}[8]{
	\setcaptionmargincm{0}
	\begin{flushleft}
		\hspace{0cm}
		\begin{tabular}{l}
			\hspace{-0.34cm}
			\begin{minipage}[t]{#1\linewidth}
				\vspace{-2em}\nobreak~ #6
			\end{minipage}
			\hspace{\columnhspace cm}
			\hspace{#4}
			\begin{minipage}[t]{#2\linewidth}
				\vspace{-2em}\nobreak~ #7
			\end{minipage}
			\hspace{\columnhspace cm}
			\hspace{#5}
			\begin{minipage}[t]{#3\linewidth}
				\vspace{-2em}\nobreak~ #8
			\end{minipage}
			\\
		\end{tabular}
		~
	\end{flushleft}
	\setcaptionmargincm{\captionlrmargin}
}
\newcommand{\createthirdcolumn}[3]{
	\createthreecolumn{0.3333}{0.3333}{0.333}{0cm}{0cm}{#1}{#2}{#3}
}
\newcommand{\createthreecolumnc}[8]{
	\createthreecolumn{#1}{#2}{#3}{#4}{#5}{\begin{center}#6\end{center}}{\begin{center}#7\end{center}}{\begin{center}#8\end{center}}
}
\newcommand{\createthirdcolumnc}[3]{
	\createthreecolumnc{0.3333}{0.3333}{0.3333}{0cm}{0cm}{#1}{#2}{#3}
}
% Template:     Informe/Reporte LaTeX
% Documento:    Definición de entornos
% Versión:      6.0.1 (21/10/2018)
% Codificación: UTF-8
%
% Autor: Pablo Pizarro R. @ppizarror
%        Facultad de Ciencias Físicas y Matemáticas
%        Universidad de Chile
%        pablo.pizarro@ing.uchile.cl, ppizarror.com
%
% Manual template: [http://latex.ppizarror.com/Template-Informe/]
% Licencia MIT:    [https://opensource.org/licenses/MIT/]

\newenvironment{references}{
\ifthenelse{\equal{\stylecitereferences}{bibtex}}{
	}{
		\throwerror{\references}{Solo se puede usar entorno references con estilo citas \noexpand\stylecitereferences=bibtex}
	}
	\begingroup
\ifthenelse{\equal{\donumrefsection}{true}}{
		\section{\namereferences}
	}{
		\sectionanum{\namereferences}
	}
	\renewcommand{\section}[2]{}
\begin{thebibliography}{99}
	}
	{
	\end{thebibliography}
\endgroup
}
\newenvironment{anexo}{
\begingroup
	\clearpage
	\phantomsection
	\ifthenelse{\equal{\showappendixsectitle}{true}}{
		\appendixpage}{
	}
\def\GLOBALsectionalph{true}
\appendixtitleon
	\appendicestocpagenum
	\appendixtitletocon
	\bookmarksetup{
		numbered,
		openlevel=0
	}
\begin{appendices}
		\bookmarksetupnext{level=part}
		\ifthenelse{\equal{\showappendixsecindex}{true}}{}{
			\belowpdfbookmark{\nameappendixsection}{contents}
		}
		\setcounter{secnumdepth}{4}
		\setcounter{tocdepth}{4}
		\ifthenelse{\equal{\appendixindepobjnum}{true}}{
			\counterwithin{equation}{section}
			\counterwithin{figure}{section}
			\counterwithin{lstlisting}{section}
			\counterwithin{table}{section}}{
		}
	}{
	\end{appendices}
\def\GLOBALsectionalph{false}
\bookmarksetupnext{level=0}
	\endgroup
}
\newcommand{\coreinitsourcecodep}[4]{
	\emptyvarerr{sourcecodep}{#2}{Estilo no definido}
	\checkvalidsourcecodestyle{#2}
	\ifthenelse{\equal{\showlinenumbers}{true}}{
		\rightlinenumbers}{
	}
	\lstset{
		backgroundcolor=\color{\sourcecodebgcolor}
	}
	\ifthenelse{\equal{\codecaptiontop}{true}}{
		\ifx\hfuzz#4\hfuzz
			\ifx\hfuzz#3\hfuzz
				\lstset{
					style=#2
				}
			\else
				\lstset{
					style=#2,
					#3
				}
			\fi
		\else
			\ifx\hfuzz#3\hfuzz
				\lstset{
					caption={#4 #1},
					captionpos=t,
					style=#2
				}
			\else
				\lstset{
					caption={#4 #1},
					captionpos=t,
					style=#2,
					#3
				}
			\fi
		\fi
	}{
		\ifx\hfuzz#4\hfuzz
			\ifx\hfuzz#3\hfuzz
				\lstset{
					style=#2
				}
			\else
				\lstset{
					style=#2,
					#3
				}
			\fi
		\else
			\ifx\hfuzz#3\hfuzz
				\lstset{
					caption={#4 #1},
					captionpos=b,
					style=#2
				}
			\else
				\lstset{
					caption={#4 #1},
					captionpos=b,
					style=#2,
					#3
				}
			\fi
		\fi	
	}
}
\lstnewenvironment{sourcecodep}[4][]{
	\coreinitsourcecodep{#1}{#2}{#3}{#4}
}{
	\ifthenelse{\equal{\showlinenumbers}{true}}{
		\leftlinenumbers}{
	}
}
\newcommand{\importsourcecodep}[5][]{
	\coreinitsourcecodep{#1}{#2}{#3}{#5}
	\inputlisting{#4}
	\ifthenelse{\equal{\showlinenumbers}{true}}{
		\leftlinenumbers}{
	}
}
\newcommand{\coreinitsourcecode}[3]{
	\emptyvarerr{\equationresize}{#2}{Estilo no definido}
	\checkvalidsourcecodestyle{#2}
	\ifthenelse{\equal{\showlinenumbers}{true}}{
		\rightlinenumbers}{
	}
	\lstset{
		backgroundcolor=\color{\sourcecodebgcolor}
	}
	\ifthenelse{\equal{\codecaptiontop}{true}}{
		\ifx\hfuzz#3\hfuzz
			\lstset{
				style=#2
			}
		\else
			\lstset{
				style=#2,
				caption={#3 #1},
				captionpos=t
			}
		\fi
	}{
		\ifx\hfuzz#3\hfuzz
			\lstset{
				style=#2
			}
		\else
			\lstset{
				style=#2,
				caption={#3 #1},
				captionpos=b
			}
		\fi
	}
}
\lstnewenvironment{sourcecode}[3][]{
	\coreinitsourcecode{#1}{#2}{#3}
}{
	\ifthenelse{\equal{\showlinenumbers}{true}}{
		\leftlinenumbers}{
	}
}
\newcommand{\importsourcecode}[4][]{
	\coreinitsourcecode{#1}{#2}{#4}
	\lstinputlisting{#3}
	\ifthenelse{\equal{\showlinenumbers}{true}}{
		\leftlinenumbers}{
	}
}
\newenvironment{itemizebf}[1][]{
	\begin{itemize}[font=\bfseries,#1]
	}{
	\end{itemize}
}
\newenvironment{enumeratebf}[1][]{
	\begin{enumerate}[font=\bfseries,#1]
	}{
	\end{enumerate}
}
\newenvironment{resumen}{
	\sectionfont{\color{\titlecolor} \fontsizetitle \styletitle \selectfont}
	\sectionanumnoiheadless{\nameabstract}}{
	\ifthenelse{\equal{\addemptypagetwosides}{true}}{
		\checkoddpage
		\ifoddpage
			\newpage
			\null
			\thispagestyle{empty}
			\newpage
			\addtocounter{page}{-1}
		\else
		\fi}{
	}
}
\newenvironment{images}[2][]{
	\def\envimageslabelvar {#1}
	\def\envimagescaptionvar {#2}
	\def\GLOBALenvimageintialized {true}
	\vspace{\margintopimages cm}
	\captionsetup{margin=\captionmarginmultimg cm}
	\begin{figure}[H] \centering
		\vspace{-\marginrightmultimage cm}
		\vspace{-\marginrightmultimage cm}
		\vspace{-\marginrightmultimage cm}
		}{
		\setcaptionmargincm{\captionlrmargin}
		\ifx\hfuzz\envimagescaptionvar\hfuzz
			\vspace{\captionlessmarginimage cm}
		\else
			\caption{\envimagescaptionvar\envimageslabelvar}
		\fi
	\end{figure}
	\setcaptionmargincm{\captionlrmargin}
	\vspace{\marginbottomimages cm}
	\def\GLOBALenvimageintialized {false}
}



% CONFIGURACIÓN INICIAL DEL DOCUMENTO
% Template:     Informe/Reporte LaTeX
% Documento:    Configuración inicial del template
% Versión:      6.0.0 (13/10/2018)
% Codificación: UTF-8
%
% Autor: Pablo Pizarro R. @ppizarror
%        Facultad de Ciencias Físicas y Matemáticas
%        Universidad de Chile
%        pablo.pizarro@ing.uchile.cl, ppizarror.com
%
% Manual template: [http://latex.ppizarror.com/Template-Informe/]
% Licencia MIT:    [https://opensource.org/licenses/MIT/]

\checkvardefined{\autordeldocumento}
\checkvardefined{\codigodelcurso}
\checkvardefined{\departamentouniversidad}
\checkvardefined{\localizacionuniversidad}
\checkvardefined{\nombredelcurso}
\checkvardefined{\nombrefacultad}
\checkvardefined{\nombreuniversidad}
\checkvardefined{\temaatratar}
\checkvardefined{\titulodelinforme}
\makeatletter
	\g@addto@macro\autordeldocumento\xspace
	\g@addto@macro\codigodelcurso\xspace
	\g@addto@macro\departamentouniversidad\xspace
	\g@addto@macro\localizacionuniversidad\xspace
	\g@addto@macro\nombredelcurso\xspace
	\g@addto@macro\nombrefacultad\xspace
	\g@addto@macro\nombreuniversidad\xspace
	\g@addto@macro\temaatratar\xspace
	\g@addto@macro\titulodelinforme\xspace
\makeatother
\ifthenelse{\isundefined{\tablaintegrantes}}{
	\errmessage{LaTeX Warning: Se borro la variable \noexpand\tablaintegrantes, creando una vacia}
	\def\tablaintegrantes {}}{
}
\ifthenelse{\equal{\cfgpdfsecnumbookmarks}{true}}{
	\bookmarksetup{numbered}}{
}
\ifthenelse{\equal{\cfgshowbookmarkmenu}{true}}{
	\def\cdfpagemodepdf {UseOutlines}
	}{
	\def\cdfpagemodepdf {UseNone}
}
\hypersetup{
	bookmarksopen={\cfgpdfbookmarkopen},
	bookmarksopenlevel={\cfgbookmarksopenlevel},
	bookmarkstype={toc},
	pdfauthor={\autordeldocumento},
	pdfcenterwindow={\cfgpdfcenterwindow},
	pdfcopyright={\cfgpdfcopyright},
	pdfcreator={LaTeX},
	pdfdisplaydoctitle={\cfgpdfdisplaydoctitle},
	pdffitwindow={\cfgpdffitwindow},
	pdfinfo={
		Curso.Codigo={\codigodelcurso},
		Curso.Nombre={\nombredelcurso},
		Documento.Autor={\autordeldocumento},
		Documento.Tema={\temaatratar},
		Documento.Titulo={\titulodelinforme},
		Template.Autor.Alias={ppizarror},
		Template.Autor.Email={pablo.pizarro@ing.uchile.cl},
		Template.Autor.Nombre={Pablo Pizarro R.},
		Template.Autor.Web={http://ppizarror.com/},
		Template.Codificacion={UTF-8},
		Template.Fecha={13/10/2018},
		Template.Latex.Compilador={pdflatex},
		Template.Licencia.Tipo={MIT},
		Template.Licencia.Web={https://opensource.org/licenses/MIT/},
		Template.Nombre={Template-Informe},
		Template.Tipo={Normal},
		Template.Version.Dev={6.0.0-2-N},
		Template.Version.Hash={5951F303B108DAD1B4F4ACEAD824F7DD},
		Template.Version.Release={6.0.0},
		Template.Web.Dev={https://github.com/Template-Latex/Template-Informe/},
		Template.Web.Manual={http://latex.ppizarror.com/Template-Informe/},
		Universidad.Departamento={\departamentouniversidad},
		Universidad.Nombre={\nombreuniversidad},
		Universidad.Ubicacion={\localizacionuniversidad}
	},
	pdfkeywords={\codigodelcurso, \nombredelcurso, \nombreuniversidad, \localizacionuniversidad},
	pdflang={\documentlang},
	pdfmenubar={\cfgpdfmenubar},
	pdfpagelayout={\cfgpdfpagemode},
	pdfpagemode={\cdfpagemodepdf},
	pdfproducer={Template-Informe v6.0.0 | (Pablo Pizarro R.) ppizarror.com},
	pdfremotestartview={Fit},
	pdfstartpage={1},
	pdfstartview={\cfgpdfpageview},
	pdfsubject={\temaatratar},
	pdftitle={\titulodelinforme},
	pdftoolbar={\cfgpdftoolbar},
	pdfview={\cfgpdfpageview}
}
\graphicspath{{./\defaultimagefolder}}
\newcounter{templatepagecounter}
\renewcommand{\baselinestretch}{\defaultinterline}
\setlength{\headheight}{64 pt}
\setcounter{MaxMatrixCols}{20}
\setlength{\footnotemargin}{4 mm}
\setlength{\columnsep}{\columnsepwidth em}
\ifthenelse{\equal{\showlinenumbers}{true}}{
	\setlength{\linenumbersep}{0.22cm}
	\renewcommand\linenumberfont{\normalfont\tiny\color{\linenumbercolor}}
	}{
}
\floatplacement{figure}{\imagedefaultplacement}
\floatplacement{table}{\tabledefaultplacement}
\floatplacement{tikz}{\tikzdefaultplacement}
\color{\maintextcolor}
\arrayrulecolor{\tablelinecolor}
\sethlcolor{\highlightcolor}
\ifthenelse{\equal{\showborderonlinks}{true}}{
	\hypersetup{
		citebordercolor=\numcitecolor,
		linkbordercolor=\linkcolor,
		urlbordercolor=\urlcolor
	}
}{
\hypersetup{
		hidelinks,
		colorlinks=true,
		citecolor=\numcitecolor,
		linkcolor=\linkcolor,
		urlcolor=\urlcolor
	}
}
\ifthenelse{\equal{\colorpage}{white}}{}{
	\pagecolor{\colorpage}
}
\setcaptionmargincm{\captionlrmargin}
\ifthenelse{\equal{\captiontextbold}{true}}{
	\renewcommand{\captiontextbold}{bf}}{
	\renewcommand{\captiontextbold}{}
}
\captionsetup{
	labelfont={color=\captioncolor, \captiontextbold},
	textfont={color=\captiontextcolor},
	singlelinecheck=on
}
\floatsetup[figure]{
	captionskip=\captiontbmarginfigure pt
}
\floatsetup[table]{
	captionskip=\captiontbmargintable pt
}
\ifthenelse{\equal{\figurecaptiontop}{true}}{
	\floatsetup[figure]{position=above}}{
}
\ifthenelse{\equal{\tablecaptiontop}{true}}{
	\floatsetup[table]{position=top}
	}{
	\floatsetup[table]{position=bottom}
}
\ifthenelse{\equal{\captionalignment}{justified}}{
	\captionsetup{
		format=plain,
		justification=justified
	}
}{
\ifthenelse{\equal{\captionalignment}{centered}}{
	\captionsetup{
		justification=centering
	}
}{
\ifthenelse{\equal{\captionalignment}{left}}{
	\captionsetup{
		justification=raggedright,
		singlelinecheck=false
	}
}{
\ifthenelse{\equal{\captionalignment}{right}}{
	\captionsetup{
		justification=raggedleft,
		singlelinecheck=false
	}
}{
	\throwbadconfig{Posicion de leyendas desconocida}{\captionalignment}{justified,centered,left,right}}}}
}
\ifthenelse{\equal{\stylecitereferences}{natbib}}{
	\bibliographystyle{apa}
	\setlength{\bibsep}{\natbibrefsep pt}
}{
\ifthenelse{\equal{\stylecitereferences}{apacite}}{
	\bibliographystyle{apacite}
	\setlength{\bibitemsep}{\apaciterefsep pt}
}{
\ifthenelse{\equal{\stylecitereferences}{bibtex}}{
	\bibliographystyle{apa}
	\newlength{\bibitemsep}
	\setlength{\bibitemsep}{.2\baselineskip plus .05\baselineskip minus .05\baselineskip}
	\newlength{\bibparskip}\setlength{\bibparskip}{0pt}
	\let\oldthebibliography\thebibliography
	\renewcommand\thebibliography[1]{
		\oldthebibliography{#1}
		\setlength{\parskip}{\bibitemsep}
		\setlength{\itemsep}{\bibparskip}
	}
	\setlength{\bibitemsep}{\bibtexrefsep pt}
}{
	\throwbadconfig{Estilo citas desconocido}{\stylecitereferences}{bibtex,apacite,natbib}}}
}
\makeatletter
\ifthenelse{\equal{\twocolumnreferences}{true}}{
	\renewenvironment{thebibliography}[1]
	{\begin{multicols}{2}[\section*{\refname}]
		\@mkboth{\MakeUppercase\refname}{\MakeUppercase\refname}
		\list{\@biblabel{\@arabic\c@enumiv}}
		{\settowidth\labelwidth{\@biblabel{#1}}
			\leftmargin\labelwidth
			\advance\leftmargin\labelsep
			\@openbib@code
			\usecounter{enumiv}
			\let\p@enumiv\@empty
			\renewcommand\theenumiv{\@arabic\c@enumiv}}
		\sloppy
		\clubpenalty 4000
		\@clubpenalty \clubpenalty
		\widowpenalty 4000
		\sfcode`\.\@m}
		{\def\@noitemerr
		{\@latex@warning{Ambiente `thebibliography' no definido}}
		\endlist\end{multicols}}}{}
\makeatother
\patchcmd{\appendices}{\quad}{\sectionappendixlastchar\quad}{}{}
\begingroup
	\makeatletter
	\let\newcounter\@gobble\let\setcounter\@gobbletwo
	\globaldefs\@ne\let\c@loldepth\@ne
	\newlistof{listings}{lol}{\lstlistlistingname}
	\newlistentry{lstlisting}{lol}{0}
	\makeatother
\endgroup
\makeatletter
	\def\ifGm@preamble#1{\@firstofone}
	\appto\restoregeometry{
		\pdfpagewidth=\paperwidth
		\pdfpageheight=\paperheight}
	\apptocmd\newgeometry{
		\pdfpagewidth=\paperwidth
		\pdfpageheight=\paperheight}{}{}
\makeatother
\hfuzz=200pt
\vfuzz=200pt
\hbadness=\maxdimen
\vbadness=\maxdimen
\strictpagecheck
\titlespacing{\section}{0pt}{20pt}{10pt}
\titlespacing{\subsection}{0pt}{15pt}{10pt}
\ttfamily \hyphenchar\the\font=`\-
\urlstyle{tt}
\ifthenelse{\equal{\portraitstyle}{style16}}{\coreimporttikz}{}
\ifthenelse{\equal{\portraitstyle}{\bgtemplatetestcode}}{\coreimporttikz}{}
\pdfminorversion=\pdfcompileversion
\setcounter{secnumdepth}{4}
\newcounter{subsubsubsection}[subsubsection]
\titleclass{\subsubsubsection}{straight}[\subsection]
\ifthenelse{\equal{\showdotontitles}{true}}{
	\renewcommand{\thesubsubsubsection}{\thesubsubsection.\arabic{subsubsubsection}.}
	\renewcommand{\theparagraph}{\thesubsubsubsection.\arabic{paragraph}.}
}{
	\renewcommand{\thesubsubsubsection}{\thesubsubsection.\arabic{subsubsubsection}}
	\renewcommand{\theparagraph}{\thesubsubsubsection.\arabic{paragraph}}
}
\makeatletter
	\renewcommand\paragraph{\@startsection{paragraph}{5}{\z@}
		{3.25ex \@plus 1ex \@minus .2ex}
		{-1em}
		{\normalfont\normalsize\bfseries}}
	\renewcommand\subparagraph{\@startsection{subparagraph}{6}{\parindent}
		{3.25ex \@plus 1ex \@minus .2ex}
		{-1em}
		{\normalfont\normalsize\bfseries}}
	\def\toclevel@subsubsubsection{4}
	\def\toclevel@paragraph{5}
	\def\toclevel@paragraph{6}
	\def\l@subsubsubsection{\@dottedtocline{4}{7em}{4em}}
	\def\l@paragraph{\@dottedtocline{5}{10em}{5em}}
	\def\l@subparagraph{\@dottedtocline{6}{14em}{6em}}
\makeatother
\setcounter{tocdepth}{\indexdepth}


% INICIO DE LAS PÁGINAS
\begin{document}

% PORTADA
% Template:     Informe/Reporte LaTeX
% Documento:    Definición de portadas
% Versión:      6.1.0 (03/11/2018)
% Codificación: UTF-8
%
% Autor: Pablo Pizarro R. @ppizarror
%        Facultad de Ciencias Físicas y Matemáticas
%        Universidad de Chile
%        pablo.pizarro@ing.uchile.cl, ppizarror.com
%
% Manual template: [https://latex.ppizarror.com/Template-Informe/]
% Licencia MIT:    [https://opensource.org/licenses/MIT/]

% Template:     Informe/Reporte LaTeX
% Documento:    Configuraciones adicionales de portadas
% Versión:      6.0.0 (13/10/2018)
% Codificación: UTF-8
%
% Autor: Pablo Pizarro R. @ppizarror
%        Facultad de Ciencias Físicas y Matemáticas
%        Universidad de Chile
%        pablo.pizarro@ing.uchile.cl, ppizarror.com
%
% Manual template: [http://latex.ppizarror.com/Template-Informe/]
% Licencia MIT:    [https://opensource.org/licenses/MIT/]

% Configuración portada style15 [A]
\def\headerimageA {departamentos/uchile}            % Imagen en el header
\def\headerimagescaleA {0.4}                        % Escala de la imagen

% Configuración portada style16 [B]
\def\portraitbackgroundimageB {ejemplos/portrait}   % Archivo de fondo
\def\portraitbackgroundcolorB {ocre}                % Color principal

% Configuración portada style17 [C]
\def\portraitimageC {img/ejemplos/test-image}       % Imagen de la portada
\def\portraitimageboxedC {true}                     % Imagen recuadrada
\def\portraitimageboxedwidthC {0.5}                 % Grosor línea recuadro
\def\portraitimagewidthC {8cm}                      % Ancho de la imagen en cm

% Configuración portada style18 [D]
\def\portraitimageD {img/ejemplos/test-image-wrap}  % Imagen de la portada
\def\portraitimageboxedD {false}                    % Imagen recuadrada
\def\portraitimageboxedwidthD {0.5}                 % Grosor línea recuadro
\def\portraitimagewidthD {4cm}                      % Ancho de la imagen en cm

\newpage
\renewcommand{\thepage}{\nameportraitpage}

\ifthenelse{\equal{\portraitstyle}{style1}}{
	\setpagemargincm{\pagemarginleft}{\firstpagemargintop}{\pagemarginright}{\pagemarginbottom}
	\pagestyle{fancy}
	\fancyhf{}
	\fancyhead[L]{
		\nombreuniversidad ~ \\
		\nombrefacultad ~ \\
		\departamentouniversidad ~ \\
		\vspace{-0.43cm}
	}
	\fancyhead[R]{
		\includegraphics[scale=\imagendepartamentoescala]{\imagendepartamento}
		\hspace{-0.255cm}
	}
	~ \\
	\vfill
	\begin{center}
		\textcolor{\portraittitlecolor}{
			{\noindent \Huge{\titulodelinforme} \vspace{0.5cm}} ~ \\
			{\noindent \large{\temaatratar}}
		}
	\end{center}
	\vfill
	\noindent
	\begin{minipage}{1.0\textwidth}
		\begin{flushright}
			\tablaintegrantes
		\end{flushright}
	\end{minipage}
}{
\ifthenelse{\equal{\portraitstyle}{style2}}{
	\setpagemargincm{\pagemarginleft}{\firstpagemargintop}{\pagemarginright}{\pagemarginbottom}
	\pagestyle{fancy}
	\fancyhf{}
	\fancyhead[L]{
		\nombreuniversidad ~ \\
		\nombrefacultad ~ \\
		\departamentouniversidad ~ \\
		\vspace{-0.43cm}
	}
	\fancyhead[R]{
		\includegraphics[scale=\imagendepartamentoescala]{\imagendepartamento}
		\hspace{-0.255cm}
	}
	~ \\
	\vfill
	\begin{center}
		{\noindent \LARGE{\nombredelcurso} \vspace{0.3cm}} ~ \\
		\vspace*{1.5cm}
		\textcolor{\portraittitlecolor}{
			{\centering \noindent \Huge{\titulodelinforme} \vspace{0.3cm}} ~ \\
			{\noindent \large{\temaatratar}}
		}
	\end{center}
	\vfill
	\noindent
	\begin{minipage}{1.0\textwidth}
		\begin{flushright}
			\tablaintegrantes
		\end{flushright}
	\end{minipage}
}{
\ifthenelse{\equal{\portraitstyle}{style3}}{
	\setpagemargincm{\pagemarginleft}{\firstpagemargintop}{\pagemarginright}{\pagemarginbottom}
	\pagestyle{fancy}
	\fancyhf{}
	\fancyhead[L]{
		\nombreuniversidad ~ \\
		\nombrefacultad ~ \\
		\departamentouniversidad ~ \\
		\vspace{-0.43cm}
	}
	\fancyhead[R]{
		\includegraphics[scale=\imagendepartamentoescala]{\imagendepartamento}
		\hspace{-0.255cm}
	}
	~ \\
	\vfill
	\begin{center}
		\vspace*{-1.0cm}
		{\noindent \huge{\nombredelcurso} \vspace{0.3cm}} ~ \\
		{\noindent \large{Código del curso: \codigodelcurso}} ~ \\
		\vspace*{1.8cm}
		\textcolor{\portraittitlecolor}{
			{\noindent \Huge{\titulodelinforme} \vspace{0.3cm}} ~ \\
			{\noindent \large{\temaatratar}}
		}
	\end{center}
	\vfill
	\noindent
	\begin{minipage}{1.0\textwidth}
		\begin{flushright}
			\tablaintegrantes
		\end{flushright}
	\end{minipage}
}{
\ifthenelse{\equal{\portraitstyle}{style4}}{
	\setpagemargincm{\pagemarginleft}{\pagemargintop}{\pagemarginright}{\pagemarginbottom}
	\thispagestyle{empty}
	\vspace*{-1.5cm}
	\noindent \includegraphics[width=1.75cm]{departamentos/uchile2}
	\hspace*{-0.15cm}
	\begin{tabular}{l}
		\small \scshape{\MakeUppercase{\nombreuniversidad}} ~ \\
		\small \scshape{\MakeUppercase{\nombrefacultad}} ~ \\
		\small \scshape{\MakeUppercase{\departamentouniversidad}} ~ \\
		\vspace*{1.25cm}\mbox{}
	\end{tabular}
	\vfill
	\begin{center}
		{\fontsize{22pt}{10pt} \selectfont
			\noindent \textcolor{\portraittitlecolor}{\titulodelinforme} \vspace*{0.35cm}} ~ \\
		{\noindent \fontsize{10pt}{5pt} \selectfont \textcolor{\portraittitlecolor}{\codigodelcurso\ - \nombredelcurso}} ~ \\
		\vspace*{3cm}
	\end{center}
	\vfill
	\noindent
	\begin{minipage}{1.0\textwidth}
		\begin{flushright}
			\tablaintegrantes
		\end{flushright}
	\end{minipage}
}{
\ifthenelse{\equal{\portraitstyle}{style5}}{
	\setpagemargincm{\pagemarginleft}{\pagemargintop}{\pagemarginright}{\pagemarginbottom}
	\thispagestyle{empty}
	\includegraphics[width=1.5cm]{departamentos/uchile3}
	\hspace{-0.2cm}
	\begin{tabular}{l}
		\small \scshape{\MakeUppercase{\nombreuniversidad}} ~ \\
		\small \scshape{\MakeUppercase{\nombrefacultad}} ~ \\
		\small \scshape{\MakeUppercase{\departamentouniversidad}} ~ \\
		\vspace*{1cm}\mbox{}
	\end{tabular}
	\vfill
	\begin{center}
		\fontsize{8mm}{9mm}\selectfont
		\textcolor{\portraittitlecolor}{
			\noindent \titulodelinforme ~ \\
			\noindent \temaatratar ~ \\
		}
		\vspace*{1cm}
		\footnotesize{\codigodelcurso\ - \nombredelcurso} ~ \\
		\vspace*{1.4cm}
	\end{center}
	\vfill
	\begin{center}
		\noindent \normalsize{\tablaintegrantes}
	\end{center}
}{
\ifthenelse{\equal{\portraitstyle}{style6}}{
	\setpagemargincm{\pagemarginleft}{\pagemargintop}{\pagemarginright}{\pagemarginbottom}
	\thispagestyle{empty}
	\begin{wrapfigure}{l}{0.3\textwidth}
		\vspace{-0.69cm}
		\noindent \hspace{-1.10cm} \includegraphics[scale=1.35]{departamentos/fcfm2}
	\end{wrapfigure}
	\hspace*{0.3cm}
	\noindent \textsc{\color{red} \hspace{-2.2cm} \departamentouniversidad} ~ \\
	\hspace*{0.3cm}
	\noindent \textsc{\color{dkgray} \hspace{-1.6cm} \nombrefacultad} ~ \\
	\hspace*{0.3cm}
	\noindent \textsc{\color{dkgray} \hspace{-1.6cm} \nombreuniversidad} ~ \\
	\hspace*{0.3cm}
	\noindent \textsc{\color{dkgray} \hspace{-1.6cm} \codigodelcurso \nombredelcurso} ~ \\
	\vfill
	\begin{center}
		\vspace*{0.5cm}
		{\color{dkgray} \Large \textbf{\MakeUppercase{\temaatratar}}} ~ \\
		\noindent \rule{\linewidth}{0.3mm} ~ \\
		\Huge \textup \bfseries \textsc{\textcolor{\portraittitlecolor}{\titulodelinforme}} ~ \\
		\noindent \rule{\linewidth}{0.3mm} ~ \\
	\end{center}
	\begin{minipage}{.5\textwidth}
		~
	\end{minipage}
	\vfill
	\begin{minipage}{1.0\textwidth}
		\begin{flushright}
			\noindent \tablaintegrantes
		\end{flushright}
	\end{minipage}
}{
\ifthenelse{\equal{\portraitstyle}{style7}}{
	\setpagemargincm{\pagemarginleft}{\pagemargintop}{\pagemarginright}{\pagemarginbottom}
	\thispagestyle{empty}
	\begin{center}
		\vspace*{-1.5cm}
		\includegraphics[scale=\imagendepartamentoescala]{\imagendepartamento}
		\hspace*{-0.15cm}
		\begin{tabular}{l}
			\vspace*{0.26cm}\mbox{} ~ \\
			\small \textsc{\MakeUppercase{\nombreuniversidad}} ~ \\
			\small \textsc{\MakeUppercase{\nombrefacultad}} ~ \\
			\small \textsc{\MakeUppercase{\departamentouniversidad}} ~ \\
			\vspace*{1.25cm}\mbox{}
		\end{tabular}
	\end{center}
	\vfill
	\begin{center}
		\noindent \rule{\textwidth}{0.4mm} \\ \vspace{0.3cm}
		{\huge \textcolor{\portraittitlecolor}{\titulodelinforme} \vspace{0.2cm} ~ \\}
		\noindent \rule{\textwidth}{0.4mm} ~ \\ \vspace{0.40cm}
		{\large \textcolor{\portraittitlecolor}{\temaatratar} ~ \\}
	\end{center}
	\vfill
	\noindent
	\begin{minipage}{1.0\textwidth}
		\begin{flushright}
			\scshape{\tablaintegrantes}
		\end{flushright}
	\end{minipage}
}{
\ifthenelse{\equal{\portraitstyle}{style8}}{
	\setpagemargincm{\pagemarginleft}{\pagemargintop}{\pagemarginright}{\pagemarginbottom}
	\thispagestyle{empty}
	\begin{center}
		\vspace*{-1.0cm}
		\begin{tabular}{c}
			\includegraphics[scale=\imagendepartamentoescala]{\imagendepartamento} \vspace{0.5cm} ~ \\
			\small \scshape{\MakeUppercase{\nombreuniversidad}} ~ \\
			\small \scshape{\MakeUppercase{\nombrefacultad}} ~ \\
			\small \scshape{\MakeUppercase{\departamentouniversidad}}
		\end{tabular}
	\end{center}
	\vfill
	\begin{center}
		\noindent \rule{\textwidth}{0.4mm} \\ \vspace{0.3cm}
		{\huge \textcolor{\portraittitlecolor}{\titulodelinforme} \vspace{0.2cm} ~ \\}
		\noindent \rule{\textwidth}{0.4mm} ~ \\ \vspace{0.40cm}
		{\large \textcolor{\portraittitlecolor}{\temaatratar} ~ \\}
	\end{center}
	\vfill
	\noindent
	\begin{minipage}{1.0\textwidth}
		\begin{flushright}
			\scshape{\tablaintegrantes}
		\end{flushright}
	\end{minipage}
}{
\ifthenelse{\equal{\portraitstyle}{style9}}{
	\setpagemargincm{\pagemarginleft}{\pagemargintop}{\pagemarginright}{\pagemarginbottom}
	\thispagestyle{empty}
	\noindent \includegraphics[scale=\imagendepartamentoescala]{\imagendepartamento}
	\vfill
	\begin{center}
		\noindent \rule{\textwidth}{0.4mm} \\ \vspace{0.3cm}
		{\huge \textcolor{\portraittitlecolor}{\titulodelinforme} \vspace{0.2cm} \\}
		\noindent \rule{\textwidth}{0.4mm} \\ \vspace{0.35cm}
		{\large \textcolor{\portraittitlecolor}{\temaatratar} \\}
	\end{center}
	\vfill
	\begin{center}
		\begin{tabular}{c}
			\small \scshape{\MakeUppercase{\nombreuniversidad}} ~ \\
			\small \scshape{\MakeUppercase{\nombrefacultad}} ~ \\
			\small \scshape{\MakeUppercase{\departamentouniversidad}}
		\end{tabular}
	\end{center}
	\vfill
	\begin{center}
		\indent \scshape{\tablaintegrantes}
	\end{center}
}{
\ifthenelse{\equal{\portraitstyle}{style10}}{
	\setpagemargincm{\pagemarginleft}{\pagemargintop}{\pagemarginright}{\pagemarginbottom}
	\thispagestyle{empty}
	~ \\
	\vfill
	\begin{center}
		\ifthenelse{\equal{\nombreuniversidad}{\xspace}}{
			\noindent {\large \textsc{\departamentouniversidad}}
		}{
			\noindent {\large \textsc{\nombreuniversidad, \departamentouniversidad}}
		}
		\vspace{1.0cm}
	\end{center}
	\vfill
	\begin{center}
		\noindent {\large \scshape{\nombredelcurso}} \vspace{0.5cm} ~ \\
		\noindent {\large \scshape{\codigodelcurso}} \vspace{0.5cm} ~ \\
		\noindent \rule{\textwidth}{0.4mm} \\ \vspace{0.3cm}
		{\huge \bfseries \textcolor{\portraittitlecolor}{\titulodelinforme} \vspace{0.2cm} \\}
		\noindent \rule{\textwidth}{0.4mm} \\ \vspace{2.5cm}
	\end{center}
	\vfill
	\begin{center}
		\indent \tablaintegrantes
	\end{center}
	\vfill
	~ \\
}{
\ifthenelse{\equal{\portraitstyle}{style11}}{
	\setpagemargincm{\pagemarginleft}{\pagemargintop}{\pagemarginright}{\pagemarginbottom}
	\thispagestyle{empty}
	\begin{center}
		\vspace*{-1.0cm}
		\scshape{\nombreuniversidad} ~ \\
		\scshape{\nombrefacultad} ~ \\
		\scshape{\departamentouniversidad}
	\end{center}
	\vfill
	\begin{center}
		{\setstretch{1.2} \fontsize{21pt}{22pt} \selectfont \textcolor{\portraittitlecolor}{\scshape{\titulodelinforme}} \vspace{0.5cm}} ~ \\
		{\fontsize{13pt}{10pt} \selectfont \textcolor{\portraittitlecolor}{\scshape{\temaatratar}}}
	\end{center}
	\vfill
	\begin{center}
		\indent \tablaintegrantes
	\end{center}
}{
\ifthenelse{\equal{\portraitstyle}{style12}}{
	\setpagemargincm{\pagemarginleft}{\pagemargintop}{\pagemarginright}{\pagemarginbottom}
	\thispagestyle{empty}
	\begin{center}
		\vspace*{-1.0cm}
		\includegraphics[scale=\imagendepartamentoescala]{\imagendepartamento}
	\end{center}
	\vfill
	\begin{center}
		{\bf \Huge \scshape{\textcolor{\portraittitlecolor}{\titulodelinforme}} \vspace{0.3cm}} \\
		{\bf \Large \textcolor{\portraittitlecolor}{\temaatratar}}
	\end{center}
	\vfill
	\begin{flushright}
		\noindent \tablaintegrantes
	\end{flushright}
	\vspace{0.5cm}
	\noindent \rule{\textwidth}{0.4mm}
	\begin{center}
		\ifthenelse{\equal{\nombreuniversidad}{\xspace}}{
			\scshape{\nombrefacultad} \\
		}{
			\scshape{\nombreuniversidad, \nombrefacultad} \\
		}
		\scshape{\departamentouniversidad}
	\end{center}
}{
\ifthenelse{\equal{\portraitstyle}{style13}}{
	\setpagemargincm{\pagemarginleft}{\pagemargintop}{\pagemarginright}{\pagemarginbottom}
	\thispagestyle{empty}
	\noindent
	\vspace*{-1.5cm}
	\begin{flushleft}
		\begin{minipage}{0.65\textwidth}
			\ifthenelse{\equal{\nombreuniversidad}{\xspace}}{
				{\fontsize{3.5mm}{0.5mm} \selectfont \noindent \textsf{\nombrefacultad}} ~ \\
			}{
				{\fontsize{3.5mm}{0.5mm} \selectfont \noindent \textsf{\nombreuniversidad, \nombrefacultad}} ~ \\
			}
			\noindent {\fontsize{3.0mm}{0.5mm} \selectfont \textsf{\departamentouniversidad} \vspace{-0.2cm}} ~ \\
			\noindent \textcolor{gray}{\rule{\textwidth}{0.3mm}}
		\end{minipage}
	\end{flushleft}
	\vspace*{-2.15cm}
	\begin{flushright}
		\begin{minipage}{0.3\textwidth}
			\noindent \includegraphics[width=1.0\textwidth]{\imagendepartamento}
		\end{minipage}
	\end{flushright}
	\vfill
	\begin{center}
		\begin{minipage}{0.9\textwidth}
			\begin{framed}
				\LARGE
				\vspace{1cm}
				\centering \textcolor{\portraittitlecolor}{\textbf{\titulodelinforme}}
				\vspace{1cm}
			\end{framed}
		\end{minipage}
	\end{center}
	\vfill
	\begin{flushright}
		\noindent \textsf{\tablaintegrantes}
	\end{flushright}
}{
\ifthenelse{\equal{\portraitstyle}{style14}}{
	\setpagemargincm{\pagemarginleft}{\pagemargintop}{\pagemarginright}{\pagemarginbottom}
	\thispagestyle{empty}
	\noindent
	\begin{flushleft}
		\vspace*{-1.0cm}
		\noindent \includegraphics[scale=\imagendepartamentoescala]{\imagendepartamento} \\
	\end{flushleft}
	\vfill
	{\bf \huge \noindent \textcolor{\portraittitlecolor}{\textsf{\MakeUppercase{\titulodelinforme}} \vspace*{0.05cm}}} \\
	{\bf \large \noindent \textcolor{\portraittitlecolor}{\textsf{\MakeUppercase{\temaatratar}}}} \\
	\vfill
	\begin{flushright}
		\noindent \textsf{\tablaintegrantes}
	\end{flushright}
}{
\ifthenelse{\equal{\portraitstyle}{style15}}{
	\setpagemargincm{\pagemarginleft}{\pagemargintop}{\pagemarginright}{\pagemarginbottom}
	\thispagestyle{empty}
	\checkextravarexist{\headerimageA}{Defina la imagen extra de la portada en el archivo lib/page/portrait-config.tex (VERSION NORMAL) o bien en el bloque PORTADA (VERSION COMPACTA)}
	\checkextravarexist{\headerimagescaleA}{Defina la escala de la imagen extra de la portada en el archivo lib/page/portrait-config.tex (VERSION NORMAL) o bien en el bloque PORTADA (VERSION COMPACTA)}
	\vspace*{-1.5cm}
	\noindent \begin{minipage}{0.8\textwidth}
		\noindent \begin{minipage}{0.22\textwidth}
			\includegraphics[scale=1.0]{departamentos/fcfm2} \\
		\end{minipage}
		\begin{minipage}{0.6\textwidth}
			\begin{flushleft}
				\textsc{
				\begin{tabular}{l}
					{\small \nombreuniversidad} ~ \\
					{\small \nombrefacultad} ~ \\
					{\small \departamentouniversidad}
				\end{tabular}
				}
			\end{flushleft}
		\end{minipage}
	\end{minipage}
	\noindent \begin{minipage}{0.2\textwidth}
		\begin{flushright}
			\ifthenelse{\isundefined{\headerimageA}}{}{
				\ifthenelse{\isundefined{\headerimagescaleA}}{}{
					\noindent \includegraphics[scale=\headerimagescaleA]{\headerimageA} \\
				}
			}
		\end{flushright}
	\end{minipage}
	\vfill
	\begin{center}
		{\fontsize{25pt}{15pt} \selectfont \textcolor{\portraittitlecolor}{\textbf{\titulodelinforme}} \vspace{0.7cm}} \\
		{\Large \textcolor{\portraittitlecolor}{\temaatratar}}
	\end{center}
	\vfill
	\begin{center}
		\noindent \tablaintegrantes
	\end{center}
}{
\ifthenelse{\equal{\portraitstyle}{style16}}{
	\setpagemargincm{\pagemarginleft}{\pagemargintop}{\pagemarginright}{\pagemarginbottom}
	\checkextravarexist{\portraitbackgroundimageB}{[portrait-style16] Defina el fondo de la portada en el archivo lib/page/portrait-config.tex (VERSION NORMAL) o bien en el bloque PORTADA (VERSION COMPACTA)}
	\checkextravarexist{\portraitbackgroundcolorB}{[portrait-style16] Defina el color del bloque del titulo de la portada en el archivo lib/page/portrait-config.tex (VERSION NORMAL) o bien en el bloque PORTADA (VERSION COMPACTA)}
	\begingroup
		\thispagestyle{empty}
		\begin{tikzpicture}[remember picture,overlay]
			\node[inner sep=0pt] (background) at (current page.center) {\includegraphics[width=\paperwidth]{\portraitbackgroundimageB}};
			\draw (current page.center) node [fill=\portraitbackgroundcolorB!30!white,fill opacity=0.6,text opacity=1,inner sep=1cm]{\Huge\centering\bfseries\sffamily\parbox[c][][t]{\paperwidth}{
					\centering \textcolor{\portraittitlecolor}{\titulodelinforme} \\ [10pt]
					{\Large \temaatratar} \\ [25pt]
					{\huge \autordeldocumento}}};
		\end{tikzpicture}
		\vfill
	\endgroup
}{
\ifthenelse{\equal{\portraitstyle}{style17}}{
	\setpagemargincm{\pagemarginleft}{\firstpagemargintop}{\pagemarginright}{\pagemarginbottom}
	\pagestyle{fancy}
	\checkextravarexist{\portraitimageC}{[portrait-style17] Defina la imagen de la portada en el archivo lib/page/portrait-config.tex (VERSION NORMAL) o bien en el bloque PORTADA (VERSION COMPACTA)}
	\checkextravarexist{\portraitimageboxedC}{[portrait-style17] Defina si la imagen de la portada se encierra en un recuadro en el archivo lib/page/portrait-config.tex (VERSION NORMAL) o bien en el bloque PORTADA (VERSION COMPACTA)}
	\checkextravarexist{\portraitimageboxedwidthC}{[portrait-style17] Defina el grosor del recuadro de la imagen de la portada en el archivo lib/page/portrait-config.tex (VERSION NORMAL) o bien en el bloque PORTADA (VERSION COMPACTA)}
	\checkextravarexist{\portraitimagewidthC}{[portrait-style17] Defina los parametros de la imagen de la portada en el archivo lib/page/portrait-config.tex (VERSION NORMAL) o bien en el bloque PORTADA (VERSION COMPACTA)}
	\fancyhf{}
	\fancyhead[L]{
		\nombreuniversidad ~ \\
		\nombrefacultad ~ \\
		\departamentouniversidad ~ \\
		\vspace{-0.43cm}
	}
	\fancyhead[R]{
		\includegraphics[scale=\imagendepartamentoescala]{\imagendepartamento}
		\hspace{-0.255cm}
	}
	~ \\
	\vfill
	\begin{center}
		\textcolor{\portraittitlecolor}{
			{\noindent \Huge{\titulodelinforme} \vspace{0.5cm}} ~ \\
			{\noindent \large{\temaatratar}}
		}
	\end{center}
	~ \\
	\ifthenelse{\equal{\portraitimageboxedC}{true}}{
		\insertimageboxed{\portraitimageC}{width=\portraitimagewidthC}{\portraitimageboxedwidthC}{}
	}{
		\insertimage{\portraitimageC}{width=\portraitimagewidthC}{}
	}
	~ \\
	\vfill
	\noindent
	\begin{minipage}{1.0\textwidth}
		\begin{flushright}
			\tablaintegrantes
		\end{flushright}
	\end{minipage}
}{
\ifthenelse{\equal{\portraitstyle}{style18}}{
	\setpagemargincm{\pagemarginleft}{\firstpagemargintop}{\pagemarginright}{\pagemarginbottom}
	\pagestyle{fancy}
	\checkextravarexist{\portraitimageD}{[portrait-style18] Defina la imagen de la portada en el archivo lib/page/portrait-config.tex (VERSION NORMAL) o bien en el bloque PORTADA (VERSION COMPACTA)}
	\checkextravarexist{\portraitimageboxedD}{[portrait-style18] Defina si la imagen de la portada se encierra en un recuadro en el archivo lib/page/portrait-config.tex (VERSION NORMAL) o bien en el bloque PORTADA (VERSION COMPACTA)}
	\checkextravarexist{\portraitimageboxedwidthD}{[portrait-style18] Defina el grosor del recuadro de la imagen de la portada en el archivo lib/page/portrait-config.tex (VERSION NORMAL) o bien en el bloque PORTADA (VERSION COMPACTA)}
	\checkextravarexist{\portraitimagewidthD}{[portrait-style18] Defina los parametros de la imagen de la portada en el archivo lib/page/portrait-config.tex (VERSION NORMAL) o bien en el bloque PORTADA (VERSION COMPACTA)}
	\fancyhf{}
	\fancyhead[L]{
		\nombreuniversidad ~ \\
		\nombrefacultad ~ \\
		\departamentouniversidad ~ \\
		\vspace{-0.43cm}
	}
	\fancyhead[R]{
		\includegraphics[scale=\imagendepartamentoescala]{\imagendepartamento}
		\hspace{-0.255cm}
	}
	~ \\
	\ifthenelse{\equal{\portraitimageboxedD}{true}}{
		\insertimageboxed{\portraitimageD}{width=\portraitimagewidthD}{\portraitimageboxedwidthD}{}
	}{
		\insertimage{\portraitimageD}{width=\portraitimagewidthD}{}
	}
	\vfill
	\begin{center}
		\textcolor{\portraittitlecolor}{
			{\noindent \Huge{\titulodelinforme} \vspace{0.5cm}} ~ \\
			{\noindent \large{\temaatratar}}
		}
	\end{center}
	\vfill
	\noindent
	\begin{minipage}{1.0\textwidth}
		\begin{flushright}
			\tablaintegrantes
		\end{flushright}
	\end{minipage}
}{
\ifthenelse{\equal{\portraitstyle}{\bgtemplatetestcode}}{
	\setpagemargincm{\pagemarginleft}{\pagemargintop}{\pagemarginright}{\pagemarginbottom}
	\pagestyle{empty}
	\pagecolor{lbrown}
	\begin{center}
		\vspace*{-1.0cm}
		\scshape{\nombreuniversidad} ~ \\
		\scshape{\nombrefacultad} ~ \\
		\scshape{\departamentouniversidad}
	\end{center}
	~ \\
	\begin{center}
		\bgtemplatetestimg
	\end{center}
	\begin{center}
		\vspace*{-6cm}
		{\setstretch{1.2} \fontsize{25pt}{22pt} \selectfont \textcolor{\portraittitlecolor}{\scshape{\titulodelinforme}} \vspace{0.5cm}} \\
		{\fontsize{15pt}{10pt} \selectfont \textcolor{\portraittitlecolor}{\scshape{\temaatratar}}}
	\end{center}
	\vfill
	\begin{flushright}
		\noindent \tablaintegrantes
	\end{flushright}
	\newpage
	\pagecolor{white}
}{
	\throwbadconfigondoc{Estilo de portada incorrecto}{\portraitstyle}{style1 .. style18}}}}}}}}}}}}}}}}}}}
}
\ifthenelse{\equal{\addemptypagetwosides}{true}}{
	\newpage
	\null
	\thispagestyle{empty}
	\renewcommand{\thepage}{}
	\newpage}{
}
 % Se puede borrar

% CONFIGURACIÓN DE PÁGINA Y ENCABEZADOS
% Template:     Informe/Reporte LaTeX
% Documento:    Configuración de página
% Versión:      5.5.5 (27/09/2018)
% Codificación: UTF-8
%
% Autor: Pablo Pizarro R. @ppizarror
%        Facultad de Ciencias Físicas y Matemáticas
%        Universidad de Chile
%        pablo.pizarro@ing.uchile.cl, ppizarror.com
%
% Manual template: [http://latex.ppizarror.com/Template-Informe/]
% Licencia MIT:    [https://opensource.org/licenses/MIT/]

\newpage\ifthenelse{\equal{\romanpageuppercase}{true}}{\pagenumbering{Roman}}{\pagenumbering{roman}}\setcounter{page}{1}\setcounter{footnote}{1}\setpagemargincm{\pagemarginleft}{\pagemargintop}{\pagemarginright}{\pagemarginbottom}\def\arraystretch {\tablepaddingv}\setlength{\tabcolsep}{\tablepaddingh em}\def\pagewidth {0.995\linewidth}\ifthenelse{\equal{\pointdecimal}{true}}{\decimalpoint}{}\renewcommand{\appendixname}{\nomltappendixsection}\renewcommand{\appendixpagename}{\nameappendixsection}\renewcommand{\appendixtocname}{\nameappendixsection}\renewcommand{\contentsname}{\nomltcont}\renewcommand{\figurename}{\nomltwfigure}\renewcommand{\listfigurename}{\nomltfigure}\renewcommand{\listtablename}{\nomlttable}\renewcommand{\lstlistingname}{\nomltwsrc}\renewcommand{\lstlistlistingname}{\nomltsrc}\renewcommand{\refname}{\namereferences}\renewcommand{\tablename}{\nomltwtable}\sectionfont{\color{\titlecolor} \fontsizetitle \styletitle \selectfont}\subsectionfont{\color{\subtitlecolor} \fontsizesubtitle \stylesubtitle \selectfont}\subsubsectionfont{\color{\subsubtitlecolor} \fontsizesubsubtitle \stylesubsubtitle \selectfont}\titleformat{\subsubsubsection}{\color{\ssstitlecolor} \normalfont \fontsizessstitle \stylessstitle}{\thesubsubsubsection}{1em}{}\titlespacing*{\subsubsubsection}{0pt}{3.25ex plus 1ex minus .2ex}{1.5ex plus .2ex}\ifthenelse{\equal{\hfstyle}{style1}}{\pagestyle{fancy} \fancyhf{}\fancyhead[L]{\nouppercase{\rightmark}}\fancyhead[R]{\small \rm \thepage}\fancyfoot[L]{\small \rm \textit{\titulodelinforme}}\fancyfoot[R]{\small \rm \textit{\codigodelcurso \nombredelcurso}}\renewcommand{\headrulewidth}{0.5pt}\renewcommand{\footrulewidth}{0.5pt}\renewcommand{\sectionmark}[1]{\markboth{#1}{}}}{\ifthenelse{\equal{\hfstyle}{style2}}{\pagestyle{fancy} \fancyhf{}\fancyhead[L]{\nouppercase{\rightmark}}\fancyhead[R]{\small \rm \thepage}\fancyfoot[L]{\small \rm \textit{\titulodelinforme}}\fancyfoot[R]{\small \rm \textit{\codigodelcurso \nombredelcurso}}\renewcommand{\headrulewidth}{0.5pt}\renewcommand{\footrulewidth}{0pt}\renewcommand{\sectionmark}[1]{\markboth{#1}{}}}{\ifthenelse{\equal{\hfstyle}{style3}}{\pagestyle{fancy} \fancyhf{}\fancyhead[L]{\small \rm \textit{\codigodelcurso \nombredelcurso}\vspace{0.04cm}}\fancyhead[R]{\includegraphics[width=1.2cm]{\imagendepartamento}\vspace{-0.10cm}}\fancyfoot[C]{\thepage}\renewcommand{\headrulewidth}{0.5pt}\renewcommand{\footrulewidth}{0pt}}{\ifthenelse{\equal{\hfstyle}{style4}}{\pagestyle{fancy} \fancyhf{}\fancyhead[L]{\nouppercase{\rightmark}}\fancyhead[R]{}\fancyfoot[C]{\small \rm \thepage}\renewcommand{\headrulewidth}{0.5pt}\renewcommand{\footrulewidth}{0pt}\renewcommand{\sectionmark}[1]{\markboth{#1}{}}}{\ifthenelse{\equal{\hfstyle}{style5}}{\pagestyle{fancy} \fancyhf{}\fancyhead[L]{\codigodelcurso \nombredelcurso}\fancyhead[R]{\nouppercase{\rightmark}}\fancyfoot[L]{\departamentouniversidad, \nombreuniversidad}\fancyfoot[R]{\small \rm \thepage}\renewcommand{\headrulewidth}{0pt}\renewcommand{\footrulewidth}{0pt}\renewcommand{\sectionmark}[1]{\markboth{#1}{}}}{\ifthenelse{\equal{\hfstyle}{style6}}{\pagestyle{fancy} \fancyhf{}\fancyfoot[L]{\departamentouniversidad}\fancyfoot[C]{\thepage}\fancyfoot[R]{\nombreuniversidad}\renewcommand{\headrulewidth}{0pt}\renewcommand{\footrulewidth}{0pt}\setlength{\headheight}{49pt}}{\ifthenelse{\equal{\hfstyle}{style7}}{\pagestyle{fancy} \fancyhf{}\fancyfoot[C]{\thepage}\renewcommand{\headrulewidth}{0pt}\renewcommand{\footrulewidth}{0pt}\setlength{\headheight}{49pt}}{\ifthenelse{\equal{\hfstyle}{style8}}{\pagestyle{fancy} \fancyhf{}\fancyfoot[R]{\thepage}\renewcommand{\headrulewidth}{0pt}\renewcommand{\footrulewidth}{0pt}\setlength{\headheight}{49pt}}{\ifthenelse{\equal{\hfstyle}{style9}}{\pagestyle{fancy} \fancyhf{}\fancyhead[L]{\nouppercase{\rightmark}}\fancyhead[R]{}\fancyfoot[L]{\small \rm \textit{\titulodelinforme}}\fancyfoot[R]{\small \rm \thepage}\renewcommand{\headrulewidth}{0.5pt}\renewcommand{\footrulewidth}{0.5pt}\renewcommand{\sectionmark}[1]{\markboth{#1}{}}}{\ifthenelse{\equal{\hfstyle}{style10}}{\pagestyle{fancy} \fancyhf{}\fancyhead[L]{\nouppercase{\rightmark}}\fancyhead[R]{\small \rm \textit{\titulodelinforme}}\fancyfoot[L]{}\fancyfoot[R]{\small \rm \thepage}\renewcommand{\headrulewidth}{0.5pt}\renewcommand{\footrulewidth}{0.5pt}\renewcommand{\sectionmark}[1]{\markboth{#1}{}}}{\ifthenelse{\equal{\hfstyle}{style11}}{\pagestyle{fancy} \fancyhf{}\fancyhead[L]{\nouppercase{\rightmark}}\fancyhead[R]{\small \rm \thepage \nomnpageof \pageref{LastPage}}\fancyfoot[L]{\small \rm \textit{\titulodelinforme}}\fancyfoot[R]{\small \rm \textit{\codigodelcurso \nombredelcurso}}\renewcommand{\headrulewidth}{0.5pt}\renewcommand{\footrulewidth}{0.5pt}\renewcommand{\sectionmark}[1]{\markboth{#1}{}}}{\ifthenelse{\equal{\hfstyle}{style12}}{\pagestyle{fancy} \fancyhf{}\fancyfoot[L]{\departamentouniversidad}\fancyfoot[C]{\thepage \nomnpageof \pageref{LastPage}}\fancyfoot[R]{\nombreuniversidad}\renewcommand{\headrulewidth}{0pt}\renewcommand{\footrulewidth}{0pt}\setlength{\headheight}{49pt}}{\ifthenelse{\equal{\hfstyle}{style13}}{\pagestyle{fancy} \fancyhf{}\fancyhead[L]{\small \rm \textit{\codigodelcurso \nombredelcurso}\vspace{0.04cm}}\fancyhead[R]{\includegraphics[width=1.2cm]{\imagendepartamento}\vspace{-0.10cm}}\fancyfoot[C]{\thepage \nomnpageof \pageref{LastPage}}\renewcommand{\headrulewidth}{0.5pt}\renewcommand{\footrulewidth}{0pt}}{\ifthenelse{\equal{\hfstyle}{style14}}{\pagestyle{fancy} \fancyhf{}\fancyhead[L]{\nouppercase{\rightmark}}\fancyhead[R]{}\fancyfoot[C]{\small \rm \thepage \nomnpageof \pageref{LastPage}}\renewcommand{\headrulewidth}{0.5pt}\renewcommand{\footrulewidth}{0pt}\renewcommand{\sectionmark}[1]{\markboth{#1}{}}}{\throwbadconfigondoc{Estilo de header-footer incorrecto}{\hfstyle}{style1 .. style14}}}}}}}}}}}}}}}\ifthenelse{\equal{\showlinenumbers}{true}}{\linenumbers}{}


% TABLA DE CONTENIDOS - ÍNDICE
% Template:     Informe/Reporte LaTeX
% Documento:    Índice
% Versión:      6.1.0 (03/11/2018)
% Codificación: UTF-8
%
% Autor: Pablo Pizarro R. @ppizarror
%        Facultad de Ciencias Físicas y Matemáticas
%        Universidad de Chile
%        pablo.pizarro@ing.uchile.cl, ppizarror.com
%
% Manual template: [https://latex.ppizarror.com/Template-Informe/]
% Licencia MIT:    [https://opensource.org/licenses/MIT/]

\ifthenelse{\equal{\showindex}{true}}{
	\newpage
	\begingroup
	\sectionfont{\color{\indextitlecolor} \fontsizetitlei \styletitlei \selectfont}
	\ifthenelse{\equal{\addemptypagetwosides}{true}}{
		\checkoddpage
		\ifoddpage
		\else
			\newpage
			\null
			\thispagestyle{empty}
			\newpage
			\addtocounter{page}{-1}
		\fi}{
	}
	\ifthenelse{\equal{\addindextobookmarks}{true}}{
		\belowpdfbookmark{\nomltcont}{contents}}{
	}
	\tocloftpagestyle{fancy}
	\ifthenelse{\equal{\showdotontitles}{true}}{
		\def\cftsecaftersnum {.}
		\def\cftsubsecaftersnum {.}
		\def\cftsubsubsecaftersnum {.}
		\def\cftsubsubsubsecaftersnum {.}
		}{
	}
	\def\cftfigaftersnum {\charafterobjectindex\enspace}
	\def\cftsubfigaftersnum {\charafterobjectindex\enspace}
	\def\cfttabaftersnum {\charafterobjectindex\enspace}
	\def\cftlstlistingaftersnum {\charafterobjectindex\enspace}
	\renewcommand{\cftdot}{\charnumpageindex}
	\ifthenelse{\equal{\showlinenumbers}{true}}{
		\nolinenumbers}{
	}
	\ifthenelse{\equal{\objectindexindent}{true}}{
		\def\cftlstlistingindent {1.495em}
	}{
		\setlength{\cfttabindent}{0in}
		\setlength{\cftfigindent}{0in}
		\setlength{\cftsubfigindent}{0in}
		\setlength{\cftfigindent}{0in}
		\def\cftlstlistingindent {0.01em}
	}
	\ifthenelse{\equal{\equalmarginnumobject}{true}}{
		\ifthenelse{\equal{\showsectioncaption}{none}}{
			\def\cftdefautnumwidth {2.3em}
		}{
		\ifthenelse{\equal{\showsectioncaption}{sec}}{
			\def\cftdefautnumwidth {3.0em}
		}{
		\ifthenelse{\equal{\showsectioncaption}{ssec}}{
			\def\cftdefautnumwidth {3.7em}
		}{
		\ifthenelse{\equal{\showsectioncaption}{sssec}}{
			\def\cftdefautnumwidth {4.4em}
		}{
		\ifthenelse{\equal{\showsectioncaption}{ssssec}}{
			\def\cftdefautnumwidth {5.1em}
		}{
			\throwbadconfig{Valor configuracion incorrecto}{\showsectioncaption}{none,sec,ssec,sssec,ssssec}}}}}
		}
		\def\cftfignumwidth {\cftdefautnumwidth}
		\def\cftsubfignumwidth {\cftdefautnumwidth}
		\def\cfttabnumwidth {\cftdefautnumwidth}
		\def\cftlstlistingnumwidth {\cftdefautnumwidth}}{
	}
	\ifthenelse{\equal{\showindexofcontents}{true}}{\tableofcontents}{}
	\iftotalfigures
		\ifthenelse{\equal{\showindexoffigures}{true}}{
			\ifthenelse{\equal{\indexforcenewpage}{true}}{\newpage}{}
			\listoffigures
		}{}
	\fi
	\iftotaltables
		\ifthenelse{\equal{\showindexoftables}{true}}{
			\ifthenelse{\equal{\indexforcenewpage}{true}}{\newpage}{}
			\listoftables
		}{}
	\fi
	\iftotallstlistings
		\ifthenelse{\equal{\showindexofcode}{true}}{
			\ifthenelse{\equal{\indexforcenewpage}{true}}{\newpage}{}
			\lstlistoflistings
		}{}
	\fi
	\endgroup
	\newpage
	\ifthenelse{\equal{\addemptypagetwosides}{true}}{
		\vfill
		\checkoddpage
		\ifoddpage
			\newpage
			\null
			\thispagestyle{empty}
			\newpage
			\addtocounter{page}{-1}
		\else
		\fi}{
	}
}{}
 % Se puede borrar

% CONFIGURACIONES FINALES
% Template:     Informe/Reporte LaTeX
% Documento:    Configuraciones finales
% Versión:      6.1.6 (14/12/2018)
% Codificación: UTF-8
%
% Autor: Pablo Pizarro R. @ppizarror
%        Facultad de Ciencias Físicas y Matemáticas
%        Universidad de Chile
%        pablo.pizarro@ing.uchile.cl, ppizarror.com
%
% Manual template: [https://latex.ppizarror.com/Template-Informe/]
% Licencia MIT:    [https://opensource.org/licenses/MIT/]

\markboth{}{}
\newpage
\ifthenelse{\equal{\disablehfrightmark}{false}}{
	\ifthenelse{\equal{\hfstyle}{style1}}{
		\fancyhead[L]{\nouppercase{\leftmark}}}{
	}
	\ifthenelse{\equal{\hfstyle}{style2}}{
		\fancyhead[L]{\nouppercase{\leftmark}}}{
	}
	\ifthenelse{\equal{\hfstyle}{style4}}{
		\fancyhead[L]{\nouppercase{\leftmark}}}{
	}
	\ifthenelse{\equal{\hfstyle}{style5}}{
		\fancyhead[R]{\nouppercase{\leftmark}}}{
	}
	\ifthenelse{\equal{\hfstyle}{style9}}{
		\fancyhead[L]{\nouppercase{\leftmark}}}{
	}
	\ifthenelse{\equal{\hfstyle}{style10}}{
		\fancyhead[L]{\nouppercase{\leftmark}}}{
	}
\ifthenelse{\equal{\hfstyle}{style11}}{
		\fancyhead[L]{\nouppercase{\leftmark}}}{
	}
\ifthenelse{\equal{\hfstyle}{style14}}{
		\fancyhead[L]{\nouppercase{\leftmark}}}{
	}}{
}
\sectionfont{\color{\titlecolor} \fontsizetitle \styletitle \selectfont}
\subsectionfont{\color{\subtitlecolor} \fontsizesubtitle \stylesubtitle \selectfont}
\subsubsectionfont{\color{\subsubtitlecolor} \fontsizesubsubtitle \stylesubsubtitle \selectfont}
\titleformat{\subsubsubsection}{\color{\ssstitlecolor} \normalfont \fontsizessstitle \stylessstitle}{\thesubsubsubsection}{1em}{}
\titlespacing*{\subsubsubsection}{0pt}{3.25ex plus 1ex minus .2ex}{1.5ex plus .2ex}
\ifthenelse{\equal{\showsectioncaption}{none}}{
}{
\ifthenelse{\equal{\showsectioncaption}{sec}}{
	\counterwithin{equation}{section}
	\counterwithin{figure}{section}
	\counterwithin{lstlisting}{section}
	\counterwithin{table}{section}
}{
\ifthenelse{\equal{\showsectioncaption}{ssec}}{
	\counterwithin{equation}{subsection}
	\counterwithin{figure}{subsection}
	\counterwithin{lstlisting}{subsection}
	\counterwithin{table}{subsection}
}{
\ifthenelse{\equal{\showsectioncaption}{sssec}}{
	\counterwithin{equation}{subsubsection}
	\counterwithin{figure}{subsubsection}
	\counterwithin{lstlisting}{subsubsection}
	\counterwithin{table}{subsubsection}
}{
\ifthenelse{\equal{\showsectioncaption}{ssssec}}{
	\counterwithin{equation}{subsubsubsection}
	\counterwithin{figure}{subsubsubsection}
	\counterwithin{lstlisting}{subsubsubsection}
	\counterwithin{table}{subsubsubsection}
}{
	\throwbadconfig{Valor configuracion incorrecto}{\showsectioncaption}{none,sec,ssec,sssec,ssssec}
}}}}}
\ifthenelse{\equal{\predocuseromannumber}{true}}{
	\renewcommand{\thepage}{\arabic{page}}}{
}
\ifthenelse{\equal{\resetpagnumafterindex}{true}}{
	\setcounter{page}{1}}{
}
\setcounter{section}{0}
\setcounter{footnote}{0}
\ifthenelse{\equal{\showlinenumbers}{true}}{
	\linenumbers}{
}


% ======================= INICIO DEL DOCUMENTO =======================

% 1 Información general edificio: zona, masa, peso, suelo, Periodos fundamentales de la estructura, aceleración. [Mauro] [OK]
\section{Información General del Edificio}
% 1 Información general edificio: zona, masa, peso, suelo, Periodos fundamentales de la estructura, aceleración. [Mauro]
    A continuación se presentan los principales parámetros usados para el análisis sísimco del edificio:
    
    \begin{table}[H]
      \centering
      \caption{Información general del edificio.}
        \begin{tabular}{cccc}
            \toprule
            \textbf{Zona} & \multicolumn{1}{c}{\textbf{Suelo}} & \textbf{Masa [tonf]} & \textbf{Peso Sísmico [Tonf]} \\
            \midrule
            3     & A     & 11434,2 & 11763,3 \\
            \bottomrule
        \end{tabular}%
      \label{info_general}%
    \end{table}%
    
    \begin{table}[H]
      \centering
      \caption{Períodos fundamentales.}
        \begin{tabular}{ccccc}
            \toprule
            \textbf{Modo} & \textbf{Tipo} & \textbf{Período (s)} & \textbf{\% De masa} & \multicolumn{1}{c}{\textbf{${S_a}_{max} [cm/s^2$]}} \\
            \midrule
                1   & Torsión z     & 1,345     & 45,52\%   & - \\
                2   & Traslación x  & 1,112     & 55,52\%   & 2,17 \\
                3   & Traslación y  & 0,744     & 56,05\%   & 3,22 \\
            \bottomrule
        \end{tabular}%
      \label{periodos}%
    \end{table}%

% 2 Espectro Nch433 para Edificio 
% 3 Factor R y R0, materiales 
\newpage

% 2 Espectro Nch433 para Edificio
% 3 Factor R y R0, materiales
\section{Parámetros de Diseño}

\subsection{Espectro de diseño NCh 433}

El espectro de diseño elástico se construyó a partir de los datos del suelo (tipo A), de la ubicación del edificio (Antofagasta, zona sísmica 3) y del uso (Residencial, categoría II). \\

La Figura \ref{fig-espectro-elastico} ilustra el espectro elástico utilizado.

\insertimage[\label{fig-espectro-elastico}]{espectro}{width=10cm}{Espectro elástico.}

Dado que el suelo de fundación de la estructura es de buena calidad (roca, tipo A) se puede considerar que no existe un contraste de impedancias alto con la roca basal, ello implica que el factor de amplificación del desplazamiento en superficie es similar a 1. Así, el espectro funciona como una envolvente de pseudoaceleraciones de los posibles sismos.

\subsection{Estructuración}

El tipo de estructuración es del tipo muro y losa de Hormigón Armado (H.A.); En este sentido el muro tiene la función de transmitir cargas tipo gravitacionales (compresión) a las fundaciones y resistir cargas cortantes, tracciones y compresiones por flexión en en caso de un sismo. Las losas por otra parte tributan las cargas al sistema de muros. \\

De acuerdo a lo estipulado en la NCh433 Tabla 5.1 y Tabla 6.1 se tienen los siguientes parámetros de modificación de acuerdo al tipo de estructuración, en donde $R$ y $R_o$ corresponden a factores de reducción e $I$ es el nivel de importancia de la estructura.

\begin{table}[H]
  \centering
  \caption{Parámetros de diseño sísmico dado la estructuración.}
    \begin{tabular}{|c|c|}
    \hline
    \textbf{Parámetro} & \textbf{Valor} \bigstrut\\
    \hline
    R     & 7 \bigstrut\\
    \hline
    $R_o$ & 11 \bigstrut\\
    \hline
    I     & 1 \bigstrut\\
    \hline
    \end{tabular}%
\end{table}%

El espesor de muros se calculó en base a una resistencia de corte admisible promedio de $\tau$=7 $kgf/cm2$, corregida por efectos de deformación sísmica considerando el espectro elástico; en la Tabla \ref{tabla-densidades-muro} se detalla la densidad de muros obtenida en cada eje.

\begin{table}[H]
  \centering
  \caption{Densidad de muros.}
  \begin{tabular}{ccccc}
    \hline
    \textbf{N° piso} & \textbf{ex [m]} & \textbf{ey [m]} & \textbf{Densidad muro x} & \textbf{Densidad muro y} \bigstrut\\
    \hline
    23    & 0.2   & 0.2   & 0.028 & 0.030 \bigstrut[t]\\
    22    & 0.2   & 0.2   & 0.028 & 0.030 \\
    21    & 0.2   & 0.2   & 0.028 & 0.030 \\
    20    & 0.2   & 0.2   & 0.028 & 0.030 \\
    19    & 0.2   & 0.2   & 0.028 & 0.030 \\
    18    & 0.2   & 0.2   & 0.028 & 0.030 \\
    17    & 0.2   & 0.2   & 0.028 & 0.030 \\
    16    & 0.2   & 0.2   & 0.028 & 0.030 \\
    15    & 0.2   & 0.2   & 0.028 & 0.030 \\
    14    & 0.2   & 0.2   & 0.034 & 0.037 \\
    13    & 0.25  & 0.25  & 0.034 & 0.037 \\
    12    & 0.25  & 0.25  & 0.034 & 0.037 \\
    11    & 0.25  & 0.25  & 0.034 & 0.037 \\
    10    & 0.25  & 0.25  & 0.034 & 0.037 \\
    9     & 0.25  & 0.25  & 0.034 & 0.037 \\
    8     & 0.25  & 0.25  & 0.034 & 0.037 \\
    7     & 0.25  & 0.25  & 0.034 & 0.037 \\
    6     & 0.25  & 0.25  & 0.034 & 0.037 \\
    5     & 0.25  & 0.25  & 0.034 & 0.037 \\
    4     & 0.25  & 0.25  & 0.034 & 0.037 \\
    3     & 0.25  & 0.25  & 0.037 & 0.037 \\
    2     & 0.25  & 0.25  & 0.041 & 0.038 \\
    1     & 0.25  & 0.3   & 0.043 & 0.041 \\
    -1    & 0.25  & 0.3   & 0.029 & 0.027 \bigstrut[b]\\
    \hline
  \end{tabular}
  \label{tabla-densidades-muro}
\end{table}

En los últimos años la densidad de muros se ha concentrado entre 0.02 y 0.035, obteniendo un buen desempeño en la estructura, por tanto, como se puede observar en la tabla anterior, en todos los pisos se obtuvo una densidad de muros superior al 0.02 y muy cercana a los 0.035 sugeridos.

\newpage
\subsection{Materiales}

\begin{itemize}
    \item \textbf{Hormigón}
    
    Se hará uso de tres tipos de hormigones para el edificio, H-20, H-30 y H-35. Esto se eligió en base a un requerimiento mínimo de resistencia a fuerzas axiales (peso), considerando la siguiente relación:
    
    \insertequationanum{N_u \geq 0.35 \cdot f'_c \cdot A_g}
    
    En donde $N_u$ corresponde a la resistencia última a compresión, $f'c$ la resistencia a compresión del hormigón y $A_g$ el área gruesa de la sección de muros resistente. Dado lo anterior se construyó la Tabla \ref{tab-g-horm-piso} para verificar el tipo de hormigón. En promedio se obtuvo un 70\% de factor de utilización.
    
\begin{table}[H]
  \centering
  \caption{Definición del tipo de hormigón para cada piso.}
  \begin{tabular}{|c|cc|c|c|c|}
    \hline
    \textbf{N° piso} & \textbf{Peso} & \textbf{Peso acum} & \boldmath{}\textbf{$f'c$}\unboldmath{} & \boldmath{}\textbf{$N_u$}\unboldmath{} & \boldmath{}\textbf{$N_u$}\unboldmath{} \bigstrut\\
    \hline
    23    & 530.45 & 530.45 & \cellcolor[rgb]{ .816,  .808,  .808}20 & 4760.3 & 11\% \bigstrut[t]\\
    22    & 435.97 & 966.41 & \cellcolor[rgb]{ .816,  .808,  .808}20 & 4760.3 & 20\% \\
    21    & 435.97 & 1402.38 & \cellcolor[rgb]{ .816,  .808,  .808}20 & 4760.3 & 29\% \\
    20    & 435.97 & 1838.35 & \cellcolor[rgb]{ .816,  .808,  .808}20 & 4760.3 & 39\% \\
    19    & 435.97 & 2274.31 & \cellcolor[rgb]{ .816,  .808,  .808}20 & 4760.3 & 48\% \\
    18    & 435.97 & 2710.28 & \cellcolor[rgb]{ .816,  .808,  .808}20 & 4760.3 & 57\% \\
    17    & 435.97 & 3146.25 & \cellcolor[rgb]{ .816,  .808,  .808}20 & 4760.3 & 66\% \\
    16    & 435.97 & 3582.21 & \cellcolor[rgb]{ .816,  .808,  .808}20 & 4760.3 & 75\% \\
    15    & 435.97 & 4018.18 & \cellcolor[rgb]{ .816,  .808,  .808}20 & 4760.3 & 84\% \\
    14    & 452.51 & 4470.69 & \cellcolor[rgb]{ .816,  .808,  .808}20 & 4760.3 & 94\% \\
    13    & 469.05 & 4939.74 & \cellcolor[rgb]{ .651,  .651,  .651}30 & 8925.5 & 55\% \\
    12    & 469.05 & 5408.79 & \cellcolor[rgb]{ .651,  .651,  .651}30 & 8925.5 & 61\% \\
    11    & 469.05 & 5877.84 & \cellcolor[rgb]{ .651,  .651,  .651}30 & 8925.5 & 66\% \\
    10    & 469.05 & 6346.89 & \cellcolor[rgb]{ .651,  .651,  .651}30 & 8925.5 & 71\% \\
    9     & 469.05 & 6815.94 & \cellcolor[rgb]{ .651,  .651,  .651}30 & 8925.5 & 76\% \\
    8     & 469.05 & 7284.99 & \cellcolor[rgb]{ .651,  .651,  .651}30 & 8925.5 & 82\% \\
    7     & 469.05 & 7754.04 & \cellcolor[rgb]{ .502,  .502,  .502}35 & 10413.1 & 74\% \\
    6     & 469.05 & 8223.09 & \cellcolor[rgb]{ .502,  .502,  .502}35 & 10413.1 & 79\% \\
    5     & 469.05 & 8692.14 & \cellcolor[rgb]{ .502,  .502,  .502}35 & 10413.1 & 83\% \\
    4     & 469.05 & 9161.19 & \cellcolor[rgb]{ .502,  .502,  .502}35 & 10413.1 & 88\% \\
    3     & 471.81 & 9633.00 & \cellcolor[rgb]{ .502,  .502,  .502}35 & 10413.1 & 93\% \\
    2     & 480.38 & 10113.38 & \cellcolor[rgb]{ .502,  .502,  .502}35 & 10765.9 & 94\% \\
    1     & 475.35 & 10588.73 & \cellcolor[rgb]{ .502,  .502,  .502}35 & 11569.5 & 92\% \\
    -1    & 670.51 & 11259.24 & \cellcolor[rgb]{ .502,  .502,  .502}35 & 12491.0 & 90\% \bigstrut[b]\\
    \hline
  \end{tabular}
  \label{tab-g-horm-piso}
\end{table}

\begin{table}[H]
  \centering
  \caption{Tensiones admisibles y características de hormigones a utilizar.}
  \begin{tabular}{cccccc}
    \hline
    \textbf{\textbf{Grado}} & \boldmath{}\textbf{\textbf{$\rho\  [tonf/m^3]$}}\unboldmath{} & \boldmath{}\textbf{\textbf{$f'c\  [Mpa]$}}\unboldmath{} & \boldmath{}\textbf{\textbf{$E\  [Mpa]$}}\unboldmath{} & \boldmath{}\textbf{\textbf{$\nu\  [-]$}}\unboldmath{} & \boldmath{}\textbf{\textbf{$G\  [Mpa]$}}\unboldmath{} \bigstrut\\
    \hline
    H20   & 2.5   & 20    & 21019.04 & 0.2   & 8757.93 \bigstrut[t]\\
    H30   & 2.5   & 30    & 25742.96 & 0.2   & 10726.23 \\
    H35   & 2.5   & 35    & 27805.57 & 0.2   & 11585.66 \bigstrut[b]\\
    \hline
  \end{tabular}
  \label{tab:addlabel}
\end{table}

\item \textbf{Acero}\\
    Se considera un acero ASTM-A36 como acero estructural y de refuerzo, el cual posee las siguientes características:
    
    \begin{table}[H]
      \centering
      \caption{Tensiones admisibles y características del acero a utilizar.}
      \resizebox{\textwidth}{!}{%
        \begin{tabular}{|c|c|c|c|c|c|}
        \hline
        \textbf{Grado} &
          \boldmath{}\textbf{$F_{y} \ [tonf/cm^2]$} &
          \boldmath{}\textbf{$F_{u} \ [tonf/cm^2]$} &
          \boldmath{}\textbf{$E \ [tonf/cm^2]$} &
          \boldmath{}\textbf{$\nu \ [-]$} &
          \boldmath{}\textbf{$G \ [tonf/cm^2]$}
          \bigstrut\\
        \hline
        A36 &
          2,53 &
          4,08 &
          2100 &
          0,29 &
          787,44
          \bigstrut\\
        \hline
        \end{tabular}}%
    \end{table}% 

\end{itemize} % [Pablo] [OK]

% 4 Calculo Factor corrección R*, para sismo X e Y [Pablo]
% 5 Chequear Masa equivalente mayor 90% en las dos direcciones. Imprimir tabla respectiva. [Pablo]
% TABLA MAGICA
% 6 Imprimir tablas: [Pablo]
% - Cortes basales
% - Periodos y participación de masas
% - Deformaciones del centro de masa
% 7 Coeficientes Sísmicos Max y Min. [Pablo]
% 8 Coeficientes Sísmicos para X e Y. [Pablo]
\newpage
\section{Resultados del Análisis Estructural}
% 4 Calculo Factor corrección R*, para sismo X e Y [Pablo]
% 5 Chequear Masa equivalente mayor 90% en las dos direcciones. Imprimir tabla respectiva. [Pablo]
% TABLA MAGICA
% 6 Imprimir tablas: [Pablo]
% - Cortes basales
% - Periodos y participación de masas
% - Deformaciones del centro de masa
% 7 Coeficientes Sísmicos Max y Min. [Pablo]
% 8 Coeficientes Sísmicos para X e Y. [Pablo]

\subsection{Cálculo factor corrección R*}

Al calcular los períodos fundamentales de la estructura en los ejes traslacionales $x$ e $y$ se obtuvo los factores de reducción $R*_x$ y $R*_y$ del pseudo-espectro de aceleración mostrados en la Tabla \ref{tab:factores-reduccion-r}. Posteriormente al obtener el corte basal en ambos ejes se calcularon los factores $f_x$ e $f_y$ para mayorar los $R*$ de cada eje.

\begin{table}[H]
  \centering
  \caption{Factores de corrección $R*$.}
  \begin{tabular}{ccccc}
    \hline
    \textbf{Eje} & \textbf{R*} & \textbf{1/R* [g] (scale factor)} & \textbf{f} & \textbf{R* mayorado} \bigstrut\\
    \hline
    x     & 10.579 & 0.927 & 3.076 & 2.851 \bigstrut[t]\\
    y     & 10.003 & 0.980 & 2.435 & 2.387 \bigstrut[b]\\
    \hline
  \end{tabular}
  \label{tab:factores-reduccion-r}
\end{table}

Los factores $R*$ mayorados fueron usados para definir los casos sísmicos en ETABS.

\begin{images}{Factores de mayoración del espectro en ETABS.}
    \addimage{sisx}{width=7cm}{Sismo en x.}
    \addimage{sisy}{width=7cm}{Sismo en y.}
\end{images}

\newpage
\subsection{Resultados valores por dirección x/y}

A partir del método modal espectral se obtuvieron los siguientes valores por cada eje X e Y :

\begin{table}[H]
  \centering
  \caption{Valores por dirección X del sismo.}
  \begin{tabular}{lrl}
    \hline
    \textbf{Sismo X} & \multicolumn{1}{c}{\textbf{Valor}} & \multicolumn{1}{c}{\textbf{Unidad}} \bigstrut\\
    \hline
    Peso sísmico & 11434.22 & [Tonf] \bigstrut[t]\\
    Corte Basal Mínimo & 705.80 & [Tonf] \\
    Corte Basal Máximo & 1482.18 & [Tonf] \\
    Período Predominante & 1.112 & [s] \\
    $R^*$ & 10.579 & [-] \\
    Factor de Mayoración & 3.076 & [-] \\
    Factor de Minoración & 1.000 & [-] \\
    Corte Basal Efectivo & 705.61 & [Tonf] \\
    Momento Volcante & 20305.03 & [Tonf-m] \\
    Brazo de Palanca & 28.78 & [m] \\
    Desplazamiento Último ($\delta ux=1.3Sd$) & 0.09  & [cm] \bigstrut[b]\\
    \hline
  \end{tabular}
\end{table}

\begin{table}[H]
  \centering
  \caption{Valores por dirección Y del sismo.}
  \begin{tabular}{lrl}
    \hline
    \textbf{Sismo Y} & \multicolumn{1}{c}{\textbf{Valor}} & \multicolumn{1}{c}{\textbf{Unidad}} \bigstrut\\
    \hline
    Peso sísmico & 11434.22 & [Tonf] \bigstrut[t]\\
    Corte Basal Mínimo & 705.80 & [Tonf] \\
    Corte Basal Máximo & 1482.18 & [Tonf] \\
    Período Predominante & 0.744 & [s] \\
    $R^*$ & 10.003 & [-] \\
    Factor de Mayoración & 2.435 & [-] \\
    Factor de Minoración & 1.000 & [-] \\
    Corte Basal Efectivo & 705.90 & [Tonf] \\
    Momento Volcante & 24152.00 & [Tonf-m] \\
    Brazo de Palanca & 34.21 & [m] \\
    Desplazamiento Último ($\delta ux=1.3Sd$) & 0.06  & [cm] \bigstrut[b]\\
    \hline
  \end{tabular}
\end{table}

\newpage
\subsection{Tablas de resultados}

\subsubsection{Periodos y participación de masas}

En la Tabla \ref{tabla-periodo-participacion} se detallan los períodos obtenidos para los modos en los cuales se alcanzó un 90\% de masa en las direcciones de análisis.

\begin{longtable}{cccccccc}
\caption{Períodos y participación de masas.}\label{tabla-periodo-participacion}\\
\hline
\multicolumn{1}{|c}{\textbf{Modo}} & \textbf{T [s]} & \textbf{\% Mx} & \textbf{\% My} & \textbf{\% Rz} & \textbf{\% \boldmath{}\textbf{$\sum$}\unboldmath{}Mx} & \textbf{\% \boldmath{}\textbf{$\sum$}\unboldmath{}My} & \textbf{\% \boldmath{}\textbf{$\sum$}\unboldmath{}Rz} \bigstrut\\
\hline
\endfirsthead
\hline
\multicolumn{1}{|c}{\textbf{Modo}} & \textbf{T (s)} & \textbf{\% Mx} & \textbf{\% My} & \textbf{\% Rz} & \textbf{\% \boldmath{}\textbf{$\sum$}\unboldmath{}Mx} & \textbf{\% \boldmath{}\textbf{$\sum$}\unboldmath{}My} & \textbf{\% \boldmath{}\textbf{$\sum$}\unboldmath{}Rz} \bigstrut\\
\hline
\endhead
\hline
\endfoot
\hline
\endlastfoot
1     & 1.315 & 4.54  & 5.42  & 45.52 & 4.54  & 5.42  & 45.52 \\
    2     & 1.112 & 55.52 & 0.78  & 1.95  & 60.06 & 6.20  & 47.47 \\
    3     & 0.744 & 0.03  & 56.05 & 6.66  & 60.08 & 62.24 & 54.13 \\
    4     & 0.382 & 0.45  & 1.06  & 8.75  & 60.54 & 63.30 & 62.89 \\
    5     & 0.272 & 15.40 & 0.13  & 0.06  & 75.94 & 63.43 & 62.95 \\
    6     & 0.216 & 0.08  & 12.12 & 1.33  & 76.02 & 75.55 & 64.28 \\
    7     & 0.185 & 0.01  & 0.70  & 3.62  & 76.03 & 76.25 & 67.90 \\
    8     & 0.124 & 5.66  & 0.11  & 0.04  & 81.69 & 76.36 & 67.94 \\
    9     & 0.111 & 0.16  & 0.00  & 2.55  & 81.85 & 76.36 & 70.50 \\
    10    & 0.107 & 0.11  & 5.11  & 0.08  & 81.96 & 81.48 & 70.58 \\
    11    & 0.077 & 1.33  & 0.00  & 1.06  & 83.29 & 81.48 & 71.64 \\
    12    & 0.073 & 1.94  & 0.32  & 0.71  & 85.23 & 81.79 & 72.35 \\
    13    & 0.07  & 0.00  & 0.00  & 0.02  & 85.23 & 81.80 & 72.37 \\
    14    & 0.069 & 0.25  & 2.83  & 0.14  & 85.48 & 84.63 & 72.50 \\
    15    & 0.067 & 0.00  & 0.00  & 0.01  & 85.49 & 84.63 & 72.52 \\
    16    & 0.056 & 0.05  & 0.04  & 1.39  & 85.53 & 84.67 & 73.91 \\
    17    & 0.052 & 0.00  & 0.02  & 0.00  & 85.54 & 84.68 & 73.91 \\
    18    & 0.052 & 2.04  & 0.33  & 0.05  & 87.57 & 85.01 & 73.96 \\
    19    & 0.049 & 0.00  & 0.01  & 0.01  & 87.58 & 85.02 & 73.97 \\
    20    & 0.049 & 0.41  & 1.75  & 0.08  & 87.99 & 86.76 & 74.05 \\
    21    & 0.048 & 0.00  & 0.00  & 0.03  & 87.99 & 86.77 & 74.08 \\
    22    & 0.046 & 0.00  & 0.00  & 0.00  & 87.99 & 86.77 & 74.08 \\
    23    & 0.043 & 0.01  & 0.07  & 1.39  & 88.00 & 86.84 & 75.47 \\
    24    & 0.04  & 0.15  & 0.01  & 0.00  & 88.15 & 86.85 & 75.47 \\
    25    & 0.039 & 1.03  & 0.47  & 0.00  & 89.18 & 87.32 & 75.48 \\
    26    & 0.038 & 0.68  & 1.04  & 0.03  & 89.86 & 88.37 & 75.50 \\
    27    & 0.035 & 0.09  & 0.08  & 1.47  & 89.95 & 88.45 & 76.97 \\
    28    & 0.035 & 0.03  & 0.00  & 0.02  & 89.99 & 88.45 & 76.99 \\
    29    & 0.033 & 0.04  & 0.00  & 0.02  & 90.02 & 88.46 & 77.01 \\
    30    & 0.032 & 0.15  & 0.85  & 0.00  & 90.18 & 89.30 & 77.01 \\
    31    & 0.031 & 0.39  & 0.21  & 0.07  & 90.57 & 89.52 & 77.07 \\
    32    & 0.031 & 0.64  & 0.05  & 0.02  & 91.21 & 89.57 & 77.09 \\
    33    & 0.03  & 0.23  & 0.00  & 0.38  & 91.43 & 89.57 & 77.47 \\
    34    & 0.029 & 0.22  & 0.08  & 1.04  & 91.65 & 89.65 & 78.50 \\
    35    & 0.029 & 0.00  & 0.00  & 0.00  & 91.65 & 89.65 & 78.50 \\
    36    & 0.028 & 0.00  & 1.08  & 0.01  & 91.65 & 90.73 & 78.52 \\
    37    & 0.027 & 0.07  & 0.01  & 0.06  & 91.72 & 90.73 & 78.57 \\
    38    & 0.026 & 0.90  & 0.00  & 0.02  & 92.62 & 90.74 & 78.59 \\
    39    & 0.026 & 0.03  & 0.06  & 0.16  & 92.65 & 90.80 & 78.75 \\
    40    & 0.025 & 0.26  & 0.10  & 1.11  & 92.91 & 90.90 & 79.85 \\
    41    & 0.025 & 0.00  & 0.00  & 0.00  & 92.91 & 90.90 & 79.86 \\
    42    & 0.025 & 0.02  & 0.34  & 0.00  & 92.93 & 91.24 & 79.86 \\
    43    & 0.024 & 0.00  & 0.01  & 0.01  & 92.93 & 91.25 & 79.86 \\
    44    & 0.024 & 0.01  & 0.73  & 0.02  & 92.94 & 91.99 & 79.89 \\
    45    & 0.022 & 0.00  & 0.00  & 0.02  & 92.94 & 91.99 & 79.91 \\
    46    & 0.022 & 0.00  & 0.22  & 1.15  & 92.94 & 92.20 & 81.06 \\
    47    & 0.022 & 0.87  & 0.15  & 0.27  & 93.81 & 92.35 & 81.33 \\
    48    & 0.022 & 0.27  & 0.06  & 0.04  & 94.08 & 92.42 & 81.37 \\
    49    & 0.021 & 0.04  & 1.30  & 0.12  & 94.12 & 93.71 & 81.48 \\
    50    & 0.021 & 0.00  & 0.02  & 0.00  & 94.12 & 93.73 & 81.49 \\
    51    & 0.02  & 0.00  & 0.00  & 0.01  & 94.12 & 93.74 & 81.50 \\
    52    & 0.02  & 0.02  & 0.32  & 1.33  & 94.14 & 94.06 & 82.83 \\
    53    & 0.02  & 0.01  & 0.01  & 0.13  & 94.15 & 94.06 & 82.95 \\
    54    & 0.02  & 0.87  & 0.01  & 0.04  & 95.02 & 94.07 & 83.00 \\
    55    & 0.019 & 0.03  & 1.48  & 0.32  & 95.05 & 95.55 & 83.31 \\
    56    & 0.018 & 0.00  & 0.01  & 0.42  & 95.05 & 95.56 & 83.73 \\
    57    & 0.018 & 0.00  & 0.00  & 0.00  & 95.05 & 95.56 & 83.73 \\
    58    & 0.018 & 0.07  & 0.28  & 1.23  & 95.11 & 95.84 & 84.97 \\
    59    & 0.017 & 0.80  & 0.01  & 0.00  & 95.92 & 95.84 & 84.97 \\
    60    & 0.017 & 0.02  & 1.00  & 0.28  & 95.94 & 96.85 & 85.25 \\
    61    & 0.016 & 0.01  & 0.00  & 1.24  & 95.95 & 96.85 & 86.49 \\
    62    & 0.016 & 0.79  & 0.00  & 0.07  & 96.75 & 96.85 & 86.56 \\
    63    & 0.016 & 0.00  & 0.44  & 0.37  & 96.75 & 97.29 & 86.92 \\
    64    & 0.015 & 0.00  & 0.01  & 0.41  & 96.75 & 97.30 & 87.34 \\
    65    & 0.015 & 0.26  & 0.02  & 0.56  & 97.01 & 97.32 & 87.90 \\
    66    & 0.015 & 0.63  & 0.00  & 0.39  & 97.63 & 97.32 & 88.29 \\
    67    & 0.014 & 0.00  & 0.16  & 0.24  & 97.64 & 97.48 & 88.53 \\
    68    & 0.014 & 0.44  & 0.10  & 0.43  & 98.08 & 97.58 & 88.96 \\
    69    & 0.014 & 0.31  & 0.11  & 0.84  & 98.38 & 97.69 & 89.80 \\
    70    & 0.013 & 0.01  & 0.00  & 0.24  & 98.39 & 97.69 & 90.04 \\
\hline
\end{longtable}

\newpage
\subsubsection{Cortes basales}

La tabla de cortes basales (Tabla \ref{tabla-cortes-basales}) fue obtenida a partir del resultado \textit{Story forces} de ETABS considerando un espectro sísmico inelástico \footnote{Reducido por el factor $1/R*$.}.

\begin{table}[H]
  \centering
  \caption{Cortes basales.}
  \resizebox{1\textwidth}{!}{
  
  \begin{tabular}{ccccccc}
    \hline
    \textbf{Combinación} & \textbf{P [tonf]} & \cellcolor[rgb]{ .949,  .949,  .949}\textbf{Vx [tonf]} & \cellcolor[rgb]{ .949,  .949,  .949}\textbf{Vy [tonf]} & \textbf{T [tonf-m]} & \textbf{Mx [tonf-m]} & \textbf{My [tonf-m]} \bigstrut\\
    \hline
    PP    & 11041.9 & \cellcolor[rgb]{ .949,  .949,  .949}0.0 & \cellcolor[rgb]{ .949,  .949,  .949}0.0 & 0.0   & 185690.8 & -175463.3 \bigstrut[t]\\
    SC    & 2885.7 & \cellcolor[rgb]{ .949,  .949,  .949}0.0 & \cellcolor[rgb]{ .949,  .949,  .949}0.0 & 0.0   & 50485.4 & -44740.7 \\
    SX Max & 0.0   & \cellcolor[rgb]{ .949,  .949,  .949}705.6 & \cellcolor[rgb]{ .949,  .949,  .949}85.4 & 12791.9 & 2422.1 & 20305.0 \\
    SY Max & 0.0   & \cellcolor[rgb]{ .949,  .949,  .949}71.5 & \cellcolor[rgb]{ .949,  .949,  .949}705.9 & 14354.7 & 24152.0 & 2140.1 \\
    PPSC  & 13927.6 & \cellcolor[rgb]{ .949,  .949,  .949}0.0 & \cellcolor[rgb]{ .949,  .949,  .949}0.0 & 0.0   & 236176.2 & -220204.0 \\
    C1    & 15458.7 & \cellcolor[rgb]{ .949,  .949,  .949}0.0 & \cellcolor[rgb]{ .949,  .949,  .949}0.0 & 0.0   & 259967.1 & -245648.6 \\
    C2    & 17867.3 & \cellcolor[rgb]{ .949,  .949,  .949}0.0 & \cellcolor[rgb]{ .949,  .949,  .949}0.0 & 0.0   & 303605.6 & -282141.1 \\
    C3 Max & 9937.7 & \cellcolor[rgb]{ .949,  .949,  .949}987.9 & \cellcolor[rgb]{ .949,  .949,  .949}119.6 & 17908.7 & 170512.6 & -129489.9 \\
    C4 Max & 9937.7 & \cellcolor[rgb]{ .949,  .949,  .949}987.9 & \cellcolor[rgb]{ .949,  .949,  .949}119.6 & 17908.7 & 170512.6 & -129489.9 \\
    C5 Max & 9937.7 & \cellcolor[rgb]{ .949,  .949,  .949}100.1 & \cellcolor[rgb]{ .949,  .949,  .949}988.3 & 20096.5 & 200934.5 & -154920.9 \\
    C6 Max & 9937.7 & \cellcolor[rgb]{ .949,  .949,  .949}100.1 & \cellcolor[rgb]{ .949,  .949,  .949}988.3 & 20096.5 & 200934.5 & -154920.9 \\
    C7 Max & 16136.0 & \cellcolor[rgb]{ .949,  .949,  .949}987.9 & \cellcolor[rgb]{ .949,  .949,  .949}119.6 & 17908.7 & 276705.2 & -226869.6 \\
    C8 Max & 16136.0 & \cellcolor[rgb]{ .949,  .949,  .949}987.9 & \cellcolor[rgb]{ .949,  .949,  .949}119.6 & 17908.7 & 276705.2 & -226869.6 \\
    C9 Max & 16136.0 & \cellcolor[rgb]{ .949,  .949,  .949}100.1 & \cellcolor[rgb]{ .949,  .949,  .949}988.3 & 20096.5 & 307127.1 & -252300.5 \\
    C10 Max & 16136.0 & \cellcolor[rgb]{ .949,  .949,  .949}100.1 & \cellcolor[rgb]{ .949,  .949,  .949}988.3 & 20096.5 & 307127.1 & -252300.5 \\
    ENVC Max & 17867.3 & \cellcolor[rgb]{ .949,  .949,  .949}987.9 & \cellcolor[rgb]{ .949,  .949,  .949}988.3 & 20096.5 & 307127.1 & -129489.9 \bigstrut[b]\\
    \hline
  \end{tabular}
  }
  \label{tabla-cortes-basales}
\end{table}


% 9 Cortes basales Max y Min [Mauro] [OK]
% 10 Cortes basales X e Y [Mauro] [OK]
% 11 Control de deformaciones en las dos direcciones drift piso [Nacho] [OK]
% Tabla deformación máxima --> 1.3Sd [Pablo]
\newpage
\section{Cortes Basales y Control de Deformaciones}

    \subsection{Cortes basales}
    
        De acuerdo a lo indicado en el punto 6.2.3 de la NCh433, se determinan los cortes basales máximos y mínimos mostrados a continuación:
        
        \begin{table}[H]
            \centering
            \caption{Cortes basales máximo y mínimo.}
                \begin{tabular}{cc}
                    \hline
                          & \textbf{Cortes basal norma [tonf]} \bigstrut\\
                    \hline
                    $Q_{min}$ & 705,800 \bigstrut[t]\\
                    $Q_{max}$ & 1482,180 \bigstrut[b]\\
                    \hline
                \end{tabular}%
            \label{cortex_max_min}%
        \end{table}%
        
        Del modelo se determina el corte basal efectivo. En la primera iteración este resultó ser menor al mínimo por lo que se corrigió el espectro, resultando finalmente lo mostrado en la siguiente tabla:
        
        \begin{table}[htbp]
          \centering
          \caption{Cortes basales efectivos.}
            \begin{tabular}{cc}
            \toprule
                  & \textbf{Corte efectivo [tonf]} \\
            \midrule
            $Q_x$ & 705,613 \\
            $Q_y$ & 705,897 \\
            \bottomrule
            \end{tabular}%
          \label{cortes}%
        \end{table}%

    \newpage
    \subsection{Control de deformaciones}
    A continuación, se muestran los desplazamientos entre pisos, sin considerar el efecto del factor de reducción, considerando la losa como un elemento completamente rígido. Se debe tener en consideración que este valor no puede superar $0.001 \cdot H_{entrepiso}$.
    
    \begin{table}[H]
  \centering
  \caption{Verificación de drifts utlizando centro de masa, sismo y desplazamiento en eje X. }
  \resizebox{\textwidth}{!}{%
  \begin{tabular}{cccccccc}
    \hline
    \multicolumn{1}{c}{\textbf{Story}} &
      \multicolumn{1}{c}{\textbf{UX [m]}} &
      \multicolumn{1}{c}{\textbf{X [m]}} &
      \multicolumn{1}{c}{\textbf{Y [m]}} &
      \multicolumn{1}{c}{\textbf{Z [m]}} &
      \multicolumn{1}{c}{\textbf{Drift [-]}} &
      \multicolumn{1}{c}{\textbf{Admisible [-]}} &
      \multicolumn{1}{c}{\textbf{Porcentaje [\%]}}
      \bigstrut\\
    \hline
    Cubierta &
      0,123 &
      18,610 &
      15,900 &
      63,38 &
      - &
      - &
      -
      \bigstrut[t]\\
    24 &
      0,118 &
      19,870 &
      16,364 &
      61,33 &
      0,002 &
      0,0021 &
      121,6
      \\
    23 &
      0,112 &
      16,314 &
      16,339 &
      58,88 &
      0,002 &
      0,0025 &
      91,4
      \\
    22 &
      0,107 &
      16,053 &
      16,409 &
      56,43 &
      0,002 &
      0,0025 &
      93,3
      \\
    21 &
      0,101 &
      16,053 &
      16,409 &
      53,98 &
      0,002 &
      0,0025 &
      93,5
      \\
    20 &
      0,095 &
      16,053 &
      16,409 &
      51,53 &
      0,002 &
      0,0025 &
      94,8
      \\
    19 &
      0,090 &
      16,053 &
      16,409 &
      49,08 &
      0,002 &
      0,0025 &
      95,9
      \\
    18 &
      0,084 &
      16,053 &
      16,409 &
      46,63 &
      0,002 &
      0,0025 &
      96,7
      \\
    17 &
      0,078 &
      16,053 &
      16,409 &
      44,18 &
      0,002 &
      0,0025 &
      97,1
      \\
    16 &
      0,072 &
      16,053 &
      16,409 &
      41,73 &
      0,002 &
      0,0025 &
      97,2
      \\
    15 &
      0,066 &
      16,053 &
      16,409 &
      39,28 &
      0,002 &
      0,0025 &
      96,9
      \\
    14 &
      0,061 &
      16,053 &
      16,409 &
      36,83 &
      0,002 &
      0,0025 &
      96,2
      \\
    13 &
      0,055 &
      16,053 &
      16,409 &
      34,38 &
      0,002 &
      0,0025 &
      95,2
      \\
    12 &
      0,049 &
      16,053 &
      16,409 &
      31,93 &
      0,002 &
      0,0025 &
      93,1
      \\
    11 &
      0,044 &
      16,053 &
      16,409 &
      29,48 &
      0,002 &
      0,0025 &
      91,5
      \\
    10 &
      0,038 &
      16,053 &
      16,409 &
      27,03 &
      0,002 &
      0,0025 &
      89,5
      \\
    9 &
      0,033 &
      16,053 &
      16,409 &
      24,58 &
      0,002 &
      0,0025 &
      87,2
      \\
    8 &
      0,028 &
      16,053 &
      16,409 &
      22,13 &
      0,002 &
      0,0025 &
      84,2
      \\
    7 &
      0,023 &
      16,062 &
      16,420 &
      19,68 &
      0,002 &
      0,0025 &
      80,4
      \\
    6 &
      0,019 &
      16,070 &
      16,430 &
      17,23 &
      0,002 &
      0,0025 &
      74,9
      \\
    5 &
      0,015 &
      16,070 &
      16,430 &
      14,78 &
      0,002 &
      0,0025 &
      70,6
      \\
    4 &
      0,011 &
      16,072 &
      16,437 &
      12,33 &
      0,002 &
      0,0025 &
      65,6
      \\
    3 &
      0,007 &
      16,070 &
      16,430 &
      9,88 &
      0,001 &
      0,0025 &
      59,4
      \\
    2 &
      0,004 &
      16,176 &
      16,550 &
      7,43 &
      0,001 &
      0,0025 &
      51,8
      \\
    1 &
      0,002 &
      11,555 &
      19,840 &
      4,98 &
      0,001 &
      0,0025 &
      39,3
      \\
    -1 &
      0,000 &
      15,476 &
      19,695 &
      2,5 &
      0,001 &
      0,0025 &
      21,4
      \bigstrut[b]\\
    \hline
  \end{tabular}}
  \label{sxCMD}
\end{table}

\begin{table}[H]
  \centering
  \caption{Verificación de drifts utlizando centro de masa, sismo y desplazamiento en eje Y.}
  \resizebox{\textwidth}{!}{%
  \begin{tabular}{cccccccc}
    \hline
    \multicolumn{1}{c}{\textbf{Story}} &
      \multicolumn{1}{c}{\textbf{UY [m]}} &
      \multicolumn{1}{c}{\textbf{X [m]}} &
      \multicolumn{1}{c}{\textbf{Y [m]}} &
      \multicolumn{1}{c}{\textbf{Z [m]}} &
      \multicolumn{1}{c}{\textbf{Drift [-]}} &
      \multicolumn{1}{c}{\textbf{Admisible [-]}} &
      \multicolumn{1}{c}{\textbf{Porcentaje [\%]}}
      \bigstrut\\
    \hline
    Cubierta &
      0,0754 &
      18,610 &
      15,900 &
      63,38 &
      - &
      - &
      -
      \bigstrut[t]\\
    24 &
      0,0778 &
      19,870 &
      16,364 &
      61,33 &
      -0,001 &
      0,00205 &
      57,1
      \\
    23 &
      0,0709 &
      16,314 &
      16,339 &
      58,88 &
      0,003 &
      0,00245 &
      115,5
      \\
    22 &
      0,0684 &
      16,053 &
      16,409 &
      56,43 &
      0,001 &
      0,00245 &
      41,3
      \\
    21 &
      0,0657 &
      16,053 &
      16,409 &
      53,98 &
      0,001 &
      0,00245 &
      45,1
      \\
    20 &
      0,0628 &
      16,053 &
      16,409 &
      51,53 &
      0,001 &
      0,00245 &
      47,8
      \\
    19 &
      0,0598 &
      16,053 &
      16,409 &
      49,08 &
      0,001 &
      0,00245 &
      50,4
      \\
    18 &
      0,0566 &
      16,053 &
      16,409 &
      46,63 &
      0,001 &
      0,00245 &
      52,9
      \\
    17 &
      0,0533 &
      16,053 &
      16,409 &
      44,18 &
      0,001 &
      0,00245 &
      55,2
      \\
    16 &
      0,0499 &
      16,053 &
      16,409 &
      41,73 &
      0,001 &
      0,00245 &
      57,3
      \\
    15 &
      0,0463 &
      16,053 &
      16,409 &
      39,28 &
      0,001 &
      0,00245 &
      59,1
      \\
    14 &
      0,0427 &
      16,053 &
      16,409 &
      36,83 &
      0,001 &
      0,00245 &
      60,6
      \\
    13 &
      0,0390 &
      16,053 &
      16,409 &
      34,38 &
      0,002 &
      0,00245 &
      61,9
      \\
    12 &
      0,0353 &
      16,053 &
      16,409 &
      31,93 &
      0,002 &
      0,00245 &
      61,5
      \\
    11 &
      0,0316 &
      16,053 &
      16,409 &
      29,48 &
      0,002 &
      0,00245 &
      61,9
      \\
    10 &
      0,0278 &
      16,053 &
      16,409 &
      27,03 &
      0,002 &
      0,00245 &
      62,1
      \\
    9 &
      0,0241 &
      16,053 &
      16,409 &
      24,58 &
      0,002 &
      0,00245 &
      61,7
      \\
    8 &
      0,0205 &
      16,053 &
      16,409 &
      22,13 &
      0,001 &
      0,00245 &
      60,8
      \\
    7 &
      0,0169 &
      16,062 &
      16,420 &
      19,68 &
      0,001 &
      0,00245 &
      59,0
      \\
    6 &
      0,0137 &
      16,070 &
      16,430 &
      17,23 &
      0,001 &
      0,00245 &
      54,8
      \\
    5 &
      0,0105 &
      16,070 &
      16,430 &
      14,78 &
      0,001 &
      0,00245 &
      52,2
      \\
    4 &
      0,0076 &
      16,072 &
      16,437 &
      12,33 &
      0,001 &
      0,00245 &
      48,8
      \\
    3 &
      0,0049 &
      16,070 &
      16,430 &
      9,88 &
      0,001 &
      0,00245 &
      44,1
      \\
    2 &
      0,0027 &
      16,176 &
      16,550 &
      7,43 &
      0,001 &
      0,00245 &
      37,6
      \\
    1 &
      0,0009 &
      11,555 &
      19,840 &
      4,98 &
      0,001 &
      0,00245 &
      30,2
      \\
    -1 &
      0,0002 &
      15,476 &
      19,695 &
      2,5 &
      0,000 &
      0,00248 &
      11,2
      \bigstrut[b]\\
    \hline
  \end{tabular}}
  \label{syCMD}
\end{table}

%poner texto
Por otro lado, utilizando la información directa de los Story Drifts obtenidos del modelo, los cuales no usan el supuesto de losa rigida, se comparan los drifts con su deformación admisible considerando este último como $0.002 \cdot H_{entrepiso}$.
\begin{table}[H]
  \centering
  \caption{Verificación de drifts utlizando deformaciones máximas y sismo en eje X.}
  \resizebox{\textwidth}{!}{%
  \begin{tabular}{ccccccc}
    \hline
    \multicolumn{1}{c}{\textbf{Story}} &
      \multicolumn{1}{c}{\textbf{X [m]}} &
      \multicolumn{1}{c}{\textbf{Y [m]}} &
      \multicolumn{1}{c}{\textbf{Z [m]}} &
      \multicolumn{1}{c}{\textbf{Drift [-]}} &
      \multicolumn{1}{c}{\textbf{Admisible [-]}} &
      \multicolumn{1}{c}{\textbf{Porcentaje [\%]}}
      \bigstrut\\
    \hline
    Cubierta &
      20,63 &
      17 &
      63,38 &
      0,002 &
      - &
      -
      \bigstrut[t]\\
    24 &
      16,59 &
      20,65 &
      61,33 &
      0,002 &
      0,0041 &
      60,3
      \\
    23 &
      17,21 &
      31,83 &
      58,88 &
      0,003 &
      0,0049 &
      69,4
      \\
    22 &
      11,14 &
      32,43 &
      56,43 &
      0,004 &
      0,0049 &
      73,6
      \\
    21 &
      11,14 &
      32,43 &
      53,98 &
      0,004 &
      0,0049 &
      77,0
      \\
    20 &
      11,14 &
      32,43 &
      51,53 &
      0,004 &
      0,0049 &
      80,0
      \\
    19 &
      11,14 &
      32,43 &
      49,08 &
      0,004 &
      0,0049 &
      82,5
      \\
    18 &
      11,14 &
      32,43 &
      46,63 &
      0,004 &
      0,0049 &
      84,4
      \\
    17 &
      11,14 &
      32,43 &
      44,18 &
      0,004 &
      0,0049 &
      85,5
      \\
    16 &
      11,14 &
      32,43 &
      41,73 &
      0,004 &
      0,0049 &
      86,0
      \\
    15 &
      11,14 &
      32,43 &
      39,28 &
      0,004 &
      0,0049 &
      85,8
      \\
    14 &
      11,14 &
      32,43 &
      36,83 &
      0,004 &
      0,0049 &
      84,8
      \\
    13 &
      11,14 &
      32,43 &
      34,38 &
      0,004 &
      0,0049 &
      82,2
      \\
    12 &
      11,14 &
      32,43 &
      31,93 &
      0,004 &
      0,0049 &
      81,1
      \\
    11 &
      11,14 &
      32,43 &
      29,48 &
      0,004 &
      0,0049 &
      79,9
      \\
    10 &
      11,14 &
      32,43 &
      27,03 &
      0,004 &
      0,0049 &
      78,4
      \\
    9 &
      11,14 &
      32,43 &
      24,58 &
      0,004 &
      0,0049 &
      76,4
      \\
    8 &
      11,14 &
      32,43 &
      22,13 &
      0,004 &
      0,0049 &
      73,5
      \\
    7 &
      11,14 &
      32,43 &
      19,68 &
      0,003 &
      0,0049 &
      69,5
      \\
    6 &
      11,14 &
      32,43 &
      17,23 &
      0,003 &
      0,0049 &
      64,9
      \\
    5 &
      11,14 &
      32,43 &
      14,78 &
      0,003 &
      0,0049 &
      58,8
      \\
    4 &
      11,14 &
      32,43 &
      12,33 &
      0,002 &
      0,0049 &
      50,7
      \\
    3 &
      20,3 &
      0 &
      9,88 &
      0,002 &
      0,0049 &
      39,9
      \\
    2 &
      15,79 &
      1,36 &
      7,43 &
      0,002 &
      0,0049 &
      31,1
      \\
    1 &
      22,95 &
      7,7 &
      4,98 &
      0,001 &
      0,0049 &
      19,5
      \\
    -1 &
      21,04 &
      39,44 &
      2,5 &
      0,000 &
      0,0050 &
      2,5
      \bigstrut[b]\\
    \hline
  \end{tabular}}
  \label{sxDD}
\end{table}


\begin{table}[H]
  \centering
  \caption{Verificación de drifts utlizando deformaciones máximas y sismo en eje Y.}
  \resizebox{\textwidth}{!}{%
  \begin{tabular}{ccccccc}
    \hline
    \multicolumn{1}{c}{\textbf{Story}} &
      \multicolumn{1}{c}{\textbf{X [m]}} &
      \multicolumn{1}{c}{\textbf{Y [m]}} &
      \multicolumn{1}{c}{\textbf{Z [m]}} &
      \multicolumn{1}{c}{\textbf{Drift [-]}} &
      \multicolumn{1}{c}{\textbf{Admisible [-]}} &
      \multicolumn{1}{c}{\textbf{Porcentaje [\%]}}
      \bigstrut\\
    \hline
    Cubierta &
      16,59 &
      17 &
      63,38 &
      0,001 &
      - &
      -
      \bigstrut[t]\\
    24 &
      16,59 &
      20,65 &
      61,33 &
      0,001 &
      0,0041 &
      24,4
      \\
    23 &
      9,03 &
      21,9 &
      58,88 &
      0,001 &
      0,0049 &
      26,2
      \\
    22 &
      9,03 &
      21,9 &
      56,43 &
      0,001 &
      0,0049 &
      27,9
      \\
    21 &
      22,95 &
      17 &
      53,98 &
      0,001 &
      0,0049 &
      30,0
      \\
    20 &
      22,95 &
      12,7 &
      51,53 &
      0,002 &
      0,0049 &
      32,6
      \\
    19 &
      22,95 &
      17 &
      49,08 &
      0,002 &
      0,0049 &
      34,8
      \\
    18 &
      22,95 &
      17 &
      46,63 &
      0,002 &
      0,0049 &
      36,7
      \\
    17 &
      22,95 &
      17 &
      44,18 &
      0,002 &
      0,0049 &
      38,2
      \\
    16 &
      22,95 &
      17 &
      41,73 &
      0,002 &
      0,0049 &
      39,4
      \\
    15 &
      22,95 &
      17 &
      39,28 &
      0,002 &
      0,0049 &
      40,1
      \\
    14 &
      22,95 &
      17 &
      36,83 &
      0,002 &
      0,0049 &
      40,2
      \\
    13 &
      22,95 &
      17 &
      34,38 &
      0,002 &
      0,0049 &
      38,4
      \\
    12 &
      22,95 &
      17 &
      31,93 &
      0,002 &
      0,0049 &
      38,3
      \\
    11 &
      22,95 &
      12,7 &
      29,48 &
      0,002 &
      0,0049 &
      38,2
      \\
    10 &
      22,95 &
      17 &
      27,03 &
      0,002 &
      0,0049 &
      38,1
      \\
    9 &
      22,95 &
      17 &
      24,58 &
      0,002 &
      0,0049 &
      37,7
      \\
    8 &
      22,95 &
      17 &
      22,13 &
      0,002 &
      0,0049 &
      37,0
      \\
    7 &
      22,95 &
      17 &
      19,68 &
      0,002 &
      0,0049 &
      35,7
      \\
    6 &
      22,95 &
      17 &
      17,23 &
      0,002 &
      0,0049 &
      34,3
      \\
    5 &
      22,95 &
      17 &
      14,78 &
      0,002 &
      0,0049 &
      32,4
      \\
    4 &
      22,95 &
      17 &
      12,33 &
      0,001 &
      0,0049 &
      29,6
      \\
    3 &
      22,95 &
      17 &
      9,88 &
      0,001 &
      0,0049 &
      25,3
      \\
    2 &
      17,21 &
      0,7 &
      7,43 &
      0,001 &
      0,0049 &
      22,3
      \\
    1 &
      22,95 &
      9,62 &
      4,98 &
      0,001 &
      0,0049 &
      14,5
      \\
    -1 &
      31,19 &
      1,36 &
      2,5 &
      0,000 &
      0,00496 &
      1,9
      \bigstrut[b]\\
    \hline
  \end{tabular}}
  \label{syDD}
\end{table}

\newpage
\subsection{Gráficos de respuesta}

\subsubsection{Corte por piso}

\begin{images}[\label{corte-piso}]{Cortes por piso.}
\addimage{corte-sismox}{width=7cm}{Sismo x.}
\addimage{corte-sismoy}{width=7cm}{Sismo y.}
\end{images}

\subsubsection{Momento volcante}

\begin{images}{Momento volcante por piso.}
\addimage{momento-sismox}{width=7cm}{Sismo x.}
\addimage{momento-sismoy}{width=7cm}{Sismo y.}
\end{images}

\newpage
\subsubsection{Desplazamiento entre piso}

\begin{images}{Desplazamiento entre piso.}
\addimage{despl-sismox}{width=7cm}{Sismo x.}
\addimage{despl-sismoy}{width=7cm}{Sismo y.}
\end{images}

\subsubsection{Drift entre piso}

\begin{images}{Drift entre piso.}
\addimage{drift-sismox}{width=7cm}{Sismo x.}
\addimage{drift-sismoy}{width=7cm}{Sismo y.}
\end{images}

% 12 Verificación de Rigidez H/T [Mauro] [OK]
% 13 Indicador acoplamiento MATRIZ [Nacho] [OK]
\newpage
\section{Indicadores de Rigidez y Acoplamiento}

    \subsection{Indicador de Rigidez}
        De la verificación de rigidez se obtiene el siguiente resultado:
        \insertequationanum{H/T= \frac{58,88}{1,345} = 43,8}
        
        \begin{table}[H]
          \centering
          \caption{Verificación de rigidez del edificio.}
            \begin{tabular}{cc}
            \toprule
            \textbf{H/T}   & \textbf{Nivel de Rigidez} \\
            \midrule
            20-40 & Flexible \\
            40-70 & Normal \\
            70-150 & Rígido \\
            < 20  & Muy Flexible \\
            > 150 & Muy Rígido \\
            \bottomrule
            \end{tabular}%
          \label{rigidez}%
        \end{table}%
        
        De acuerdo a la tabla \ref{rigidez} se determina que el edificio es de rigidez \textit{Normal}.
        
    \subsection{Indicador de Acoplamiento}
    De la verificación de Acoplamiento se obtiene el siguiente resultado:
    
    \begin{table}[H]
  \centering
  \caption{Indicadores de acoplamiento.}
  \begin{tabular}{cc}
    \hline
    \textbf{Acoplamiento} &
      \textbf{Valor (-)}
      \bigstrut\\
    \hline
    T1/T2 &
      1,8
      \bigstrut[t]\\
    T1/T3 &
      1,2
      \\
    T3/T2 &
      1,5
      \bigstrut[b]\\
    \hline
  \end{tabular}
  \label{IndAcpl}
\end{table}

Como se puede apreciar en la tabla anterior, todos los factores obtenidos son mayores o iguales que 1.2 como lo establece la norma.

% 14 Entrega modelo sísmico funcionando. [OK]
% 15 Conclusiones y Comentarios para al menos: Deformaciones, Rigidez, Acoplamiento, Cortes basales
% - concluir drift piso < 0.002
% - periodo torsional, concluir FT suelo roca, poca amplificacion, respuesta periodo similar
% - 
\newpage
\section{Comentarios y Conclusiones}

En cuanto al período predominante se obtuvo un primer modo torsional, si bien esto no es lo deseable hay que destacar que el período obtenido está lejos del período en que el suelo presenta mayores aceleraciones (el cual es cercano a los 0.2s), por tanto el contenido de frecuencias de un sismo no proporcionará energía suficiente a la estructura para excitar un modo torsional. Por otro lado en cuanto al período en $X$ e $Y$ se obtuvieron igualmente períodos altos en comparación a la zona de amplificación, lo cual es deseable dado que la mayor masa del edificio se tributará en periodos para los cuales la estructura recibe menos energía (aceleración).
    
\insertimage{periodos}{width=12cm}{Períodos de los primeros períodos junto al espectro de aceleraciones reducido.}

Para lograr el 90\% de masa trasladada en cada dirección de análisis se necesitó de un total de 70 modos, esto conversa muy bien con la hipótesis de diseño inicial la cual consideró un diafragma rígido, requiriendo así de un total de tres grados de libertad por cada piso para representar los desplazamientos de la estructura ($u_x$, $u_y$ desplazamientos horizontales en planta y $\theta$ ángulo de giro) sin mayores diferencias con la realidad.  Así, teóricamente con un total de $3 \cdot 23 = 69$ modos se logra un buen nivel de confianza, resultado muy cercano al obtenido en la práctica. \\

Con respecto a los desplazamientos obtenidos por piso, se puede apreciar como estos son mayores por efecto del sismo en el eje $X$. Lo anterior se debe a la baja inercia que posee la estructura en este eje en comparación con el eje $Y$. Sin embargo, es posible apreciar que estos desplazamientos son menores que los admisibles o solicitados por norma.

Los casos puntuales de los pisos más superiores donde esto no se cumple es prácticamente por la condición de apoyo libre que existe a nivel de la cubierta. \\
    
Dado que los factores de acoplamiento obtenidos en la Tabla \ref{IndAcpl} son mayores o iguales que lo indicado en la norma, se puede concluir que no habrá efecto de acoplamiento entre los modos obtenidos. \\

Al analizar los gráficos de distribución del cortante por piso (Figuras \ref{corte-piso}.a y \ref{corte-piso}.b) es posible observar que en cuanto al sismo en el eje $Y$ la distribución es relativamente homogénea, ello significa que el corte se tributa de igual manera en cada piso, por tanto en dicho eje la rigidez (dado el hormigón y espesor de muros elegidos) fue la adecuada. \\

Distinto es en el caso del sismo en x, en donde en los pisos intermedios de la estructura (concretamente entre el 17 y el 12) toman poco corte, tributándolo a los pisos inferiores de la estructura. Esto indica que dichos pisos en ese eje resultaron muy rígidos, disminuyendo por tanto la distorsión angular (Mayor módulo de rigidez, mayor módulo de corte, menor distorsión angular a un mismo desplazamiento), tomando así menos corte dado la relación $\tau = G \cdot \gamma$. Una posible solución es disminuir el espesor de los muros, verificando que dicho cambio no afecte al drift entre pisos.

% FIN DEL DOCUMENTO
\end{document}
