\newpage
\section{Indicadores de Rigidez y Acoplamiento}

    \subsection{Indicador de Rigidez}
        De la verificación de rigidez se obtiene el siguiente resultado:
        \insertequationanum{H/T= \frac{58,88}{1,345} = 43,8}
        
        \begin{table}[H]
          \centering
          \caption{Verificación de rigidez del edificio.}
            \begin{tabular}{cc}
            \toprule
            \textbf{H/T}   & \textbf{Nivel de Rigidez} \\
            \midrule
            20-40 & Flexible \\
            40-70 & Normal \\
            70-150 & Rígido \\
            < 20  & Muy Flexible \\
            > 150 & Muy Rígido \\
            \bottomrule
            \end{tabular}%
          \label{rigidez}%
        \end{table}%
        
        De acuerdo a la tabla \ref{rigidez} se determina que el edificio es de rigidez \textit{Normal}.
        
    \subsection{Indicador de Acoplamiento}
    De la verificación de Acoplamiento se obtiene el siguiente resultado:
    
    \begin{table}[H]
  \centering
  \caption{Indicadores de acoplamiento.}
  \begin{tabular}{cc}
    \hline
    \textbf{Acoplamiento} &
      \textbf{Valor (-)}
      \bigstrut\\
    \hline
    T1/T2 &
      1,8
      \bigstrut[t]\\
    T1/T3 &
      1,2
      \\
    T3/T2 &
      1,5
      \bigstrut[b]\\
    \hline
  \end{tabular}
  \label{IndAcpl}
\end{table}

Como se puede apreciar en la tabla anterior, todos los factores obtenidos son mayores o iguales que 1.2 como lo establece la norma.